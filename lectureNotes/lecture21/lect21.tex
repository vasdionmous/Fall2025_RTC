\documentclass[12pt,a4paper,reqno]{amsart}

% language
\usepackage[greek,english]{babel}
\usepackage[utf8]{inputenc}
\usepackage{alphabeta}

% change default names to greek
\addto\captionsenglish{
    \renewcommand{\contentsname}{Περιεχόμενα}
    \renewcommand{\refname}{Βιβλιογραφία}
    \renewcommand{\datename}{Ημερομηνία:}
    \renewcommand{\urladdrname}{Ιστοσελίδα}
}

% math 
\usepackage{amsmath,amsthm,amssymb,amscd}

% font
\usepackage[cal=euler]{mathalfa}
\usepackage{libertinus-type1}
% \usepackage{txfonts} % for upright greek letters
\usepackage{bm} % for bold symbols
\usepackage{bbm} % for the simply-looking bb symbols

% miscellaneous 
\usepackage{changepage} %for indenting environments
\usepackage{csquotes} % example: \textcquote{}
\usepackage{blkarray}
\setcounter{MaxMatrixCols}{20} % default for pmatrix is 10!!
\usepackage{ytableau}
\usepackage{array} %needed to increase the vertical length in a tabular

% drawing
\usepackage{tikz,tikz-cd}
\usetikzlibrary{shapes.misc, patterns, matrix, calc, intersections,positioning}
\usepackage{graphics,graphicx}
\usepackage{float} % provides enhanced control and customization options for floating objects such as figures and tables

% colors
\usepackage{xcolor}
\definecolor{darkcandyapplered}{rgb}{0.64, 0.0, 0.0}
\definecolor{midnightblue}{rgb}{0.1, 0.1, 0.44}
\definecolor{mylightblue}{HTML}{336699}
\definecolor{burntorange}{rgb}{0.8, 0.33, 0.0}
\definecolor{iceberg}{rgb}{0.44, 0.65, 0.82}
\definecolor{applegreen}{rgb}{0.55, 0.71, 0.0}
\definecolor{canaryyellow}{rgb}{1.0, 0.94, 0.0}

% hrefs
\usepackage{hyperref}
\usepackage[noabbrev,capitalize]{cleveref}
\hypersetup{
    pdftoolbar=true,        
    pdfmenubar=true,        
    pdffitwindow=false,     
    pdfstartview={FitH},  % fits the width of the page to the window
    pdftitle={},
    pdfauthor={},
    pdfsubject={},
    pdfkeywords={},
    pdfnewwindow=true,  % links in new window
    colorlinks=true,  % false: boxed links; true: colored links
    linkcolor=darkcandyapplered,   % color of internal links
    citecolor=midnightblue,  % color of links to bibliography
    urlcolor=cyan,  % color of external links
    linktocpage=true  % changes the links from the section body to the page number
    }

% geometry
\textwidth=16cm 
\textheight=21cm 
\hoffset=-55pt 
\footskip=25pt

% thm envs (you might need to change the path)
% In this macro I define all the theorem environments

\theoremstyle{definition}
\newtheorem{theorem}{Θεώρημα}
\newtheorem{proposition}[theorem]{Πρόταση}
\newtheorem{lemma}[theorem]{Λήμμα}
\newtheorem{corollary}[theorem]{Πόρισμα}
\newtheorem{conjecture}[theorem]{Εικασία}
\newtheorem{problem}[theorem]{Πρόβλημα}
\newtheorem*{claim}{Ισχυρισμός}
\newtheorem{observation}[theorem]{Παρατήρηση}
\newtheorem{definition}[theorem]{Ορισμός}
\newtheorem{question}[theorem]{Ερώτηση}
\newtheorem*{questions}{Ερωτήματα}
\newtheorem{example}[theorem]{Παράδειγμα}
\newtheorem{exercise}{Άσκηση}

\newtheorem*{combInterlude}{Ιντερλούδιο Συνδυαστικής}
\newtheorem*{example_cont}{Παράδειγμα~6.6}
\newtheorem*{digression_la}{Παρέκβαση Γραμμικής Άλγεβρας}
\newtheorem*{thm}{Θεώρημα}

\theoremstyle{remark}
\newtheorem*{remark}{Παρατήρηση}

% fixes the correct numbering of environments
\numberwithin{theorem}{section}
\numberwithin{exercise}{section}
\numberwithin{equation}{section}

% math ops (you might need to change the path)
% In this macro I define all of my math operators

% fields
\newcommand{\NN}{\mathbbmss{N}} 
\newcommand{\ZZ}{\mathbbmss{Z}} 
\newcommand{\QQ}{\mathbbmss{Q}} 
\newcommand{\RR}{\mathbbmss{R}} 
\newcommand{\CC}{\mathbbmss{C}} 
\newcommand{\KK}{\mathbbmss{K}} 
\newcommand{\FF}{\mathbbmss{F}} 

% symmetric group
\newcommand{\fS}{\mathfrak{S}}  

% calligraphic 
\newcommand{\aA}{\mathcal{A}} 
\newcommand{\bB}{\mathcal{B}}
\newcommand{\cC}{\mathcal{C}}
\newcommand{\dD}{\mathcal{D}}
\newcommand{\eE}{\mathcal{E}}
\newcommand{\fF}{\mathcal{F}}
\newcommand{\hH}{\mathcal{H}}
\newcommand{\iI}{\mathcal{I}}
\newcommand{\lL}{\mathcal{L}}
\newcommand{\oO}{\mathcal{O}}
\newcommand{\pP}{\mathcal{P}}
\newcommand{\sS}{\mathcal{S}}
\newcommand{\mM}{\mathcal{M}}
\newcommand{\uU}{\mathcal{U}}

% bold
\newcommand{\bfa}{\mathbf{a}}
\newcommand{\bfe}{\mathbf{e}}
\newcommand{\bfF}{\pmb{F}}
\newcommand{\bfR}{\pmb{R}}
\newcommand{\bfv}{\mathbf{v}}
%\newcommand{\bfx}{\bm{x}}
%\newcommand{\bfx}{\mathbf{x}} 
\newcommand{\bfx}{\pmb{x}}
\newcommand{\bfX}{\pmb{X}}
\newcommand{\bfy}{\pmb{y}}
\newcommand{\bfz}{\pmb{z}}

% roman
\newcommand{\rmA}{\mathrm{A}}
\newcommand{\rmB}{\mathrm{B}}
\newcommand{\rmC}{\mathrm{C}}
\newcommand{\rmD}{\mathrm{D}} 
\newcommand{\rmI}{\mathrm{I}} 
\newcommand{\rmK}{\mathrm{K}}
\newcommand{\rmM}{\mathrm{M}}
\newcommand{\rmP}{\mathrm{P}}  
\newcommand{\rmp}{\mathrm{p}}  
\newcommand{\rmQ}{\mathrm{Q}}  
\newcommand{\rmR}{\mathrm{R}}
\newcommand{\rmS}{\mathrm{S}}
\newcommand{\rmT}{\mathrm{T}}
\newcommand{\rmU}{\mathrm{U}}
\newcommand{\rmV}{\mathrm{V}}
\newcommand{\rmY}{\mathrm{Y}}
\newcommand{\rmZ}{\mathrm{Z}}
\newcommand{\rmz}{\mathrm{z}}

% greek letters
% I'm renewing some commands in order to appear in upright font
% If I want to change it later, I don't have to do it manually, I just change it from here.
% \newcommand{\uaa}{\alphaup}
% \renewcommand{\alpha}{\alphaup}
% \renewcommand{\beta}{\betaup}
% \renewcommand{\gamma}{\gammaup}
% \renewcommand{\delta}{\deltaup}
% \renewcommand{\epsilon}{\epsilonup}
% \newcommand{\ee}{\epsilon}
% \renewcommand{\varepsilon}{\varepsilonup}
% \renewcommand{\theta}{\thetaup}
% \renewcommand{\lambda}{\lambdaup}
% \newcommand{\ull}{\lambda}
% \renewcommand{\mu}{\muup}
% \renewcommand{\nu}{\nuup}
% \renewcommand{\pi}{\piup}
% \renewcommand{\rho}{\rhoup}
% \renewcommand{\varrho}{\varrhoup}
% \renewcommand{\sigma}{\sigmaup}
% \renewcommand{\tau}{\tauup} 
% \renewcommand{\phi}{\phiup}
% \renewcommand{\chi}{\chiup}
% \renewcommand{\psi}{\psiup}
% \renewcommand{\omega}{\omegaup}

% arrows and symbols 
\renewcommand{\to}{\rightarrow}
\newcommand{\toto}{\longrightarrow}
\newcommand{\mapstoto}{\longmapsto}
\newcommand{\then}{\Rightarrow}
\newcommand{\IFF}{\Leftrightarrow}
\newcommand{\tl}{\tilde}
\newcommand{\wtl}{\widetilde}
\newcommand{\ol}{\overline}
\newcommand{\ul}{\underline}
\newcommand{\oldemptyset}{\emptyset}
\renewcommand{\emptyset}{\varnothing}
\DeclareMathSymbol{\Arg}{\mathbin}{AMSa}{"39} % for arguments 
\newcommand{\onto}{\ensuremath{\twoheadrightarrow}}
\newcommand{\tle}{\trianglelefteq}
\newcommand{\tge}{\trianglerighteq}

% absolute value symbol
\usepackage{mathtools} 
\DeclarePairedDelimiter\abs{\lvert}{\rvert}%
\DeclarePairedDelimiter\norm{\lVert}{\rVert}%
\makeatletter
\let\oldabs\abs
\def\abs{\@ifstar{\oldabs}{\oldabs*}}

% tensor symbol
\newcommand{\tensor}[1]{%
  \mathbin{\mathop{\otimes}\limits_{#1}}%
}

% permutation cycle notation
\ExplSyntaxOn
\NewDocumentCommand{\cycle}{ O{\;} m }
 {
  (
  \alec_cycle:nn { #1 } { #2 }
  )
 }

\seq_new:N \l_alec_cycle_seq
\cs_new_protected:Npn \alec_cycle:nn #1 #2
 {
  \seq_set_split:Nnn \l_alec_cycle_seq { , } { #2 }
  \seq_use:Nn \l_alec_cycle_seq { #1 }
 }
\ExplSyntaxOff

% setminus symbol
\newcommand{\mysetminusD}{\hbox{\tikz{\draw[line width=0.6pt,line cap=round] (3pt,0) -- (0,6pt);}}}
\newcommand{\mysetminusT}{\mysetminusD}
\newcommand{\mysetminusS}{\hbox{\tikz{\draw[line width=0.45pt,line cap=round] (2pt,0) -- (0,4pt);}}}
\newcommand{\mysetminusSS}{\hbox{\tikz{\draw[line width=0.4pt,line cap=round] (1.5pt,0) -- (0,3pt);}}}
\newcommand{\sm}{\mathbin{\mathchoice{\mysetminusD}{\mysetminusT}{\mysetminusS}{\mysetminusSS}}}

% custom math operators
\newcommand{\Des}{\mathrm{Des}} 
\newcommand{\des}{\mathrm{des}} 
\newcommand{\Asc}{\mathrm{Asc}}
\newcommand{\asc}{\mathrm{asc}} 
\newcommand{\inv}{\mathrm{inv}}
\newcommand{\Inv}{\mathrm{Inv}}
\newcommand{\maj}{\mathrm{maj}} 
\newcommand{\comaj}{\mathrm{comaj}} 
\newcommand{\fix}{\mathrm{fix}} 
\newcommand{\Sym}{\mathrm{Sym}} 
\newcommand{\QSym}{\mathrm{QSym}}
\newcommand{\FQSym}{\mathrm{FQSym}} 
\newcommand{\End}{\mathrm{End}} 
\newcommand{\Rad}{\mathrm{Rad}} 
\newcommand{\rmMat}{\mathrm{Mat}} 
\newcommand{\rmdim}{\mathrm{dim}} 
\newcommand{\rmTop}{\mathrm{Top}} 
\newcommand{\rmCF}{\mathrm{CF}} 
\newcommand{\rmId}{\mathrm{Id}}
\newcommand{\rmid}{\mathrm{id}}
\newcommand{\rmtw}{\mathrm{tw}}
\newcommand{\trace}{\mathrm{tr}}
\newcommand{\Irr}{\mathrm{Irr}}
\newcommand{\Ind}{\mathrm{Ind}} % induction
\newcommand{\Res}{\mathrm{Res}} % restriction
\newcommand{\triv}{\mathrm{triv}} % trivial rep
\newcommand{\rmdef}{\mathrm{def}} % defining rep
\newcommand{\dom}{\triangleleft}
\newcommand{\domeq}{\trianglelefteq}
\newcommand{\lex}{\mathrm{lex}}
\newcommand{\sign}{\mathrm{sign}}
\newcommand{\SYT}{\mathrm{SYT}}
\renewcommand{\Im}{\mathrm{Im}}
\newcommand{\Ker}{\mathrm{Ker}}
\newcommand{\GL}{\mathrm{GL}}
\newcommand{\FL}{\mathrm{FL}}
\newcommand{\Span}{\mathrm{span}}
\newcommand{\pos}{\mathrm{pos}}
\newcommand{\Comp}{\mathrm{Comp}}
\newcommand{\Set}{\mathrm{Set}}
\newcommand{\std}{\mathrm{std}}
\newcommand{\cont}{\mathrm{cont}} %content of a SSYT
\newcommand{\SSYT}{\mathrm{SSYT}}
\newcommand{\ct}{\mathrm{ct}} % cycle type
\newcommand{\ch}{\mathrm{ch}} % Frobenius characteristic map
\newcommand{\height}{\mathrm{ht}}
\newcommand{\FPS}{\CC[\![\bfx]\!]} % formal power series
\newcommand{\FPSS}{\CC[\![\bfx,\bfy]\!]}
\newcommand{\reg}{\mathrm{reg}}
\newcommand{\hook}{\mathrm{h}}
\newcommand{\weight}{\mathrm{wt}}
\newcommand{\co}{\mathrm{co}}
\newcommand{\ps}{\mathrm{ps}}
\newcommand{\rmsum}{\mathrm{sum}}
\newcommand{\NSym}{\mathrm{NSym}}
\newcommand{\Hom}{\mathrm{Hom}}
\newcommand{\proj}{\mathrm{proj}}
\newcommand{\stat}{\mathrm{stat}}
\newcommand{\Par}{\mathrm{Par}}
\newcommand{\rmset}{\mathrm{set}}
\newcommand{\comp}{\mathrm{comp}}

% miscellaneous commands
\newcommand{\defn}[1]{{\color{mylightblue}{#1}}}
\newcommand{\toDo}{{\bf\color{red} TODO}}
\newcommand{\toCite}{{\bf\color{green} CITE}}
\newcommand*{\vertbar}{\rule[-1ex]{0.5pt}{2.5ex}} % for matrices with column vectors
\newcommand*{\horzbar}{\rule[.5ex]{2.5ex}{0.5pt}} % for matrices with row vectors
\newcommand{\myblue}[1]{{\color{iceberg}{#1}}}
\newcommand{\myorange}[1]{{\color{burntorange}{#1}}}
\newcommand{\mygreen}[1]{{\color{applegreen}{#1}}}
\newcommand{\myred}[1]{{\color{darkcandyapplered}{#1}}}

% ferrer's diagram
\newcommand{\fdiagram}[1]{
    \begin{tikzpicture}[scale=.7]
        \fill foreach \Z [count=\Y] in {#1}
        {foreach \X in {1,...,\Z} 
        {(\X,-\Y) circle[radius=3pt]}};
    \end{tikzpicture}
}

%
\newcommand{\tcbo}[1]{\textcolor{burntorange}{#1}}

% 
\newenvironment{nouppercase}{%
  \let\uppercase\relax%
  \renewcommand{\uppercasenonmath}[1]{}}{}

% titlepage
\title{Θ2.04: Θεωρία Αναπαραστάσεων και Συνδυαστική}
\author[Β.~Δ. Μουστακας]{Βασίλης Διονύσης Μουστάκας \\ Πανεπιστήμιο Κρήτης}
\date{11 Δεκεμβρίου 2025}
% \urladdr{\href{https://sites.google.com/view/vasmous}{https://sites.google.com/view/vasmous}}

\begin{document}

\begingroup
\def\uppercasenonmath#1{} % this disables uppercase title
\let\MakeUppercase\relax % this disables uppercase authors
\maketitle
\endgroup


\setcounter{section}{16}
\setcounter{theorem}{5}
\setcounter{equation}{1}
\begin{center}
    \textbf{16. Οι συναρτήσεις Schur
} (Συνέχεια)
\end{center}

\begin{corollary}
    \label{cor:schur_basis}
    Αν $\lambda \vdash n$, τότε 
    \begin{equation}
        \label{eq:s_to_m}
        s_\lambda = \sum_{\mu \tle \lambda} \rmK_{\lambda\mu} m_\mu.
    \end{equation}
    Ειδικότερα, το σύνολο $\{s_\lambda : \lambda \vdash n\}$ αποτελεί βάση του $\Sym_n$.
\end{corollary}

\begin{proof}[Απόδειξη]
    Από το Θεώρημα~16.5 έπεται ότι 
    \[
    s_\lambda = \sum_{\mu \vdash n} \rmK_{\lambda\mu} m_\mu.
    \]
    Το ζητούμενο έπεται από τον Κανόνα του Young και το Πόρισμα 11.13.
\end{proof}

Για $\lambda, \mu \vdash n$, έστω $\chi^\lambda(\mu)$ η τιμή του χαρακτήρα του προτύπου Specht $\sS^\lambda$ στην κλάση συζυγίας της $\fS_n$ που αντιστοιχεί στη διαμέριση $\mu$.
\begin{theorem}
    \label{thm:s_to_p}
    Αν $\lambda \vdash n$, τότε 
    \begin{equation}
        \label{eq:s_to_p}
    s_\lambda 
    = \sum_{\mu\vdash{n}} \frac{1}{\rmz_\mu}\chi^\lambda(\mu)p_\mu
    = \frac{1}{n!}\sum_{\pi \in \fS_n}\chi^\lambda(\pi)p_{\ct(\pi)},
    \end{equation}
    όπου $\ct(\pi)$ είναι ο κυκλικός τύπος της $\pi$.
\end{theorem}

\begin{proof}[Απόδειξη]
    Η δεύτερη ισότητα έπεται από την Πρόταση~9.5. Για την πρώτη ισότητα, από τον Κανόνα του Young έπεται ότι 
    \begin{equation}
        \label{eq:s_to_p_help}
        \rmK_{\lambda\mu} = (\varphi^\mu, \chi^\lambda) = \sum_{\nu \vdash n} \frac{1}{\rmz_\nu} \varphi^\mu(\nu)\chi^\lambda(\nu),
    \end{equation}
    όπου η δεύτερη ισότητα έπεται από την Ταυτότητα~(7.3) και το γεγονός ότι μια μετάθεση και η αντίστροφή της έχουν τον ίδιο κυκλικό τύπο (γιατί;). Υπολογίζουμε,
    \begin{align*}
        s_\lambda
        &= \sum_{\mu \vdash n} \rmK_{\lambda\mu} m_\mu \\
        &= \sum_{\mu \vdash n} \left(\sum_{\nu \vdash n} \frac{1}{\rmz_\nu} \varphi^\mu(\nu)\chi^\lambda(\nu)\right) m_\mu \\
        &= \sum_{\nu \vdash n} \frac{1}{\rmz_\nu}\chi^\lambda(\nu)\left(\sum_{\mu \vdash n}\varphi^\mu(\nu)m_\mu\right) \\
        &= \sum_{\nu \vdash n} \frac{1}{\rmz_\nu}\chi^\lambda(\nu)p_\nu,
    \end{align*}
    όπου η πρώτη και δεύτερη ισότητα έπονται από τις Ταυτότητες~\eqref{eq:s_to_m} και \eqref{eq:s_to_p_help}, αντίστοιχα, ενώ η τελευταία ισότητα έπεται από το Θεώρημα~15.12. 
\end{proof}

Το Θεώρημα~\ref{thm:s_to_p} μετατρέπει το πρόβλημα υπολογισμού του χαρακτήρα μιας ανάγωγης αναπαράστασης της $\fS_n$ σε πρόβλημα εξαγωγής συντελεστών του αναπτύγματος των συνα\-ρτήσεων Schur στην βάση των power sum συμμετρικών συναρτήσεων. Το παρακάτω αποτέ\-λεσμα έπεται άμεσα από την Ταυτότητα~\eqref{eq:s_to_p} και το Παράδειγμα~16.4 (1).

\begin{corollary}
    \label{cor:hnen_to_p}
    Ισχύουν οι εξής ταυτότητες 
    \begin{align*}
        h_n &= \sum_{\mu \vdash n} \frac{1}{\rmz_\mu} p_\mu \\
        e_n &= \sum_{\mu \vdash n} \frac{\sign(\mu)}{\rmz_\mu} p_\mu, 
    \end{align*}
    όπου $\sign(\mu) \coloneqq (-1)^{n - \ell(\mu)}$.
\end{corollary}

\newpage 

\setcounter{section}{17}
\setcounter{theorem}{0}
\setcounter{equation}{0}
\begin{center}
    \textbf{17. Η χαρακτηριστική απεικόνιση Frobenius
} 
\end{center}

Η συζήτηση των προηγούμενων παραγράφων υποδεικνύει ότι υπάρχει μια σύνδεση μεταξύ των αναπαραστάσεων της συμμετρικής ομάδας και των συμμετρικών συναρτήσεων. Αυτή παίρνει μορφή μέσω της παρακάτω απεικόνισης. Έστω $\rmCF_n \coloneqq \rmCF(\fS_n)$.

\begin{definition}
    \label{def:frob_char}
    Η απεικόνιση\footnote{Συχνά συμβολίζεται και ως $\mathrm{Frob}$.} $\ch_n : \rmCF_n \to \Sym_n$ που ορίζεται θέτοντας 
    \begin{equation}
        \label{eq:frob_char}
        \ch_n(f) \coloneqq \frac{1}{n!} \sum_{\pi\in\fS_n}f(\pi)p_{\ct(\pi)} = \sum_{\mu \vdash n} \frac{1}{\rmz_\mu}f(\mu) p_\mu,
    \end{equation}
    για κάθε $f \in \rmCF_n$, όπου $f(\mu)$ είναι η τιμή της $f$ στην κλάση συζυγίας της $\fS_n$ που αντιστοιχεί στην $\mu$, ονομάζεται \defn{χαρακτηριστική Frobenius}.
\end{definition}

Η χαρακτηριστική απεικόνιση Frobenius είναι προφανώς γραμμική και από το Θεώρημα 16.7 έπεται ότι 
\[
\ch_n(\chi^\lambda) = s_\lambda
\]
για κάθε $\lambda \vdash n$. Με άλλα λόγια, απεικονίζει την βάση των ανάγωγων χαρακτήρων του $\rmCF_n$ στην βάση των συναρτήσεων Schur του $\Sym_n$ και γι αυτό είναι γραμμικός ισομορφισμός. Ειδικότερα, 
\[
\ch_n(\chi^\triv) = h_n \quad \text{και} \quad \ch_n(\chi^\sign) = e_n,
\]
όπου $\chi^\triv$ και $\chi^\sign$ είναι οι χαρακτήρες της τετριμμένης αναπαράστασης και της αναπαράστασης προσήμου της $\fS_n$, αντίστοιχα.

Επιπλέον, θεωρώντας το (μοναδικό) εσωτερικό γινόμενο\footnote{Αυτό συνήθως ονομάζεται \defn{εσωτερικό γινόμενο Hall}.} $(\Arg, \Arg) : \Sym_n\times\Sym_n \to \CC$ για το οποίο 
\[
(s_\lambda, s_\mu) = 
\begin{cases}
    1, &\ \text{αν $\lambda = \mu$} \\
    0, &\ \text{διαφορετικά}, 
\end{cases}
\]
και επεκτείνοντας γραμμικά ως προς την πρώτη μεταβλητή και συζυγώς-γραμμικά ως προς την δεύτερη, τότε η χαρακτηριστική απεικόνιση Frobenius είναι ισομετρία (γιατί;).

\begin{example}
    \label{ex:ch_1}
    Για $\lambda \vdash n$, θεωρούμε την χαρακτηριστική συνάρτηση $1_\lambda$ της κλάσης συζυγίας της $\fS_n$ που αντιστοιχεί στην $\lambda$, δηλαδή 
    \[
    1_\lambda(\mu) = 
    \begin{cases}
        1, &\ \text{αν $\lambda = \mu$} \\
        0, &\ \text{διαφορετικά}
    \end{cases}
    \]
    (βλ. συζύτηση πριν τον Ορισμό~5.4). Τότε 
    \[
    \ch_n(1_\lambda) = \sum_{\mu\vdash n}\frac{1}{\rmz_\mu}1_\lambda(\mu) p_\mu = \frac{1}{\rmz_\lambda}p_\lambda.
    \]
    Συνεπώς, για κάθε $\lambda, \mu \vdash n$, 
    \begin{align*}
        (p_\lambda, p_\mu)_{\Sym_n}
        &= \left(
            \rmz_\lambda\ch_n(1_\lambda), \rmz_\mu\ch_n(1_\mu)
        \right)_{\Sym_n} \\
        &= \rmz_\lambda\rmz_\mu \left(
            \ch_n(1_\lambda), \ch_n(1_\mu)
        \right)_{\Sym_n} \\ 
        &= \rmz_\lambda\rmz_\mu \left(
            1_\lambda, 1_\mu
        \right)_{\rmCF_n} \\
        &= \rmz_\lambda\rmz_\mu \sum_{\nu \vdash n} \frac{1}{\rmz_\nu} 1_\lambda(\nu)1_\mu(\nu) \\
        &= \begin{cases}
            \rmz_\lambda, &\ \text{αν $\lambda = \mu$} \\
            0, &\ \text{διαφορετικά}
        \end{cases},
    \end{align*}
    όπου η τρίτη ισότητα έπεται από το ότι η χαρακτηριστική Frobenius είναι ισομετρία. Άρα, το σύνολο των power sum συμμετρικών συναρτήσεων αποτελεί \emph{ορθογώνια} βάση του $\Sym_n$ ως προς το εσωτερικό γινόμενο Hall.
\end{example}

Θεωρούμε τον (διαβαθμισμένο) διανυσματικό χώρο 
\[
\rmCF \coloneqq \CC \oplus \rmCF_1 \oplus \rmCF_2 \oplus \cdots.
\]
Επεκτείνοντας την χαρακτηριστική Frobenius στο $\rmCF$ με το να θέσουμε 
\[
\ch(f_0 + f_1 + f_2 + \cdots) \coloneqq f_0 + \ch_1(f_1) + \ch_2(f_2) + \cdots 
\]
για κάθε $f = f_0 + f_1 + f_2 + \cdots \in \rmCF$, προκύπτει ένας γραμμικός ισομορφισμός $\ch : \rmCF \to \Sym$. Επεκτείνοντας και το εσωτερικό γινόμενο Hall στο $\rmCF$, θέτοντας $(f, g) = 0$ αν $f \in \rmCF_n$ και $g \in \rmCF_m$ με $n\neq m$, τότε η $\ch$ είναι ισομετρία.

Όπως είδαμε στην Παράγραφο 15, ο χώρος των συμμετρικών συναρτήσεων $\Sym$ διαθέτει έναν \textquote{φυσικό} πολλαπλασιασμό που τον καθιστά δακτύλιο. Πως θα μπορούσαμε να πολλαπλασιάσουμε δυο $f \in \rmCF_n$ και $g \in \rmCF_m$ έτσι ώστε το γινόμενό τους να είναι ένα στοιχείο $f\circ{g} \in \rmCF_{n+m}$;

\begin{digression_a}
Έστω $G$ και $G'$ δυο πεπερασμένες ομάδες. Αν $V$ και $V'$ είναι $G$- και $G'$-πρότυπα αντίστοιχα, τότε το τανυστικό τους γινόμενο $V \otimes V'$ γίνεται $G\times{G'}$-πρότυπο θέτοντας 
\[
(g, g') \cdot v \otimes v' \coloneqq gv \otimes g'v'
\]
(γιατί;). Σε αντίθεση με την καταστασκευή της Παραγράφου 6, εδώ το $V \otimes V'$ δέχεται δομή προτύπου από δυο διαφορετικές ομάδες. Γι αυτό τον λόγο συχνά ονομάζεται \emph{εξωτερικό τανυστικό γινόμενο} (outer tensor product). Γι αυτά τα πρότυπα ισχούν τα παρακάτω.
\begin{itemize}
    \item $\chi^{V\otimes{V'}}(g,g') = \chi^V(g)\chi^{V'}(g')$
    \item $\left(\chi^{V_1\otimes{V_1'}}, \chi^{V_2\otimes{V_2'}}\right)_{G\times{G'}} = \left(
        \chi^{V_1},\chi^{V_2}
    \right)_G\left(
        \chi^{V_1'},\chi^{V_2'}
    \right)_{G'}$
    \item Αν $\{V_1, V_2, \dots, V_n\}$ και $\{V_1', V_2', \dots, V_m'\}$ είναι σύνολα ανά δύο διακεκριμένων ανάγωγων προτύπων της $G$ και $G'$, αντίστοιχα, τότε το \[\{V_i \otimes V_j' : 1\le i\le n, 1 \le j \le m\}\] είναι ένα σύνολο ανά δύο μη ισόμορφων ανάγωγων $G\times{G'}$-προτύπων.
\end{itemize}
\end{digression_a}

Έστω $\chi$ και $\psi$ (ανάγωγοι) χαρακτήρες της $\fS_n$ και $\fS_m$, αντίστοιχα. Από την παραπάνω συζήτηση προκύπτει ότι η απεικόνιση $\chi\times\psi : \fS_n\times\fS_m \to \CC$ που ορίζεται θέτοντας 
\[
\chi\times\psi(\pi,\sigma) \coloneqq \chi(\pi)\psi(\sigma)
\]
για κάθε $(\pi,\sigma) \in \fS_n\times\fS_m$ είναι χαρακτήρας της $\fS_n\times\fS_m$. Τώρα, μπορούμε να \textquote{ανέβουμε} στην $\fS_{n+m}$ θεωρώντας την επαγωγή αυτού του χαρακτήρα:
\[
\chi\circ\psi \coloneqq \left(\chi\times\psi\right)\uparrow_{\fS_n\times\fS_m}^{\fS_{n+m}}.
\]
Αυτό ονομάζεται \defn{γινόμενο επαγωγής} (induction product) των $\chi$ και $\psi$. Επεκτείνοντας διγραμμικά το $\circ$ σε ολόκληρο το $\rmCF$ αυτό παίρνει τη δομή δακτύλιου.

\begin{theorem}
    \label{thm:frobenius}
    Η χαρακτηριστική απεικόνιση Frobenius $\ch : \rmCF \to \Sym$ είναι ισομορφισμός δακτυλίων.
\end{theorem}

Για την απόδειξη του Θεωρήματος~\ref{thm:frobenius} χρειαζόμαστε μια γενικότερη μορφή του νόμου αντιστροφής Frobenius (βλ. Θεωρήμα~8.8), η οποία αποδεικνύεται ουσιαστικά με τον ίδιο τρόπο.
\begin{lemma}
    \label{lem:frobenius_reciprocity_law}
    Έστω $G$ πεπερασμένη ομάδα και $H$ υποομάδα της. Αν $A$ είναι μια άλγεβρα πάνω από το $\CC$ και $\chi : G \to A$ μια συνάρτηση η οποία έχει σταθερή τιμή στις κλάσεις συζυγίας της $G$, τότε 
    \[
    \left(
        \psi\uparrow_H^G, \chi
    \right)_G' =
    \left(
        \psi, \chi\downarrow_H^G
    \right)_H'
    \]
    για κάθε $\psi : H \to A$, όπου 
    \begin{itemize}
        \item για κάθε $h \in H$, $\chi\downarrow_H^G(h) \coloneqq \chi(h)$,
        \item για κάθε $g \in G$, $\psi\uparrow_H^G(g) \coloneqq \frac{1}{\abs{H}}\sum_{x^{-1}gx \in H} \psi(x^{-1}gx)$
        \item για κάθε $\alpha, \beta : G \to A$, $(\alpha,\beta)' \coloneqq \frac{1}{\abs{G}}\sum_{g \in G} \ol{\alpha(g)}\beta(g)$.
    \end{itemize}
\end{lemma}

\begin{proof}[Απόδειξη του Θεωρήματος~\ref{thm:frobenius}]
    Αρκεί να αποδείξουμε ότι η $\ch$ είναι ομομορφισμός δακτυλίων. Γι αυτό, παρατηρούμε ότι 
    \begin{align*}
    \ch_n(\chi) 
    &= \frac{1}{n!} \sum_{\pi \in \fS_n} \chi(\pi) p_{\ct(\pi)} \\
    &= \frac{1}{n!} \sum_{\pi \in \fS_n} \chi(\pi^{-1}) p_{\ct(\pi^{-1})} \\
    &= \frac{1}{n!} \sum_{\pi \in \fS_n} \ol{\chi(\pi)} p_{\ct(\pi)} \\
    &= (\chi, P)_{\fS_n}', 
    \end{align*}
    όπου $P : \fS_n \to \Sym$ είναι η συνάρτηση που ορίζεται θέτοντας $P(\pi) \coloneqq p_{\ct(\pi)}$. Προφανώς, η συνάρτηση $P$ έχει σταθερή τιμή στις κλάσεις συζυγίας της $\fS_n$. Συνεπώς, για $f \in \rmCF_n$ και $g \in \rmCF_m$ υπολογίζουμε 
    \begin{align*}
        \ch(f \circ g)
        &= \ch_{n+m}(\left(
            f\times{g}
        \right)\uparrow_{\fS_n\times\fS_m}^{\fS_{n+m}}) \\ 
        &= \left(\left(
            f\times{g}
        \right)\uparrow_{\fS_n\times\fS_m}^{\fS_{n+m}}, P
        \right)_{\fS_{n+m}}' \\
        &= \left(
            f\times{g}, P\downarrow_{\fS_n\times\fS_m}^{\fS_{n+m}}
        \right)'_{\fS_n\times\fS_m} \\
        &= \frac{1}{n!m!} \sum_{(\pi, \sigma) \in \fS_n\times\fS_m} \ol{f\times{g}(\pi,\sigma)} P(\pi,\sigma) \\
        &= \frac{1}{n!m!} \sum_{(\pi, \sigma) \in \fS_n\times\fS_m} \ol{f(\pi)}\ol{g(\sigma)} p_{\ct(\pi,\sigma)} \\
        &=  \left(
            \frac{1}{n!}\sum_{\pi\in\fS_n} \ol{f(\pi)}p_{\ct(\pi)}
        \right)
        \left(
            \frac{1}{m!}\sum_{\sigma\in\fS_m} \ol{g(\sigma)}p_{\ct(\sigma)}
        \right) \\
        &= \ch_n(f)\ch_m(g),
    \end{align*}
    όπου η τρίτη ισότητα έπεται από το Λήμμα~\ref{lem:frobenius_reciprocity_law} και η προτελευταία ισότητα έπεται από την πολλαπλασιαστικότητα των power sum συμμετρικών συναρτήσεων και το γεγονός ότι ο κυκλικός τύπος του στοιχείου $(\pi,\sigma)$ όταν το βλέπουμε ως μετάθεση της $\fS_{n+m}$ είναι ουσιαστικά\footnote{Για παράδειγμα, αν $\pi = \cycle{1,2}\cycle{3} \in \fS([3])\cong\fS_3$ με κυκλικό τύπο $(2,1)$ και $\sigma = \cycle{4,5} \in \fS([4,5])\cong \fS_2$ με κυκλικό τύπο $(2)$, τότε το στοιχείο $(\pi,\sigma)$ ταυτίζεται με την μετάθεση $\cycle{1,2}\cycle{3}\cycle{4,5}$ στην $\fS_5$ και γι αυτό έχει κυκλικό τύπο $(2,2,1)$. Συνεπώς, $p_{\ct(\pi,\sigma)} = p_{(2,2,1)} = p_2p_2p_1 = p_{(2,1)}p_{(2)} = p_{\ct(\pi)}p_{\ct(\sigma)}$.} η \textquote{ένωση} των κυκλικών τύπων της $\pi$ και της $\sigma$.
\end{proof}
\end{document}