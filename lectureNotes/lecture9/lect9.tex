\documentclass[12pt,a4paper,reqno]{amsart}

% section handling
\usepackage{subfiles} 

% language
\usepackage[greek,english]{babel}
\usepackage[utf8]{inputenc}
\usepackage{alphabeta}

% change default names to greek
\addto\captionsenglish{
    \renewcommand{\contentsname}{Περιεχόμενα}
    \renewcommand{\refname}{Βιβλιογραφία}
    \renewcommand{\datename}{Ημερομηνία:}
    \renewcommand{\urladdrname}{Ιστοσελίδα}
}

% math 
\usepackage{amsmath,amsthm,amssymb,amscd}

% font
\usepackage[cal=euler]{mathalfa}
\usepackage{libertinus-type1}
% \usepackage{txfonts} % for upright greek letters
\usepackage{bm} % for bold symbols
\usepackage{bbm} % for the simply-looking bb symbols

% miscellaneous 
\usepackage{changepage} %for indenting environments
\usepackage{csquotes} % example: \textcquote{}

% drawing
\usepackage{tikz,tikz-cd}
\usetikzlibrary{shapes.misc, patterns, matrix, calc, intersections,positioning}
\usepackage{graphics,graphicx}
\usepackage{float} % provides enhanced control and customization options for floating objects such as figures and tables

% colors
\usepackage{xcolor}
\definecolor{darkcandyapplered}{rgb}{0.64, 0.0, 0.0}
\definecolor{midnightblue}{rgb}{0.1, 0.1, 0.44}
\definecolor{mylightblue}{HTML}{336699}
\definecolor{burntorange}{rgb}{0.8, 0.33, 0.0}
\definecolor{iceberg}{rgb}{0.44, 0.65, 0.82}
\definecolor{applegreen}{rgb}{0.55, 0.71, 0.0}
\definecolor{canaryyellow}{rgb}{1.0, 0.94, 0.0}

% hrefs
\usepackage{hyperref}
\usepackage[noabbrev,capitalize]{cleveref}
\hypersetup{
    pdftoolbar=true,        
    pdfmenubar=true,        
    pdffitwindow=false,     
    pdfstartview={FitH},  % fits the width of the page to the window
    pdftitle={},
    pdfauthor={},
    pdfsubject={},
    pdfkeywords={},
    pdfnewwindow=true,  % links in new window
    colorlinks=true,  % false: boxed links; true: colored links
    linkcolor=darkcandyapplered,   % color of internal links
    citecolor=midnightblue,  % color of links to bibliography
    urlcolor=cyan,  % color of external links
    linktocpage=true  % changes the links from the section body to the page number
    }

% geometry
\textwidth=16cm 
\textheight=21cm 
\hoffset=-55pt 
\footskip=25pt

% thm envs (you might need to change the path)
% In this macro I define all the theorem environments

\theoremstyle{definition}
\newtheorem{theorem}{Θεώρημα}
\newtheorem{proposition}[theorem]{Πρόταση}
\newtheorem{lemma}[theorem]{Λήμμα}
\newtheorem{corollary}[theorem]{Πόρισμα}
\newtheorem{conjecture}[theorem]{Εικασία}
\newtheorem{problem}[theorem]{Πρόβλημα}
\newtheorem*{claim}{Ισχυρισμός}
\newtheorem{observation}[theorem]{Παρατήρηση}
\newtheorem{definition}[theorem]{Ορισμός}
\newtheorem{question}[theorem]{Ερώτηση}
\newtheorem{example}[theorem]{Παράδειγμα}
\newtheorem{exercise}{Άσκηση}

\theoremstyle{remark}
\newtheorem*{remark}{Παρατήρηση}

% fixes the correct numbering of environments
\numberwithin{theorem}{section}
\numberwithin{exercise}{section}
\numberwithin{equation}{section}

% math ops (you might need to change the path)
% In this macro I define all of my math operators

% fields
\newcommand{\NN}{\mathbbmss{N}} 
\newcommand{\ZZ}{\mathbbmss{Z}} 
\newcommand{\QQ}{\mathbbmss{Q}} 
\newcommand{\RR}{\mathbbmss{R}} 
\newcommand{\CC}{\mathbbmss{C}} 
\newcommand{\KK}{\mathbbmss{K}} 
\newcommand{\FF}{\mathbbmss{F}} 

% symmetric group
\newcommand{\fS}{\mathfrak{S}}  

% calligraphic 
\newcommand{\aA}{\mathcal{A}} 
\newcommand{\bB}{\mathcal{B}}
\newcommand{\cC}{\mathcal{C}}
\newcommand{\dD}{\mathcal{D}}
\newcommand{\eE}{\mathcal{E}}
\newcommand{\fF}{\mathcal{F}}
\newcommand{\hH}{\mathcal{H}}
\newcommand{\iI}{\mathcal{I}}
\newcommand{\lL}{\mathcal{L}}
\newcommand{\oO}{\mathcal{O}}
\newcommand{\pP}{\mathcal{P}}
\newcommand{\sS}{\mathcal{S}}
\newcommand{\mM}{\mathcal{M}}
\newcommand{\uU}{\mathcal{U}}

% bold
\newcommand{\bfa}{\mathbf{a}}
\newcommand{\bfe}{\mathbf{e}}
\newcommand{\bfF}{\pmb{F}}
\newcommand{\bfR}{\pmb{R}}
\newcommand{\bfv}{\mathbf{v}}
%\newcommand{\bfx}{\bm{x}}
%\newcommand{\bfx}{\mathbf{x}} 
\newcommand{\bfx}{\pmb{x}}
\newcommand{\bfX}{\pmb{X}}
\newcommand{\bfy}{\pmb{y}}
\newcommand{\bfz}{\pmb{z}}

% roman
\newcommand{\rmB}{\mathrm{B}}
\newcommand{\rmC}{\mathrm{C}}
\newcommand{\rmD}{\mathrm{D}} 
\newcommand{\rmI}{\mathrm{I}} 
\newcommand{\rmK}{\mathrm{K}}
\newcommand{\rmM}{\mathrm{M}}
\newcommand{\rmP}{\mathrm{P}}  
\newcommand{\rmQ}{\mathrm{Q}}  
\newcommand{\rmR}{\mathrm{R}}
\newcommand{\rmS}{\mathrm{S}}
\newcommand{\rmT}{\mathrm{T}}
\newcommand{\rmU}{\mathrm{U}}
\newcommand{\rmV}{\mathrm{V}}
\newcommand{\rmY}{\mathrm{Y}}
\newcommand{\rmZ}{\mathrm{Z}}

% greek letters
% I'm renewing some commands in order to appear in upright font
% If I want to change it later, I don't have to do it manually, I just change it from here.
% \newcommand{\uaa}{\alphaup}
% \renewcommand{\alpha}{\alphaup}
% \renewcommand{\beta}{\betaup}
% \renewcommand{\gamma}{\gammaup}
% \renewcommand{\delta}{\deltaup}
% \renewcommand{\epsilon}{\epsilonup}
% \newcommand{\ee}{\epsilon}
% \renewcommand{\varepsilon}{\varepsilonup}
% \renewcommand{\theta}{\thetaup}
% \renewcommand{\lambda}{\lambdaup}
% \newcommand{\ull}{\lambda}
% \renewcommand{\mu}{\muup}
% \renewcommand{\nu}{\nuup}
% \renewcommand{\pi}{\piup}
% \renewcommand{\rho}{\rhoup}
% \renewcommand{\varrho}{\varrhoup}
% \renewcommand{\sigma}{\sigmaup}
% \renewcommand{\tau}{\tauup} 
% \renewcommand{\phi}{\phiup}
% \renewcommand{\chi}{\chiup}
% \renewcommand{\psi}{\psiup}
% \renewcommand{\omega}{\omegaup}

% arrows and symbols 
\renewcommand{\to}{\rightarrow}
\newcommand{\toto}{\longrightarrow}
\newcommand{\mapstoto}{\longmapsto}
\newcommand{\then}{\Rightarrow}
\newcommand{\IFF}{\Leftrightarrow}
\newcommand{\tl}{\tilde}
\newcommand{\wtl}{\widetilde}
\newcommand{\ol}{\overline}
\newcommand{\ul}{\underline}
\newcommand{\oldemptyset}{\emptyset}
\renewcommand{\emptyset}{\varnothing}
\DeclareMathSymbol{\Arg}{\mathbin}{AMSa}{"39} % for arguments 
\newcommand{\onto}{\ensuremath{\twoheadrightarrow}}

% absolute value symbol
\usepackage{mathtools} 
\DeclarePairedDelimiter\abs{\lvert}{\rvert}%
\DeclarePairedDelimiter\norm{\lVert}{\rVert}%
\makeatletter
\let\oldabs\abs
\def\abs{\@ifstar{\oldabs}{\oldabs*}}

% tensor symbol
\newcommand{\tensor}[1]{%
  \mathbin{\mathop{\otimes}\limits_{#1}}%
}

% permutation cycle notation
\ExplSyntaxOn
\NewDocumentCommand{\cycle}{ O{\;} m }
 {
  (
  \alec_cycle:nn { #1 } { #2 }
  )
 }

\seq_new:N \l_alec_cycle_seq
\cs_new_protected:Npn \alec_cycle:nn #1 #2
 {
  \seq_set_split:Nnn \l_alec_cycle_seq { , } { #2 }
  \seq_use:Nn \l_alec_cycle_seq { #1 }
 }
\ExplSyntaxOff

% setminus symbol
\newcommand{\mysetminusD}{\hbox{\tikz{\draw[line width=0.6pt,line cap=round] (3pt,0) -- (0,6pt);}}}
\newcommand{\mysetminusT}{\mysetminusD}
\newcommand{\mysetminusS}{\hbox{\tikz{\draw[line width=0.45pt,line cap=round] (2pt,0) -- (0,4pt);}}}
\newcommand{\mysetminusSS}{\hbox{\tikz{\draw[line width=0.4pt,line cap=round] (1.5pt,0) -- (0,3pt);}}}
\newcommand{\sm}{\mathbin{\mathchoice{\mysetminusD}{\mysetminusT}{\mysetminusS}{\mysetminusSS}}}

% custom math operators
\newcommand{\Des}{\mathrm{Des}} 
\newcommand{\des}{\mathrm{des}} 
\newcommand{\Asc}{\mathrm{Asc}}
\newcommand{\asc}{\mathrm{asc}} 
\newcommand{\inv}{\mathrm{inv}}
\newcommand{\Inv}{\mathrm{Inv}}
\newcommand{\maj}{\mathrm{maj}} 
\newcommand{\comaj}{\mathrm{comaj}} 
\newcommand{\fix}{\mathrm{fix}} 
\newcommand{\Sym}{\mathrm{Sym}} 
\newcommand{\QSym}{\mathrm{QSym}}
\newcommand{\FQSym}{\mathrm{FQSym}} 
\newcommand{\End}{\mathrm{End}} 
\newcommand{\Rad}{\mathrm{Rad}} 
\newcommand{\rmMat}{\mathrm{Mat}} 
\newcommand{\rmdim}{\mathrm{dim}} 
\newcommand{\rmTop}{\mathrm{Top}} 
\newcommand{\rmCF}{\mathrm{CF}} 
\newcommand{\rmId}{\mathrm{Id}}
\newcommand{\rmid}{\mathrm{id}}
\newcommand{\rmtw}{\mathrm{tw}}
\newcommand{\trace}{\mathrm{tr}}
\newcommand{\Irr}{\mathrm{Irr}}
\newcommand{\Ind}{\mathrm{Ind}} % induction
\newcommand{\Res}{\mathrm{Res}} % restriction
\newcommand{\triv}{\mathrm{triv}} % trivial rep
\newcommand{\rmdef}{\mathrm{def}} % defining rep
\newcommand{\dom}{\triangleleft}
\newcommand{\domeq}{\trianglelefteq}
\newcommand{\lex}{\mathrm{lex}}
\newcommand{\sign}{\mathrm{sign}}
\newcommand{\SYT}{\mathrm{SYT}}
\renewcommand{\Im}{\mathrm{Im}}
\newcommand{\Ker}{\mathrm{Ker}}
\newcommand{\GL}{\mathrm{GL}}
\newcommand{\FL}{\mathrm{FL}}
\newcommand{\Span}{\mathrm{span}}
\newcommand{\pos}{\mathrm{pos}}
\newcommand{\Comp}{\mathrm{Comp}}
\newcommand{\Set}{\mathrm{Set}}
\newcommand{\std}{\mathrm{std}}
\newcommand{\cont}{\mathrm{cont}} %content of a SSYT
\newcommand{\SSYT}{\mathrm{SSYT}}
\newcommand{\rmz}{\mathrm{z}}
\newcommand{\ct}{\mathrm{ct}} % cycle type
\newcommand{\ch}{\mathrm{ch}} % Frobenius characteristic map
\newcommand{\height}{\mathrm{ht}}
\newcommand{\FPS}{\CC[\![\bfx]\!]} % formal power series
\newcommand{\FPSS}{\CC[\![\bfx,\bfy]\!]}
\newcommand{\reg}{\mathrm{reg}}
\newcommand{\hook}{\mathrm{h}}
\newcommand{\weight}{\mathrm{wt}}
\newcommand{\co}{\mathrm{co}}
\newcommand{\ps}{\mathrm{ps}}
\newcommand{\rmsum}{\mathrm{sum}}
\newcommand{\NSym}{\mathrm{NSym}}
\newcommand{\Hom}{\mathrm{Hom}}
\newcommand{\proj}{\mathrm{proj}}
\newcommand{\stat}{\mathrm{stat}}

% miscellaneous commands
\newcommand{\defn}[1]{{\color{mylightblue}{#1}}}
\newcommand{\toDo}{{\bf\color{red} TODO}}
\newcommand{\toCite}{{\bf\color{green} CITE}}
\newcommand*{\vertbar}{\rule[-1ex]{0.5pt}{2.5ex}} % for matrices with column vectors
\newcommand*{\horzbar}{\rule[.5ex]{2.5ex}{0.5pt}} % for matrices with row vectors

% 
\newenvironment{nouppercase}{%
  \let\uppercase\relax%
  \renewcommand{\uppercasenonmath}[1]{}}{}

% titlepage
\title{Θ2.04: Θεωρία Αναπαραστάσεων και Συνδυαστική}
\author[Β.~Δ. Μουστακας]{Βασίλης Διονύσης Μουστάκας \\ Πανεπιστήμιο Κρήτης}
\date{30 Οκτωβρίου 2025}
% \urladdr{\href{https://sites.google.com/view/vasmous}{https://sites.google.com/view/vasmous}}

\begin{document}

\begingroup
\def\uppercasenonmath#1{} % this disables uppercase title
\let\MakeUppercase\relax % this disables uppercase authors
\maketitle
\endgroup


\setcounter{section}{7}
\setcounter{theorem}{5}
\begin{center}
    \textbf{7. Χαρακτήρες ομάδων: Σχέσεις ορθογωνιότητας
} (Συνέχεια)
\end{center}

\begin{proof}[Απόδειξη του Πορίσματος~7.5]
    Το (1) έπεται άμεσα από την Πρόταση 6.6 (1). Για το (2), υπολογίζουμε
    \[
    (\chi^V, \chi^{V_i}) = (\left(\sum_{j=1}^n m_j \chi^{V_j}\right), \chi^{V_i}) = \sum_{j=1}^n m_j \, (\chi^{V_j}, \chi^{V_i}) = m_i 
    \]
    όπου η τελευταία ισότητα έπεται από το Πόρισμα 7.4. Για το (3), υπολογίζουμε 
    \[
    (\chi^V, \chi^V) = (\left(\sum_{i=1}^n m_i \chi^{V_i}\right), \left(\sum_{j=1}^n m_j \chi^{V_j}\right)) 
    = \sum_{1 \le i, j \le n} m_i\ol{m_j} \, (\chi^{V_i}, \chi^{V_j}) 
    = \sum_{i=1}^n m_i^2,
    \]
    όπου η τελευταία ισότητα έπεται από το Πόρισμα 7.4. 
    
    Τέλος, το $V$ είναι ανάγωγο $G$-πρότυπο αν και μόνο αν υπάρχει μοναδικό $1 \le i \le n$ για το οποίο $m_i = 1$ και $m_j = 0$ για κάθε $j \neq i$, ή ισοδύναμα, από το (3), αν και μόνο αν $(\chi^V, \chi^V)=1$. 
\end{proof}

Το Πόρισμα 7.5 μας δίνει ένα ισχυρό κριτήριο για το πότε ένα $G$-πρότυπο είναι ανάγωγο. Για παράδειγμα, ξεχνώντας προσωρινά το Παράδειγμα 6.7, μπορούμε να δείξουμε ότι η συνήθης αναπαράσταση της $\fS_3$ είναι ανάγωγη υπολογίζοντας το $(\chi^\std, \chi^\std)$. Πράγματι, 
    \[
    \renewcommand{\arraystretch}{1.2} 
    \begin{array}{c|c|c|c}
              & \rmK_{111} & \rmK_{21} & \rmK_{3} \\ \hline
    \chi^\std & 2          & 0       & -1        
    \end{array}
    \]
και γι αυτό 
    \[
    (\chi^\std, \chi^\std) = \frac{1}{6}(2^2 + 3\cdot0 + 2\cdot(-1)^2) = 1
    \]
    όπως περιμέναμε.
Τι γίνεται για μεγαλύτερα $n$ (βλ. Άσκηση 2.4);

Ομοίως, για την αναπαράσταση $\rho$ της Άσκησης 1.4, υπολογίζουμε 
    \[
    \renewcommand{\arraystretch}{1.2} 
    \begin{array}{c|c|c|c|c|c}
              & \{\epsilon\} & \{r^2\} & \{r, r^3\} & \{s, sr^2\} & \{sr, sr^3\} \\ \hline
    \chi^\rho & 2            & -2      & 0          & 0           & 0     
    \end{array}
    \]
    και γι αυτό 
    \[
    (\chi^\rho, \chi^\rho) = \frac{1}{8}(2^2 + (-2)^2 + 2\cdot0 + 2\cdot0 + 2\cdot0) = 1.
    \]
Για τον πίνακα χαρακτήρων της $\rmD_{2n}$ δείτε την Άσκηση 2.1.

\begin{corollary}
    \label{cor:representation_isomorphism_criterion}
    Δυο αναπαραστάσεις είναι ισόμορφες αν και μόνο αν έχουν τον ίδιο χαρακτήρα.
\end{corollary}

\begin{proof}
    Την κατεύθυνση \textquote{$\then$} την είδαμε στην Πρόταση 5.3 (3). Για την άλλη κατεύθυνση, αν δυο πρότυπα έχουν τον ίδιο χαρακτήρα, τότε από το Πόρισμα 7.5 έπεται ότι τα ανάγωγα πρότυπα που εμφανίζονται στην ισοτυπική τους διάσπαση έχουν τις ίδιες πολλαπλότητες. Τo ζητούμενο έπεται από το Πόρισμα~3.5.
\end{proof}

Τα Πορίσματα 7.5 και 7.6 απαντάνε πλήρως στο Ερώτημα (3) της Παραγράφου 4. Το επόμενο αποτέλεσμα απαντάει στο Ερώτημα (1).
\begin{theorem}
    \label{thm:character_basis}
    Το σύνολο 
    \[
    \{\chi^V : \ \text{$V$ είναι ανάγωγο $G$-πρότυπο}\}
    \]
    αποτελεί ορθοκανονική βάση του $\rmCF(G)$ ως προς το εσωτερικό γινόμενο $(\Arg,\Arg)_G$. Ειδικότερα, το πλήθος των (διακεκριμένων) ανάγωγων $G$-προτύπων ισούται με το πλήθος των κλάσεων συζυγίας της $G$.
\end{theorem}

Από τις σχέσεις ορθογωνιότητας Ι, γνωρίζουμε ότι το σύνολο όλων των $\chi^V $ για ανάγωγα $G$-πρότυπα $V$ είναι ορθοκανονικό και κατά συνέπεια γραμμικώς ανεξάρτητο στο $\rmCF(G)$ (γιατί;). Αρκεί λοιπόν να δείξουμε ότι παράγει τον $\rmCF(G)$. Γι αυτό αρκεί να δέιξουμε ότι το ορθογώνιο συμπλήρωμα του υπόχωρου που παράγει είναι τετριμμένο (γιατί;). Για να το κάνουμε αυτό θα χρησιμοποιήσουμε το εξής.
\begin{lemma}
    \label{lem:character_basis}
    Έστω $(\rho,V)$ μια αναπαράσταση της $G$ και $\alpha \in \rmCF(G)$. 
    \begin{itemize}
        \item[(1)] Η απεικόνιση $\rho_\alpha : V \to V$ που ορίζεται θέτοντας 
        \[
        \rho_\alpha(v) \coloneqq \frac{1}{\abs{G}} \sum_{g \in G} \ol{\alpha(g)} \rho(g)(v)
        \]
        για κάθε $v \in V$, είναι $G$-ομομορφισμός.
        \item[(2)] Αν η $(\rho, V)$ είναι ανάγωγη, τότε 
        \begin{equation}
            \label{eq:character_basis}
            \rho_\alpha = \frac{(\chi^{\rho,V}, \alpha)}{\dim(V)} \, \rmid_V.
        \end{equation}
    \end{itemize}
\end{lemma}

\begin{proof}[Απόδειξη]
    Για το (1), η γραμμικότητα της $\rho_\alpha$ έπεται από την γραμμικότητα της $\rho$ (γιατί;). Για να επαληθεύσουμε την Ταυτότητα~(3.1) για την $\rho_\alpha$ παρατηρούμε ότι 
    \begin{align*}
        \rho(x^{-1})\left(\rho_\alpha(\rho(x)(v))\right)
        &= \frac{1}{\abs{G}} \sum_{g \in G} \ol{\alpha(g)}\rho(x^{-1}gx)(v) \\
        &= \frac{1}{\abs{G}} \sum_{g \in G} \ol{\alpha(x^{-1}gx)}\rho(x^{-1}gx)(v) \\
        &= \frac{1}{\abs{G}} \sum_{h \in G} \ol{\alpha(h)}\rho(h)(v) \\ 
        &= \rho_\alpha(v)
    \end{align*}
    για κάθε $x \in G$ και $v \in V$, όπου η πρώτη ισότητα έπεται από το ότι η $(\rho, V)$ είναι αναπαράσταση, η δεύτερη από το ότι η $\alpha$ είναι συνάρτηση κλάσης και η τρίτη ισότητα προκύπτει από την αλλαγή μεταβλητής $h \to x^{-1}gx$.

    Για το (2), από το Λήμμα του Schur (βλ. Θεώρημα 3.2) υπάρχει $c \in \CC$ τέτοιο ώστε $\rho_\alpha = c\,\rmid_V$. Υπολογίζοντας το ίχνος και στα δυο μέλη, 
    \begin{align*}
    c \dim(V) 
    &= \trace(\rho_\alpha) \\
    &= \trace\left(\frac{1}{\abs{G}} \sum_{g \in G} \ol{\alpha(g)} \rho(g)(v)\right) \\ 
    &= \frac{1}{\abs{G}} \sum_{g \in G} \ol{\alpha(g)} \chi^{\rho,V}(g) \\
    &= (\chi^{\rho,V}, \alpha),
    \end{align*}
    και το ζητούμενο έπεται.
\end{proof}

\begin{proof}[Απόδειξη του Θεωρήματος 7.7]
    Έστω $\alpha \in \rmCF(G)$ τέτοια ώστε $(\alpha, \chi^{\rho,V}) = 0$ ή ισοδύναμα $(\chi^{\rho,V},\alpha) = 0$ για κάθε ανάγωγη ανάγωγη αναπαράσταση $(\rho,V)$ της $G$. Από την Ταυτότητα \eqref{eq:character_basis} έπεται ότι 
    \begin{equation}
        \label{eq:character_basis_proof}
        \sum_{g \in G} \ol{\alpha(G)}\rho(g)(v) = 0,
    \end{equation}
    για κάθε $v \in V$, για κάθε ανάγωγη αναπαράσταση $(\rho,V)$ της $G$. Αφού η Ταυτότητα~\eqref{eq:character_basis_proof} ισχύει για κάθε ανάγωγη αναπαράσταση, θα ισχύει και για κάθε αναπαράσταση της $G$ (γιατί;).

    Για την κανονική αναπαράσταση $(\rho^\reg, \CC[G])$, η Ταυτότητα \eqref{eq:character_basis_proof} γίνεται\footnote{Αυτή είναι μια ισότητα διανυσμάτων στον χώρο $\CC[G]$.} 
    \[
    0 = \sum_{g \in G} \ol{\alpha(G)}\rho^\reg(g)(h) 
    = \sum_{g \in G} \ol{\alpha(G)}gh 
    \]
    για κάθε $h \in G$. Για το ταυτοτικό στοιχείο $\epsilon$ της $G$
    παίρνουμε
    \[
    \sum_{g \in G} \ol{\alpha(G)}g = 0
    \]
    στον $\CC[G]$. Εξ' ορισμού τα στοιχεία του $G$ αποτελούν βάση του $\CC[G]$ και γι αυτό έπεται ότι $\ol{\alpha(g)} = 0$ για κάθε $g \in G$ (γιατί;). Με άλλα λόγια, η $\alpha$ είναι η μηδενική απεικόνιση και η απόδειξη ολοκληρώνεται.
\end{proof}

Μια ακόμη συνέπεια του Θεωρήματος~\ref{thm:character_basis} είναι ότι ο πίνακας χαρακτήρων της $G$ είναι τετραγωνικός. Η συζήτηση περί ορθογωνιότητας ξεκίνησε στην αρχή της παραγράφου όταν αποφασίσαμε να φανταστούμε τις γραμμές του πίνακα χαρακτήρων ως διανύσματα κάποιου $\CC^m$ και να υπολογίσουμε το σύνηθες εσωτερικό γινόμενό τους. Τι γίνεται αν κάνουμε το ίδιο και για τις στήλες;

Ας θυμιθούμε τον πίνακα χαρακτήρων της $\fS_3$ 
\[
\renewcommand{\arraystretch}{1.2}
\begin{array}{l|c|c|c}
               & K_{111} & K_{21} & K_3 \\ \hline
    \chi^\triv & 1       & 1      & 1 \\ \hline
    \chi^\sign & 1       & -1     & 1 \\ \hline
    \chi^\std  & 2       & 0      & -1 
\end{array} \ .
\]
Αν προσωρινά φανταστούμε 
\[
\rmK_{111} = (1,1,2), \ \rmK_{21} = (1,-1,0), \ \rmK_3 = (1,1,-1),
\]
τότε παρατηρούμε ότι 
\begin{align*}
    \rmK_{111}\cdot\rmK_{111} &= 1^2 + 1^2 + 2^2 = 6 = \frac{\abs{\fS_3}}{\abs{\rmK_{111}}} \\ 
    \rmK_{21}\cdot\rmK_{21} &= 1^2 + (-1)^2 + 0 = 2 = \frac{\abs{\fS_3}}{\abs{\rmK_{21}}} \\ 
    \rmK_3\cdot\rmK_3 &= 2^2 + 0 + (-1)^2 = 3 = \frac{\abs{\fS_3}}{\abs{\rmK_3}} \\ 
    \rmK_{111}\cdot\rmK_{21} &= \rmK_{111}\cdot\rmK_{3} = \rmK_{21}\cdot\rmK_{3} = 0. 
\end{align*}
Με άλλα λόγια, το σύνολο $\{\rmK_{111},\rmK_{21}, \rmK_3\}$ είναι ορθογώνιο ως προς το σύνηθες εσωτερικό γινόμενο του $\CC^3$. Αυτό ισχύει γενικότερα.
\begin{theorem}{\rm(Σχέσεις ορθογωνιότητας ΙΙ)}
    \label{thm:orthogonality_relations_II}
    Αν $K$ και $L$ είναι δύο κλάσεις συζυγίας της $G$, τότε 
    %
    \begin{equation}
        \label{eq:orthogonality_relations_II}
        \sum_{\chi} \chi(K)\ol{\chi(L)} = 
        \begin{cases}
            \frac{\abs{G}}{\abs{K}}, &\ \text{αν $K = L$} \\
            0, &\ \text{διαφορετικά}
        \end{cases}
    \end{equation}
    όπου το $\chi$ στο άθροισμα διατρέχει όλους τους (διακεκριμένους) ανάγωγους χαρακτήρες της $G$. 
\end{theorem}
\begin{proof}[Απόδειξη]
    Η Ταυτότητα~\eqref{eq:orthogonality_relations_II} είναι ισοδύναμη με το να έχει ο (κανονικοποιημένος) πίνακας χαρακτήρων 
    \[
    \left( \sqrt{\frac{\abs{K}}{\abs{G}}} \chi(K)\right)_{\chi, K}
    \]
    ορθοκανονικές στήλες, όπου το $\chi$ διατρέχει τους (διακεκριμένους) ανάγωγους χαρακτήρες της $G$ (γραμμές του πίνακα) και το $K$ διατρέχει τις κλάσεις συζυγίας της $G$ (στήλες του πίνακα) (γιατί;). Αυτό είναι ισοδύναμο με το να έχει ο πίνακας αυτός ορθοκανονικές γραμμές (βλ. παρέκβαση Γραμμικής Άλγεβρας μετά το τέλος της απόδειξης). Αν $\chi$ και $\psi$ είναι δυο ανάγωγοι χαρακτήρες της $G$, τότε αυτό είναι ισοδύναμο με την ταυτότητα 
    \begin{equation}
        \label{eq:orthogonality_relations_I}
        \sum_{K} \abs{K} \, \chi(K)\ol{\psi(K)} = \sum_{g \in G} \chi(g)\ol{\psi(g)} = 
        \begin{cases}
            \abs{G}, &\ \text{αν $\chi = \psi$} \\
            0, &\ \text{διαφορετικά}
        \end{cases}
    \end{equation}
    όπου το $K$ στο άθροισμα διατρέχει όλες τις κλάσεις συζυγίας της $G$ (γιατί;). Όμως, αυτές είναι ακριβώς οι σχέσεις ορθογωνιότητας Ι (γιατί;) και η απόδειξη ολοκληρώνεται.
\end{proof}

\begin{digression_la}
Αν θεωρήσουμε τα στοιχεία του $\CC^n$ ως πίνακες-στήλη, τότε το σύνηθες εσωτερικό γινόμενο παίρνει την μορφή
\[
v \cdot w = w^\ast v,
\]
όπου $w^\ast \coloneqq \ol{w}^\top$, ενώ αν τα θεωρήσουμε ως πίνακες-γραμμή, τότε παίρνει την μορφή 
\[
v \cdot w = vw^\ast.
\]
Έχοντας αυτό κατά νου, ένας $(n\times n)$-πίνακας 
\[
A = 
\begin{pmatrix}
\vertbar & \vertbar &        &  \vertbar \\
v_1      & v_2      & \cdots & v_n \\
\vertbar & \vertbar &        & \vertbar 
\end{pmatrix}
\] 
έχει ορθοκανονικές στήλες αν και μόνο αν 
\[
v_j^\ast v_i = \delta_{ij} \coloneqq
\begin{cases}
    1, &\ \text{αν $i=j$} \\
    0, &\ \text{διαφορετικά}
\end{cases}
\]
για κάθε $1 \le i, j \le n$, ή ισοδύναμα αν
\[
A^\ast A = \rmI_n
\]
όπου $A^\ast \coloneqq \ol{A}^\top$ είναι ο προσηρτημένος του $A$ (έννοια, την οποία συναντήσαμε και στην Παράγραφο 2). Αυτό με τη σειρά του είναι ισοδύναμο με το 
\[
AA^\ast = \rmI_n,
\]
καθώς ο πίνακας $A$ είναι αντιστρέψιμος με αντίστροφο $A^\ast$ (γιατί;). Αν 
\[
A = 
\begin{pmatrix}
    \horzbar & w_1    & \horzbar \\
    \horzbar & w_2    & \horzbar \\
             & \vdots &  \\
    \horzbar & w_n    & \horzbar
\end{pmatrix}
\]
τότε η συνθήκη $AA^\ast = \rmI_n$ είναι ισοδύναμη με το 
\[
w_iw_j^\ast = \delta_{ij}
\]
για κάθε $1 \le i \le n$, δηλαδή αν και μόνο αν ο $A$ έχει ορθοκανονικές γραμμές.
\end{digression_la}

Εφαρμόζοντας τις σχέσεις ορθογωνιότητας ΙΙ στην πρώτη στήλη του πίνακα χαρακτήρων προκύπτει ο τύπος διάστασης (βλ. Ταυτότητα 4.1). Γενικά, οι σχέσεις ορθογωνιότητας μας βοηθούν να συμπληρώσουμε πίνακες χαρακτήρων ομάδων των οποίων λείπουν ορισμένα στοιχεία. Για παράδειγμα, αν $G$ είναι ομάδα τάξης 12 με πίνακα χαρακτήρων 
\[
\renewcommand{\arraystretch}{1.2}
\begin{array}{l|c|c|c|c}
           & K_1 & K_2 & K_3     & K_4 \\
           & 1   & 3   & 4       & 4       \\ \hline 
    \chi_1 & 1   & 1   & 1       & 1       \\ \hline
    \chi_2 & 1   & 1   & \zeta   & \zeta^2  \\ \hline
    \chi_3 & 1   & 1   & \zeta^2 & \zeta   \\ \hline
    \chi_4 & a   & b   & c & d
\end{array},
\]
με τέσσερις κλάσεις συζυγίας με 1, 3, 4 και 4 στοιχεία αντίστοιχα, τότε από τις σχέσεις ορθογωνιότητας στην πρώτη στήλη προκύπτει ότι $a = 3$ (γιατί;), ενώ στην πρώτη και τη δεύτερη στήλη προκύπτει ότι $b=-1$ (γιατί;) κ.ο.κ. Ποιά είναι τα $c$ και $d$; 
\end{document}