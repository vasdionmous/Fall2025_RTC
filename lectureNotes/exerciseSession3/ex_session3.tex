\documentclass[12pt,a4paper,reqno]{amsart}

% language
\usepackage[greek,english]{babel}
\usepackage[utf8]{inputenc}
\usepackage{alphabeta}

% change default names to greek
\addto\captionsenglish{
    \renewcommand{\contentsname}{Περιεχόμενα}
    \renewcommand{\refname}{Βιβλιογραφία}
    \renewcommand{\datename}{Ημερομηνία:}
    \renewcommand{\urladdrname}{Ιστοσελίδα}
}

% math 
\usepackage{amsmath,amsthm,amssymb,amscd}

% font
\usepackage[cal=euler]{mathalfa}
\usepackage{libertinus-type1}
% \usepackage{txfonts} % for upright greek letters
\usepackage{bm} % for bold symbols
\usepackage{bbm} % for the simply-looking bb symbols

% miscellaneous 
\usepackage{changepage} %for indenting environments
\usepackage{csquotes} % example: \textcquote{}
\usepackage{stmaryrd} % needed for \mapsfrom
\usepackage{multirow}


% drawing
\usepackage{tikz,tikz-cd}
\usetikzlibrary{shapes.misc, patterns, matrix, calc, intersections,positioning}
\usepackage{graphics,graphicx}
\usepackage{float} % provides enhanced control and customization options for floating objects such as figures and tables

% colors
\usepackage{xcolor}
\definecolor{darkcandyapplered}{rgb}{0.64, 0.0, 0.0}
\definecolor{midnightblue}{rgb}{0.1, 0.1, 0.44}
\definecolor{mylightblue}{HTML}{336699}
\definecolor{burntorange}{rgb}{0.8, 0.33, 0.0}
\definecolor{iceberg}{rgb}{0.44, 0.65, 0.82}
\definecolor{applegreen}{rgb}{0.55, 0.71, 0.0}
\definecolor{canaryyellow}{rgb}{1.0, 0.94, 0.0}

% hrefs
\usepackage{hyperref}
\usepackage[noabbrev,capitalize]{cleveref}
\hypersetup{
    pdftoolbar=true,        
    pdfmenubar=true,        
    pdffitwindow=false,     
    pdfstartview={FitH},  % fits the width of the page to the window
    pdftitle={},
    pdfauthor={},
    pdfsubject={},
    pdfkeywords={},
    pdfnewwindow=true,  % links in new window
    colorlinks=true,  % false: boxed links; true: colored links
    linkcolor=darkcandyapplered,   % color of internal links
    citecolor=midnightblue,  % color of links to bibliography
    urlcolor=cyan,  % color of external links
    linktocpage=true  % changes the links from the section body to the page number
    }

% geometry
\textwidth=16cm 
\textheight=21cm 
\hoffset=-55pt 
\footskip=25pt

% thm envs
\theoremstyle{definition}
\newtheorem*{example}{Παράδειγμα}
\newtheorem*{exercise}{Άσκηση}
\newtheorem*{solution}{Λύση}

% math ops (you might need to change the path)
% In this macro I define all of my math operators

% fields
\newcommand{\NN}{\mathbbmss{N}} 
\newcommand{\ZZ}{\mathbbmss{Z}} 
\newcommand{\QQ}{\mathbbmss{Q}} 
\newcommand{\RR}{\mathbbmss{R}} 
\newcommand{\CC}{\mathbbmss{C}} 
\newcommand{\KK}{\mathbbmss{K}} 
\newcommand{\FF}{\mathbbmss{F}} 

% symmetric group
\newcommand{\fS}{\mathfrak{S}}  

% calligraphic 
\newcommand{\aA}{\mathcal{A}} 
\newcommand{\bB}{\mathcal{B}}
\newcommand{\cC}{\mathcal{C}}
\newcommand{\dD}{\mathcal{D}}
\newcommand{\eE}{\mathcal{E}}
\newcommand{\fF}{\mathcal{F}}
\newcommand{\hH}{\mathcal{H}}
\newcommand{\iI}{\mathcal{I}}
\newcommand{\lL}{\mathcal{L}}
\newcommand{\oO}{\mathcal{O}}
\newcommand{\pP}{\mathcal{P}}
\newcommand{\sS}{\mathcal{S}}
\newcommand{\mM}{\mathcal{M}}
\newcommand{\uU}{\mathcal{U}}

% bold
\newcommand{\bfa}{\mathbf{a}}
\newcommand{\bfe}{\mathbf{e}}
\newcommand{\bfF}{\pmb{F}}
\newcommand{\bfR}{\pmb{R}}
\newcommand{\bfv}{\mathbf{v}}
%\newcommand{\bfx}{\bm{x}}
%\newcommand{\bfx}{\mathbf{x}} 
\newcommand{\bfx}{\pmb{x}}
\newcommand{\bfX}{\pmb{X}}
\newcommand{\bfy}{\pmb{y}}
\newcommand{\bfz}{\pmb{z}}

% roman
\newcommand{\rmA}{\mathrm{A}}
\newcommand{\rmB}{\mathrm{B}}
\newcommand{\rmC}{\mathrm{C}}
\newcommand{\rmD}{\mathrm{D}} 
\newcommand{\rmI}{\mathrm{I}} 
\newcommand{\rmK}{\mathrm{K}}
\newcommand{\rmM}{\mathrm{M}}
\newcommand{\rmP}{\mathrm{P}}  
\newcommand{\rmp}{\mathrm{p}}  
\newcommand{\rmQ}{\mathrm{Q}}  
\newcommand{\rmR}{\mathrm{R}}
\newcommand{\rmS}{\mathrm{S}}
\newcommand{\rmT}{\mathrm{T}}
\newcommand{\rmU}{\mathrm{U}}
\newcommand{\rmV}{\mathrm{V}}
\newcommand{\rmY}{\mathrm{Y}}
\newcommand{\rmZ}{\mathrm{Z}}
\newcommand{\rmz}{\mathrm{z}}

% greek letters
% I'm renewing some commands in order to appear in upright font
% If I want to change it later, I don't have to do it manually, I just change it from here.
% \newcommand{\uaa}{\alphaup}
% \renewcommand{\alpha}{\alphaup}
% \renewcommand{\beta}{\betaup}
% \renewcommand{\gamma}{\gammaup}
% \renewcommand{\delta}{\deltaup}
% \renewcommand{\epsilon}{\epsilonup}
% \newcommand{\ee}{\epsilon}
% \renewcommand{\varepsilon}{\varepsilonup}
% \renewcommand{\theta}{\thetaup}
% \renewcommand{\lambda}{\lambdaup}
% \newcommand{\ull}{\lambda}
% \renewcommand{\mu}{\muup}
% \renewcommand{\nu}{\nuup}
% \renewcommand{\pi}{\piup}
% \renewcommand{\rho}{\rhoup}
% \renewcommand{\varrho}{\varrhoup}
% \renewcommand{\sigma}{\sigmaup}
% \renewcommand{\tau}{\tauup} 
% \renewcommand{\phi}{\phiup}
% \renewcommand{\chi}{\chiup}
% \renewcommand{\psi}{\psiup}
% \renewcommand{\omega}{\omegaup}

% arrows and symbols 
\renewcommand{\to}{\rightarrow}
\newcommand{\toto}{\longrightarrow}
\newcommand{\mapstoto}{\longmapsto}
\newcommand{\then}{\Rightarrow}
\newcommand{\IFF}{\Leftrightarrow}
\newcommand{\tl}{\tilde}
\newcommand{\wtl}{\widetilde}
\newcommand{\ol}{\overline}
\newcommand{\ul}{\underline}
\newcommand{\oldemptyset}{\emptyset}
\renewcommand{\emptyset}{\varnothing}
\DeclareMathSymbol{\Arg}{\mathbin}{AMSa}{"39} % for arguments 
\newcommand{\onto}{\ensuremath{\twoheadrightarrow}}
\newcommand{\tle}{\trianglelefteq}
\newcommand{\tge}{\trianglerighteq}

% absolute value symbol
\usepackage{mathtools} 
\DeclarePairedDelimiter\abs{\lvert}{\rvert}%
\DeclarePairedDelimiter\norm{\lVert}{\rVert}%
\makeatletter
\let\oldabs\abs
\def\abs{\@ifstar{\oldabs}{\oldabs*}}

% tensor symbol
\newcommand{\tensor}[1]{%
  \mathbin{\mathop{\otimes}\limits_{#1}}%
}

% permutation cycle notation
\ExplSyntaxOn
\NewDocumentCommand{\cycle}{ O{\;} m }
 {
  (
  \alec_cycle:nn { #1 } { #2 }
  )
 }

\seq_new:N \l_alec_cycle_seq
\cs_new_protected:Npn \alec_cycle:nn #1 #2
 {
  \seq_set_split:Nnn \l_alec_cycle_seq { , } { #2 }
  \seq_use:Nn \l_alec_cycle_seq { #1 }
 }
\ExplSyntaxOff

% setminus symbol
\newcommand{\mysetminusD}{\hbox{\tikz{\draw[line width=0.6pt,line cap=round] (3pt,0) -- (0,6pt);}}}
\newcommand{\mysetminusT}{\mysetminusD}
\newcommand{\mysetminusS}{\hbox{\tikz{\draw[line width=0.45pt,line cap=round] (2pt,0) -- (0,4pt);}}}
\newcommand{\mysetminusSS}{\hbox{\tikz{\draw[line width=0.4pt,line cap=round] (1.5pt,0) -- (0,3pt);}}}
\newcommand{\sm}{\mathbin{\mathchoice{\mysetminusD}{\mysetminusT}{\mysetminusS}{\mysetminusSS}}}

% custom math operators
\newcommand{\Des}{\mathrm{Des}} 
\newcommand{\des}{\mathrm{des}} 
\newcommand{\Asc}{\mathrm{Asc}}
\newcommand{\asc}{\mathrm{asc}} 
\newcommand{\inv}{\mathrm{inv}}
\newcommand{\Inv}{\mathrm{Inv}}
\newcommand{\maj}{\mathrm{maj}} 
\newcommand{\comaj}{\mathrm{comaj}} 
\newcommand{\fix}{\mathrm{fix}} 
\newcommand{\Sym}{\mathrm{Sym}} 
\newcommand{\QSym}{\mathrm{QSym}}
\newcommand{\FQSym}{\mathrm{FQSym}} 
\newcommand{\End}{\mathrm{End}} 
\newcommand{\Rad}{\mathrm{Rad}} 
\newcommand{\rmMat}{\mathrm{Mat}} 
\newcommand{\rmdim}{\mathrm{dim}} 
\newcommand{\rmTop}{\mathrm{Top}} 
\newcommand{\rmCF}{\mathrm{CF}} 
\newcommand{\rmId}{\mathrm{Id}}
\newcommand{\rmid}{\mathrm{id}}
\newcommand{\rmtw}{\mathrm{tw}}
\newcommand{\trace}{\mathrm{tr}}
\newcommand{\Irr}{\mathrm{Irr}}
\newcommand{\Ind}{\mathrm{Ind}} % induction
\newcommand{\Res}{\mathrm{Res}} % restriction
\newcommand{\triv}{\mathrm{triv}} % trivial rep
\newcommand{\rmdef}{\mathrm{def}} % defining rep
\newcommand{\dom}{\triangleleft}
\newcommand{\domeq}{\trianglelefteq}
\newcommand{\lex}{\mathrm{lex}}
\newcommand{\sign}{\mathrm{sign}}
\newcommand{\SYT}{\mathrm{SYT}}
\renewcommand{\Im}{\mathrm{Im}}
\newcommand{\Ker}{\mathrm{Ker}}
\newcommand{\GL}{\mathrm{GL}}
\newcommand{\FL}{\mathrm{FL}}
\newcommand{\Span}{\mathrm{span}}
\newcommand{\pos}{\mathrm{pos}}
\newcommand{\Comp}{\mathrm{Comp}}
\newcommand{\Set}{\mathrm{Set}}
\newcommand{\std}{\mathrm{std}}
\newcommand{\cont}{\mathrm{cont}} %content of a SSYT
\newcommand{\SSYT}{\mathrm{SSYT}}
\newcommand{\ct}{\mathrm{ct}} % cycle type
\newcommand{\ch}{\mathrm{ch}} % Frobenius characteristic map
\newcommand{\height}{\mathrm{ht}}
\newcommand{\FPS}{\CC[\![\bfx]\!]} % formal power series
\newcommand{\FPSS}{\CC[\![\bfx,\bfy]\!]}
\newcommand{\reg}{\mathrm{reg}}
\newcommand{\hook}{\mathrm{h}}
\newcommand{\weight}{\mathrm{wt}}
\newcommand{\co}{\mathrm{co}}
\newcommand{\ps}{\mathrm{ps}}
\newcommand{\rmsum}{\mathrm{sum}}
\newcommand{\NSym}{\mathrm{NSym}}
\newcommand{\Hom}{\mathrm{Hom}}
\newcommand{\proj}{\mathrm{proj}}
\newcommand{\stat}{\mathrm{stat}}
\newcommand{\Par}{\mathrm{Par}}
\newcommand{\rmset}{\mathrm{set}}
\newcommand{\comp}{\mathrm{comp}}

% miscellaneous commands
\newcommand{\defn}[1]{{\color{mylightblue}{#1}}}
\newcommand{\toDo}{{\bf\color{red} TODO}}
\newcommand{\toCite}{{\bf\color{green} CITE}}

% 
\newenvironment{nouppercase}{%
  \let\uppercase\relax%
  \renewcommand{\uppercasenonmath}[1]{}}{}

\newcommand{\tcbo}[1]{\textcolor{burntorange}{#1}}

% titlepage
\title{Θ2.04: Θεωρία Αναπαραστάσεων και Συνδυαστική}
\author[Β.~Δ. Μουστακας]{Βασίλης Διονύσης Μουστάκας \\ Πανεπιστήμιο Κρήτης}
\date{24 Οκτωβρίου 2025}
% \urladdr{\href{https://sites.google.com/view/vasmous}{https://sites.google.com/view/vasmous}}

\begin{document}

\begingroup
\def\uppercasenonmath#1{} % this disables uppercase title
\let\MakeUppercase\relax % this disables uppercase authors
\maketitle
\endgroup

% \setcounter{section}{}
\thispagestyle{empty}

Στην παράγραφο 7, λίγο αργότερα, θα αποδείξουμε τα εξής:
\begin{itemize}
    \item Δυο αναπαραστάσεις είναι ισόμορφες αν και μόνο αν έχουν τον ίδιο χαρακτήρα.
    \item Το πλήθος των ανάγωγων χαρακτήρων μιας ομάδας ισούται με το πλήθος των κλάσεων συζυγίας της.
\end{itemize}
Ας υπολογίσουμε τον πίνακα χαρακτήρων της $\fS_4$, χρησιμοποιώντας τα εργαλεία που έχουμε αναπτύξει έως τώρα.

\begin{example}{\rm(Πίνακας χαρακτήρων της $\fS_4$)}
    Η $\fS_4$ έχει πέντε κλάσεις συζυγίας 
    \[
    \renewcommand{\arraystretch}{1.2} 
    \begin{array}{c|c|c|c|c|c}
                        & \rmK_{1111}  & \rmK_{211}  & \rmK_{22}              & \rmK_{31}     & \rmK_4          \\ \hline
    \text{αντιπρόσωπος} & \epsilon     & \cycle{1,2} & \cycle{1,2}\cycle{3,4} & \cycle{1,2,3} & \cycle{1,2,3,4} \\ \hline
    \text{πληθάριθμος} & 1            & 6           & 3                      & 8             &  6        
    \end{array}\ .
    \]
    Στην Παράγραφο 9, θα βρούμε έναν τύπο για το πληθάριθμο μιας αυθαίρετης κλάσης συζυγίας της $\fS_n$.

    Έχουμε συναντήσει τρεις ανάγωγους χαρακτήρες της $\fS_4$, τους χαρακτήρες της τετριμμένης αναπαράστασης, της αναπαράστασης προσήμου και της συνήθους αναπαράστασης 
    \[
    \renewcommand{\arraystretch}{1.2} 
    \begin{array}{c|c|c|c|c|c}
               & \rmK_{1111}  & \rmK_{211}  & \rmK_{22} & \rmK_{31} & \rmK_4  \\ \hline
    \chi^\triv & 1            & 1           & 1         & 1         & 1 \\ \hline 
    \chi^\sign & 1            & -1          & 1         & 1         & -1 \\ \hline 
    \chi^\std  & 3            & 1           & -1        & 0         & -1 
    \end{array}\ .
    \]
    Ψάχνουμε δύο ακόμα, έστω $\chi_4$ και $\chi_5$. Αν $a$ και $b$ είναι οι διαστάσεις τους, ο τύπος διάστασης
    \[
    24 = 1^2 + 1^2 + 3^2 + a^2 + b^2 \ \then a^2 + b^2 = 13
    \]
    μας πληροφορεί ότι $a = 3$ και $b = 2$. 

    Πως θα μπορούσαμε να βρούμε έναν χαρακτήρα διάστασης 3; Από την Πρόταση 6.6 γνωρίζουμε ότι το γινόμενο δυο χαρακτήρων είναι και αυτός χαρακτήρας. Προφανώς, ο πολλαπλασιασμός με τον τετριμμένο χαρακτήρα δεν προσφέρει κάτι καινούργιο (γιατί;). Όμως, 
    \[
    \renewcommand{\arraystretch}{1.2} 
    \begin{array}{c|c|c|c|c|c}
                        & \rmK_{1111}  & \rmK_{211}  & \rmK_{22} & \rmK_{31} & \rmK_4  \\ \hline
    \chi^\std\chi^\sign & 3            & -1          & -1        & 0         & 1 
    \end{array}\ .
    \]
    Θα μπορούσε να είναι ανάγωγος; Υπολογίζουμε 
    \[
    (\chi^\std\chi^\sign, \chi^\std\chi^\sign) = \frac{1}{24}(3^2 + 6\cdot(-1)^2 + 3\cdot(-1)^2 +8\cdot0 + 6\cdot1) = 1.
    \]
    Συνεπώς, από το Πόρισμα 7.5, ο $\chi^\std\chi^\sign$ είναι ανάγωγος και κατά συνέπεια βρήκαμε τον $\chi_4$.
\end{example}

Για να βρούμε τον $\chi_4$, παρατηρήσαμε ότι πολλαπλασιάζοντας έναν ανάγωγο χαρακτήρα με έναν χαρακτήρα διάστασης 1 προέκυψε ένας ακόμη ανάγωγος χαρακτήρας. Αυτό ισχύει γενικότερα.
\begin{exercise}
    Έστω $G$ πεπερασμένη ομάδα. Αν $\chi$ είναι ένας ανάγωγος χαρακτήρας της $G$ και $\psi$ είναι ένας χαρακτήρας διάστασης 1 της $G$, τότε ο $\chi\psi$ είναι ανάγωγος χαρακτήρας.
\end{exercise}

\begin{solution}
    Έχουμε 
    \[
    (\chi\psi,\chi\psi) 
    = \frac{1}{\abs{G}} \sum_{g \in G} \chi(g)\psi(g)\ol{\chi(g)}\ol{\psi(g)} 
    = \frac{1}{\abs{G}} \sum_{g \in G} \chi(g)\ol{\chi(g)}\left(\psi(g)\ol{\psi(g)}\right).
    \]
    Αφού όμως $\psi(\epsilon) = 1$, έπεται ότι
    \[
    \psi(g)\ol{\psi(g)} = \psi(g)\psi(g^{-1}) = \psi(gg^{-1}) = \psi(\epsilon) = 1,
    \]
    όπου η πρώτη ισότητα έπεται από την Άσκηση 2.2 (2) και η δεύτερη ισότητα έπεται από το ότι ο χαρακτήρας μια αναπαράστασης διάστασης 1 είναι ουσιαστικά το ίδιο με την αντίστοιχη αναπαράσταση. Συνεπώς, 
    \[
    (\chi\psi,\chi\psi) = \sum_{g \in G} \chi(g)\ol{\chi(g)} = (\chi,\chi) = 1,
    \]
    όπου η τελευταία ισότητα έπεται από το Πόρισμα 7.5.
\end{solution}

\begin{example}{\rm(Πίνακας χαρακτήρων της $\fS_4$, συνέχεια)}
    Συνεχίζοντας, για να βρούμε τον τελευταίο χαρακτήρα, το Πόρισμα 4.2 μας πληροφορεί ότι ο $\chi_5$ εμφανίζεται στην ισοτυπική διάσπαση του χαρακτήρα της κανονικής αναπαράστασης της $\fS_4$
    \[
    \renewcommand{\arraystretch}{1.2} 
    \begin{array}{c|c|c|c|c|c}
              & \rmK_{1111}  & \rmK_{211}  & \rmK_{22} & \rmK_{31} & \rmK_4  \\ \hline
    \chi^\reg & 6            & 0           & 0         & 0         & 0 
    \end{array}\ .
    \]
    Από το Πόρισμα 7.5, έπεται ότι 
    \[
    \chi^\reg = \chi^\triv + \chi^\sign + 3\chi^\std + 3\chi_4 + 2\chi_5 \ \then \ 
    \chi_5 = \left(\chi^\reg - \chi^\triv - \chi^\sign - 3\chi^\std - 3\chi_4\right)/2
    \]
    και γι αυτό 
    \[
    \renewcommand{\arraystretch}{1.2} 
    \begin{array}{c|c|c|c|c|c}
           & \rmK_{1111}  & \rmK_{211}  & \rmK_{22} & \rmK_{31} & \rmK_4  \\ \hline
    \chi_5 & 2            & 0           & 2         & -1        & 0 
    \end{array}\ .
    \]

    Συμπερασματικά, ο πίνακας χαρακτήρων της $\fS_4$ είναι 
    \[
    \renewcommand{\arraystretch}{1.2} 
    \begin{array}{c|c|c|c|c|c}
               & \rmK_{1111}  & \rmK_{211}  & \rmK_{22} & \rmK_{31} & \rmK_4  \\ \hline
    \chi^\triv & 1            & 1           & 1         & 1         & 1 \\ \hline 
    \chi^\sign & 1            & -1          & 1         & 1         & -1 \\ \hline 
    \chi^\std  & 3            & 1           & -1        & 0         & -1  \\ \hline 
    \chi_4     & 3            & -1          & -1        & 0         & 1 \\ \hline 
    \chi_5     & 2            & 0           & 2         & -1        & 0
    \end{array}\ .
    \]
    Τι παρατηρείτε;
\end{example}

\emph{Διαμέριση} του συνόλου $[n]$ ονομάζεται μια συλλογή μη κενών υποσυνόλων $S_1, S_2, \dots, S_k$ του $[n]$, τα οποία λέγονται \emph{μέρη}, έτσι ώστε κάθε στοιχείο του $[n]$ να εμφανίζεται ακριβώς μια φορά σε κάποιο μέρος. Μια τέτοια διαμέριση τη συμβολίζουμε με $S_1 \vert S_2 \vert \cdots \vert S_k$. Για παράδειγμα, για $n=4$
\[
\renewcommand{\arraystretch}{1.2} 
\begin{array}{c|c|c|c|c}
k & 1 & 2 & 3 & 4 \\ \hline
\multirow{7}{*}{\text{διαμέριση με $k$ μέρη}} 
  & 1234 & 1\vert234 & 12\vert3\vert4 & 1\vert2\vert3\vert4 \\ 
  &      & 2\vert134 & 13\vert2\vert4 & \\ 
  &      & 3\vert124 & 14\vert2\vert3 & \\ 
  &      & 4\vert123 & 23\vert1\vert4 & \\ 
  &      & 12\vert34 & 24\vert1\vert3 & \\ 
  &      & 13\vert24 & 34\vert1\vert2 & \\ 
  &      & 14\vert23 &                & \\ \hline
\text{πλήθος} & 1 & 7 & 6 & 1 
\end{array} \ .
\]

Το πλήθος των διαμερίσεων του $[n]$ με $k$ μέρη συμβολίζεται με $\rmS(n,k)$ και ονομάζεται \emph{αριθμός Stirling δευτέρου είδους}. Το πλήθος όλων των διαμερίσεων του $[n]$ συμβολίζεται με $\rmB(n)$ και ονομάζεται \emph{αριθμός Bell}. Οι δυο αυτές ακολουθίες αριθμών εμφανίζονται συχνά στα Διακριτά Μαθηματικά και σχετίζονται ως εξής 
\[
\rmB(n) = \rmS(n,1) + \rmS(n,2) + \cdots + \rmS(n,n).
\]

\begin{example}
    Έστω $S = \{12\vert34,13\vert24,14\vert23\}$ το σύνολο των διαμερίσεων του $[4]$ με δύο και δύο μέρη. Η δράση καθορισμού της $\fS_4$ στο $[4]$ επάγει μια δράση στο $S$ με τον προφανή τρόπο. Έστω $\chi$ ο χαρακτήρας της αντίστοιχης αναπαράστασης μεταθέσεων. Χρησιμοποιώντας την Άσκηση 2.3 (1) υπολογίζουμε 
    \[
    \renewcommand{\arraystretch}{1.2} 
    \begin{array}{c|c|c|c|c|c}
                        & \rmK_{1111}      & \rmK_{211}       & \rmK_{22}              & \rmK_{31}     & \rmK_4         \\ \hline
    \text{αντιπρόσωπος} & \epsilon         & \cycle{1,2}      & \cycle{1,2}\cycle{3,4} & \cycle{1,2,3} & \cycle{1,2,3,4} \\ \hline
    12\vert34           & \tcbo{12\vert34} & \tcbo{12\vert34} & \tcbo{12\vert34}       & 14\vert23     & 14\vert23        \\ \hline 
    13\vert24           & \tcbo{13\vert24} & 14\vert23        & \tcbo{13\vert24}       & 12\vert34     & \tcbo{13\vert24} \\ \hline 
    14\vert23           & \tcbo{14\vert23} & 13\vert24        & \tcbo{14\vert23}       & 13\vert24     & 12\vert34        \\ \hline 
    \chi                & 3                & 1                & 3                      & 0             & 1     
    \end{array}\ .
    \]

Ας υπολογίσουμε την ισοτυπική διάσπαση του $\chi$. Έχουμε την εξής διάσπαση 
\[
\CC[S] = \underbrace{\CC[12\vert34 + 13\vert24 + 14\vert23]}_{\cong \, V^\triv} \oplus \, \left(\CC[12\vert34 + 13\vert24 + 14\vert23]\right)^\perp,
\]
όπως και κάθε αναπαράσταση μεταθέσεων. Αν $\chi^\perp$ είναι ο χαρακτήρας του δεύτερου προσθεταίου της διάσπασης αυτής, τότε από την Πρόταση 6.6 υπολογίζουμε
\[
\renewcommand{\arraystretch}{1.2} 
\begin{array}{c|c|c|c|c|c}
            & \rmK_{1111}  & \rmK_{211}  & \rmK_{22} & \rmK_{31} & \rmK_4  \\ \hline
\chi^\perp & 2            & 0           & 2         & -1        & 0 
\end{array}
\]
και γι αυτό 
\[
(\chi^\perp,\chi^\perp) = \frac{1}{24}(2^2 + 6\cdot0 + 3\cdot4 + 8\cdot(-1)^2 + 6\cdot0) = 1.
\]
Άρα, από το πόρισμα 7.5, ο $\chi^\perp$ είναι ανάγωγος και η ισοτυπική του διάσπαση είναι 
\[
\chi = \chi^\triv + \chi_5,
\]
με τον συμβολισμό του προηγούμενου παραδείγματος.

Ας υπολογίσουμε το $\Hom_{\fS_4}\left(\CC[S],V^\rmdef\right)$, όπου $V^\rmdef$ είναι η αναπαράσταση καθορισμού της $\fS_4$. Από το Θεώρημα 7.2 γνωρίζουμε ότι 
\begin{align*}
\dim\left(\Hom_{\fS_4}\left(\CC[S],V^\rmdef\right)\right)
&= (\chi^\rmdef,\chi) \\
&= \frac{1}{24}(3\cdot4 + 6(2\cdot1) + 3(3\cdot0) + 8(0\cdot1) + 6(1\cdot0)) \\
&= 1,
\end{align*}
όπου $\chi^\rmdef$ είναι ο χαρακτήρας του $V^\rmdef$
\[
\renewcommand{\arraystretch}{1.2} 
\begin{array}{c|c|c|c|c|c}
            & \rmK_{1111}  & \rmK_{211}  & \rmK_{22} & \rmK_{31} & \rmK_4  \\ \hline
\chi^\rmdef & 4            & 2           & 0         & 1         & 0 
\end{array}\ .
\]
Άρα, υπάρχει μοναδικός $\fS_4$-ομομορφισμός $\CC[S] \to V^\rmdef$ ως προς βαθμωτά πολλαπλάσια. Ένας τέτοιος δίνεται από 
\begin{align*}
    \CC[12\vert34 + 13\vert24 + 14\vert23] \oplus \, \left(\CC[12\vert34 + 13\vert24 + 14\vert23]\right)^\perp &\to V^\triv \oplus V^\std \\
    (v,w) &\mapsto (v,0)
\end{align*}
(γιατί;).
\end{example}
\end{document}