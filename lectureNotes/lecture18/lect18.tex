\documentclass[12pt,a4paper,reqno]{amsart}

% language
\usepackage[greek,english]{babel}
\usepackage[utf8]{inputenc}
\usepackage{alphabeta}

% change default names to greek
\addto\captionsenglish{
    \renewcommand{\contentsname}{Περιεχόμενα}
    \renewcommand{\refname}{Βιβλιογραφία}
    \renewcommand{\datename}{Ημερομηνία:}
    \renewcommand{\urladdrname}{Ιστοσελίδα}
}

% math 
\usepackage{amsmath,amsthm,amssymb,amscd}

% font
\usepackage[cal=euler]{mathalfa}
\usepackage{libertinus-type1}
% \usepackage{txfonts} % for upright greek letters
\usepackage{bm} % for bold symbols
\usepackage{bbm} % for the simply-looking bb symbols

% miscellaneous 
\usepackage{changepage} %for indenting environments
\usepackage{csquotes} % example: \textcquote{}
\usepackage{blkarray}
\setcounter{MaxMatrixCols}{20} % default for pmatrix is 10!!
\usepackage{ytableau}

% drawing
\usepackage{tikz,tikz-cd}
\usetikzlibrary{shapes.misc, patterns, matrix, calc, intersections,positioning}
\usepackage{graphics,graphicx}
\usepackage{float} % provides enhanced control and customization options for floating objects such as figures and tables

% colors
\usepackage{xcolor}
\definecolor{darkcandyapplered}{rgb}{0.64, 0.0, 0.0}
\definecolor{midnightblue}{rgb}{0.1, 0.1, 0.44}
\definecolor{mylightblue}{HTML}{336699}
\definecolor{burntorange}{rgb}{0.8, 0.33, 0.0}
\definecolor{iceberg}{rgb}{0.44, 0.65, 0.82}
\definecolor{applegreen}{rgb}{0.55, 0.71, 0.0}
\definecolor{canaryyellow}{rgb}{1.0, 0.94, 0.0}

% hrefs
\usepackage{hyperref}
\usepackage[noabbrev,capitalize]{cleveref}
\hypersetup{
    pdftoolbar=true,        
    pdfmenubar=true,        
    pdffitwindow=false,     
    pdfstartview={FitH},  % fits the width of the page to the window
    pdftitle={},
    pdfauthor={},
    pdfsubject={},
    pdfkeywords={},
    pdfnewwindow=true,  % links in new window
    colorlinks=true,  % false: boxed links; true: colored links
    linkcolor=darkcandyapplered,   % color of internal links
    citecolor=midnightblue,  % color of links to bibliography
    urlcolor=cyan,  % color of external links
    linktocpage=true  % changes the links from the section body to the page number
    }

% geometry
\textwidth=16cm 
\textheight=21cm 
\hoffset=-55pt 
\footskip=25pt

% thm envs (you might need to change the path)
% In this macro I define all the theorem environments

\theoremstyle{definition}
\newtheorem{theorem}{Θεώρημα}
\newtheorem{proposition}[theorem]{Πρόταση}
\newtheorem{lemma}[theorem]{Λήμμα}
\newtheorem{corollary}[theorem]{Πόρισμα}
\newtheorem{conjecture}[theorem]{Εικασία}
\newtheorem{problem}[theorem]{Πρόβλημα}
\newtheorem*{claim}{Ισχυρισμός}
\newtheorem{observation}[theorem]{Παρατήρηση}
\newtheorem{definition}[theorem]{Ορισμός}
\newtheorem{question}[theorem]{Ερώτηση}
\newtheorem*{questions}{Ερωτήματα}
\newtheorem{example}[theorem]{Παράδειγμα}
\newtheorem{exercise}{Άσκηση}

\newtheorem*{combInterlude}{Ιντερλούδιο Συνδυαστικής}
\newtheorem*{example_cont}{Παράδειγμα~6.6}
\newtheorem*{digression_la}{Παρέκβαση Γραμμικής Άλγεβρας}
\newtheorem*{thm}{Θεώρημα}

\theoremstyle{remark}
\newtheorem*{remark}{Παρατήρηση}

% fixes the correct numbering of environments
\numberwithin{theorem}{section}
\numberwithin{exercise}{section}
\numberwithin{equation}{section}

% math ops (you might need to change the path)
% In this macro I define all of my math operators

% fields
\newcommand{\NN}{\mathbbmss{N}} 
\newcommand{\ZZ}{\mathbbmss{Z}} 
\newcommand{\QQ}{\mathbbmss{Q}} 
\newcommand{\RR}{\mathbbmss{R}} 
\newcommand{\CC}{\mathbbmss{C}} 
\newcommand{\KK}{\mathbbmss{K}} 
\newcommand{\FF}{\mathbbmss{F}} 

% symmetric group
\newcommand{\fS}{\mathfrak{S}}  

% calligraphic 
\newcommand{\aA}{\mathcal{A}} 
\newcommand{\bB}{\mathcal{B}}
\newcommand{\cC}{\mathcal{C}}
\newcommand{\dD}{\mathcal{D}}
\newcommand{\eE}{\mathcal{E}}
\newcommand{\fF}{\mathcal{F}}
\newcommand{\hH}{\mathcal{H}}
\newcommand{\iI}{\mathcal{I}}
\newcommand{\lL}{\mathcal{L}}
\newcommand{\oO}{\mathcal{O}}
\newcommand{\pP}{\mathcal{P}}
\newcommand{\sS}{\mathcal{S}}
\newcommand{\mM}{\mathcal{M}}
\newcommand{\uU}{\mathcal{U}}

% bold
\newcommand{\bfa}{\mathbf{a}}
\newcommand{\bfe}{\mathbf{e}}
\newcommand{\bfF}{\pmb{F}}
\newcommand{\bfR}{\pmb{R}}
\newcommand{\bfv}{\mathbf{v}}
%\newcommand{\bfx}{\bm{x}}
%\newcommand{\bfx}{\mathbf{x}} 
\newcommand{\bfx}{\pmb{x}}
\newcommand{\bfX}{\pmb{X}}
\newcommand{\bfy}{\pmb{y}}
\newcommand{\bfz}{\pmb{z}}

% roman
\newcommand{\rmA}{\mathrm{A}}
\newcommand{\rmB}{\mathrm{B}}
\newcommand{\rmC}{\mathrm{C}}
\newcommand{\rmD}{\mathrm{D}} 
\newcommand{\rmI}{\mathrm{I}} 
\newcommand{\rmK}{\mathrm{K}}
\newcommand{\rmM}{\mathrm{M}}
\newcommand{\rmP}{\mathrm{P}}  
\newcommand{\rmp}{\mathrm{p}}  
\newcommand{\rmQ}{\mathrm{Q}}  
\newcommand{\rmR}{\mathrm{R}}
\newcommand{\rmS}{\mathrm{S}}
\newcommand{\rmT}{\mathrm{T}}
\newcommand{\rmU}{\mathrm{U}}
\newcommand{\rmV}{\mathrm{V}}
\newcommand{\rmY}{\mathrm{Y}}
\newcommand{\rmZ}{\mathrm{Z}}
\newcommand{\rmz}{\mathrm{z}}

% greek letters
% I'm renewing some commands in order to appear in upright font
% If I want to change it later, I don't have to do it manually, I just change it from here.
% \newcommand{\uaa}{\alphaup}
% \renewcommand{\alpha}{\alphaup}
% \renewcommand{\beta}{\betaup}
% \renewcommand{\gamma}{\gammaup}
% \renewcommand{\delta}{\deltaup}
% \renewcommand{\epsilon}{\epsilonup}
% \newcommand{\ee}{\epsilon}
% \renewcommand{\varepsilon}{\varepsilonup}
% \renewcommand{\theta}{\thetaup}
% \renewcommand{\lambda}{\lambdaup}
% \newcommand{\ull}{\lambda}
% \renewcommand{\mu}{\muup}
% \renewcommand{\nu}{\nuup}
% \renewcommand{\pi}{\piup}
% \renewcommand{\rho}{\rhoup}
% \renewcommand{\varrho}{\varrhoup}
% \renewcommand{\sigma}{\sigmaup}
% \renewcommand{\tau}{\tauup} 
% \renewcommand{\phi}{\phiup}
% \renewcommand{\chi}{\chiup}
% \renewcommand{\psi}{\psiup}
% \renewcommand{\omega}{\omegaup}

% arrows and symbols 
\renewcommand{\to}{\rightarrow}
\newcommand{\toto}{\longrightarrow}
\newcommand{\mapstoto}{\longmapsto}
\newcommand{\then}{\Rightarrow}
\newcommand{\IFF}{\Leftrightarrow}
\newcommand{\tl}{\tilde}
\newcommand{\wtl}{\widetilde}
\newcommand{\ol}{\overline}
\newcommand{\ul}{\underline}
\newcommand{\oldemptyset}{\emptyset}
\renewcommand{\emptyset}{\varnothing}
\DeclareMathSymbol{\Arg}{\mathbin}{AMSa}{"39} % for arguments 
\newcommand{\onto}{\ensuremath{\twoheadrightarrow}}
\newcommand{\tle}{\trianglelefteq}
\newcommand{\tge}{\trianglerighteq}

% absolute value symbol
\usepackage{mathtools} 
\DeclarePairedDelimiter\abs{\lvert}{\rvert}%
\DeclarePairedDelimiter\norm{\lVert}{\rVert}%
\makeatletter
\let\oldabs\abs
\def\abs{\@ifstar{\oldabs}{\oldabs*}}

% tensor symbol
\newcommand{\tensor}[1]{%
  \mathbin{\mathop{\otimes}\limits_{#1}}%
}

% permutation cycle notation
\ExplSyntaxOn
\NewDocumentCommand{\cycle}{ O{\;} m }
 {
  (
  \alec_cycle:nn { #1 } { #2 }
  )
 }

\seq_new:N \l_alec_cycle_seq
\cs_new_protected:Npn \alec_cycle:nn #1 #2
 {
  \seq_set_split:Nnn \l_alec_cycle_seq { , } { #2 }
  \seq_use:Nn \l_alec_cycle_seq { #1 }
 }
\ExplSyntaxOff

% setminus symbol
\newcommand{\mysetminusD}{\hbox{\tikz{\draw[line width=0.6pt,line cap=round] (3pt,0) -- (0,6pt);}}}
\newcommand{\mysetminusT}{\mysetminusD}
\newcommand{\mysetminusS}{\hbox{\tikz{\draw[line width=0.45pt,line cap=round] (2pt,0) -- (0,4pt);}}}
\newcommand{\mysetminusSS}{\hbox{\tikz{\draw[line width=0.4pt,line cap=round] (1.5pt,0) -- (0,3pt);}}}
\newcommand{\sm}{\mathbin{\mathchoice{\mysetminusD}{\mysetminusT}{\mysetminusS}{\mysetminusSS}}}

% custom math operators
\newcommand{\Des}{\mathrm{Des}} 
\newcommand{\des}{\mathrm{des}} 
\newcommand{\Asc}{\mathrm{Asc}}
\newcommand{\asc}{\mathrm{asc}} 
\newcommand{\inv}{\mathrm{inv}}
\newcommand{\Inv}{\mathrm{Inv}}
\newcommand{\maj}{\mathrm{maj}} 
\newcommand{\comaj}{\mathrm{comaj}} 
\newcommand{\fix}{\mathrm{fix}} 
\newcommand{\Sym}{\mathrm{Sym}} 
\newcommand{\QSym}{\mathrm{QSym}}
\newcommand{\FQSym}{\mathrm{FQSym}} 
\newcommand{\End}{\mathrm{End}} 
\newcommand{\Rad}{\mathrm{Rad}} 
\newcommand{\rmMat}{\mathrm{Mat}} 
\newcommand{\rmdim}{\mathrm{dim}} 
\newcommand{\rmTop}{\mathrm{Top}} 
\newcommand{\rmCF}{\mathrm{CF}} 
\newcommand{\rmId}{\mathrm{Id}}
\newcommand{\rmid}{\mathrm{id}}
\newcommand{\rmtw}{\mathrm{tw}}
\newcommand{\trace}{\mathrm{tr}}
\newcommand{\Irr}{\mathrm{Irr}}
\newcommand{\Ind}{\mathrm{Ind}} % induction
\newcommand{\Res}{\mathrm{Res}} % restriction
\newcommand{\triv}{\mathrm{triv}} % trivial rep
\newcommand{\rmdef}{\mathrm{def}} % defining rep
\newcommand{\dom}{\triangleleft}
\newcommand{\domeq}{\trianglelefteq}
\newcommand{\lex}{\mathrm{lex}}
\newcommand{\sign}{\mathrm{sign}}
\newcommand{\SYT}{\mathrm{SYT}}
\renewcommand{\Im}{\mathrm{Im}}
\newcommand{\Ker}{\mathrm{Ker}}
\newcommand{\GL}{\mathrm{GL}}
\newcommand{\FL}{\mathrm{FL}}
\newcommand{\Span}{\mathrm{span}}
\newcommand{\pos}{\mathrm{pos}}
\newcommand{\Comp}{\mathrm{Comp}}
\newcommand{\Set}{\mathrm{Set}}
\newcommand{\std}{\mathrm{std}}
\newcommand{\cont}{\mathrm{cont}} %content of a SSYT
\newcommand{\SSYT}{\mathrm{SSYT}}
\newcommand{\ct}{\mathrm{ct}} % cycle type
\newcommand{\ch}{\mathrm{ch}} % Frobenius characteristic map
\newcommand{\height}{\mathrm{ht}}
\newcommand{\FPS}{\CC[\![\bfx]\!]} % formal power series
\newcommand{\FPSS}{\CC[\![\bfx,\bfy]\!]}
\newcommand{\reg}{\mathrm{reg}}
\newcommand{\hook}{\mathrm{h}}
\newcommand{\weight}{\mathrm{wt}}
\newcommand{\co}{\mathrm{co}}
\newcommand{\ps}{\mathrm{ps}}
\newcommand{\rmsum}{\mathrm{sum}}
\newcommand{\NSym}{\mathrm{NSym}}
\newcommand{\Hom}{\mathrm{Hom}}
\newcommand{\proj}{\mathrm{proj}}
\newcommand{\stat}{\mathrm{stat}}
\newcommand{\Par}{\mathrm{Par}}
\newcommand{\rmset}{\mathrm{set}}
\newcommand{\comp}{\mathrm{comp}}

% miscellaneous commands
\newcommand{\defn}[1]{{\color{mylightblue}{#1}}}
\newcommand{\toDo}{{\bf\color{red} TODO}}
\newcommand{\toCite}{{\bf\color{green} CITE}}
\newcommand*{\vertbar}{\rule[-1ex]{0.5pt}{2.5ex}} % for matrices with column vectors
\newcommand*{\horzbar}{\rule[.5ex]{2.5ex}{0.5pt}} % for matrices with row vectors
\newcommand{\myblue}[1]{{\color{iceberg}{#1}}}
\newcommand{\myorange}[1]{{\color{burntorange}{#1}}}
\newcommand{\mygreen}[1]{{\color{applegreen}{#1}}}
\newcommand{\myred}[1]{{\color{darkcandyapplered}{#1}}}

% ferrer's diagram
\newcommand{\fdiagram}[1]{
    \begin{tikzpicture}[scale=.7]
        \fill foreach \Z [count=\Y] in {#1}
        {foreach \X in {1,...,\Z} 
        {(\X,-\Y) circle[radius=3pt]}};
    \end{tikzpicture}
}

%
\newcommand{\tcbo}[1]{\textcolor{burntorange}{#1}}

% 
\newenvironment{nouppercase}{%
  \let\uppercase\relax%
  \renewcommand{\uppercasenonmath}[1]{}}{}

% titlepage
\title{Θ2.04: Θεωρία Αναπαραστάσεων και Συνδυαστική}
\author[Β.~Δ. Μουστακας]{Βασίλης Διονύσης Μουστάκας \\ Πανεπιστήμιο Κρήτης}
\date{2 Δεκεμβρίου 2025}
% \urladdr{\href{https://sites.google.com/view/vasmous}{https://sites.google.com/view/vasmous}}

\begin{document}

\begingroup
\def\uppercasenonmath#1{} % this disables uppercase title
\let\MakeUppercase\relax % this disables uppercase authors
\maketitle
\endgroup

\setcounter{section}{14}
\setcounter{theorem}{0}
\begin{center}
    \textbf{14. Στοιχεία αλγεβρικής συνδυαστικής: Γεννήτριες συναρτήσεις και τυπικές δυναμοσειρές
} 
\end{center}


Στην Παράγραφο 9 είδαμε ότι δεν γνωρίζουμε ούτε κλειστό τύπο για το πλήθος $\rmp(n)$ των διαμερίσεων του $n$, αλλά ούτε και αναδρομικό τύπο. Τι μπορούμε να κάνουμε σε αυτή την περίπτωση; Μια ιδέα είναι αντί να ψάχνουμε κάθε $\rmp(1), \rmp(2), \rmp(3), \dots$ ξεχωριστά, να \textquote{οργανώσουμε} όλους αυτούς τους (άπειρους το πλήθος) αριθμούς σε μια συνάρτηση.

\begin{definition}
    \label{def:generating_function}
    Έστω $(a_n)_{n\ge0}$ μια ακολουθία (μιγαδικών) αριθμών. Η δυναμοσειρά 
    \[
    \sum_{n\ge0} a_n x^n = a_0 + a_1x + a_2x^2 + \cdots 
    \]
    ονομάζεται \defn{γεννήτρια συνάρτηση} (generating function) της ακολουθίας $(a_n)_{n\ge0}$.
\end{definition}

Για παράδειγμα, για κάποιο $n \ge 0$, έστω $a_k = \binom{n}{k}$, για κάθε $k \ge 0$. Η γεννήτρια συνάρτηση της ακολουθίας $(a_k)_{k \ge0}$ δίνεται από 
\[
\sum_{k\ge0} \binom{n}{k} x^k = 
\sum_{k=0}^n \binom{n}{k} x^k = 
(1+x)^n,
\]
όπου η τελευταία ισότητα είναι γνωστή ως \emph{Διωνυμικό Θεώρημα}.

Για ένα ακόμα παράδειγμα, έστω $(a_n)_{n\ge0}$ η ακολουθία αριθμών\footnote{Οι αριθμοί αυτοί ονομάζονται \emph{αριθμοί Fibonacci}.} που ορίζεται από τον αναδρομικό τύπο
\begin{equation}
    \label{eq:fibonacci}
    a_n = a_{n-1} + a_{n-2}, 
\end{equation}
για κάθε $n \ge 2$, όπου $a_0=a_1=1$. Αν συμβολίσουμε με $F(x)$ την γεννήτρια συνάρτηση της $(a_n)_{n\ge0}$, τότε 
\begin{align*}
    F(x) 
    &= \sum_{n \ge 0} a_n x^n \\
    &= 1 + x + \sum_{n \ge 2} a_n x^n \\
    &= 1 + x + \sum_{n \ge 2} (a_{n-1} + a_{n-2}) x^n \\
    &= 1 + x + \sum_{n \ge 2} a_{n-1}x^n + \sum_{n\ge 2}a_{n-2} x^n \\
    &= 1 + x + x\sum_{n \ge 2} a_{n-1}x^{n-1} + x^2\sum_{n\ge 2}a_{n-2} x^{n-2} \\
    &= 1 + x + x(F(x)-1) + x^2F(x) \\
    &= 1 + (x + x^2)F(x) \\
\end{align*}
και γι αυτό 
\[
F(x) = \frac{1}{1-x-x^2} 
= \frac{1}{x\sqrt{5}}\left(\frac{1}{1-\varphi{x}} + \frac{1}{1-\ol{\varphi}{x}}\right),
\]
όπου $\varphi = \frac{1 + \sqrt{5}}{2}$ και $\ol{\varphi} = \frac{1 + \sqrt{5}}{2}$. Άρα, 
\[
F(x) = \frac{1}{x\sqrt{5}}\sum_{n\ge0}(\varphi^n - \ol{\varphi}^n)x^n
\]
και γι αυτό 
\[
a_n = \frac{\varphi^{n+1} - \ol{\varphi}^{n+1}}{\sqrt{5}}.
\]

Τι ακριβώς είναι μια γεννήτρια συνάρτηση; Μια γεννήτρια συνάρτηση δεν είναι δυναμοσειρά με την αναλυτική έννοια. Πρόκειτα για μια \textquote{τυπική} δυναμοσειρά, δηλαδή μια έκφραση της μορφής 
\[
a_0 + a_1x + a_2x^2 + \cdots, 
\]
για κάποιους αριθμούς $a_0, a_1, a_2, \dots$. Συνεπώς, εξαρτάται από τους συντελεστές της και δεν είναι (απαραίτητα) ίση με την σειρά Taylor κάποιας συνάρτησης. 

Ας το κάνουμε πιο συγκεκριμένο. Έστω $\bfx = (x_1, x_2, \dots)$ μια (αριθμήσιμα άπειρη) ακολουθία μεταβλητών που μετατίθενται. Αν $\alpha = (\alpha_1, \alpha_2, \dots, \alpha_k)$ είναι μια ακολουθία μη αρνητικών ακεραίων, τότε το 
\[
\bfx^\alpha \coloneqq x_1^{\alpha_1}x_2^{\alpha_2}\cdots x_k^{\alpha_k}
\]
ονομάζεται \defn{μονώνυμο} με \defn{εκθέτες}  $\alpha_1, \alpha_2, \dots, \alpha_k$.
\begin{definition}
\label{def:formal_power_series} 
Μια έκφραση της μορφής 
\[
\sum_{\alpha} c_\alpha \bfx^\alpha
\]
για $c_\alpha \in \CC$, όπου στο άθροισμα το $\alpha$ διατρέχει όλες τις ακολουθίες $(\alpha_1, \alpha_2, \dots)$ μη αρνητικών ακεραίων, οι οποίοι είναι όλοι μηδέν, εκτός από το πολύ πεπερασμένο πλήθος αυτών, ονομάζεται \defn{τυπική δυναμοσειρά} (formal power series).
\end{definition}

Με άλλα λόγια, μια τυπική δυναμοσειρά είναι ένας γραμμικός συνδυασμός (άπειρων το πλήθος) μονωνύμων με συντελεστές στο $\CC$. Το σύνολο $\CC[\![\bfx]\!]$ όλων των τυπικών δυναμοσειρών αποτελεί διανυσματικό χώρο με την προφανή πράξη της πρόσθεσης, δηλαδή 
\begin{equation}
    \label{eq:FPS_addition}
\left(\sum_{\alpha} c_\alpha \bfx^\alpha\right) + 
\left(\sum_{\alpha} d_\alpha \bfx^\alpha\right) = 
\sum_{\alpha} (c_\alpha + d_\alpha) \bfx^\alpha.
\end{equation}
Επίσης, θέτοντας 
\begin{equation} 
    \label{eq:FPS_multiplication}
\left(\sum_{\alpha} c_\alpha \bfx^\alpha\right) + 
\left(\sum_{\beta} d_\beta \bfx^\beta\right) = 
\sum_{\gamma}\left(\sum_{\alpha + \beta = \gamma} c_\alpha c_\beta\right) \bfx^\gamma, 
\end{equation}
όπου $\alpha + \beta \coloneqq (\alpha_1 + \beta_1, \alpha_2 + \beta_2, \dots)$ το $\CC[\![\bfx]\!]$ αποκτά την δομή \emph{δακτύλιου}. 

Για παράδειγμα, αν έχουμε μόνο μία μεταβλητή\footnote{Για να κάνουμε τον διαχωρισμό ανάμεσα σε μια ακολουθία μεταβλητών $\bfx$ και σε μια μεμονωμένη μεταβλητή $x$ χρησιμοποιούμε έντονη γραφή.} $x$ οι Ταυτότητες~\eqref{eq:FPS_addition} και \eqref{eq:FPS_multiplication} γίνονται 
\begin{align*}
    \left(\sum_{n\ge0} c_n x^n\right) + 
    \left(\sum_{n\ge0} d_n x^n\right) &= 
    \sum_{n\ge0} (c_n + d_n) x^n  \\
    \left(\sum_{n\ge0} c_n x^n\right)\left(\sum_{n\ge0} d_n x^n\right) &= 
    \sum_{n\ge0}\left(\sum_{k=0}^n c_kd_{n-k}\right) x^n.
\end{align*}

Οι γεννήτριες συναρτήσεις είναι τυπικές δυναμοσειρές μια μεταβλητής με την έννοια του Ορισμού~\ref{def:formal_power_series}. Το πλεονέκτημα των τυπικών δυναμοσειρών έναντι των \textquote{αναλυτικών} είναι ότι δεν μας απασχολεί πότε ή που συγκλίνουν, αρκεί να είναι καλά ορισμένα στοιχεία του $\CC[\![\bfx]\!]$. 

Για παράδειγμα, μια (αναλυτική) συνάρτηση όπως η $\exp(\Arg)$ μπορεί να θεωρηθεί ως τυπική δυναμοσειρά ορίζοντας 
\[
\exp(x) \coloneqq \sum_{n\ge0}\frac{x^n}{n!}.
\]
Έτσι, η \textquote{συνάρτηση} $\exp(x)$ μπορεί να θεωρηθεί ως η γεννήτρια συνάρτηση της ακολουθίας $(\frac{1}{n!})_{n\ge0}$. Ομοίως και για άλλες γνωστές (αναλυτικές) συναρτήσεις.

Τέλος, θα θέλαμε να ισχύουν οι γνωστές ταυτότητες των (αναλυτικών) δυναμοσειρών όπως, για παράδειγμα, η 
\begin{equation}
    \label{eq:geometric_series}
1 + x + x^2 + \cdots = \frac{1}{1-x}.
\end{equation}
Παρόλα αυτά, το δεξί μέλος της Ταυτότητας~\eqref{eq:geometric_series} δεν έχει (από μόνο του) νόημα στον δακτύλιο $\CC[\![x]\!]$. Όμως, παρατηρούμε ότι 
\[
(1-x)(1+x+x^2+\cdots) = 1 + x + x^2 + \cdots - x - x^2 - \cdots = 1.
\]
Συνεπώς, το $1-x$ είναι αντιστρέψιμο στο $\CC[\![x]\!]$ και το αντίστροφό του, το οποίο συμβολίζουμε με $\frac{1}{1-x}$, ισούται με το $\sum_{n\ge0}x^n$. 

Επιστρέφοντας στην αρχική συζήτηση αυτής τηε παραγράφου, η γεννήτρια συνάρτηση της $(\rmp(n))_{n\ge0}$, όπου $\rmp(0)\coloneqq 1$ έχει μια απροσδόκητα απλή μορφή.

\begin{theorem}{\rm(Euler 1748)}
    \label{thm:Euler}
    Ισχύει ότι 
    \begin{equation}
        \label{eq:Euler}
        \sum_{n\ge0}\rmp(n) x^n = \prod_{i\ge1}\frac{1}{1-x^i}.
    \end{equation}
\end{theorem}
\begin{proof}[Απόδειξη]
    Θεωρούμε το (άπειρο) σύνολο\footnote{Το συναντήσαμε ξανά στην Παράγραφο 9 ως διάταξη Young.} 
    \[
    \Par \coloneqq \bigcup_{n\ge0} \Par(n).
    \]
    Το αριστερό μέλος της Ταυτότητας~\eqref{eq:Euler} γράφεται ισοδύναμα 
    \[
    \sum_{n\ge0}\rmp(n) x^n = \sum_{\lambda \in \Par} x^{\abs{\lambda}},
    \]
    όπου με $\abs{\lambda}$ συμβολίζουμε το άθροισμα των μερών της διαμέρισης $\lambda$. Για παράδειγμα, αν $\lambda = (4,2,2,1)$, τότε $
    \abs{\lambda} = 4 + 2 + 2 + 1 = 9$.

    Έχουμε δει ότι κάθε διαμέριση $\lambda \vdash n$ μπορεί να παρασταθεί από μία διατεταγμένη $n$-άδα της μορφής $(1^{m_1},2^{m_2},\dots,n^{m_n})$, όπου $m_i$ είναι η πολλαπλότητα εμφάνισης του μέρους $i$ στην $\lambda$, έτσι ώστε
    \[
    x^{\abs{\lambda}} = x^{m_1 + 2m_2 + \cdots + nm_n}.
    \]
    Η διαμέριση του παραδείγματος αντοιστοιχεί στην ακολουθία 
    \[
    (1^1, 2^2, 3^0, 4^1, 5^0, 6^0, 7^0, 8^0, 9^0).
    \]
    Γενικότερα, ένα στοιχείο του $\Par$ μπορεί να παρασταθεί ως $(1^{m_1}, 2^{m_2}, \dots)$ για κάποια $m_i \ge 0$, τα οποία είναι όλα μηδέν, εκτός από το πολύ πεπερασμένο πλήθος αυτών.
    
    Με άλλα λόγια, έχουμε μια αμφιμονοσήμαντη αντιστοιχία μεταξύ του $\Par$ και του συνόλου 
    \[
    \left(\bigcup_{m_1\ge0}(1^{m_1})\right)\times
    \left(\bigcup_{m_2\ge0}(2^{m_2})\right)\times\cdots.
    \]
    Σε επίπεδο γεννητριών συναρτήσεων το $\cup$ (αντ. $\times$) μεταφράζεται σε $\sum$ (αντ. $\prod$) και γι αυτό 
    \begin{align*} 
    \sum_{\lambda \in \Par} x^{\abs{\lambda}} 
    &= \prod_{i\ge1}\left(\sum_{m_i \ge 0} (x^i)^{m_i}\right) \\
    &= \prod_{i\ge1}\left(1 + x^i + (x^i)^2 + \cdots\right) \\
    &= \prod_{i\ge1}\frac{1}{1-x^i},\\
    \end{align*}
    όπου η τελευταία ισότητα έπεται από την Ταυτότητα~\eqref{eq:geometric_series}.
\end{proof}

Κάποιος που συναντά για πρώτη φορά τυπικές δυναμοσειρές θα πρέπει να είναι σκεπτικός για το αν το δεξί μέλος της Ταυτότητας~\eqref{eq:Euler} είναι καλώς ορισμένο στοιχείο του $\CC[\![x]\!]$. Όμως, οι όροι $\frac{1}{1-x^i}$ του γινομένου με $i \ge n+1$ στην πραγματικότητα δε συνεισφέρουν στον συντελεστή του $x^n$ και κατά συνέπεια στον υπολογισμό του $\rmp(n)$. Για παράδειγμα, ο συντελεστής του $x^3$ στο 
\begin{align*}
\prod_{i\ge1}\frac{1}{1-x^i} &= 
\left(\frac{1}{1-x}\right)
\left(\frac{1}{1-x^2}\right)
\left(\frac{1}{1-x^3}\right)\cdots \\
&= (\myblue{1} + \mygreen{x} + x^2 + \myorange{x^3} + \cdots)(\myblue{1} + \mygreen{x^2} + x^4 + \cdots)(1 + \myblue{x^3} + x^6 + \cdots)\cdots
\end{align*}
ισούται με 3 και γι αυτό $\rmp(3) = 3$. 

Για περισσότερες πληροφορίες σχετικά με την βασική θεωρία των τυπικών δυναμοσειρών παραπέμπουμε τον ενδιαφερόμενο αναγνώστη στο \cite[Ενότητα~1.1.3]{AthAEC}.

\newpage

\setcounter{section}{15}
\setcounter{theorem}{0}
\begin{center}
    \textbf{15. Η άλγεβρα των συμμετρικών συναρτήσεων
} 

\end{center}

Έστω $\bfx = (x_1, x_2, \dots)$ μια (αριθμήσιμα άπειρη) ακολουθία μεταβλητών που μετατίθενται. Για ένα μονώνυμο $\bfx^\alpha = x_1^{\alpha_1}x_2^{\alpha_2}\cdots{x_k^{\alpha_k}}$, το $\alpha_1 + \alpha_2 + \cdots + \alpha_k$ ονομάζεται \defn{βαθμός}. Μια τυπική δυναμοσειρά $f \in \CC[\![\bfx]\!]$ λέμε ότι είναι \defn{ομογενής βαθμού} $n$ αν κάθε μονώνυμο στο ανάπτυγμά της έχει βαθμό $n$. Για παράδειγμα, η 
\[
x_1x_2^2 + x_1x_3^2 + \cdots + x_2x_3^2 + x_2x_4^2 + \cdots
\]
είναι ομογενής βαθμού $3$, ενώ η 
\[
x_1x_2^4 + x_1^2 + x_3x_4x_7
\]
δεν είναι ομογενής.

\begin{definition}
    \label{def:symmetric_function}
    Μια ομογενής τυπική δυναμοσειρά $f \in \CC[\![x]\!]$ ονομάζεται \defn{συμμετρική συνά\-ρτηση} αν 
    \[
    f(x_{\pi(1)}, x_{\pi(2)}, \dots) = f(x_1, x_2, \dots)
    \]
    για κάθε μετάθεση $\pi$ των θετικών ακεραίων, η οποία αφήνει όλα τα στοιχεία σταθερά, εκτός από το πολύ πεπερασμένο πλήθος αυτών.
\end{definition}

Με άλλα λόγια, η $f$ είναι συμμετρική συνάρτηση αν και μόνο αν ο συντελεστής του 
\[
x_{i_1}^{\lambda_{i_1}}x_{i_2}^{\lambda_{i_2}}\cdots x_{i_k}^{\lambda_{i_k}}
\]
στην $f$ είναι ο ίδιος για κάθε διακεκριμένους $i_1, i_2, \dots, i_k \in \ZZ_{>0}$ και αυθαίρετους εκθέτες $\lambda_{i_1}, \lambda_{i_2}, \dots, \lambda_{i_k}$ (γιατί;). Για μονώνυμο $\bfx^\alpha$, θα γράφουμε $[\bfx^\alpha]f(\bfx)$ για τον συντελεστή του $\bfx^{\alpha}$ στην $f(\bfx)$.

\begin{example}
    \label{ex:symmetric_functions}
    \leavevmode
    \begin{itemize}
        \item[(1)] Ένα παράδειγμα συμμετρικής συνάρτησης βαθμού 1 είναι η 
        \[
        \sum_{i\ge1} x_i = x_1 + x_2 + \cdots.
        \]
        \item[(2)]
        Παραδείγματα συμμετρικών συναρτήσεων βαθμού 2 αποτελούν οι 
        \begin{align*}
            \sum_{i\ge1}x_i^2 &= x_1^2 + x_2^2 + \cdots \\
            \sum_{1 \le i <j} x_ix_j &= x_1x_2 + x_1x_3 + \cdots + x_2x_3 + x_2x_4 + \cdots.
        \end{align*}
        \item[(3)] Η τυπική δυναμοσειρά
        \[
        f(\bfx) = x_1x_2^2 + x_1x_3^2 + \cdots + x_2x_3^2 + x_2x_4^2 + \cdots
        \]
        που συναντήσαμε στην αρχή της παραγράφου, δεν είναι συμμετρική συνάρτηση, καθώς, για παράδειγμα, αν $\pi = \cycle{1,2}$, τότε 
        \[
        [x_{\pi(1)}x_{\pi(2)}^2]f(\bfx) = [x_2x_1^2]f(\bfx) = 0 \neq 1 = [x_1x_2^2]f(\bfx).
        \] 
        Ποιά ομογενής τυπική δυναμοσειρά βαθμού 2 \textquote{λείπει} από την $f(\bfx)$ ώστε να γίνει συμμετρική συνάρτηση;
    \end{itemize}
\end{example}

Έχοντας την παρατήρηση μετά τον Ορισμό~\ref{def:symmetric_function} κατά νου, ο πιο φυσικός τρόπος να κατα\-σκευάσουμε μια συμμετρική συνάρτηση βαθμού $n$ είναι ο εξής: Θεωρούμε μη αρνητικούς ακεραίους $\lambda_1, \lambda_2, \dots, \lambda_k$, έτσι ώστε $\lambda_1 + \lambda_2 + \cdots + \lambda_k = n$ και συμμετρικοποιούμε το μονώνυμο 
\[
x_1^{\lambda_1}x_2^{\lambda_2}\cdots x_k^{\lambda_k}
\]
με εκθέτες $\lambda_1, \lambda_2, \dots, \lambda_k$, δηλαδή παίρνουμε το άθροισμα όλων των διακεκριμένων μονωνύμων με αυτούς τους εκθέτες. Για την ακολουθία των εκθετών μπορούμε να υποθέσουμε ότι $\lambda_1 \ge \lambda_2 \ge \cdots \ge \lambda_k$ (γιατί;). Η διαδικασία αυτή οδηγεί στον ακόλουθο ορισμό.

\begin{definition}
    \label{def:monomial}
    Έστω $\lambda = (\lambda_1, \lambda_2, \dots, \lambda_k) \vdash n$. Το στοιχείο
    \[
    m_\lambda = m_\lambda(\bfx) \coloneqq \sum x_{i_1}^{\lambda_{i_1}}x_{i_2}^{\lambda_{i_2}}\cdots x_{i_k}^{\lambda_{i_k}},
    \]
    όπου το άθροισμα διατρέχει όλα τα διακεκριμένα μονώνυμα με εκθέτες $\lambda_1, \lambda_2, \dots, \lambda_k$ ονομάζεται \defn{μονωνυμική συμμετρική συνάρτηση} (monomial symmetric functions) που αντιστοιχεί στην $\lambda$.
\end{definition}

Στα Παραδείγματα~\ref{ex:symmetric_functions} (1) και (2), είδαμε τα στοιχεία $m_{(1)}, m_{(2)}$ και $m_{(1,1)}$, αντίστοιχα. Για $n=3$, έχουμε 
\begin{align*}
    m_{(3)}(\bfx) &= \sum_{i\ge1} x_i^3 = x_1^3 + x_2^3 + \cdots\\
    m_{(2,1)}(\bfx) &= \sum_{i \neq j} x_i^2x_j \\
    &= \sum_{i < j}x_i^2x_j + \sum_{i < j}x_ix_j^2 \\
    &= x_1^2x_2 + x_1^2x_3 + \cdots + x_2^2x_3 + x_2^2x_4 + \cdots  \\
    &\quad+ x_1x_2^2 + x_1x_3^2 + \cdots + x_2x_3^2 + x_2x_4^2 + \cdots \\
    m_{(1,1,1)}(\bfx) 
    &= \sum_{i < j < k} x_ix_jx_k \\
    &= x_1x_2x_3 + x_1x_2x_4 + \cdots + x_1x_3x_4 + x_1x_3x_5 + \cdots + x_2x_3x_4 + \cdots \\
\end{align*}
(γιατί;).

\begin{thebibliography}{99}
    \bibitem{AthAEC}
    Χ.~Α.~Αθανασιάδης,
    \emph{Αλγεβρική και Απαριθµητική Συνδυαστική}, 
    Σηµειώσεις, Εθνικό και Καποδιστριακό Πανεπιστήµιο Αθηνών. (διαθέσιμο \href{http://users.uoa.gr/~caath/book.pdf}{εδώ})
\end{thebibliography}
\end{document}