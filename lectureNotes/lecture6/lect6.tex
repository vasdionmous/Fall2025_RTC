\documentclass[12pt,a4paper,reqno]{amsart}

% section handling
\usepackage{subfiles} 

% language
\usepackage[greek,english]{babel}
\usepackage[utf8]{inputenc}
\usepackage{alphabeta}

% change default names to greek
\addto\captionsenglish{
    \renewcommand{\contentsname}{Περιεχόμενα}
    \renewcommand{\refname}{Βιβλιογραφία}
    \renewcommand{\datename}{Ημερομηνία:}
    \renewcommand{\urladdrname}{Ιστοσελίδα}
}

% math 
\usepackage{amsmath,amsthm,amssymb,amscd}

% font
\usepackage[cal=euler]{mathalfa}
\usepackage{libertinus-type1}
% \usepackage{txfonts} % for upright greek letters
\usepackage{bm} % for bold symbols
\usepackage{bbm} % for the simply-looking bb symbols

% miscellaneous 
\usepackage{changepage} %for indenting environments
\usepackage{csquotes} % example: \textcquote{}

% drawing
\usepackage{tikz,tikz-cd}
\usetikzlibrary{shapes.misc, patterns, matrix, calc, intersections,positioning}
\usepackage{graphics,graphicx}
\usepackage{float} % provides enhanced control and customization options for floating objects such as figures and tables

% colors
\usepackage{xcolor}
\definecolor{darkcandyapplered}{rgb}{0.64, 0.0, 0.0}
\definecolor{midnightblue}{rgb}{0.1, 0.1, 0.44}
\definecolor{mylightblue}{HTML}{336699}
\definecolor{burntorange}{rgb}{0.8, 0.33, 0.0}
\definecolor{iceberg}{rgb}{0.44, 0.65, 0.82}

% hrefs
\usepackage{hyperref}
\usepackage[noabbrev,capitalize]{cleveref}
\hypersetup{
    pdftoolbar=true,        
    pdfmenubar=true,        
    pdffitwindow=false,     
    pdfstartview={FitH},  % fits the width of the page to the window
    pdftitle={},
    pdfauthor={},
    pdfsubject={},
    pdfkeywords={},
    pdfnewwindow=true,  % links in new window
    colorlinks=true,  % false: boxed links; true: colored links
    linkcolor=darkcandyapplered,   % color of internal links
    citecolor=midnightblue,  % color of links to bibliography
    urlcolor=cyan,  % color of external links
    linktocpage=true  % changes the links from the section body to the page number
    }

% geometry
\textwidth=16cm 
\textheight=21cm 
\hoffset=-55pt 
\footskip=25pt

% thm envs (you might need to change the path)
% In this macro I define all the theorem environments

\theoremstyle{definition}
\newtheorem{theorem}{Θεώρημα}
\newtheorem{proposition}[theorem]{Πρόταση}
\newtheorem{lemma}[theorem]{Λήμμα}
\newtheorem{corollary}[theorem]{Πόρισμα}
\newtheorem{conjecture}[theorem]{Εικασία}
\newtheorem{problem}[theorem]{Πρόβλημα}
\newtheorem*{claim}{Ισχυρισμός}
\newtheorem{observation}[theorem]{Παρατήρηση}
\newtheorem{definition}[theorem]{Ορισμός}
\newtheorem{question}[theorem]{Ερώτηση}
\newtheorem{example}[theorem]{Παράδειγμα}
\newtheorem{exercise}{Άσκηση}

\theoremstyle{remark}
\newtheorem*{remark}{Παρατήρηση}

% fixes the correct numbering of environments
\numberwithin{theorem}{section}
\numberwithin{exercise}{section}
\numberwithin{equation}{section}

% math ops (you might need to change the path)
% In this macro I define all of my math operators

% fields
\newcommand{\NN}{\mathbbmss{N}} 
\newcommand{\ZZ}{\mathbbmss{Z}} 
\newcommand{\QQ}{\mathbbmss{Q}} 
\newcommand{\RR}{\mathbbmss{R}} 
\newcommand{\CC}{\mathbbmss{C}} 
\newcommand{\KK}{\mathbbmss{K}} 
\newcommand{\FF}{\mathbbmss{F}} 

% symmetric group
\newcommand{\fS}{\mathfrak{S}}  

% calligraphic 
\newcommand{\aA}{\mathcal{A}} 
\newcommand{\bB}{\mathcal{B}}
\newcommand{\cC}{\mathcal{C}}
\newcommand{\dD}{\mathcal{D}}
\newcommand{\eE}{\mathcal{E}}
\newcommand{\fF}{\mathcal{F}}
\newcommand{\hH}{\mathcal{H}}
\newcommand{\iI}{\mathcal{I}}
\newcommand{\lL}{\mathcal{L}}
\newcommand{\oO}{\mathcal{O}}
\newcommand{\pP}{\mathcal{P}}
\newcommand{\sS}{\mathcal{S}}
\newcommand{\mM}{\mathcal{M}}
\newcommand{\uU}{\mathcal{U}}

% bold
\newcommand{\bfa}{\mathbf{a}}
\newcommand{\bfe}{\mathbf{e}}
\newcommand{\bfF}{\pmb{F}}
\newcommand{\bfR}{\pmb{R}}
\newcommand{\bfv}{\mathbf{v}}
%\newcommand{\bfx}{\bm{x}}
%\newcommand{\bfx}{\mathbf{x}} 
\newcommand{\bfx}{\pmb{x}}
\newcommand{\bfX}{\pmb{X}}
\newcommand{\bfy}{\pmb{y}}
\newcommand{\bfz}{\pmb{z}}

% roman
\newcommand{\rmB}{\mathrm{B}}
\newcommand{\rmC}{\mathrm{C}}
\newcommand{\rmD}{\mathrm{D}} 
\newcommand{\rmI}{\mathrm{I}} 
\newcommand{\rmK}{\mathrm{K}}
\newcommand{\rmM}{\mathrm{M}}
\newcommand{\rmP}{\mathrm{P}}  
\newcommand{\rmQ}{\mathrm{Q}}  
\newcommand{\rmR}{\mathrm{R}}
\newcommand{\rmS}{\mathrm{S}}
\newcommand{\rmT}{\mathrm{T}}
\newcommand{\rmU}{\mathrm{U}}
\newcommand{\rmV}{\mathrm{V}}
\newcommand{\rmY}{\mathrm{Y}}
\newcommand{\rmZ}{\mathrm{Z}}

% greek letters
% I'm renewing some commands in order to appear in upright font
% If I want to change it later, I don't have to do it manually, I just change it from here.
% \newcommand{\uaa}{\alphaup}
% \renewcommand{\alpha}{\alphaup}
% \renewcommand{\beta}{\betaup}
% \renewcommand{\gamma}{\gammaup}
% \renewcommand{\delta}{\deltaup}
% \renewcommand{\epsilon}{\epsilonup}
% \newcommand{\ee}{\epsilon}
% \renewcommand{\varepsilon}{\varepsilonup}
% \renewcommand{\theta}{\thetaup}
% \renewcommand{\lambda}{\lambdaup}
% \newcommand{\ull}{\lambda}
% \renewcommand{\mu}{\muup}
% \renewcommand{\nu}{\nuup}
% \renewcommand{\pi}{\piup}
% \renewcommand{\rho}{\rhoup}
% \renewcommand{\varrho}{\varrhoup}
% \renewcommand{\sigma}{\sigmaup}
% \renewcommand{\tau}{\tauup} 
% \renewcommand{\phi}{\phiup}
% \renewcommand{\chi}{\chiup}
% \renewcommand{\psi}{\psiup}
% \renewcommand{\omega}{\omegaup}

% arrows and symbols 
\renewcommand{\to}{\rightarrow}
\newcommand{\toto}{\longrightarrow}
\newcommand{\mapstoto}{\longmapsto}
\newcommand{\then}{\Rightarrow}
\newcommand{\IFF}{\Leftrightarrow}
\newcommand{\tl}{\tilde}
\newcommand{\wtl}{\widetilde}
\newcommand{\ol}{\overline}
\newcommand{\ul}{\underline}
\newcommand{\oldemptyset}{\emptyset}
\renewcommand{\emptyset}{\varnothing}
\DeclareMathSymbol{\Arg}{\mathbin}{AMSa}{"39} % for arguments 
\newcommand{\onto}{\ensuremath{\twoheadrightarrow}}

% absolute value symbol
\usepackage{mathtools} 
\DeclarePairedDelimiter\abs{\lvert}{\rvert}%
\DeclarePairedDelimiter\norm{\lVert}{\rVert}%
\makeatletter
\let\oldabs\abs
\def\abs{\@ifstar{\oldabs}{\oldabs*}}

% tensor symbol
\newcommand{\tensor}[1]{%
  \mathbin{\mathop{\otimes}\limits_{#1}}%
}

% permutation cycle notation
\ExplSyntaxOn
\NewDocumentCommand{\cycle}{ O{\;} m }
 {
  (
  \alec_cycle:nn { #1 } { #2 }
  )
 }

\seq_new:N \l_alec_cycle_seq
\cs_new_protected:Npn \alec_cycle:nn #1 #2
 {
  \seq_set_split:Nnn \l_alec_cycle_seq { , } { #2 }
  \seq_use:Nn \l_alec_cycle_seq { #1 }
 }
\ExplSyntaxOff

% setminus symbol
\newcommand{\mysetminusD}{\hbox{\tikz{\draw[line width=0.6pt,line cap=round] (3pt,0) -- (0,6pt);}}}
\newcommand{\mysetminusT}{\mysetminusD}
\newcommand{\mysetminusS}{\hbox{\tikz{\draw[line width=0.45pt,line cap=round] (2pt,0) -- (0,4pt);}}}
\newcommand{\mysetminusSS}{\hbox{\tikz{\draw[line width=0.4pt,line cap=round] (1.5pt,0) -- (0,3pt);}}}
\newcommand{\sm}{\mathbin{\mathchoice{\mysetminusD}{\mysetminusT}{\mysetminusS}{\mysetminusSS}}}

% custom math operators
\newcommand{\Des}{\mathrm{Des}} 
\newcommand{\des}{\mathrm{des}} 
\newcommand{\Asc}{\mathrm{Asc}}
\newcommand{\asc}{\mathrm{asc}} 
\newcommand{\inv}{\mathrm{inv}}
\newcommand{\Inv}{\mathrm{Inv}}
\newcommand{\maj}{\mathrm{maj}} 
\newcommand{\comaj}{\mathrm{comaj}} 
\newcommand{\fix}{\mathrm{fix}} 
\newcommand{\Sym}{\mathrm{Sym}} 
\newcommand{\QSym}{\mathrm{QSym}}
\newcommand{\FQSym}{\mathrm{FQSym}} 
\newcommand{\End}{\mathrm{End}} 
\newcommand{\Rad}{\mathrm{Rad}} 
\newcommand{\rmMat}{\mathrm{Mat}} 
\newcommand{\rmdim}{\mathrm{dim}} 
\newcommand{\rmTop}{\mathrm{Top}} 
\newcommand{\rmCF}{\mathrm{CF}} 
\newcommand{\rmId}{\mathrm{Id}}
\newcommand{\rmid}{\mathrm{id}}
\newcommand{\rmtw}{\mathrm{tw}}
\newcommand{\trace}{\mathrm{tr}}
\newcommand{\Irr}{\mathrm{Irr}}
\newcommand{\Ind}{\mathrm{Ind}} % induction
\newcommand{\Res}{\mathrm{Res}} % restriction
\newcommand{\triv}{\mathrm{triv}} % trivial rep
\newcommand{\rmdef}{\mathrm{def}} % defining rep
\newcommand{\dom}{\triangleleft}
\newcommand{\domeq}{\trianglelefteq}
\newcommand{\lex}{\mathrm{lex}}
\newcommand{\sign}{\mathrm{sign}}
\newcommand{\SYT}{\mathrm{SYT}}
\renewcommand{\Im}{\mathrm{Im}}
\newcommand{\Ker}{\mathrm{Ker}}
\newcommand{\GL}{\mathrm{GL}}
\newcommand{\FL}{\mathrm{FL}}
\newcommand{\Span}{\mathrm{span}}
\newcommand{\pos}{\mathrm{pos}}
\newcommand{\Comp}{\mathrm{Comp}}
\newcommand{\Set}{\mathrm{Set}}
\newcommand{\std}{\mathrm{std}}
\newcommand{\cont}{\mathrm{cont}} %content of a SSYT
\newcommand{\SSYT}{\mathrm{SSYT}}
\newcommand{\rmz}{\mathrm{z}}
\newcommand{\ct}{\mathrm{ct}} % cycle type
\newcommand{\ch}{\mathrm{ch}} % Frobenius characteristic map
\newcommand{\height}{\mathrm{ht}}
\newcommand{\FPS}{\CC[\![\bfx]\!]} % formal power series
\newcommand{\FPSS}{\CC[\![\bfx,\bfy]\!]}
\newcommand{\reg}{\mathrm{reg}}
\newcommand{\hook}{\mathrm{h}}
\newcommand{\weight}{\mathrm{wt}}
\newcommand{\co}{\mathrm{co}}
\newcommand{\ps}{\mathrm{ps}}
\newcommand{\rmsum}{\mathrm{sum}}
\newcommand{\NSym}{\mathrm{NSym}}
\newcommand{\Hom}{\mathrm{Hom}}
\newcommand{\proj}{\mathrm{proj}}
\newcommand{\stat}{\mathrm{stat}}

% miscellaneous commands
\newcommand{\defn}[1]{{\color{mylightblue}{#1}}}
\newcommand{\toDo}{{\bf\color{red} TODO}}
\newcommand{\toCite}{{\bf\color{green} CITE}}

% 
\newenvironment{nouppercase}{%
  \let\uppercase\relax%
  \renewcommand{\uppercasenonmath}[1]{}}{}

% titlepage
\title{Θ2.04: Θεωρία Αναπαραστάσεων και Συνδυαστική}
\author[Β.~Δ. Μουστακας]{Βασίλης Διονύσης Μουστάκας \\ Πανεπιστήμιο Κρήτης}
\date{14 Οκτωβρίου 2025}
% \urladdr{\href{https://sites.google.com/view/vasmous}{https://sites.google.com/view/vasmous}}

\begin{document}

\begingroup
\def\uppercasenonmath#1{} % this disables uppercase title
\let\MakeUppercase\relax % this disables uppercase authors
\maketitle
\endgroup

\setcounter{section}{5}
\setcounter{theorem}{2}
\begin{center}
    \textbf{5. Χαρακτήρες ομάδων: Εισαγωγή (συνέχεια)
}
\end{center}

\begin{definition}
    \label{def:class_function}
    Μια απεικόνιση $\alpha : G \to \CC$ η οποία έχει σταθερή τιμή στις κλάσεις συζυγίας της $G$, δηλαδή 
    \[
    \alpha(g) = \alpha(xgx^{-1})
    \]
    για κάθε $x \in G$ και $g \in G$ ονομάζεται \defn{συνάρτηση κλάσης} (class function).
\end{definition}

Η Πρόταση~5.2~(2) μας πληροφορεί ότι κάθε χαρακτήρας της $G$ είναι συνάρτηση κλάσης. Το σύνολο $\rmCF_\CC(G)$ (ή πιο απλά $\rmCF(G)$) όλων των συναρτήσεων κλάσης της $G$ αποτελεί διανυσματικό χώρο με πράξη την πρόσθεση συναρτήσεων 
\[
(\alpha+\beta)(g) \coloneqq \alpha(g) + \beta(g)
\]
για κάθε $\alpha, \beta \in \rmCF(G)$ και $g \in G$ (γιατί;). Μια προφανής βάση του χώρου αυτού αποτελείται από τις χαρακτηριστικές συναρτήσεις στις κλάσεις συζυγίας $\rmK$ της $G$, δηλαδή 
\[
1_K(g) \coloneqq 
\begin{cases}
    1, &\ \text{αν $g \in K$} \\
    0, &\ \text{διαφορετικά}.
\end{cases}
\]
Συνεπώς, 
\begin{equation}
    \label{eq:CF_dimension}
    \dim\left(\rmCF(G)\right) = \ \text{πλήθος των κλάσεων συζυγίας της $G$}.
\end{equation}

\begin{definition}
    \label{def:character_table}
    Ο πίνακας 
    \[
    \begin{array}{c | c}
                              & \text{κλάσεις συζυγίας της $G$} \\ 
    \hline 
    \text{ανάγωγοι}           &        \\
    \text{χαρακτήρες}  &  \vdots    \\
    \text{μη ισόμορφων}       & \cdots \ \chi(K) \ \cdots\\  
    \text{αναπαραστάσεων}     & \vdots \\
    \text{της $G$}
    \end{array}
    \]
    ονομάζεται \defn{πίνακας χαρακτήρων} (character table) της $G$. Στην πρώτη στήλη θα αναγράφεται πάντα η κλάση συζυγίας του ταυτοτικού στοιχείου και στην πρώτη γραμμή ο χαρακτήρας της τετριμμένης αναπαράστασης.
\end{definition}

Ας (προσπαθήσουμε) να υπολογίσουμε τον πίνακα χαρακτήρων της συμμετρικής ομάδας $\fS_3$. Οι κλάσεις συζυγίας της $\fS_3$ είναι 
\[
Κ_1 = \{\cycle{1}\cycle{2}\cycle{3}\}, \ 
K_2 = \{\cycle{1,2}\cycle{3}, \cycle{1,3}\cycle{2}, \cycle{2,3}\cycle{1}\}, \ \text{και} \ \
K_3= \{\cycle{1,2,3}, \cycle{1,3,2}\}.
\]
Στην Παράγραφο~4 είδαμε ότι η $\fS_3$ έχει τρεις διαφορετικές ανάγωγες αναπαραστάσεις: 
\begin{itemize}
    \item την τετριμμένη αναπαράσταση,
    \item την αναπαράσταση προσήμου,
\end{itemize}  
των οποίων τους χαρακτήρες συμβολίζουμε με $\chi^\rmdef$ και $\chi^\sign$, αντίστοιχα και 
\begin{itemize}
    \item μια ανάγωγη αναπαράσταση διάστασης 2.
\end{itemize} 
Συνεπώς, ο πίνακας χαρακτήρων της $\fS_3$ έχει την εξής μορφή
\renewcommand{\arraystretch}{1.2} % increases row height by 20%
\[
\begin{array}{l|c|c|c}
                  & K_1 & K_2 & K_3 \\ \hline
    \chi^\triv    & 1   & 1   & 1 \\ \hline
    \chi^\sign    & 1   & -1  & 1 \\ \hline
    \chi^\text{;} & 2   & \text{;}   & \text{;} 
\end{array}.
\]
Ποιός είναι ο ανάγωγος χαρακτήρας που λείπει;

Ας κάνουμε το ίδιο για την κυκλική ομάδας $\rmC_3 = \{\epsilon, g, g^2\}$ τάξης 3. Αφού η $\rmC_3$ είναι αβελιανή ομάδα, κάθε στοιχείο της σχηματίζει τη δική του κλάση συζυγίας. Στην Άσκηση~1.6 είδαμε ότι η $\rmC_3$ έχει τρεις ανάγωγες αναπαραστάσεις με χαρακτήρες\footnote{Στην περίπτωση όπου η αναπαράσταση έχει διάσταση 1, τότε η έννοια του χαρακτήρα και της αναπαράστασης ταυτίζονται.}
\[
\chi_1(g) = 1, \quad \chi_2(g) = \zeta, \quad \chi_3(g) = \zeta^2,
\]
όπου $\zeta$ είναι μια τρίτη ρίζα της μονάδας. Συνεπώς, ο πίνακας χαρακτήρων της $\rmC_3$ είναι 
\[
\begin{array}{l|c|c|c}
           & \{\epsilon\} & \{g\}   & \{g^2\} \\ \hline
    \chi_1 & 1            & 1       & 1 \\ \hline
    \chi_2 & 1            & \zeta   & \zeta^2 \\ \hline
    \chi_3 & 1            & \zeta^2 & \zeta
\end{array}.
\]
Ομοίως, o πίνακας χαρακτήρων της κυκλικής ομάδας τάξης 4 είναι 
\[
\begin{array}{l|c|c|c|c}
           & \{\epsilon\} & \{g\}   & \{g^2\} & \{g^3\} \\ \hline
    \chi_1 & 1            & 1       & 1       & 1       \\ \hline
    \chi_2 & 1            & -1      & 1       & -1      \\ \hline
    \chi_3 & 1            & i       & -1      & -i      \\ \hline 
    \chi_4 & 1            & -i      & -1      & i      
\end{array}
\]
(γιατί;).

\newpage

\setcounter{section}{6}
\setcounter{theorem}{0}
\begin{center}
    \textbf{6. Χαρακτήρες ομάδων: Κατασκευές προτύπων
}
\end{center}

Το ευθύ άρθοισμα $V\oplus W$ δυο διανυσματικών χώρων $V$ και $W$ γίνεται $G$-πρότυπο\footnote{Όπως είδαμε και στην περίπτωση του ευθέως αθροίσματος δυο υπόχωρων.} με τη \emph{διαγώνια δράση}:
\[
g\cdot(v,w) \coloneqq (gv,gw).
\]
Το ευθύ άθροισμα είναι μια κατασκευή μεταξύ διανυσματικών χώρων όπου τα στοιχεία του δεν \textquote{ανακατεύονται}. Για παράδειγμα, 
\[
\underbrace{\begin{pmatrix}
    a \\ 
    b
\end{pmatrix}}_{\in \, \RR^2}
\oplus
\underbrace{\begin{pmatrix}
    c \\ 
    d \\ 
    e
\end{pmatrix}}_{\in \, \RR^3}
=
\begin{pmatrix}
    a \\ 
    b \\ 
    c \\ 
    d \\ 
    e
\end{pmatrix} \ \in \RR^5
\]
Όπως θα δούμε παρακάτω, ο χαρακτήρας του ευθέως αθροίσματος ισούται με το \emph{άθροισμα} των χαρακτήρων των αντίστοιχων προτύπων.

Πώς θα μπορούσαμε να κατασκευάσουμε ένα πρότυπο του οποίου τα στοιχεία θα \textquote{ανακατεύονται} (ή \emph{πολλαπλασιάζονται}) και σε επίπεδο χαρακτήρων αυτό να μεταφράζεται σε \emph{πολλαπλασιασμό}; Για παράδειγμα, θα θέλαμε 
\[
\underbrace{\begin{pmatrix}
    a \\ 
    b
\end{pmatrix}}_{\in \, \RR^2}
\ \text{\textquote{$\times$}} \
\underbrace{\begin{pmatrix}
    c \\ 
    d \\ 
    e
\end{pmatrix}}_{\in \, \RR^3}
\ \text{\textquote{$=$}} \
\begin{pmatrix}
    ac \\ 
    ad \\ 
    ae \\ 
    bc \\ 
    bd \\
    be
\end{pmatrix} \ \in \RR^6.
\]

Όπως και στο ευθύ άθροισμα, ξεκινάμε με το $V\times{W}$. Στο $V \oplus W$ όμως, η πρόσθεση ορίζεται κατά συντεταγμένες. Για παράδειγμα, στο $\RR^2 \oplus \RR^2$ έχουμε 
\[
((1,0) + (0,1), (1,2)) = ((1,1), (1,2)).
\]
Στον δικό μας χώρο όμως, θα θέλαμε ο \textquote{πολλαπλασιασμός} να είναι επιμεριστικός ως προς την πρόσθεση:
\[
(v + v') \ \text{\textquote{$\times$}} \ w = 
v \ \text{\textquote{$\times$}} \ w + v' \ \text{\textquote{$\times$}} \ w,
\]
το οποίο δε συμβαίνει στο $V\oplus W$. Στο παράδειγμα, 
\[
((1,0), (1,2)) + ((0,1),(1,2)) = ((1,1), (2,4)) \neq ((1,1), (1,2)).
\] 
Για να \textquote{λύσουμε} αυτό το πρόβλημα, κατασκευάζουμε έναν χώρο, όπου \emph{εξαναγκάζουμε} τα στοιχεία του να ικανοποιούν την επιμεριστική ιδιότητα.

\begin{definition}
    \label{def:tensor_product}
    Έστω $V$ και $W$ δυο διανυσματικοί χώρου. Ο χώρος πηλίκο
    \[
    V \otimes W \coloneqq \FF[V \times W] / U,
    \]
    όπου $U$ είναι ο υπόχωρος που παράγεται από όλα τα στοιχεία της μορφής 
    \begin{itemize}
        \item $(v + v', w) - (v,w) - (v',w)$
        \item $(v, w+w') - (v,w) - (v,w')$
        \item $(cv, w) - c(v,w)$
        \item $(v, cw) - c(v,w)$,
    \end{itemize}
    για κάθε $v, v' \in V$, $w, w' \in W$ και $c \in \FF$ ονομάζεται \defn{τανυστικό γινόμενο} (tensor product) των $V$ και $W$.
\end{definition}

\begin{remark}
    Ο ορισμός του τανυστικού γινομένου εξαρτάται σε μεγάλο βαθμό πάνω από πιο σώμα το ορίζουμε. Γι αυτό συνήθως χρησιμοποιείται ο συμβολισμός $V\otimes_{\FF}W$. Για παράδειγμα, ως διανυσματικοί χώροι $\CC\otimes_\CC\CC \ncong \CC\otimes_\RR\CC$, καθώς ο πρώτος έχει διάσταση 1 (βλ. Παράδειγμα~6.4~(1)), ενώ ο δεύτερος έχει διάσταση 2 (βλ. Θεώρημα~\ref{thm:tensor_product_basis}), αλλά τέτοιες περιπτώσεις δε θα μας απασχολήσουν σε αυτό το μάθημα.
\end{remark}

Το τανυστικό γινόμενο αποτελείται από όλους γραμμικούς συνδυασμούς 
\[
\sum_{(v,w) \in V\times W} c_{v,w}(v,w)
\]
όπου όλα τα $c_{v,w}=0$ εκτός από ένα πεπερασμένο πλήθος που είναι μη μηδενικά, modulo τις σχέσεις του $U$. Την κλάση ενός στοιχείου $(v,w)$ τη συμβολίζουμε με $v \otimes w$ και ονομάζεται \defn{στοιχειώδης τανυστής} (simple tensor).

Πρέπει να είναι κανείς προσεκτικός όταν δουλεύει με στοιχείωδεις τανυστές καθώς μπορεί να έχουν πολλές \textquote{διαφορετικές} μορφές. Για παράδειγμα, 
\[
(2\bfe_1) \otimes \bfe_2 = 2(\bfe_1 \otimes \bfe_2) = \bfe_1 \otimes (2\bfe_2) = 4(\bfe_1 \otimes (\frac{1}{2}\bfe_2)) = \cdots. 
\]

Το τανυστικό γινόμενο μας επιτρέπει να \textquote{πολλαπλασιάσουμε} στοιχεία διανυσματικών χώρων όπου εκ των προτέρων δεν υπάρχει κάποιος προφανής \textquote{πολλαπλασιασμός}. Για παράδειγμα\footnote{Με $\FF[t]$ και $\rmMat_n(\FF)$ συμβολίζουμε τον χώρο των πολυωνύμων στο $t$ με συντελεστές στο $\FF$ και τον χώρο των $(n\times n)$-πινάκων με στοιχεία από το $\FF$, αντίστοιχα.}, στο $\RR[t] \otimes \rmMat_2(\RR)$,
\begin{align*}
(1 + 2t) \otimes \begin{pmatrix}
    0 & 1 \\
    2 & 0 
\end{pmatrix} &= 
1 \otimes \begin{pmatrix}
    0 & 1 \\
    2 & 0 
\end{pmatrix} + 2t \otimes \begin{pmatrix}
    0 & 1 \\
    2 & 0 
\end{pmatrix} \\ 
&= 
1 \otimes \begin{pmatrix}
    0 & 1 \\
    2 & 0 
\end{pmatrix} + t \otimes \begin{pmatrix}
    0 & 2 \\
    4 & 0 
\end{pmatrix} \\
&= \cdots 
\end{align*}
Σε κάποιες περιπτώσεις όμως που υπάρχει ένας προφανής \textquote{πολλαπλασιασμός}, παίρνουμε διαφορετικά στοιχεία από αυτά που περιμέναμε. Για παράδειγμα, στο $\RR[t] \otimes \RR[t]$, 
\[
1 \otimes t^2 \neq t \otimes t \neq t^2 \otimes 1.
\]

\begin{definition}
    \label{def:bilinear}
    Έστω $V, W$ και $U$ διανυσματικοί χώροι. Μια απεικόνιση $f: V\times W \to U$ ονομάζεται \defn{διγραμμική} αν 
    \begin{align*}
        f(c_1v_1 + c_2v_2, w) &= c_1f(v_1,w) + c_2f(v_2,w) \\
        f(v, cw_1+dw_2) &= c_1f(v,w_1) + c_2f(v,w_2) \\
    \end{align*}
    για κάθε $v_1, v_2 \in V$, $w_1, w_2 \in W$ και $c_1, c_2 \in \FF$.
\end{definition}

Οι σχέσεις που \textquote{επιβάλλαμε} στο $\FF[V\times{W}]$ στον Ορισμό~\ref{def:tensor_product} έχουν ως συνέπεια η απεικόνιση
\begin{align*}
\proj : V \times W &\to V \otimes W \\
(v,w) &\mapsto v \otimes w
\end{align*}
να είναι διγραμμική (γιατί;). Το επόμενο αποτέλεσμα εκφράζει την \emph{καθολική ιδιότητα} του τανυστικού γινομένου ως προς τις διγραμμικές απεικονίσης.
\begin{theorem}
    \label{thm:tensor_product_universal_property}
    Έστω $V, W$ και $U$ διανυσματικοί χώροι. Για κάθε διγραμμική απεικόνιση $f: V \times W \to U$, υπάρχει μοναδική γραμμική απεικόνιση $\ol{f}: V \otimes W \to U$, τέτοια ώστε
    \[
    f = \ol{f} \circ \proj
    \]
    ή ισοδύναμα το διάγραμμα
    \[
    \begin{tikzcd}
    V\times W \arrow[r, "f"] \arrow[d, "\proj"'] & U \\
    V \otimes W   \arrow[ur, "\ol{f}"']          & 
    \end{tikzcd}
    \]
    είναι μεταθετικό. Ειδικότερα, υπάρχει μια αμφιμονοσήμαντη απεικόνιση 
    \[
    \left\{
        \begin{array}{c}
        \text{διγραμμικές απεικονίσεις} \\
        V \times W \to U
    \end{array}
    \right\} 
    \to
    \left\{
    \begin{array}{c}
        \text{γραμμικές απεικονίσεις} \\
        V \otimes W \to U
    \end{array}
    \right\}.    
    \]
\end{theorem}

Το Θεώρημα~\ref{thm:tensor_product_universal_property}, η απόδειξη του οποίου παραλλείπεται, μας επιτρέπει να \textquote{αναγνωρίσουμε} τανυστικά γινόμενα ως εξής: αν θέλουμε να ορίσουμε μια γραμμική απεικόνιση $V\otimes{W} \to U$, αρκεί να βρούμε μια διγραμμική απεικόνιση $V\times{W} \to U$.

\begin{example}
    \leavevmode
    \begin{itemize}
        \item[(1)] Για κάθε διανυσματικό χώρο $V$, ισχύει ότι $\FF \otimes V \cong V$. Πράγματι, η απεικόνιση 
        \begin{align*}
            \FF\times{V} &\to V \\
            (c,v) &\mapsto cv
        \end{align*}
        είναι διγραμμική (γιατί;) και από το Θεώρημα~\ref{thm:tensor_product_universal_property} επάγεται μια γραμμική απεικόνιση 
        \begin{align*}
            \FF\otimes{V} &\to V \\
            c\otimes{v} &\mapsto cv.
        \end{align*}
        Η αντίστροφή της δίνεται από $v \mapsto 1 \otimes v$ (γιατί;).
        \item[(2)] Έστω $\KK$ ένα σώμα το οποίο περιέχει το $\FF$. Το τανυστικό γινόμενο $\KK \otimes V$ (πάνω από το $\FF$) γίνεται διανυσματικός χώρος πάνω από το $\KK$ θέτοντας 
        \[
        c \cdot \left(\sum_{i} a_i \otimes v\right) \coloneqq \sum_{i} (ca_i) \otimes v,
        \]
        για κάθε $c \in \KK$. Η κατασκευή αυτή ονομάζεται \emph{επέκταση βαθμωτών} από το $\FF$ στο $\KK$. 

        Για παράδειγμα, αν $\KK = \CC$ και $\FF = \RR$, τότε το τανυστικό γινόμενο $\CC \otimes \RR$ ως διανυσματικός χώρος πάνω από το $\CC$ είναι ισόμορφος με το ίδιο το $\CC$, μέσω της ταύτισης
        \[
        a + ib \ \mapsto \ 1 \otimes a + i \otimes b 
        \]
        (γιατί;).
    \end{itemize}
\end{example}
\end{document}