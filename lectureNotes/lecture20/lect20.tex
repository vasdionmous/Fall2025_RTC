\documentclass[12pt,a4paper,reqno]{amsart}

% language
\usepackage[greek,english]{babel}
\usepackage[utf8]{inputenc}
\usepackage{alphabeta}

% change default names to greek
\addto\captionsenglish{
    \renewcommand{\contentsname}{Περιεχόμενα}
    \renewcommand{\refname}{Βιβλιογραφία}
    \renewcommand{\datename}{Ημερομηνία:}
    \renewcommand{\urladdrname}{Ιστοσελίδα}
}

% math 
\usepackage{amsmath,amsthm,amssymb,amscd}

% font
\usepackage[cal=euler]{mathalfa}
\usepackage{libertinus-type1}
% \usepackage{txfonts} % for upright greek letters
\usepackage{bm} % for bold symbols
\usepackage{bbm} % for the simply-looking bb symbols

% miscellaneous 
\usepackage{changepage} %for indenting environments
\usepackage{csquotes} % example: \textcquote{}
\usepackage{blkarray}
\setcounter{MaxMatrixCols}{20} % default for pmatrix is 10!!
\usepackage{ytableau}
\usepackage{array} %needed to increase the vertical length in a tabular

% drawing
\usepackage{tikz,tikz-cd}
\usetikzlibrary{shapes.misc, patterns, matrix, calc, intersections,positioning}
\usepackage{graphics,graphicx}
\usepackage{float} % provides enhanced control and customization options for floating objects such as figures and tables

% colors
\usepackage{xcolor}
\definecolor{darkcandyapplered}{rgb}{0.64, 0.0, 0.0}
\definecolor{midnightblue}{rgb}{0.1, 0.1, 0.44}
\definecolor{mylightblue}{HTML}{336699}
\definecolor{burntorange}{rgb}{0.8, 0.33, 0.0}
\definecolor{iceberg}{rgb}{0.44, 0.65, 0.82}
\definecolor{applegreen}{rgb}{0.55, 0.71, 0.0}
\definecolor{canaryyellow}{rgb}{1.0, 0.94, 0.0}

% hrefs
\usepackage{hyperref}
\usepackage[noabbrev,capitalize]{cleveref}
\hypersetup{
    pdftoolbar=true,        
    pdfmenubar=true,        
    pdffitwindow=false,     
    pdfstartview={FitH},  % fits the width of the page to the window
    pdftitle={},
    pdfauthor={},
    pdfsubject={},
    pdfkeywords={},
    pdfnewwindow=true,  % links in new window
    colorlinks=true,  % false: boxed links; true: colored links
    linkcolor=darkcandyapplered,   % color of internal links
    citecolor=midnightblue,  % color of links to bibliography
    urlcolor=cyan,  % color of external links
    linktocpage=true  % changes the links from the section body to the page number
    }

% geometry
\textwidth=16cm 
\textheight=21cm 
\hoffset=-55pt 
\footskip=25pt

% thm envs (you might need to change the path)
% In this macro I define all the theorem environments

\theoremstyle{definition}
\newtheorem{theorem}{Θεώρημα}
\newtheorem{proposition}[theorem]{Πρόταση}
\newtheorem{lemma}[theorem]{Λήμμα}
\newtheorem{corollary}[theorem]{Πόρισμα}
\newtheorem{conjecture}[theorem]{Εικασία}
\newtheorem{problem}[theorem]{Πρόβλημα}
\newtheorem*{claim}{Ισχυρισμός}
\newtheorem{observation}[theorem]{Παρατήρηση}
\newtheorem{definition}[theorem]{Ορισμός}
\newtheorem{question}[theorem]{Ερώτηση}
\newtheorem*{questions}{Ερωτήματα}
\newtheorem{example}[theorem]{Παράδειγμα}
\newtheorem{exercise}{Άσκηση}

\newtheorem*{combInterlude}{Ιντερλούδιο Συνδυαστικής}
\newtheorem*{example_cont}{Παράδειγμα~6.6}
\newtheorem*{digression_la}{Παρέκβαση Γραμμικής Άλγεβρας}
\newtheorem*{thm}{Θεώρημα}

\theoremstyle{remark}
\newtheorem*{remark}{Παρατήρηση}

% fixes the correct numbering of environments
\numberwithin{theorem}{section}
\numberwithin{exercise}{section}
\numberwithin{equation}{section}

% math ops (you might need to change the path)
% In this macro I define all of my math operators

% fields
\newcommand{\NN}{\mathbbmss{N}} 
\newcommand{\ZZ}{\mathbbmss{Z}} 
\newcommand{\QQ}{\mathbbmss{Q}} 
\newcommand{\RR}{\mathbbmss{R}} 
\newcommand{\CC}{\mathbbmss{C}} 
\newcommand{\KK}{\mathbbmss{K}} 
\newcommand{\FF}{\mathbbmss{F}} 

% symmetric group
\newcommand{\fS}{\mathfrak{S}}  

% calligraphic 
\newcommand{\aA}{\mathcal{A}} 
\newcommand{\bB}{\mathcal{B}}
\newcommand{\cC}{\mathcal{C}}
\newcommand{\dD}{\mathcal{D}}
\newcommand{\eE}{\mathcal{E}}
\newcommand{\fF}{\mathcal{F}}
\newcommand{\hH}{\mathcal{H}}
\newcommand{\iI}{\mathcal{I}}
\newcommand{\lL}{\mathcal{L}}
\newcommand{\oO}{\mathcal{O}}
\newcommand{\pP}{\mathcal{P}}
\newcommand{\sS}{\mathcal{S}}
\newcommand{\mM}{\mathcal{M}}
\newcommand{\uU}{\mathcal{U}}

% bold
\newcommand{\bfa}{\mathbf{a}}
\newcommand{\bfe}{\mathbf{e}}
\newcommand{\bfF}{\pmb{F}}
\newcommand{\bfR}{\pmb{R}}
\newcommand{\bfv}{\mathbf{v}}
%\newcommand{\bfx}{\bm{x}}
%\newcommand{\bfx}{\mathbf{x}} 
\newcommand{\bfx}{\pmb{x}}
\newcommand{\bfX}{\pmb{X}}
\newcommand{\bfy}{\pmb{y}}
\newcommand{\bfz}{\pmb{z}}

% roman
\newcommand{\rmA}{\mathrm{A}}
\newcommand{\rmB}{\mathrm{B}}
\newcommand{\rmC}{\mathrm{C}}
\newcommand{\rmD}{\mathrm{D}} 
\newcommand{\rmI}{\mathrm{I}} 
\newcommand{\rmK}{\mathrm{K}}
\newcommand{\rmM}{\mathrm{M}}
\newcommand{\rmP}{\mathrm{P}}  
\newcommand{\rmp}{\mathrm{p}}  
\newcommand{\rmQ}{\mathrm{Q}}  
\newcommand{\rmR}{\mathrm{R}}
\newcommand{\rmS}{\mathrm{S}}
\newcommand{\rmT}{\mathrm{T}}
\newcommand{\rmU}{\mathrm{U}}
\newcommand{\rmV}{\mathrm{V}}
\newcommand{\rmY}{\mathrm{Y}}
\newcommand{\rmZ}{\mathrm{Z}}
\newcommand{\rmz}{\mathrm{z}}

% greek letters
% I'm renewing some commands in order to appear in upright font
% If I want to change it later, I don't have to do it manually, I just change it from here.
% \newcommand{\uaa}{\alphaup}
% \renewcommand{\alpha}{\alphaup}
% \renewcommand{\beta}{\betaup}
% \renewcommand{\gamma}{\gammaup}
% \renewcommand{\delta}{\deltaup}
% \renewcommand{\epsilon}{\epsilonup}
% \newcommand{\ee}{\epsilon}
% \renewcommand{\varepsilon}{\varepsilonup}
% \renewcommand{\theta}{\thetaup}
% \renewcommand{\lambda}{\lambdaup}
% \newcommand{\ull}{\lambda}
% \renewcommand{\mu}{\muup}
% \renewcommand{\nu}{\nuup}
% \renewcommand{\pi}{\piup}
% \renewcommand{\rho}{\rhoup}
% \renewcommand{\varrho}{\varrhoup}
% \renewcommand{\sigma}{\sigmaup}
% \renewcommand{\tau}{\tauup} 
% \renewcommand{\phi}{\phiup}
% \renewcommand{\chi}{\chiup}
% \renewcommand{\psi}{\psiup}
% \renewcommand{\omega}{\omegaup}

% arrows and symbols 
\renewcommand{\to}{\rightarrow}
\newcommand{\toto}{\longrightarrow}
\newcommand{\mapstoto}{\longmapsto}
\newcommand{\then}{\Rightarrow}
\newcommand{\IFF}{\Leftrightarrow}
\newcommand{\tl}{\tilde}
\newcommand{\wtl}{\widetilde}
\newcommand{\ol}{\overline}
\newcommand{\ul}{\underline}
\newcommand{\oldemptyset}{\emptyset}
\renewcommand{\emptyset}{\varnothing}
\DeclareMathSymbol{\Arg}{\mathbin}{AMSa}{"39} % for arguments 
\newcommand{\onto}{\ensuremath{\twoheadrightarrow}}
\newcommand{\tle}{\trianglelefteq}
\newcommand{\tge}{\trianglerighteq}

% absolute value symbol
\usepackage{mathtools} 
\DeclarePairedDelimiter\abs{\lvert}{\rvert}%
\DeclarePairedDelimiter\norm{\lVert}{\rVert}%
\makeatletter
\let\oldabs\abs
\def\abs{\@ifstar{\oldabs}{\oldabs*}}

% tensor symbol
\newcommand{\tensor}[1]{%
  \mathbin{\mathop{\otimes}\limits_{#1}}%
}

% permutation cycle notation
\ExplSyntaxOn
\NewDocumentCommand{\cycle}{ O{\;} m }
 {
  (
  \alec_cycle:nn { #1 } { #2 }
  )
 }

\seq_new:N \l_alec_cycle_seq
\cs_new_protected:Npn \alec_cycle:nn #1 #2
 {
  \seq_set_split:Nnn \l_alec_cycle_seq { , } { #2 }
  \seq_use:Nn \l_alec_cycle_seq { #1 }
 }
\ExplSyntaxOff

% setminus symbol
\newcommand{\mysetminusD}{\hbox{\tikz{\draw[line width=0.6pt,line cap=round] (3pt,0) -- (0,6pt);}}}
\newcommand{\mysetminusT}{\mysetminusD}
\newcommand{\mysetminusS}{\hbox{\tikz{\draw[line width=0.45pt,line cap=round] (2pt,0) -- (0,4pt);}}}
\newcommand{\mysetminusSS}{\hbox{\tikz{\draw[line width=0.4pt,line cap=round] (1.5pt,0) -- (0,3pt);}}}
\newcommand{\sm}{\mathbin{\mathchoice{\mysetminusD}{\mysetminusT}{\mysetminusS}{\mysetminusSS}}}

% custom math operators
\newcommand{\Des}{\mathrm{Des}} 
\newcommand{\des}{\mathrm{des}} 
\newcommand{\Asc}{\mathrm{Asc}}
\newcommand{\asc}{\mathrm{asc}} 
\newcommand{\inv}{\mathrm{inv}}
\newcommand{\Inv}{\mathrm{Inv}}
\newcommand{\maj}{\mathrm{maj}} 
\newcommand{\comaj}{\mathrm{comaj}} 
\newcommand{\fix}{\mathrm{fix}} 
\newcommand{\Sym}{\mathrm{Sym}} 
\newcommand{\QSym}{\mathrm{QSym}}
\newcommand{\FQSym}{\mathrm{FQSym}} 
\newcommand{\End}{\mathrm{End}} 
\newcommand{\Rad}{\mathrm{Rad}} 
\newcommand{\rmMat}{\mathrm{Mat}} 
\newcommand{\rmdim}{\mathrm{dim}} 
\newcommand{\rmTop}{\mathrm{Top}} 
\newcommand{\rmCF}{\mathrm{CF}} 
\newcommand{\rmId}{\mathrm{Id}}
\newcommand{\rmid}{\mathrm{id}}
\newcommand{\rmtw}{\mathrm{tw}}
\newcommand{\trace}{\mathrm{tr}}
\newcommand{\Irr}{\mathrm{Irr}}
\newcommand{\Ind}{\mathrm{Ind}} % induction
\newcommand{\Res}{\mathrm{Res}} % restriction
\newcommand{\triv}{\mathrm{triv}} % trivial rep
\newcommand{\rmdef}{\mathrm{def}} % defining rep
\newcommand{\dom}{\triangleleft}
\newcommand{\domeq}{\trianglelefteq}
\newcommand{\lex}{\mathrm{lex}}
\newcommand{\sign}{\mathrm{sign}}
\newcommand{\SYT}{\mathrm{SYT}}
\renewcommand{\Im}{\mathrm{Im}}
\newcommand{\Ker}{\mathrm{Ker}}
\newcommand{\GL}{\mathrm{GL}}
\newcommand{\FL}{\mathrm{FL}}
\newcommand{\Span}{\mathrm{span}}
\newcommand{\pos}{\mathrm{pos}}
\newcommand{\Comp}{\mathrm{Comp}}
\newcommand{\Set}{\mathrm{Set}}
\newcommand{\std}{\mathrm{std}}
\newcommand{\cont}{\mathrm{cont}} %content of a SSYT
\newcommand{\SSYT}{\mathrm{SSYT}}
\newcommand{\ct}{\mathrm{ct}} % cycle type
\newcommand{\ch}{\mathrm{ch}} % Frobenius characteristic map
\newcommand{\height}{\mathrm{ht}}
\newcommand{\FPS}{\CC[\![\bfx]\!]} % formal power series
\newcommand{\FPSS}{\CC[\![\bfx,\bfy]\!]}
\newcommand{\reg}{\mathrm{reg}}
\newcommand{\hook}{\mathrm{h}}
\newcommand{\weight}{\mathrm{wt}}
\newcommand{\co}{\mathrm{co}}
\newcommand{\ps}{\mathrm{ps}}
\newcommand{\rmsum}{\mathrm{sum}}
\newcommand{\NSym}{\mathrm{NSym}}
\newcommand{\Hom}{\mathrm{Hom}}
\newcommand{\proj}{\mathrm{proj}}
\newcommand{\stat}{\mathrm{stat}}
\newcommand{\Par}{\mathrm{Par}}
\newcommand{\rmset}{\mathrm{set}}
\newcommand{\comp}{\mathrm{comp}}

% miscellaneous commands
\newcommand{\defn}[1]{{\color{mylightblue}{#1}}}
\newcommand{\toDo}{{\bf\color{red} TODO}}
\newcommand{\toCite}{{\bf\color{green} CITE}}
\newcommand*{\vertbar}{\rule[-1ex]{0.5pt}{2.5ex}} % for matrices with column vectors
\newcommand*{\horzbar}{\rule[.5ex]{2.5ex}{0.5pt}} % for matrices with row vectors
\newcommand{\myblue}[1]{{\color{iceberg}{#1}}}
\newcommand{\myorange}[1]{{\color{burntorange}{#1}}}
\newcommand{\mygreen}[1]{{\color{applegreen}{#1}}}
\newcommand{\myred}[1]{{\color{darkcandyapplered}{#1}}}

% ferrer's diagram
\newcommand{\fdiagram}[1]{
    \begin{tikzpicture}[scale=.7]
        \fill foreach \Z [count=\Y] in {#1}
        {foreach \X in {1,...,\Z} 
        {(\X,-\Y) circle[radius=3pt]}};
    \end{tikzpicture}
}

%
\newcommand{\tcbo}[1]{\textcolor{burntorange}{#1}}

% 
\newenvironment{nouppercase}{%
  \let\uppercase\relax%
  \renewcommand{\uppercasenonmath}[1]{}}{}

% titlepage
\title{Θ2.04: Θεωρία Αναπαραστάσεων και Συνδυαστική}
\author[Β.~Δ. Μουστακας]{Βασίλης Διονύσης Μουστάκας \\ Πανεπιστήμιο Κρήτης}
\date{9 Δεκεμβρίου 2025}
% \urladdr{\href{https://sites.google.com/view/vasmous}{https://sites.google.com/view/vasmous}}

\begin{document}

\begingroup
\def\uppercasenonmath#1{} % this disables uppercase title
\let\MakeUppercase\relax % this disables uppercase authors
\maketitle
\endgroup

\setcounter{section}{15}
\setcounter{theorem}{11}
\begin{center}
    \textbf{15. Η άλγεβρα των συμμετρικών συναρτήσεων
} (Συνέχεια)
\end{center}

Για $\lambda, \mu \vdash n$, έστω $\varphi^\lambda(\mu)$ η τιμή του χαρακτήρα του προτύπου Young $\rmM^\lambda$ στην κλάση συζυγίας της $\fS_n$ που αντιστοιχεί στη διαμέριση $\mu$.
\begin{theorem}
    \label{thm:p_to_h}
    Αν $\lambda \vdash n$, τότε 
    \[
    p_\lambda = \sum_{\mu \, \tge \lambda} \varphi^\mu(\lambda) m_\mu.
    \]
\end{theorem}

\begin{proof}[Απόδειξη]
    Έστω $\ell(\lambda) = \ell$ και $\ell(\mu) = k$. Ο συντελεστής του $\bfx^\mu = x_1^{\mu_1}x_2^{\mu_2}\cdots x_k^{\mu_k}$ στο ανάπτυγμα του δεξιού μέλους της 
    \[
    p_\lambda(\bfx) = 
    \left(
        x_1^{\lambda_1} + x_2^{\lambda_1} + \cdots
    \right)
    \left(
        x_1^{\lambda_2} + x_2^{\lambda_2} + \cdots
    \right) 
    \cdots 
    \left(
        x_1^{\lambda_\ell} + x_2^{\lambda_\ell} + \cdots
    \right) 
    \]
    ισούται με το πλήθος των παραγοντοποιήσεων του $\bfx^\mu$ ως $x_{i_1}^{\lambda_1}x_{i_2}^{\lambda_2}\cdots x_{i_\ell}^{\lambda_\ell}$. Για παράδειγμα, για $\lambda = (2,1,1,1)$ και $\mu = (3,2)$, έχουμε τις τέσσερις παραγοντοποιήσεις 
    \[
    x_1^3x_2^2 
    = x_1^2x_1x_2x_2
    = x_1^2x_2x_1x_2
    = x_1^2x_2x_2x_1
    = x_2^2x_1x_1x_1.
    \]

    Αναπαριστούμε κάθε τέτοια παραγοντοποίηση με ένα \emph{ταμπλοειδές} σχήματος $\mu$ έτσι ώστε οι αριθμοί 
    \[
    [\lambda_1 + \lambda_2 + \cdots + \lambda_{j-1}+1, 
    \lambda_1 + \lambda_2 + \cdots + \lambda_j]
    \]
    να βρίσκονται στην $i_j$-οστή γραμμή, για κάθε $1 \le j \le \ell$. Με άλλα λόγια, οι γραμμές του ταμπλοειδούς αντιστοιχούν στις μεταβλητές $x_1, x_2, \dots, x_k$, ενώ τα στοιχεία του ταμπλοειδούς αντιστοιχούν στη θέση που καταλαμβάνει κάθε μεταβλητή στο ανάπτυγμα. Τα ταμπλοειδή που αντιστοιχούν στις παραγοντοποιήσεις του παραδείγματος είναι 
    \[
    \ytableausetup{centertableaux,tabloids}
    \ytableaushort{123,45} \ , \ 
    \ytableaushort{124,35} \ , \ 
    \ytableaushort{125,34} \ , \ 
    \ytableaushort{345,12} \ , \ 
    \]
    αντίστοιχα.

    Τα ταμπλοειδή που αντιστοιχούν στις παραγοντοποιήσεις αυτές είναι ακριβώς τα ταμπλοειδή σχήματος $\mu$ που μένουν σταθερά από τη δράση της μετάθεσης 
    \[
    \cycle{1,2,\cdots,\lambda_1}
    \cycle{\lambda_1+1,\lambda_1+2,\cdots,\lambda_1+\lambda_2}
    \cdots 
    \cycle{\lambda_1+\lambda_2+\cdots+\lambda_{\ell-1}+1,\cdots,n},
    \]
    η οποία έχει κυκλικό τύπο $\lambda$ (γιατί;). Άρα, ο συντελεστής του $\bfx^\mu$ στο $p_\lambda$ ισούται με $\varphi^\lambda(\mu)$. Τέλος, από το Πρόταση~11.11, αν $\varphi^\lambda(\mu) \neq 0$, τότε $\mu \tge \lambda$ και το ζητούμενο έπεται.
\end{proof}

\begin{corollary}
    Αν $\lambda \vdash n$, τότε 
    \[
    \varphi^\lambda(\lambda) = \prod_{i=1}^n m_i!,
    \]
    όπου $m_i$ είναι η πολλαπλότητα εμφάνισης του $i$ στα μέρη της $\lambda$. Ειδικότερα, το σύνολο $\{p_\lambda : \lambda \vdash n\}$ αποτελεί βάση του $\Sym_n$ και ο $\Sym$ παράγεται ως δακτύλιος από τα $p_1, p_2, \dots$.
\end{corollary}

\begin{proof}[Απόδειξη]
    Από την απόδειξη του Θεωρήματος~\ref{thm:p_to_h}, προκύπτει ότι η τιμή του $\varphi^\lambda(\lambda)$ ισούται με το πλήθος των ταμπλοειδών σχήματος $\lambda$ των οποίων η $i$-οστή γραμμή περιέχει τους αριθμούς 
    \[
    [\lambda_1 + \lambda_2 + \cdots + \lambda_{i-1}+1, \lambda_1 + \lambda_2 + \cdots + \lambda_i]
    \]
    για κάθε $i \ge 1$ (γιατί;) και ο πρώτος ισχυρισμός έπεται (γιατί;). Ο δεύτερος ισχυρισμός έπεται με τον ίδιο τρόπο όπως στο Πόρισμα 15.7.

    Για παράδειγμα, αν $\lambda = (2,1,1,1)$, τότε έχουμε τα εξής ταμπλοειδή της παραπάνω μορφής 
    \[
    \ytableaushort{12,3,4,5} \ , \
    \ytableaushort{12,3,5,4} \ , \
    \ytableaushort{12,4,3,5} \ , \
    \ytableaushort{12,4,5,3} \ , \
    \ytableaushort{12,5,3,4} \ , \
    \ytableaushort{12,5,4,3} \ , 
    \]
    και γι αυτό $\varphi^\lambda(\lambda) = 6$. (Ποιές είναι οι αντίστοιχες παραγοντοποιήσεις του $x_1^2x_2x_3x_4$;)
\end{proof}

Για $n=3$, υπολογίζουμε
\begin{alignat*}{4}
p_{(1,1,1)} &=\;& 6m_{(1,1,1)} &+&\; 3m_{(2,1)} &+&\; m_{(3)} \\
p_{(2,1)}   &=\;&              & &    m_{(2,1)} &+&\; m_{(3)}\\
p_{(3)}     &=\;&              & &              & &\; m_{(3)}.
\end{alignat*}
Σε αντίθεση με την βάση των στοιχειωδών συμμετρικών συναρτήσεων, όπου στην κύρια διαγώνιο είχαμε 1, εδώ οι πίνακες μετάβασης είναι αντιστρέψιμοι πάνω από το $\QQ$. Για παράδειγμα, 
\[
m_{(1,1)} = \frac{1}{2}(p_{(1,1)} + p_{(2)}).
\]

\newpage

\setcounter{section}{16}
\setcounter{theorem}{0}
\begin{center}
    \textbf{16. Οι συναρτήσεις Schur
} 
\end{center}

Στην Παράγραφο 15 είδαμε τέσσερις βάσεις του χώρου των συμμετρικών συναρτήσεων: την μονωνυμική, την στοιχειώδη, την πλήρως ομογενή και την power sum. Η πρώτη από αυτές προέκυψε με φυσικό τρόπο από τη διαδικασία συμμετρικοποίησης ενός μονωνύμου. Οι υπόλοιπες τρεις σχετίζονται με την μονωνυμική και αποτελούν παραδείγματα πολλαπλασιαστικών συμμετρικών συναρτήσεων.

Σε αυτή την παράγραφο θα δούμε ένα ακόμα παράδειγμα, ίσως το πιο σημαντικό, βάσης του χώρου των συμμετρικών συναρτήσεων. Η βάση αυτή θα προκύψει\footnote{Υπάρχουν διάφοροι ισοδύναμοι τρόποι να εισαγάγει κανείς τις συναρτήσεις Schur. Εδώ παρουσιάζεται ο πιο συνδυαστικός.} ως η \textquote{γεννήτρια συνάρτηση} μιας οικογένειας συνδυαστικών αντικειμένων με κάποιο κατάλληλο \textquote{βάρος}. 

\begin{definition}
    \label{def:SSYT}
    Έστω $\lambda$ μια διαμέριση ακεραίου. \defn{Ημισύνηθες Young ταμπλώ} (semistandard Young tableau) σχήματος $\lambda$ ονομάζεται μια αμφιμονοσήμαντη αντιστοιχία του συνόλου των τετραγώνων του $\rmY_\lambda$ και του $\ZZ_{>0}$, τέτοια ώστε 
    \begin{itemize}
        \item τα στοιχεία κάθε γραμμής να αυξάνουν \emph{ασθενώς} από τα αριστερά προς τα δεξιά 
        \item τα στοιχεία κάθε στήλης να αυξάνουν \emph{αυστηρά} από πάνω προς τα κάτω. 
    \end{itemize}
    Το σύνολο όλων των ημισυνήθων ταμπλώ σχήματος $\lambda$ συμβολίζεται με $\SSYT(\lambda)$.
\end{definition}

Ένα παράδειγμα ημισυνήθους ταμπλώ σχήματος $\lambda = (4,2,2,1)$ είναι το εξής 
\[
\ytableausetup{notabloids}
\ytableaushort{1133,22,34,6} \ .
\]

Για $T \in \SSYT(\lambda)$, έστω $\cont(T) = (\alpha_1, \alpha_2, \dots)$, όπου $\alpha_i$ είναι η πολλαπλότητα εμφάνισης του ακεραίου $i$ στο $T$. Το $\cont(T)$ ονομάζεται \defn{τύπος} του $T$. Ο τύπος του ημισυνήθους ταμπλώ του παραδείγματος είναι 
\[
(2, 2, 3, 1, 0, 1, 0, 0, \dots).
\]
Παρατηρούμε ότι $2 + 2 + 3 + 1 + 0 + 1 + 0 + \cdots = 9$, δηλαδή το πλήθος των τετραγώνων του $\rmY_{(4,2,2,1)}$ ή ισοδύναμα το $\abs{\lambda} = 9$. Αν $\lambda \vdash n$, τότε ο τύπος ενός ημισυνήθους ταμπλώ σχήματος $\lambda$ είναι παράδειγμα \emph{ασθενούς σύνθεσης}\footnote{Μια ακολουθία $(\alpha_1, \alpha_2, \dots)$ μη αρνητικών ακεραίων τέτοια ώστε $\alpha_1 + \alpha_2 + \cdots = n$ ονομάζεται \defn{ασθενής σύνθεση} (weak composition) του $n$ (συγκρίνετε με τον Ορισμό~9.8).} του $n$. Σε κάθε ομογενή τυπική δυναμοσειρά βαθμού $n$, η ακολουθία των εκθετών  είναι επίσης παράδειγμα ασθενούς σύνθεσης του $n$.

Το παρακάτω αποτέλεσμα, η απόδειξη του οποίου παραλλείπεται, παρέχει μια συνδυαστική ερμηνεία των αριθμών Kostka, δηλαδή των πολλαπλοτήτων εμφάνισης των ανάγωγων αναπαραστάσεων στην ισοτυπική διάσπαση του προτύπου Young, που αφορά ημισυ\-νήθη ταμπλώ.
\begin{theorem}{\rm(Κανόνας του Young)}
    \label{thm:young_rule}
    Για κάθε $\mu \vdash n$, 
    \begin{equation}
        \label{eq:young_rule}
        \rmM^\mu \cong_{\fS_n} \bigoplus_{\lambda \vdash n}\, (\sS^\lambda)^{\rmK_{\lambda\mu}},
    \end{equation}
    όπου 
    \[
    \rmK_{\lambda\mu} = \abs{\{T \in \SSYT(\lambda) : \cont(\lambda)=\mu\}}.
    \]
\end{theorem}

Για παράδειγμα, από το Πόρισμα 11.13, για $\mu = (2,2,1) \vdash 5$, γνωρίζουμε ότι 
\[
\rmM^{(2,2,1)} \cong_{\fS_5} (\sS^{(5)})^a \oplus (\sS^{(4,1)})^b \oplus (\sS^{(3,2)})^c \oplus (\sS^{(3,1,1)})^d \oplus \sS^{(2,2,1)}.
\]
για κάποια $a, b, c, d \in \NN$. Όμως, 
\[
\ytableausetup{smalltableaux}
\renewcommand{\arraystretch}{1.2}
\setlength{\extrarowheight}{8pt}
\begin{tabular}{c|c|cc|cc|c|c}
\lambda & (5) & \multicolumn{2}{c|}{(4,1)} & \multicolumn{2}{c|}{(3,2)} & (3,1,1) & (2,2,1) \\
\hline
$T$     
& \ytableaushort{11223} 
& \ytableaushort{1122,3}  & \ytableaushort{1123,2}  
& \ytableaushort{112,23}  & \ytableaushort{113,22} 
& \ytableaushort{112,2,3} & \ytableaushort{11,22,3} 
\end{tabular}
\]
και από τον Κανόνα του Young βρίσκουμε
\[
\rmM^{(2,2,1)} \cong_{\fS_5} \sS^{(5)} \oplus (\sS^{(4,1)})^2 \oplus (\sS^{(3,2)})^2 \oplus \sS^{(3,1,1)} \oplus \sS^{(2,2,1)}.
\]

Παρατηρούμε τα εξής:
\begin{itemize}
    \item 
    $\rmK_{\mu\mu} =1$, για κάθε $\mu \vdash n$, σε συμφωνία με τον ισχυρισμό του Πορίσματος~11.13. 
    \item 
    $\rmK_{(n)\mu} = 1$, για κάθε $\mu \vdash n$, σε συμφωνία με το γεγονός ότι η αναπαράσταση συ\-μπλόκου της υποομάδας Young $\fS_\mu$ περιέχει ακριβώς ένα αντίγραφο της τετριμμένης αναπαράστασης της $\fS_n$.
    \item 
    Ένα ημισύνηθες ταμπλώ σχήματος $\lambda$ και περιεχομένου $(1^n)$ δεν είναι άλλο παρά ένα σύνηθες ταμπλώ (γιατί;) και γι αυτό $\rmK_{\lambda(1^n)} = f^\lambda$, για κάθε $\lambda \vdash n$. Συνεπώς, από τον Κανόνα του Young, έπεται ότι
\[
\rmM^{(1^n)} \cong_{\fS_n} \bigoplus_{\lambda \vdash n} \, (\sS^\lambda)^{f^\lambda},
\]
σε συμφωνία με τον υπολογισμό στην αρχή της Παραγράφου~11.
\end{itemize}
Τι μας πληροφορεί ο Κανόνας του Young για την περίπτωση $\mu = (n-1,1)$;

Επιστρέφοντας στον \text{κόσμο} των συμμετρικών συναρτήσεων, για $T \in \SSYT(\lambda)$ τύπου $\cont(T) = (\alpha_1, \alpha_2, \dots)$ θεωρούμε το μονώνυμο
\[
\bfx^T \coloneqq x_1^{\alpha_1}x_2^{\alpha_2}\cdots.
\]
Για παράδειγμα, το μονώνυμο που αντιστοιχεί στο ημισύνηθες ταμπλώ 
\[
\ytableausetup{nosmalltableaux}
Τ = \ytableaushort{1133,22,34,6} \ \in \ \SSYT(4,2,2,1)
\]
είναι 
\[
\bfx^T = x_1^2x_2^2x_3^3x_4x_6.
\]
\begin{definition}
    \label{def:schur_function}
    Έστω $\lambda \vdash n$. Η τυπική δυναμοσειρά 
    \[
    s_\lambda = s_\lambda(\bfx) \coloneqq \sum_{T \in \SSYT(\lambda)} \bfx^T
    \]
    ονομάζεται \defn{συνάρτηση Schur} που αντιστοιχεί στην $\lambda$.
\end{definition}

Με άλλα λόγια, η συνάρτηση Schur είναι η \textquote{γεννήτρια συνάρτηση} των ημισυνήθων Young ταμπλώ $T$ ως προς το \textquote{βάρος} $\bfx^T$. Σε αντίθεση με όλα τα προηγούμενα παραδείγματα οικογενειών συμμετρικών συναρτήσεων που είδαμε μέχρι στιγμής, οι συναρτήσεις Schur δεν έχουν εκ των προτέρων κανένα λόγο να είναι συμμετρικές. 

Για ασθενή σύνθεση $\alpha$, έστω $\lambda(\alpha)$ η διαμέριση που προκύπτει από την $\alpha$ ξεχνώντας τα μηδενικά και αναδιατάσσοντας τα μέρη της σε (ασθενώς) αύξουσα σειρά.

\begin{example}
    \label{ex:schur}
    \leavevmode
    \begin{itemize}
        \item[(1)] Έχουμε $s_{(n)} = h_n$ και $s_{(1^n)} = e_n$. Συνεπώς η συνάρτηση Schur παρεμβάλλεται μεταξύ της πλήρους ομογενούς και της στοιχειώδους συμμετρικής συνάρτησης.
        \item[(2)] Έστω ότι
        \[
        s_{(2,1)} = c_{(3)}m_{(3)} + c_{(2,1)}m_{(2,1)} + c_{(1,1,1)}m_{(1,1,1)}
        \]
        για κάποια $c_\mu \in \CC$. Ας υπολογίσουμε τα $c_\mu$ του παραπάνω αναπτύγματος.

        Παρατηρούμε ότι δεν υπάρχει ημισύνηθες ταμπλώ σχήματος $(2,1)$ και τύπου $\alpha$ με $\lambda(\alpha) = (3)$ (γιατί;) και γι αυτό $c_{(3)} = 0$. Αν $0 < a < b < c$, τότε υπάρχουν ακριβώς δυο διαφορετικά ημισυνήθη ταμπλώ σχήματος $(2,1)$ και τύπου $\alpha$ με $\lambda(\alpha) = (1,1,1)$:
        \[
        \ytableaushort{ab,c}  \quad \text{και} \quad \ytableaushort{ac,b}
        \]
        και γι αυτό $c_{(1,1,1)} = 2$. Τέλος, αν $a\neq b$ δυο θετικοί ακέραιοι, τότε υπάρχει ακριβώς ένα ημισύνηθες ταμπλώ $(2,1)$ και τύπου $\alpha$ με $\lambda(\alpha) = (2,1)$:
        \[
        \begin{cases}
            \ \ytableaushort{aa,b}\, , &\ \text{αν $a < b$}, \vspace{10pt}\\
            \ \ytableaushort{bb,a}\, , &\ \text{αν $b < a$}
        \end{cases}
        \]
        και γι αυτό $c_{(2,1)} = 1$. Συμπερασματικά, λοιπόν, έχουμε 
        \[
        s_{(2,1)} = m_{(2,1)} + 2m_{(1,1,1)}
        \]
        \item[(3)] Γενικεύοντας τον υπολογισμό του $c_{(2,1)}$ στο (2), ο συντελεστής του $x_1x_2\cdots{x_n}$ στην συνάρτηση Schur που αντιστοιχεί στην διαμέριση $\lambda \vdash n$ ισούται με $f^\lambda$ (γιατί;).
    \end{itemize}
\end{example}

Όπως υποννοεί το παράδειγμα~\ref{ex:schur}, οι συναρτήσεις Schur είναι συμμετρικές και \textquote{γνωρίζουν} πολλά από αυτά που έχουμε δει μέχρι στιγμής.
\begin{theorem}
    \label{thm:schur_symmetric}
    Για κάθε $\lambda \vdash n$, $s_\lambda \in \Sym_n$.
\end{theorem}

Θα δούμε δύο αποδείξεις του Θεωρήματος~\ref{thm:schur_symmetric}: μια αλγεβρική και μια συνδυαστική. Για την αλγεβρική, θυμίζουμε ότι το πρότυπο Young $\rmM^\lambda$ είναι ισόμορφο με την επαγωγή της τετριμμένης αναπαράστασης $V^\triv$ της υποομάδας Young $\fS_\lambda$ στην $\fS_n$ (βλ. Πρόταση~10.4). Όμως, $\fS_\alpha \cong \fS_{\lambda(\alpha)}$ για κάθε ασθενή σύνθεση $\alpha$ του $n$, όπου $\fS_0 \coloneqq \{\epsilon\}$ και γι αυτό 
\[
V^\triv\uparrow_{\fS_\alpha}^{\fS_n} \cong_{\fS_n} 
V^\triv\uparrow_{\fS_{\lambda(\alpha)}}^{\fS_n} \cong_{\fS_n}
\rmM^{\lambda(\alpha)}.  
\]

\begin{proof}[Απόδειξη του Θεωρήματος~\ref{thm:schur_symmetric} (Αλγεβρική)]
    Από τον Ορισμό~\ref{def:schur_function},
    \[
    s_\lambda = \sum_\alpha \rmK_{\lambda\alpha} m_\alpha,
    \]
    όπου στο άθροισμα το $\alpha$ διατρέχει όλες τις ασθενείς συνθέσεις του $n$. Αρκεί, λοιπόν, να δείξουμε ότι 
    \[
    \rmK_{\lambda\alpha} = \rmK_{\lambda\lambda(\alpha)}
    \]
    για κάθε ασθενή σύνθεση $\alpha$ του $n$. Το ζητούμενο έπεται από την συζήτηση που προηγείται της απόδειξης, τον Κανόνα του Young και την μοναδικότητα της ισοτυπικής διάσπασης (γιατί;).
\end{proof}

Για την συνδυαστική απόδειξη, θα επαληθεύσουμε τον Ορισμό~15.1, δηλαδή θα δείξουμε ότι η συνάρτηση Schur είναι αναλλοίωτη από οποιαδήποτε μετάθεση των μεταβλητών της. Από την Άσκηση 3.5 (1), αρκεί να εξετάσουμε την δράση κάθε γειτονικής αντιμεταθέσης $\cycle{i, i+1}$ για $i \ge 1$.
\begin{proof}[Απόδειξη του Θεωρήματος~\ref{thm:schur_symmetric} (Συνδυαστική)]
    Από την συζήτηση που προηγήθηκε αρκεί να δείξουμε ότι 
    \[
    \rmK_{\lambda\alpha} = \rmK_{\lambda\hat{\alpha}},
    \]
    όπου $\hat{\alpha} = (\alpha_1, \dots, \alpha_{i+1}, \alpha_i, \dots, )$ για κάθε ασθενή σύνθεση $\alpha = (\alpha_1, \alpha_2, \dots)$ του $n$. Για τον λόγο αυτό αρκεί να βρούμε μια αμφιμονοσήμαντη αντιστοιχία 
    \[
    \{T \in \SSYT(\lambda) : \cont(T) = \alpha\} \to 
    \{T \in \SSYT(\lambda) : \cont(T) = \hat{\alpha}\}.  
    \]

    Έστω $T \in \SSYT(\lambda)$ με τύπο $\alpha$. Θεωρούμε τις εμφανίσεις των $i$ και $i+1$ σε κάθε γραμμή του $T$, αγνοώντας εκείνες όπου τα $i$ και $i+1$ βρίσκονται διαδοχικά στην ίδια στήλη, δηλαδή 
    \[
    \ytableaushort{i,{\scriptstyle i+1}}\, ,
    \]
    καθώς αυτά δεν μπορούν να εναλλαγούν (γιατί;). Για παράδειγμα, για $\lambda = (9,5,4,3,2)$ και $i=5$ έχουμε 
    \[
    Τ = \
    \begin{ytableau}
        1 & 1 & 1 & 2 & 3 & *(iceberg) 5 & *(applegreen) 6 & *(applegreen) 6 & *(applegreen) 6 \\
        2 & 3 & 4 & *(iceberg) 5 & *(applegreen) 6 \\
        3 & 4 & *(burntorange) 5 & 7 \\ 
        4 & *(applegreen) 6 & *(burntorange) 6 \\
        *(iceberg) 5 & 7
    \end{ytableau}
    \]
    με $\cont(T) = (3,2,3,3, \textcolor{iceberg}{3} + \textcolor{burntorange}{1}, \textcolor{applegreen}{5} + \textcolor{burntorange}{1}, 2)$.

    Έστω $\phi(T)$ το ταμπλώ που προκύπτει από το $T$ εναλλάσσοντας σε κάθε γραμμή το πλήθος των εμφανίσεων του $i$ με το πλήθος των εμφανίσεων του $i+1$. Το $\phi(T)$ είναι ένα ημισύνηθες ταμπλώ σχήματος $\lambda$ και τύπου $\hat{\alpha}$ (γιατί;) και η απεικόνιση $\phi$ είναι η ζητούμενη αμφιμονοσήμαντη αντιστοιχία (γιατί;). Στο παράδειγμα, έχουμε 
    \[
    \phi(Τ) = \
    \begin{ytableau}
        1 & 1 & 1 & 2 & 3 & *(applegreen) 5 & *(applegreen) 5 & *(applegreen) 5 & *(iceberg) 6 \\
        2 & 3 & 4 & *(applegreen) 5 & *(iceberg) 6 \\
        3 & 4 & *(burntorange) 5 & 7 \\ 
        4 & *(applegreen) 5 & *(burntorange) 6 \\
        *(iceberg) 6 & 7
    \end{ytableau}
    \]
    με $\cont(\phi(T)) = (3,2,3,3, \textcolor{applegreen}{5} + \textcolor{burntorange}{1}, \textcolor{iceberg}{3} + \textcolor{burntorange}{1}, 2)$.
\end{proof}
\end{document}