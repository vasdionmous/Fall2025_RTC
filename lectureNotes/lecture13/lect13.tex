\documentclass[12pt,a4paper,reqno]{amsart}

% language
\usepackage[greek,english]{babel}
\usepackage[utf8]{inputenc}
\usepackage{alphabeta}

% change default names to greek
\addto\captionsenglish{
    \renewcommand{\contentsname}{Περιεχόμενα}
    \renewcommand{\refname}{Βιβλιογραφία}
    \renewcommand{\datename}{Ημερομηνία:}
    \renewcommand{\urladdrname}{Ιστοσελίδα}
}

% math 
\usepackage{amsmath,amsthm,amssymb,amscd}

% font
\usepackage[cal=euler]{mathalfa}
\usepackage{libertinus-type1}
% \usepackage{txfonts} % for upright greek letters
\usepackage{bm} % for bold symbols
\usepackage{bbm} % for the simply-looking bb symbols

% miscellaneous 
\usepackage{changepage} %for indenting environments
\usepackage{csquotes} % example: \textcquote{}
\usepackage{blkarray}
\setcounter{MaxMatrixCols}{20} % default for pmatrix is 10!!
\usepackage{ytableau}

% drawing
\usepackage{tikz,tikz-cd}
\usetikzlibrary{shapes.misc, patterns, matrix, calc, intersections,positioning}
\usepackage{graphics,graphicx}
\usepackage{float} % provides enhanced control and customization options for floating objects such as figures and tables

% colors
\usepackage{xcolor}
\definecolor{darkcandyapplered}{rgb}{0.64, 0.0, 0.0}
\definecolor{midnightblue}{rgb}{0.1, 0.1, 0.44}
\definecolor{mylightblue}{HTML}{336699}
\definecolor{burntorange}{rgb}{0.8, 0.33, 0.0}
\definecolor{iceberg}{rgb}{0.44, 0.65, 0.82}
\definecolor{applegreen}{rgb}{0.55, 0.71, 0.0}
\definecolor{canaryyellow}{rgb}{1.0, 0.94, 0.0}

% hrefs
\usepackage{hyperref}
\usepackage[noabbrev,capitalize]{cleveref}
\hypersetup{
    pdftoolbar=true,        
    pdfmenubar=true,        
    pdffitwindow=false,     
    pdfstartview={FitH},  % fits the width of the page to the window
    pdftitle={},
    pdfauthor={},
    pdfsubject={},
    pdfkeywords={},
    pdfnewwindow=true,  % links in new window
    colorlinks=true,  % false: boxed links; true: colored links
    linkcolor=darkcandyapplered,   % color of internal links
    citecolor=midnightblue,  % color of links to bibliography
    urlcolor=cyan,  % color of external links
    linktocpage=true  % changes the links from the section body to the page number
    }

% geometry
\textwidth=16cm 
\textheight=21cm 
\hoffset=-55pt 
\footskip=25pt

% thm envs (you might need to change the path)
% In this macro I define all the theorem environments

\theoremstyle{definition}
\newtheorem{theorem}{Θεώρημα}
\newtheorem{proposition}[theorem]{Πρόταση}
\newtheorem{lemma}[theorem]{Λήμμα}
\newtheorem{corollary}[theorem]{Πόρισμα}
\newtheorem{conjecture}[theorem]{Εικασία}
\newtheorem{problem}[theorem]{Πρόβλημα}
\newtheorem*{claim}{Ισχυρισμός}
\newtheorem{observation}[theorem]{Παρατήρηση}
\newtheorem{definition}[theorem]{Ορισμός}
\newtheorem{question}[theorem]{Ερώτηση}
\newtheorem*{questions}{Ερωτήματα}
\newtheorem{example}[theorem]{Παράδειγμα}
\newtheorem{exercise}{Άσκηση}

\newtheorem*{combInterlude}{Ιντερλούδιο Συνδυαστικής}
\newtheorem*{example_cont}{Παράδειγμα~6.6}
\newtheorem*{digression_la}{Παρέκβαση Γραμμικής Άλγεβρας}
\newtheorem*{thm}{Θεώρημα}

\theoremstyle{remark}
\newtheorem*{remark}{Παρατήρηση}

% fixes the correct numbering of environments
\numberwithin{theorem}{section}
\numberwithin{exercise}{section}
\numberwithin{equation}{section}

% math ops (you might need to change the path)
% In this macro I define all of my math operators

% fields
\newcommand{\NN}{\mathbbmss{N}} 
\newcommand{\ZZ}{\mathbbmss{Z}} 
\newcommand{\QQ}{\mathbbmss{Q}} 
\newcommand{\RR}{\mathbbmss{R}} 
\newcommand{\CC}{\mathbbmss{C}} 
\newcommand{\KK}{\mathbbmss{K}} 
\newcommand{\FF}{\mathbbmss{F}} 

% symmetric group
\newcommand{\fS}{\mathfrak{S}}  

% calligraphic 
\newcommand{\aA}{\mathcal{A}} 
\newcommand{\bB}{\mathcal{B}}
\newcommand{\cC}{\mathcal{C}}
\newcommand{\dD}{\mathcal{D}}
\newcommand{\eE}{\mathcal{E}}
\newcommand{\fF}{\mathcal{F}}
\newcommand{\hH}{\mathcal{H}}
\newcommand{\iI}{\mathcal{I}}
\newcommand{\lL}{\mathcal{L}}
\newcommand{\oO}{\mathcal{O}}
\newcommand{\pP}{\mathcal{P}}
\newcommand{\sS}{\mathcal{S}}
\newcommand{\mM}{\mathcal{M}}
\newcommand{\uU}{\mathcal{U}}

% bold
\newcommand{\bfa}{\mathbf{a}}
\newcommand{\bfe}{\mathbf{e}}
\newcommand{\bfF}{\pmb{F}}
\newcommand{\bfR}{\pmb{R}}
\newcommand{\bfv}{\mathbf{v}}
%\newcommand{\bfx}{\bm{x}}
%\newcommand{\bfx}{\mathbf{x}} 
\newcommand{\bfx}{\pmb{x}}
\newcommand{\bfX}{\pmb{X}}
\newcommand{\bfy}{\pmb{y}}
\newcommand{\bfz}{\pmb{z}}

% roman
\newcommand{\rmA}{\mathrm{A}}
\newcommand{\rmB}{\mathrm{B}}
\newcommand{\rmC}{\mathrm{C}}
\newcommand{\rmD}{\mathrm{D}} 
\newcommand{\rmI}{\mathrm{I}} 
\newcommand{\rmK}{\mathrm{K}}
\newcommand{\rmM}{\mathrm{M}}
\newcommand{\rmP}{\mathrm{P}}  
\newcommand{\rmp}{\mathrm{p}}  
\newcommand{\rmQ}{\mathrm{Q}}  
\newcommand{\rmR}{\mathrm{R}}
\newcommand{\rmS}{\mathrm{S}}
\newcommand{\rmT}{\mathrm{T}}
\newcommand{\rmU}{\mathrm{U}}
\newcommand{\rmV}{\mathrm{V}}
\newcommand{\rmY}{\mathrm{Y}}
\newcommand{\rmZ}{\mathrm{Z}}
\newcommand{\rmz}{\mathrm{z}}

% greek letters
% I'm renewing some commands in order to appear in upright font
% If I want to change it later, I don't have to do it manually, I just change it from here.
% \newcommand{\uaa}{\alphaup}
% \renewcommand{\alpha}{\alphaup}
% \renewcommand{\beta}{\betaup}
% \renewcommand{\gamma}{\gammaup}
% \renewcommand{\delta}{\deltaup}
% \renewcommand{\epsilon}{\epsilonup}
% \newcommand{\ee}{\epsilon}
% \renewcommand{\varepsilon}{\varepsilonup}
% \renewcommand{\theta}{\thetaup}
% \renewcommand{\lambda}{\lambdaup}
% \newcommand{\ull}{\lambda}
% \renewcommand{\mu}{\muup}
% \renewcommand{\nu}{\nuup}
% \renewcommand{\pi}{\piup}
% \renewcommand{\rho}{\rhoup}
% \renewcommand{\varrho}{\varrhoup}
% \renewcommand{\sigma}{\sigmaup}
% \renewcommand{\tau}{\tauup} 
% \renewcommand{\phi}{\phiup}
% \renewcommand{\chi}{\chiup}
% \renewcommand{\psi}{\psiup}
% \renewcommand{\omega}{\omegaup}

% arrows and symbols 
\renewcommand{\to}{\rightarrow}
\newcommand{\toto}{\longrightarrow}
\newcommand{\mapstoto}{\longmapsto}
\newcommand{\then}{\Rightarrow}
\newcommand{\IFF}{\Leftrightarrow}
\newcommand{\tl}{\tilde}
\newcommand{\wtl}{\widetilde}
\newcommand{\ol}{\overline}
\newcommand{\ul}{\underline}
\newcommand{\oldemptyset}{\emptyset}
\renewcommand{\emptyset}{\varnothing}
\DeclareMathSymbol{\Arg}{\mathbin}{AMSa}{"39} % for arguments 
\newcommand{\onto}{\ensuremath{\twoheadrightarrow}}
\newcommand{\tle}{\trianglelefteq}
\newcommand{\tge}{\trianglerighteq}

% absolute value symbol
\usepackage{mathtools} 
\DeclarePairedDelimiter\abs{\lvert}{\rvert}%
\DeclarePairedDelimiter\norm{\lVert}{\rVert}%
\makeatletter
\let\oldabs\abs
\def\abs{\@ifstar{\oldabs}{\oldabs*}}

% tensor symbol
\newcommand{\tensor}[1]{%
  \mathbin{\mathop{\otimes}\limits_{#1}}%
}

% permutation cycle notation
\ExplSyntaxOn
\NewDocumentCommand{\cycle}{ O{\;} m }
 {
  (
  \alec_cycle:nn { #1 } { #2 }
  )
 }

\seq_new:N \l_alec_cycle_seq
\cs_new_protected:Npn \alec_cycle:nn #1 #2
 {
  \seq_set_split:Nnn \l_alec_cycle_seq { , } { #2 }
  \seq_use:Nn \l_alec_cycle_seq { #1 }
 }
\ExplSyntaxOff

% setminus symbol
\newcommand{\mysetminusD}{\hbox{\tikz{\draw[line width=0.6pt,line cap=round] (3pt,0) -- (0,6pt);}}}
\newcommand{\mysetminusT}{\mysetminusD}
\newcommand{\mysetminusS}{\hbox{\tikz{\draw[line width=0.45pt,line cap=round] (2pt,0) -- (0,4pt);}}}
\newcommand{\mysetminusSS}{\hbox{\tikz{\draw[line width=0.4pt,line cap=round] (1.5pt,0) -- (0,3pt);}}}
\newcommand{\sm}{\mathbin{\mathchoice{\mysetminusD}{\mysetminusT}{\mysetminusS}{\mysetminusSS}}}

% custom math operators
\newcommand{\Des}{\mathrm{Des}} 
\newcommand{\des}{\mathrm{des}} 
\newcommand{\Asc}{\mathrm{Asc}}
\newcommand{\asc}{\mathrm{asc}} 
\newcommand{\inv}{\mathrm{inv}}
\newcommand{\Inv}{\mathrm{Inv}}
\newcommand{\maj}{\mathrm{maj}} 
\newcommand{\comaj}{\mathrm{comaj}} 
\newcommand{\fix}{\mathrm{fix}} 
\newcommand{\Sym}{\mathrm{Sym}} 
\newcommand{\QSym}{\mathrm{QSym}}
\newcommand{\FQSym}{\mathrm{FQSym}} 
\newcommand{\End}{\mathrm{End}} 
\newcommand{\Rad}{\mathrm{Rad}} 
\newcommand{\rmMat}{\mathrm{Mat}} 
\newcommand{\rmdim}{\mathrm{dim}} 
\newcommand{\rmTop}{\mathrm{Top}} 
\newcommand{\rmCF}{\mathrm{CF}} 
\newcommand{\rmId}{\mathrm{Id}}
\newcommand{\rmid}{\mathrm{id}}
\newcommand{\rmtw}{\mathrm{tw}}
\newcommand{\trace}{\mathrm{tr}}
\newcommand{\Irr}{\mathrm{Irr}}
\newcommand{\Ind}{\mathrm{Ind}} % induction
\newcommand{\Res}{\mathrm{Res}} % restriction
\newcommand{\triv}{\mathrm{triv}} % trivial rep
\newcommand{\rmdef}{\mathrm{def}} % defining rep
\newcommand{\dom}{\triangleleft}
\newcommand{\domeq}{\trianglelefteq}
\newcommand{\lex}{\mathrm{lex}}
\newcommand{\sign}{\mathrm{sign}}
\newcommand{\SYT}{\mathrm{SYT}}
\renewcommand{\Im}{\mathrm{Im}}
\newcommand{\Ker}{\mathrm{Ker}}
\newcommand{\GL}{\mathrm{GL}}
\newcommand{\FL}{\mathrm{FL}}
\newcommand{\Span}{\mathrm{span}}
\newcommand{\pos}{\mathrm{pos}}
\newcommand{\Comp}{\mathrm{Comp}}
\newcommand{\Set}{\mathrm{Set}}
\newcommand{\std}{\mathrm{std}}
\newcommand{\cont}{\mathrm{cont}} %content of a SSYT
\newcommand{\SSYT}{\mathrm{SSYT}}
\newcommand{\ct}{\mathrm{ct}} % cycle type
\newcommand{\ch}{\mathrm{ch}} % Frobenius characteristic map
\newcommand{\height}{\mathrm{ht}}
\newcommand{\FPS}{\CC[\![\bfx]\!]} % formal power series
\newcommand{\FPSS}{\CC[\![\bfx,\bfy]\!]}
\newcommand{\reg}{\mathrm{reg}}
\newcommand{\hook}{\mathrm{h}}
\newcommand{\weight}{\mathrm{wt}}
\newcommand{\co}{\mathrm{co}}
\newcommand{\ps}{\mathrm{ps}}
\newcommand{\rmsum}{\mathrm{sum}}
\newcommand{\NSym}{\mathrm{NSym}}
\newcommand{\Hom}{\mathrm{Hom}}
\newcommand{\proj}{\mathrm{proj}}
\newcommand{\stat}{\mathrm{stat}}
\newcommand{\Par}{\mathrm{Par}}
\newcommand{\rmset}{\mathrm{set}}
\newcommand{\comp}{\mathrm{comp}}

% miscellaneous commands
\newcommand{\defn}[1]{{\color{mylightblue}{#1}}}
\newcommand{\toDo}{{\bf\color{red} TODO}}
\newcommand{\toCite}{{\bf\color{green} CITE}}
\newcommand*{\vertbar}{\rule[-1ex]{0.5pt}{2.5ex}} % for matrices with column vectors
\newcommand*{\horzbar}{\rule[.5ex]{2.5ex}{0.5pt}} % for matrices with row vectors
\newcommand{\myblue}[1]{{\color{iceberg}{#1}}}
\newcommand{\myorange}[1]{{\color{burntorange}{#1}}}
\newcommand{\mygreen}[1]{{\color{applegreen}{#1}}}
\newcommand{\myred}[1]{{\color{darkcandyapplered}{#1}}}

% ferrer's diagram
\newcommand{\fdiagram}[1]{
    \begin{tikzpicture}[scale=.7]
        \fill foreach \Z [count=\Y] in {#1}
        {foreach \X in {1,...,\Z} 
        {(\X,-\Y) circle[radius=3pt]}};
    \end{tikzpicture}
}

%
\newcommand{\tcbo}[1]{\textcolor{burntorange}{#1}}

% 
\newenvironment{nouppercase}{%
  \let\uppercase\relax%
  \renewcommand{\uppercasenonmath}[1]{}}{}

% titlepage
\title{Θ2.04: Θεωρία Αναπαραστάσεων και Συνδυαστική}
\author[Β.~Δ. Μουστακας]{Βασίλης Διονύσης Μουστάκας \\ Πανεπιστήμιο Κρήτης}
\date{19 Νοεμβρίου 2025}
% \urladdr{\href{https://sites.google.com/view/vasmous}{https://sites.google.com/view/vasmous}}

\begin{document}

\begingroup
\def\uppercasenonmath#1{} % this disables uppercase title
\let\MakeUppercase\relax % this disables uppercase authors
\maketitle
\endgroup

\setcounter{section}{10}
\setcounter{theorem}{3}
\begin{center}
    \textbf{10. Πρότυπα Young
} (Συνέχεια)
\end{center}

Ας υπολογίσουμε τον χαρακτήρα $\psi^{(2,1)}$ του προτύπου Young 
\[
\rmM^{(2,1)} = 
\CC\left[\,
    \ytableausetup{tabloids,centertableaux,smalltableaux}
    \begin{ytableau}
        1 & 2 \\
        3
    \end{ytableau} \, , \ 
    \begin{ytableau}
        1 & 3 \\
        2
    \end{ytableau}\, , \ 
    \begin{ytableau}
        2 & 3 \\
        3
    \end{ytableau}
\right]
\]
που αντιστοιχεί στην διαμέριση $\lambda = (2,1)$ του $n=3$. Έχουμε
\[
    \renewcommand{\arraystretch}{2} 
    \begin{array}{c|c|c|c|}
                        & \rmK_{111}      & \rmK_{21}       & \rmK_{3}              \\ \hline
    \text{αντιπρόσωπος} & \epsilon         & \cycle{1,2}      & \cycle{1,2,3} \\ \hline
    \begin{ytableau}
        1 & 2 \\
        3
    \end{ytableau}      & \tcbo{\begin{ytableau}
        1 & 2 \\
        3
    \end{ytableau}} & \tcbo{\begin{ytableau}
        1 & 2 \\
        3
    \end{ytableau}} & \begin{ytableau}
        2 & 3 \\
        3
    \end{ytableau}       \\ \hline 
    \begin{ytableau}
        1 & 3 \\
        2
    \end{ytableau}           & \tcbo{\begin{ytableau}
        1 & 3 \\
        2
    \end{ytableau}} & \begin{ytableau}
        2 & 3 \\
        3
    \end{ytableau}        & \begin{ytableau}
        1 & 2 \\
        3
    \end{ytableau}      \\ \hline 
    \begin{ytableau}
        2 & 3 \\
        3
    \end{ytableau}           & \tcbo{\begin{ytableau}
        2 & 3 \\
        3
    \end{ytableau}} & \begin{ytableau}
        1 & 3 \\
        2
    \end{ytableau}        & \begin{ytableau}
        1 & 3 \\
        2
    \end{ytableau}     \\ \hline 
    \psi^{(2,1)}                & 3                & 1                & 0                       
    \end{array}\ .
    \]
Τι σας θυμίζει;

\begin{proposition}
    \label{prop:young_module}
    Για κάθε $\lambda \vdash n$, 
    \begin{equation}
        \label{eq:young_module}
        \rmM^\lambda \cong_{\fS_n} V^\triv\uparrow_{\fS_{\lambda}}^{\fS_n},
    \end{equation}
    όπου $V^\triv$ είναι το τετριμμένο $\fS_\lambda$-πρότυπο. Ειδικότερα, 
    \[
    \dim(\rmM^\lambda) = \frac{n!}{\lambda_1!\lambda_2!\cdots\lambda_{\ell(\lambda)}!}.
    \]
\end{proposition}

\begin{proof}[Απόδειξη]
    Ο δεύτερος ισχυρισμός έπεται άμεσα απαριθμώντας το πλήθος των ταμπλοειδών σχήματος $\lambda$ (γιατί;). Εναλλακτικά, υπολογίζοντας τις διαστάσεις και στα δυο μέλη του ισομορφισμού \ref{eq:young_module}, έπεται ότι 
    \begin{align*}
    \dim(\rmM^\lambda) &= 
    \dim\left(
        V^\triv\uparrow_{\fS_{\lambda}}^{\fS_n}
    \right) \\
    &= \chi^\triv\uparrow_{\fS_{\lambda}}^{\fS_n}(\epsilon) \\
    &= \frac{\abs{\{\pi \in \fS_n : \pi^{-1}\epsilon\pi \in \fS_n\}}}{\abs{\fS_\lambda}} \\ 
    &= \frac{n!}{\lambda_1!\lambda_2!\cdots\lambda_{\ell(\lambda)}!},
    \end{align*}
    όπου η τρίτη ισότητα έπεται από το Πόρισμα 8.6. 

    Για τον πρώτο ισχυρισμό, αρκεί να δείξουμε ότι το $\rmM^\lambda$ είναι ισόμορφο με την αναπαράσταση συμπλόκου της $\fS_\lambda$ στην $\fS_n$ (γιατί;). Πράγματι, θεωρούμε την αντιστοιχία που στέλνει το σύμπλοκο $\epsilon\fS_\lambda$ στο \textquote{σύνηθες} ταμπλοειδές
    \[
    \ytableausetup{nosmalltableaux}
    \ytableausetup{boxsize=2em} 
    [T] \ = \ \begin{ytableau}
    1                                                                &&& 2                         &&              & \cdots &&&& \lambda_1 \\
    \scriptstyle \lambda_1 + 1                                       &&& \scriptstyle\lambda_1 + 2 &&  & \cdots &&& \scriptstyle \lambda_1 + \lambda_2 \\
    \vdots                                                           &&& \vdots && &  &  \\
     &\scriptstyle \lambda_1 + \cdots + \lambda_{\ell(\lambda)-1} + 1&               &&\scriptstyle \lambda_1 + \cdots + \lambda_{\ell(\lambda)-1} + 2  &             &
    \cdots &
    n
    \end{ytableau} 
    \]
    και κατά συνέπεια το σύμπλοκο $\pi\fS_n$ στο ταμπλοειδές $\pi[T]$. Η αντιστοιχεία αυτή είναι καλά ορισμένη. Πράγματι, αν $\pi\fS_\lambda = \sigma\fS_\lambda$, τότε $\pi = \sigma\tau$ για κάποια $\tau \in \fS_\lambda$ ή ισοδύναμα για κάποια $\tau \in \rmR(T)$. Όμως, 
    \[
    \pi = \left(\sigma\tau\sigma^{-1}\right) \sigma
    \]
    όπου $\sigma\tau\sigma^{-1} \in \rmR(\sigma T)$. Συνεπώς, $\pi = w\sigma$, για κάποια $w \in \rmR(\sigma T)$ και γι αυτό $\pi[T] = \sigma[T]$, όπως θέλαμε. Επεκτείνοντας γραμμικά προκύπτει ο ζητούμενος $\fS_n$-ισομορφισμός (γιατί;).
\end{proof}

\newpage
\setcounter{section}{11}
\setcounter{theorem}{0}
\begin{center}
    \textbf{11. Πρότυπα Specht
}
\end{center}

Όπως είδαμε στο τέλος της προηγούμενης παραγράφου τα πρότυπα Young δεν είναι ανάγωγα για κάθε διαμέριση $\lambda$. Για την ακρίβεια, αν $\lambda = (n)$, τότε 
\[
\ytableausetup{nosmalltableaux,boxsize=normal}
\rmM^{(n)} = \CC\left[\ \begin{ytableau}
        1 & 2 & \cdots & n 
    \end{ytableau} \ \right]
\]
είναι το τετριμμένο $\fS_n$-πρότυπο. Αυτή είναι και η μόνη περίπτωση όπου το $\rmM^\lambda$ είναι ανάγωγο. 

Αν $\lambda = (1^n)$ είναι η διαμέριση με $n$ μέρη ίσα με 1, τότε υπάρχουν $n!$ το πλήθος ταμπλοειδή σχήματος $(1^n)$ και κάθε τέτοιο αντιστοιχεί σε μια μετάθεση της $\fS_n$ ως εξής 
\[
\begin{ytableau}
    \pi_1 \\
    \pi_2 \\
    \vdots \\
    \pi_n 
\end{ytableau} \ \mapsto \ \pi_1\pi_2\cdots\pi_n.
\]
Συνεπώς, το $\rmM^{(1^n)}$ δεν είναι άλλο από το πρότυπο της κανονικής αναπαράστασης της $\fS_n$ (γιατί;).

Τέλος, αν $\lambda = (n-1,1)$, τότε υπάρχουν $n$ το πλήθος ταμπλοειδή σχήματος $(n-1,1)$ και κάθε ένα από αυτά αντιστοιχεί σε ένα στοιχείο του $[n]$ ως εξής 
\[
\begin{ytableau}
    \ast & \ast & \cdots & \ast\\ 
    i
\end{ytableau} \ \mapsto \ i.
\]
Συνεπώς, το $\rmM^{(n-1,1)}$ δεν είναι άλλο από το πρότυπο της αναπαράστασης καθορισμού της $\fS_n$ (γιατί;).

Επιστρέφοντας στον στόχο μας, δηλαδή να κατασκευάσουμε ένα ανάγωγο $\fS_n$-πρότυπο $\sS^\lambda$ για κάθε $\lambda \vdash n$, συνδυάζοντας τους παραπάνω υπολογισμούς με αυτά που έχουμε δει σε προηγούμενες παραγράφους βρίσκουμε
%
\begin{align*}
    \rmM^{(3)}   &\cong \sS^{(3)} \\ 
    \rmM^{(2,1)} &\cong \sS^{(3)} \oplus \sS^{(2,1)} \\ 
    \rmM^{(1,1,1)} &\cong \sS^{(3)} \oplus \left(\sS^{(2,1)}\right)^2 \oplus \sS^{(1,1,1)},
\end{align*}
%
όπου με $\sS^{(3)}, \sS^{(2,1)}$ και $\sS^{(1,1,1)}$ συμβολίσαμε τα πρότυπα της τετριμμένης, της συνήθους και της αναπαράστασης προσήμου, αντίστοιχα.

Η παρατήρηση αυτή γεννάει την εξής ιδέα: Να βρούμε μια ολική διάταξη\footnote{\emph{Ολική διάταξη} ονομάζεται μια μερική διάταξη $(P,\le)$ όπου για κάθε δυο στοιχεία $x, y \in P$ ισχύει ότι $x < y$ ή $y < x$.} 
\[
\lambda^{(1)} > \lambda^{(2)} > \cdots >\lambda^{(\rmp(n))}
\]
στο σύνολο $\Par(n)$ των διαμερίσων του $n$ τέτοια ώστε το $\rmM^{\lambda^{(1)}}$ να είναι ανάγωγο, έστω $\sS^{\lambda^{(1)}}$, το $\rmM^{\lambda^{(2)}}$ να αποτελείται από κάποια αντίγραφα του $\sS^{\lambda^{(1)}}$ και \emph{ακριβώς} ένα αντίγραφο ενός νέου ανάγωγου προτύπου, έστω $\sS^{\lambda^{(2)}}$ κ.ο.κ. μέχρι να εξαντλήσουμε όλες τις διαμερίσεις του $n$. Για $n=3$, η ζητούμενη ολική διάταξη είναι 
\[
(3) > (2,1) > (1,1,1).
\]

Προς αυτή την κατεύθυνση, θα εξερευνήσουμε τις συμμετρίες των Young ταμπλώ.

\begin{definition}
    \label{def:anti-symmetrizer}
    Έστω $T$ ταμπλώ σχήματος $\lambda \vdash n$. Τα στοιχεία 
    \begin{align*}
        \nabla_T^+ &\coloneqq \sum_{\pi \in \rmR(T)} \pi \\
        \nabla_T^- &\coloneqq \sum_{\pi \in \rmC(T)} \sign(\pi)\pi \\
    \end{align*}
    του $\CC[\fS_n]$ ονομάζονται \defn{συμμετρικοποιητής γραμμών} (row symmetrizer) και \defn{αντισυμμετρικοποιητής στηλών} (column antisymmetrizer) του $T$, αντίστοιχα.
\end{definition}

Για παράδειγμα, αν
\[
\ytableausetup{notabloids}
T = \
\begin{ytableau}
    3 & 1 & 4 \\
    5 & 2 
\end{ytableau}
\]
τότε 
\begin{align*}
    \rmR(T) &= \fS(\{1,3,4\})\times\fS(\{2,5\}) \\
    \rmC(T) &= \fS(\{3,5\})\times\fS(\{1,2\})\times\fS(\{4\}) \\
\end{align*}
και γι αυτό 
\begin{align*}
\nabla_T^+ &= (\epsilon + \cycle{1,3} + \cycle{3,4} + \cycle{1,3,4} + \cycle{1,4,3})(\epsilon  + \cycle{2,5}) \\
\nabla_T^- &= \epsilon -\cycle{3,5} - \cycle{1,2} + \cycle{3,5}\cycle{1,2}.
\end{align*}

\begin{definition}
    \label{def:polytabloid}
    Έστω $T$ ένα ταμπλώ σχήματος $\lambda \vdash n$. Το  
    \[
    \bfe_T \coloneqq \nabla_T^- [T] \ \in \ \rmM^\lambda
    \]
    ονομάζεται \defn{πολυταμπλοειδές} (polytabloid) και ο υπόχωρος $\sS^\lambda$ που παράγεται από όλα τα $\bfe_T$, καθώς το $T$ διατρέχει τα ταμπλώ σχήματος $\lambda$ ονομάζεται \defn{πρότυπο Specht} (Specht module).
\end{definition}

Κάθε ταμπλοειδές είναι ουσιαστικά γραμμικός συνδυασμός (με συντελεστές στο $\ZZ$) ταμπλοειδών εντός του αντίστοιχου προτύπου Young. Στο παράδειγμα, 
\begin{align*}
    \ytableausetup{smalltableaux,notabloids}
    \bfe_{\ytableaushort{314, 52}} 
    \ytableausetup{nosmalltableaux,tabloids}
    &= \left(\epsilon -\cycle{3,5} - \cycle{1,2} + \cycle{3,5}\cycle{1,2}\right)\ytableaushort{314, 52} \\
    &= \ytableaushort{314, 52} - \ytableaushort{514, 32} - \ytableaushort{324, 51} + \ytableaushort{524, 31} \\
    &= \ytableaushort{134, 25} - \ytableaushort{145, 23} - \ytableaushort{234, 15} + \ytableaushort{245, 13} \ \in \rmM^{(3,2)}.
\end{align*}
Παρατηρήστε ότι κάθε πολυταμπλοειδές εξαρτάται από το ταμπλώ $T$ και όχι από το ταμπλοειδές $[T]$ με αντιπρόσωπο $T$. Ποιό είναι, για παράδειγμα, το πολυταμπλοειδές 
\[
\ytableausetup{smalltableaux,notabloids}
\bfe_{\ytableaushort{134, 52}}\, ;
\]
Επίσης, από τον Ορισμό~\ref{def:polytabloid} δεν έπεται ότι το πρότυπο Specht είναι όντως $\fS_n$-πρότυπο. 
\end{document}