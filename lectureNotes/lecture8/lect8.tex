\documentclass[12pt,a4paper,reqno]{amsart}

% section handling
\usepackage{subfiles} 

% language
\usepackage[greek,english]{babel}
\usepackage[utf8]{inputenc}
\usepackage{alphabeta}

% change default names to greek
\addto\captionsenglish{
    \renewcommand{\contentsname}{Περιεχόμενα}
    \renewcommand{\refname}{Βιβλιογραφία}
    \renewcommand{\datename}{Ημερομηνία:}
    \renewcommand{\urladdrname}{Ιστοσελίδα}
}

% math 
\usepackage{amsmath,amsthm,amssymb,amscd}

% font
\usepackage[cal=euler]{mathalfa}
\usepackage{libertinus-type1}
% \usepackage{txfonts} % for upright greek letters
\usepackage{bm} % for bold symbols
\usepackage{bbm} % for the simply-looking bb symbols

% miscellaneous 
\usepackage{changepage} %for indenting environments
\usepackage{csquotes} % example: \textcquote{}

% drawing
\usepackage{tikz,tikz-cd}
\usetikzlibrary{shapes.misc, patterns, matrix, calc, intersections,positioning}
\usepackage{graphics,graphicx}
\usepackage{float} % provides enhanced control and customization options for floating objects such as figures and tables

% colors
\usepackage{xcolor}
\definecolor{darkcandyapplered}{rgb}{0.64, 0.0, 0.0}
\definecolor{midnightblue}{rgb}{0.1, 0.1, 0.44}
\definecolor{mylightblue}{HTML}{336699}
\definecolor{burntorange}{rgb}{0.8, 0.33, 0.0}
\definecolor{iceberg}{rgb}{0.44, 0.65, 0.82}
\definecolor{applegreen}{rgb}{0.55, 0.71, 0.0}
\definecolor{canaryyellow}{rgb}{1.0, 0.94, 0.0}

% hrefs
\usepackage{hyperref}
\usepackage[noabbrev,capitalize]{cleveref}
\hypersetup{
    pdftoolbar=true,        
    pdfmenubar=true,        
    pdffitwindow=false,     
    pdfstartview={FitH},  % fits the width of the page to the window
    pdftitle={},
    pdfauthor={},
    pdfsubject={},
    pdfkeywords={},
    pdfnewwindow=true,  % links in new window
    colorlinks=true,  % false: boxed links; true: colored links
    linkcolor=darkcandyapplered,   % color of internal links
    citecolor=midnightblue,  % color of links to bibliography
    urlcolor=cyan,  % color of external links
    linktocpage=true  % changes the links from the section body to the page number
    }

% geometry
\textwidth=16cm 
\textheight=21cm 
\hoffset=-55pt 
\footskip=25pt

% thm envs (you might need to change the path)
% In this macro I define all the theorem environments

\theoremstyle{definition}
\newtheorem{theorem}{Θεώρημα}
\newtheorem{proposition}[theorem]{Πρόταση}
\newtheorem{lemma}[theorem]{Λήμμα}
\newtheorem{corollary}[theorem]{Πόρισμα}
\newtheorem{conjecture}[theorem]{Εικασία}
\newtheorem{problem}[theorem]{Πρόβλημα}
\newtheorem*{claim}{Ισχυρισμός}
\newtheorem{observation}[theorem]{Παρατήρηση}
\newtheorem{definition}[theorem]{Ορισμός}
\newtheorem{question}[theorem]{Ερώτηση}
\newtheorem{example}[theorem]{Παράδειγμα}
\newtheorem{exercise}{Άσκηση}

\theoremstyle{remark}
\newtheorem*{remark}{Παρατήρηση}

% fixes the correct numbering of environments
\numberwithin{theorem}{section}
\numberwithin{exercise}{section}
\numberwithin{equation}{section}

% math ops (you might need to change the path)
% In this macro I define all of my math operators

% fields
\newcommand{\NN}{\mathbbmss{N}} 
\newcommand{\ZZ}{\mathbbmss{Z}} 
\newcommand{\QQ}{\mathbbmss{Q}} 
\newcommand{\RR}{\mathbbmss{R}} 
\newcommand{\CC}{\mathbbmss{C}} 
\newcommand{\KK}{\mathbbmss{K}} 
\newcommand{\FF}{\mathbbmss{F}} 

% symmetric group
\newcommand{\fS}{\mathfrak{S}}  

% calligraphic 
\newcommand{\aA}{\mathcal{A}} 
\newcommand{\bB}{\mathcal{B}}
\newcommand{\cC}{\mathcal{C}}
\newcommand{\dD}{\mathcal{D}}
\newcommand{\eE}{\mathcal{E}}
\newcommand{\fF}{\mathcal{F}}
\newcommand{\hH}{\mathcal{H}}
\newcommand{\iI}{\mathcal{I}}
\newcommand{\lL}{\mathcal{L}}
\newcommand{\oO}{\mathcal{O}}
\newcommand{\pP}{\mathcal{P}}
\newcommand{\sS}{\mathcal{S}}
\newcommand{\mM}{\mathcal{M}}
\newcommand{\uU}{\mathcal{U}}

% bold
\newcommand{\bfa}{\mathbf{a}}
\newcommand{\bfe}{\mathbf{e}}
\newcommand{\bfF}{\pmb{F}}
\newcommand{\bfR}{\pmb{R}}
\newcommand{\bfv}{\mathbf{v}}
%\newcommand{\bfx}{\bm{x}}
%\newcommand{\bfx}{\mathbf{x}} 
\newcommand{\bfx}{\pmb{x}}
\newcommand{\bfX}{\pmb{X}}
\newcommand{\bfy}{\pmb{y}}
\newcommand{\bfz}{\pmb{z}}

% roman
\newcommand{\rmB}{\mathrm{B}}
\newcommand{\rmC}{\mathrm{C}}
\newcommand{\rmD}{\mathrm{D}} 
\newcommand{\rmI}{\mathrm{I}} 
\newcommand{\rmK}{\mathrm{K}}
\newcommand{\rmM}{\mathrm{M}}
\newcommand{\rmP}{\mathrm{P}}  
\newcommand{\rmQ}{\mathrm{Q}}  
\newcommand{\rmR}{\mathrm{R}}
\newcommand{\rmS}{\mathrm{S}}
\newcommand{\rmT}{\mathrm{T}}
\newcommand{\rmU}{\mathrm{U}}
\newcommand{\rmV}{\mathrm{V}}
\newcommand{\rmY}{\mathrm{Y}}
\newcommand{\rmZ}{\mathrm{Z}}

% greek letters
% I'm renewing some commands in order to appear in upright font
% If I want to change it later, I don't have to do it manually, I just change it from here.
% \newcommand{\uaa}{\alphaup}
% \renewcommand{\alpha}{\alphaup}
% \renewcommand{\beta}{\betaup}
% \renewcommand{\gamma}{\gammaup}
% \renewcommand{\delta}{\deltaup}
% \renewcommand{\epsilon}{\epsilonup}
% \newcommand{\ee}{\epsilon}
% \renewcommand{\varepsilon}{\varepsilonup}
% \renewcommand{\theta}{\thetaup}
% \renewcommand{\lambda}{\lambdaup}
% \newcommand{\ull}{\lambda}
% \renewcommand{\mu}{\muup}
% \renewcommand{\nu}{\nuup}
% \renewcommand{\pi}{\piup}
% \renewcommand{\rho}{\rhoup}
% \renewcommand{\varrho}{\varrhoup}
% \renewcommand{\sigma}{\sigmaup}
% \renewcommand{\tau}{\tauup} 
% \renewcommand{\phi}{\phiup}
% \renewcommand{\chi}{\chiup}
% \renewcommand{\psi}{\psiup}
% \renewcommand{\omega}{\omegaup}

% arrows and symbols 
\renewcommand{\to}{\rightarrow}
\newcommand{\toto}{\longrightarrow}
\newcommand{\mapstoto}{\longmapsto}
\newcommand{\then}{\Rightarrow}
\newcommand{\IFF}{\Leftrightarrow}
\newcommand{\tl}{\tilde}
\newcommand{\wtl}{\widetilde}
\newcommand{\ol}{\overline}
\newcommand{\ul}{\underline}
\newcommand{\oldemptyset}{\emptyset}
\renewcommand{\emptyset}{\varnothing}
\DeclareMathSymbol{\Arg}{\mathbin}{AMSa}{"39} % for arguments 
\newcommand{\onto}{\ensuremath{\twoheadrightarrow}}

% absolute value symbol
\usepackage{mathtools} 
\DeclarePairedDelimiter\abs{\lvert}{\rvert}%
\DeclarePairedDelimiter\norm{\lVert}{\rVert}%
\makeatletter
\let\oldabs\abs
\def\abs{\@ifstar{\oldabs}{\oldabs*}}

% tensor symbol
\newcommand{\tensor}[1]{%
  \mathbin{\mathop{\otimes}\limits_{#1}}%
}

% permutation cycle notation
\ExplSyntaxOn
\NewDocumentCommand{\cycle}{ O{\;} m }
 {
  (
  \alec_cycle:nn { #1 } { #2 }
  )
 }

\seq_new:N \l_alec_cycle_seq
\cs_new_protected:Npn \alec_cycle:nn #1 #2
 {
  \seq_set_split:Nnn \l_alec_cycle_seq { , } { #2 }
  \seq_use:Nn \l_alec_cycle_seq { #1 }
 }
\ExplSyntaxOff

% setminus symbol
\newcommand{\mysetminusD}{\hbox{\tikz{\draw[line width=0.6pt,line cap=round] (3pt,0) -- (0,6pt);}}}
\newcommand{\mysetminusT}{\mysetminusD}
\newcommand{\mysetminusS}{\hbox{\tikz{\draw[line width=0.45pt,line cap=round] (2pt,0) -- (0,4pt);}}}
\newcommand{\mysetminusSS}{\hbox{\tikz{\draw[line width=0.4pt,line cap=round] (1.5pt,0) -- (0,3pt);}}}
\newcommand{\sm}{\mathbin{\mathchoice{\mysetminusD}{\mysetminusT}{\mysetminusS}{\mysetminusSS}}}

% custom math operators
\newcommand{\Des}{\mathrm{Des}} 
\newcommand{\des}{\mathrm{des}} 
\newcommand{\Asc}{\mathrm{Asc}}
\newcommand{\asc}{\mathrm{asc}} 
\newcommand{\inv}{\mathrm{inv}}
\newcommand{\Inv}{\mathrm{Inv}}
\newcommand{\maj}{\mathrm{maj}} 
\newcommand{\comaj}{\mathrm{comaj}} 
\newcommand{\fix}{\mathrm{fix}} 
\newcommand{\Sym}{\mathrm{Sym}} 
\newcommand{\QSym}{\mathrm{QSym}}
\newcommand{\FQSym}{\mathrm{FQSym}} 
\newcommand{\End}{\mathrm{End}} 
\newcommand{\Rad}{\mathrm{Rad}} 
\newcommand{\rmMat}{\mathrm{Mat}} 
\newcommand{\rmdim}{\mathrm{dim}} 
\newcommand{\rmTop}{\mathrm{Top}} 
\newcommand{\rmCF}{\mathrm{CF}} 
\newcommand{\rmId}{\mathrm{Id}}
\newcommand{\rmid}{\mathrm{id}}
\newcommand{\rmtw}{\mathrm{tw}}
\newcommand{\trace}{\mathrm{tr}}
\newcommand{\Irr}{\mathrm{Irr}}
\newcommand{\Ind}{\mathrm{Ind}} % induction
\newcommand{\Res}{\mathrm{Res}} % restriction
\newcommand{\triv}{\mathrm{triv}} % trivial rep
\newcommand{\rmdef}{\mathrm{def}} % defining rep
\newcommand{\dom}{\triangleleft}
\newcommand{\domeq}{\trianglelefteq}
\newcommand{\lex}{\mathrm{lex}}
\newcommand{\sign}{\mathrm{sign}}
\newcommand{\SYT}{\mathrm{SYT}}
\renewcommand{\Im}{\mathrm{Im}}
\newcommand{\Ker}{\mathrm{Ker}}
\newcommand{\GL}{\mathrm{GL}}
\newcommand{\FL}{\mathrm{FL}}
\newcommand{\Span}{\mathrm{span}}
\newcommand{\pos}{\mathrm{pos}}
\newcommand{\Comp}{\mathrm{Comp}}
\newcommand{\Set}{\mathrm{Set}}
\newcommand{\std}{\mathrm{std}}
\newcommand{\cont}{\mathrm{cont}} %content of a SSYT
\newcommand{\SSYT}{\mathrm{SSYT}}
\newcommand{\rmz}{\mathrm{z}}
\newcommand{\ct}{\mathrm{ct}} % cycle type
\newcommand{\ch}{\mathrm{ch}} % Frobenius characteristic map
\newcommand{\height}{\mathrm{ht}}
\newcommand{\FPS}{\CC[\![\bfx]\!]} % formal power series
\newcommand{\FPSS}{\CC[\![\bfx,\bfy]\!]}
\newcommand{\reg}{\mathrm{reg}}
\newcommand{\hook}{\mathrm{h}}
\newcommand{\weight}{\mathrm{wt}}
\newcommand{\co}{\mathrm{co}}
\newcommand{\ps}{\mathrm{ps}}
\newcommand{\rmsum}{\mathrm{sum}}
\newcommand{\NSym}{\mathrm{NSym}}
\newcommand{\Hom}{\mathrm{Hom}}
\newcommand{\proj}{\mathrm{proj}}
\newcommand{\stat}{\mathrm{stat}}

% miscellaneous commands
\newcommand{\defn}[1]{{\color{mylightblue}{#1}}}
\newcommand{\toDo}{{\bf\color{red} TODO}}
\newcommand{\toCite}{{\bf\color{green} CITE}}

% 
\newenvironment{nouppercase}{%
  \let\uppercase\relax%
  \renewcommand{\uppercasenonmath}[1]{}}{}

% titlepage
\title{Θ2.04: Θεωρία Αναπαραστάσεων και Συνδυαστική}
\author[Β.~Δ. Μουστακας]{Βασίλης Διονύσης Μουστάκας \\ Πανεπιστήμιο Κρήτης}
\date{23 Οκτωβρίου 2025}
% \urladdr{\href{https://sites.google.com/view/vasmous}{https://sites.google.com/view/vasmous}}

\begin{document}

\begingroup
\def\uppercasenonmath#1{} % this disables uppercase title
\let\MakeUppercase\relax % this disables uppercase authors
\maketitle
\endgroup

\setcounter{section}{6}
\setcounter{theorem}{6}
\begin{center}
    \textbf{6. Χαρακτήρες ομάδων: Κατασκευές προτύπων
} (Συνέχεια)
\end{center}

\begin{proof}[Απόδειξη Πρότασης 6.6]
    Έστω $(\rho, V)$ και $(\sigma, W)$ οι αντίστοιχες αναπαραστάσεις της $G$ και έστω $\{v_1, v_2, \dots, v_n\}$, $\{w_1, w_2, \dots, w_m\}$ βάσεις των $V$ και $W$, αντίστοιχα. Για την Ταυτότητα (6.2), αρκεί να παρατηρήσουμε ότι ο πίνακας της δράσης του $g$ στο $V \oplus W$ έχει την μορφή
    \[
    \begin{pmatrix}
        \rho(g) & 0 \\
        0       & \sigma(g)
    \end{pmatrix}
    \]
    (γιατί;). 
    
    Για την Ταυτότητα~(6.3), από το Θεώρημα~6.5 το σύνολο $\{v_i \otimes w_j : 1 \le i \le n, \, 1 \le j \le m\}$ αποτελεί βάση του $V \otimes W$. Πρέπει να υπολογίσουμε τον πίνακα του $\rho(g)$ ως προς αυτή τη βάση. Αν 
    \[
    \rho(g) = (\alpha_{ij})_{1 \le i, j \le n} \quad \text{και} \quad \sigma(g) = (\beta_{ij})_{1 \le i,j \le m}
    \]
    τότε 
    \begin{align*}
        g \cdot (v_i \otimes w_j)
        &= \rho(g)(v_i) \otimes \sigma(g)(w_j) \\
        &= \left(
            \alpha_{1i}v_1 + \cdots + \alpha_{ni}v_n 
        \right) \otimes 
        \left(
            \beta_{1j}w_1 + \cdots + \beta_{mj}w_m
        \right) \\ 
        &= \sum_{\substack{1 \le k \le n \\ 1 \le \ell \le m}} \alpha_{ki}\beta_{\ell{j}} \, v_k \otimes w_\ell,
    \end{align*}
    για κάθε $ 1 \le i \le n$ και $1 \le j \le m$. Με άλλα λόγια, ο πίνακας της δράσης του $g$ στο $V\otimes{W}$ είναι το γινόμενο Kronecker των $\rho(g)$ και $\sigma(g)$ και γι' αυτό
    \begin{align*}
        \chi^{V\otimes{W}}(g) 
        &= \sum_{i=1}^n \left(\alpha_{ii}\left(\sum_{j=1}^m\beta_{jj}\right)\right) \\
        &= \left(\sum_{i=1}^n\alpha_{ii}\right)\left(\sum_{j=1}^m\beta_{jj}\right) \\
        &= \trace(\rho(g))\trace(\sigma(g)) \\
        &= \chi^V(g)\chi^W(g),
    \end{align*}
    και το ζητούμενο έπεται.    

    Για την Ταυτότητα~(6.4), πρέπει να υπολογίσουμε τον πίνακα του $\rho(g)$ ως προς αυτή τη δυϊκή βάση $\{v_1^\ast, v_2^\ast, \dots, v_n^\ast\}$. Υποθέτουμε ότι  
    \[
    \rho(g^{-1}) = 
    \begin{pmatrix}
        \gamma_{11} & \gamma_{12} & \cdots & \gamma_{1n} \\
        \gamma_{21} & \gamma_{22} & \cdots & \gamma_{2n} \\
        \vdots      & \vdots      &        & \vdots \\
        \gamma_{n1} & \gamma_{n2} & \cdots & \gamma_{nn} \\
    \end{pmatrix}
    \]
    για $g \in G$ και γι αυτό
    \[
    \rho(g^{-1})(v_i) = \gamma_{1i}v_1 + \gamma_{2i}v_2 + \cdots + \gamma_{ni}v_n.
    \]
    Συνεπώς,
    \[
    \left(g \cdot v_j^\ast\right)(v_i) 
    = v_j^\ast\left(\rho(g^{-1})(v_i)\right) 
    = v_j^\ast(\gamma_{1i}v_1 + \gamma_{2i}v_2 + \cdots + \gamma_{ni}v_n) 
    = \gamma_{ji}
    \]
    για κάθε $1 \le i, j \le n$. Με άλλα λόγια ο πίνακας της δράσης του $g$ στο $V^\ast$ είναι ο ανάστροφος του $\rho(g^{-1})$ και γι αυτό 
    \[
    \chi^{V^\ast}(g) = \trace\left(\rho(g^{-1})^\top\right) = \chi^V(g^{-1}) = \ol{\chi^V(g)},
    \]
    όπου η τελευταία ισότητα έπεται από την Άσκηση~2.2~(2).

    Τέλος, η Ταυτότητα~(6.5) έπεται από τις Ταυτότητες~(6.3) και (6.4), σε συνδυασμό με το Θεώρημα~6.5 και την Πρόταση~5.2~(3). 
\end{proof}

Οι πρώτες δυο ταυτότητες της Πρότασης~6.6 μας πληροφορούν ότι το άθροισμα και το γινόμενο δυο χαρακτήρων είναι και αυτοί χαρακτήρες κάποιων αναπαραστάσεων, γεγονός το οποίο δεν είναι καθόλου προφανές εκ των προτέρων.
\begin{example}
    Από την συζήτηση της Πραγράφου~2, έπεται ότι η αναπαράσταση καθορισμού $V^{\rmdef}$ της $\fS_n$ διασπάται ως 
    \begin{align}
    V^{\rmdef} &= \CC[\bfe_1 + \cdots + \bfe_n] \oplus \left(\CC[\bfe_1 + \cdots + \bfe_n]\right)^\perp \notag \\
    &= \underbrace{\CC[\bfe_1 + \cdots + \bfe_n]}_{\cong \, V^{\triv}} \oplus \underbrace{\CC[\bfe_1-\bfe_2, \bfe_1 - \bfe_3, \dots, \bfe_1 - \bfe_n]}_{\coloneqq \, V^{\std}}. \label{eq:def_isotypic}
    \end{align}
    Η αναπαράσταση $V^\std$ ονομάζεται \defn{συνήθης} (standard) αναπαράσταση της $\fS_n$ και έχει διάσταση $n-1$. Από την Πρόταση~6.6~(1), ο χαρακτήρας της είναι 
    \[
    \chi^\std(\pi) = \fix(\pi) - 1,
    \]
    για κάθε $\pi \in \fS_n$.

    H αναπαράσταση αυτή είναι \emph{ανάγωγη} και κατά συνέπεια, η Διάσπαση~\eqref{eq:def_isotypic} είναι η ισοτυπική διάσπαση της αναπαράστασης καθορισμού. Έστω $v = (v_1, v_2, \dots, v_n) \in V^\std$ τέτοιο ώστε $v \neq 0$. Θα δείξουμε ότι ο υπόχωρος που παράγεται από την τροχιά του $v$ περιέχει τα στοιχεία $\bfe_1 - \bfe_2, \bfe_1 - \bfe_3, \dots, \bfe_1-\bfe_n$ και κατά συνέπεια ολόκληρο τον $V^\std$. Με άλλα λόγια, ο $V^\std$ δεν περιέχει γνήσιους $\fS_n$-αναλλοίωτους υπόχωρους διαφορετικούς από το $\{0\}$. 

    Πράγματι, αφού $v \neq 0$ έπεται ότι  $v_i \neq v_{i+1}$ για κάποιο $1 \le i \le n-1$ (γιατί;). Αν $s_j \coloneqq \cycle{j,j+1}$ είναι η αντιμετάθεση που εναλλάσει τα $j$ και $j+1$ και αφήνει όλα τα άλλα σταθερά για κάθε $1 \le j \le n-1$, τότε 
    \[ 
    s_i\cdot{v} = (v_1, \dots, v_{i+1}, v_i, \dots, v_n)
    \]
    (γιατί;) και γι' αυτό 
    \begin{align*}
    v - s_i\cdot{v} 
    &= (0, \dots, \underbrace{v_i - v_{i+1}}_{\text{$i$-οστή θέση}}, \underbrace{v_{i+1} - v_i}_{\text{$(i+1)$-οστή θέση}}, \dots, 0) \\ 
    &= \underbrace{(v_i - v_{i+1})}_{\neq \, 0}(\bfe_i - \bfe_{i+1}).
    \end{align*}
    Άρα, $\bfe_i - \bfe_{i+1} \in \CC[\oO_v]$ και κατά συνέπεια 
    \[
    \bfe_1 - \bfe_{i+1} = (s_1s_2\cdots{s_{i-1}})\cdot{\bfe_i - \bfe_{i+1}} \in \oO_v
    \]
    (γιατί;) και 
    \[
    \bfe_1 - \bfe_k = 
    \begin{cases}
        (s_ks_{k+1}\cdots s_i)(\bfe_1-\bfe_{i+1}), &\ \text{αν $2\le k \le i$} \\ 
        (s_ks_{k+1}\cdots s_{i+1})(\bfe_1-\bfe_{i+1}), &\ \text{αν $i+1 \le k \le n$} \\ 
    \end{cases} \quad \in \oO_v,
    \]
    που ήταν το ζητούμενο. Τώρα, μπορούμε να ολοκληρώσουμε τον πίνακα χαρακτήρων της $\fS_3$
    \renewcommand{\arraystretch}{1.2} 
    \[
    \begin{array}{l|c|c|c}
                   & K_1 & K_2 & K_3 \\ \hline
        \chi^\triv & 1   & 1   & 1 \\ \hline
        \chi^\sign & 1   & -1  & 1 \\ \hline
        \chi^\std  & 2   & 0   & -1 
    \end{array} \ .
    \]
\end{example}

\vspace{1cm}
\setcounter{section}{7}
\setcounter{theorem}{0}
\begin{center}
    \textbf{7. Χαρακτήρες ομάδων: Σχέσεις ορθογωνιότητας
}
\end{center}

Στην παράγραφο αυτή θα χρησιμοποιήσουμε τα αποτελέσματα των προηγούμενων παραγράφων για να απαντήσουμε στα δυο από τα τρία ερωτήματα που τέθηκαν στην Παράγραφο~4. Σε ότι ακολουθεί υποθέτουμε ότι $G$ είναι μια πεπερασμένη ομάδα και όλοι οι διανυσματικοί χώροι είναι πεπερασμένης διάστασης πάνω από το $\CC$.

Ας θυμιθούμε τον πίνακα χαρακτήρων της $\rmC_3$
\renewcommand{\arraystretch}{1.2}
\[
\begin{array}{l|c|c|c}
           & \{\epsilon\} & \{g\}   & \{g^2\} \\ \hline
    \chi_1 & 1            & 1       & 1 \\ \hline
    \chi_2 & 1            & \zeta   & \zeta^2 \\ \hline
    \chi_3 & 1            & \zeta^2 & \zeta
\end{array}\ ,
\]
όπου $\zeta$ είναι μια τρίτη ρίζα της μονάδας. Αν φανταστούμε τους ανάγωγους χαρακτήρες της $\rmC_3$ ως διανύσματα του $\CC^3$, δηλαδή
\[
\chi_1 = (1,1,1), \chi_2=(1,\zeta,\zeta^2), \ \text{και} \ \chi_3 = (1,\zeta^2,\zeta),
\]
και υπολογίσουμε το εσωτερικό τους γινόμενο, τότε παρατηρούμε ότι 
\begin{align*}
\chi_1\cdot\chi_2 &= \chi_1\cdot\chi_3 = \chi_2\cdot\chi_3 = 0 \\
\chi_1\cdot\chi_1 &= \chi_2\cdot\chi_2 = \chi_3\cdot\chi_3 = 3. 
\end{align*}
Με άλλα λόγια, το σύνολο $\{\chi_1, \chi_2, \chi_3\}$ είναι ορθογώνιο ως προς το σύνηθες εσωτερικό γινόμενο του $\CC^3$. Η παρατήρηση αυτή μας οδηγεί στον να ορίσουμε το εξής.
\begin{definition}
    \label{def:character_inner_product}
    Για δυο συναρτήσεις $\alpha, \beta : G \to \CC$ (όχι απαραίτητα χαρακτήρες της $G$), θέτουμε 
    \begin{equation}
    \label{eq:character_inner_product_def}
    (\alpha, \beta)_G \coloneqq  \frac{1}{\abs{G}} \sum_{g \in G} \alpha(g)\ol{\beta(g)},
    \end{equation}
    ή απλούστερα $(\alpha, \beta)$.
\end{definition}
Η απεικόνιση $(\Arg, \Arg)$ είναι εσωτερικό γινόμενο στον χώρο των συναρτήσεων κλάσης $\rmCF(G)$ (γιατί;). Από την Άσκηση~2.2~(2), αν $\chi$ και $\psi$ είναι δυο χαρακτήρες της $G$, τότε 
\begin{equation}
    \label{eq:character_bilinear}
    (\chi, \psi)_G = \frac{1}{\abs{G}} \sum_{g \in G} \chi(g)\psi(g^{-1}).
\end{equation}
Αν ορίζαμε το $(\Arg,\Arg)$ από το δεξί μέλος της Ταυτότητας~\eqref{eq:character_bilinear}, αντί γι αυτό της Ταυτότητας~\eqref{eq:character_inner_product_def}, τότε θα είχαμε μια διγραμμική απεικόνιση $\rmCF(G) \times \rmCF(G) \to \CC$, αντί για εσωτερικό γινόμενο. Παρόλα αυτά, στους χαρακτήρες της $G$ οι δυο τύποι συμφωνούν.

\begin{theorem}
    \label{thm:inner_prod_of_characters}
    Αν $V, W$ είναι δυο $G$-πρότυπα, τότε 
    \begin{equation}
        \label{eq:inner_prod_of_characters}
        (\chi^V,\chi^W) = \dim\left(\Hom_G(W,V)\right) = \dim\left(\Hom_G(V,W)\right).
    \end{equation}
\end{theorem}

Η Ταυτότητα~\eqref{eq:inner_prod_of_characters} εξηγεί την συμμετρία που παρατηρήσαμε μετά το Λήμμα~3.6. Η απόδειξη του Θεωρήματος~\ref{thm:inner_prod_of_characters} βασίζεται στο παρακάτω αποτέλεσμα, την βασική ιδέα του οποία συναντήσαμε στην απόδειξη της Πρότασης 2.3.

\begin{lemma}
    \label{lem:projection}
    Έστω $(\rho,V)$ μια αναπαράσταση της $G$ και 
    \[
    V^G \coloneqq \{v \in V : \rho(g)(v) = v, \, \text{για κάθε $g \in G$}\}.
    \]
    \begin{itemize}
        \item[(1)] Το $V^G$ είναι υποαναπαράσταση της $V$.
        \item[(2)] Η απεικόνιση $\tl{\rho} : V \to V$ που ορίζεται θέτοντας 
        \[
        \tl{\rho}(v) = \frac{1}{\abs{G}} \sum_{g \in G} \rho(g)(v)
        \]
        για κάθε $v \in V$ είναι $G$-ομομορφισμός.
        \item[(3)] Ισχύει ότι $\mathrm{Im}(\tl{\rho}) = V^G$.
        \item[(4)] Ισχύει ότι 
        \[
        \dim(V^G) = \frac{1}{\abs{G}}\sum_{g \in G} \chi^{\rho,V}(g).
        \]
    \end{itemize}
\end{lemma}

\begin{proof}
    Για το (1), αν $x \in G$ και $v \in V^G$, τότε για κάθε $g \in G$
    \[
    \rho(g)\left(\rho(x)(v)\right) = \rho(gx)(v) = v,
    \]
    όπου η πρώτη ισότητα έπεται από το ότι η $\rho$ είναι ομομορφισμός ομάδων και η δεύτερη ισότητα από το ότι $v \in V^G$. Συνεπώς, το $V^G$ είναι $G$-αναλλοίωτο.

    Για το (2), η γραμμικότητα της $\tl{\rho}$ έπεται άμεσα από την γραμμικότητα της $\rho$. Για να δείξουμε ότι είναι $G$-ομομορφισμός, πρέπει να δείξουμε ότι 
    \[
    \tl{\rho}\left(\rho(x)(v)\right) = \rho(x)\left(\tl{\rho}(v)\right) 
    \]
    ή ισοδύναμα ότι 
    \[
    \rho(x^{-1})\left(\tl{\rho}\left(\rho(x)(v)\right)\right) = \tl{\rho}(v)
    \]
    για κάθε $x\in G$ και $v \in V$. Πράγματι, 
    \begin{align*}
        \rho(x^{-1})\left(\tl{\rho}\left(\rho(x)(v)\right)\right) 
        &= \rho(x^{-1})\left(
            \frac{1}{\abs{G}} \sum_{g \in G} \rho(g)\left(\rho(x)(v)\right)
        \right) \\ 
        &= \frac{1}{\abs{G}} \sum_{g \in G} \rho(x^{-1})\left(\rho(g)\left(\rho(x)(v)\right)\right) \\
        &= \frac{1}{\abs{G}}\sum_{g \in G} \rho(x^{-1}gx)(v) \\
        &= \frac{1}{\abs{G}}\sum_{h \in G} \rho(h)(v) \\
        &= \tl{\rho}(v),
    \end{align*}
    όπου η δεύτερη (αντ. τρίτη) ισότητα έπεται από το ότι η $\rho$ είναι γραμμική απεικόνιση (αντ. ομομορφισμός ομάδων) και η τέταρτη από αλλαγή μεταβλητής $h \to x^{-1}gx$ στο άθροισμα.

    Για το (3), αν $v \in \mathrm{Im}(\tl{\rho})$, τότε $v = \tl{\rho}(v_0)$, για κάποιο $v_0 \in V$ και γι αυτό 
    \begin{align*}
        \rho(x)(\tl{\rho}(v_0)) 
        &= \rho(x)\left(\frac{1}{\abs{G}} \sum_{g \in G} \rho(g)(v_0)\right) \\
        &= \frac{1}{\abs{G}} \sum_{g \in G} \rho(x)\left(\rho(g)(v_0)\right) \\ 
        &= \frac{1}{\abs{G}} \sum_{g \in G} \rho(xg)(v_0) \\ 
        &= \frac{1}{\abs{G}}\sum_{h \in G} \rho(h)(v_0) \\
        &= \tl{\rho}(v_0)
    \end{align*}
    για κάθε $x \in G$, όπου η δεύτερη (αντ. τρίτη) ισότητα έπεται από το ότι η $\rho$ είναι γραμμική απεικόνιση (αντ. ομομορφισμός ομάδων) και η τέταρτη από αλλαγή μεταβλητής $h \to xg$ στο άθροισμα. Άρα, $v \in V^G$ και γι' αυτό $\mathrm{Im}(\tl{\rho}) \subseteq V^G$. Για τον άλλο εγκλεισμό, αν $v \in V^G$, τότε
    \[
    \tl{\rho}(v) = \frac{1}{\abs{G}}\sum_{g \in G} \rho(g)(v) = \frac{1}{\abs{G}} \sum_{g \in G} v = v
    \]
    που σημαίνει ότι $v \in \mathrm{Im}(\tl{\rho})$ και γι αυτό $V^G \subseteq \mathrm{Im}(\tl{\rho})$.

    Τέλος, για το (4), παρατηρούμε ότι η $\tl{\rho}$ είναι \emph{προβολή}, δηλαδή $\tl{\rho}\circ\tl{\rho} = \tl{\rho}$. Πράγματι,
    \begin{align*}
        \tl{\rho}\left(\tl{\rho}(v)\right)
        &= \tl{\rho}\left(\frac{1}{\abs{G}} \sum_{g \in G} \rho(g)(v)\right) \\
        &= \frac{1}{\abs{G}} \sum_{g \in G} \tl{\rho}\left(\rho(g)(v)\right) \\
        &= \frac{1}{\abs{G}} \sum_{g \in G} \rho(g)\left(\tl{\rho}(v)\right) \\
        &=\tl{\rho}(v),
    \end{align*}
    όπου η δεύτερη ισότητα έπεται από την γραμμικότητα της $\tl{\rho}$ και η τρίτη (αντ. τέταρτη) ισότητα έπεται από το (2) (αντ. (3)). Αυτό έχει ως συνέπεια\footnote{Γενικότερα, αν έχουμε μια προβολή $p : V\to{V}$ σε ένα διανυσματικό χώρο $V$, τότε $V = \mathrm{Im}(p) \oplus \Ker(p)$.},
    \[
    V = \mathrm{Im}(\tl{\rho}) \oplus \Ker(\tl{\rho}),
    \]
    και γι αυτό ο πίνακας του $(\tl{\rho})$ έχει την εξής μορφή 
    \[
    \begin{pmatrix}
        \mathrm{I}_{\dim(\mathrm{Im}(\tl{\rho}))} & 0 \\ 
        0 & 0
    \end{pmatrix},
    \]
    όπου $\mathrm{I}_k$ είναι ο ταυτοτικός $(k\times{k})$-πίνακας. 
    Υπολογίζοντας το ίχνος του έπεται ότι 
    \[
    \dim(V^G) = \dim(\mathrm{Im}(\tl{\rho})) = \trace\left(\tl{\rho}\right) = \frac{1}{\abs{G}} \sum_{g \in G} \trace\left(\rho(g)\right) = \frac{1}{\abs{G}}\sum_{g \in G} \chi^{\rho,V}(g),
    \]
    όπου η πρώτη ισότητα έπεται από το (3), και η απόδειξη ολοκληρώθηκε.
\end{proof}

\begin{proof}[Απόδειξη του Θεωρήματος~\ref{thm:inner_prod_of_characters}]
    Για την πρώτη ισότητα της Ταυτότητας~\ref{eq:inner_prod_of_characters},
    \begin{align*}
        (\chi^V,\chi^W) 
        &= \frac{1}{\abs{G}}\sum_{g\in{G}} \chi^V(g)\ol{\chi^W(g)} \\ 
        &= \frac{1}{\abs{G}}\sum_{g\in{G}}\chi^{\Hom(W,V)}(g) \\ 
        &= \dim\left(\Hom(W,V)^G\right) \\
        &= \dim\left(\Hom_G(W,V)\right),
    \end{align*}
    όπου η δεύτερη ισότητα έπεται από την Ταυτότητα~(6.5), η τρίτη ισότητα έπεται από το Λήμμα~\ref{lem:projection} και η τέταρτη ισότητα από την Άσκηση~1.4~(2).

    Για την δεύτερη ισότητα της Ταυτότητας~\ref{eq:inner_prod_of_characters},
    \[
    (\chi^V,\chi^W) = \ol{(\chi^V,\chi^W)} = (\chi^W,\chi^V) = \dim\left(\Hom_G(V,W)\right),
    \]
    όπου η πρώτη ισότητα έπεται από την συζυγή συμμετρία του εσωτερικού γινομένου.
\end{proof}

Συνδυάζοντας το Θεώρημα~\ref{thm:inner_prod_of_characters} και το Πόρισμα~3.3 προκύπτει ότι το σύνολο των ανάγωγων χαρακτήρων της $G$ είναι ορθοκανονικό ως προς το εσωτερικό γινόμενο που ορίσαμε.

\begin{theorem}{\rm(Σχέσεις ορθογωνιότητας Ι)}
    \label{thm:orthogonality_rels_I}
    Αν $V$ και $W$ είναι ανάγωγα $G$-πρότυπα, τότε 
    \[
    (\chi^V,\chi^W) = 
    \begin{cases}
        1, &\ \text{αν $V\cong_G{W}$} \\
        0, &\ \text{διαφορετικά}
    \end{cases}.
    \]
\end{theorem}

\begin{corollary}
    \label{cor:orthogonality_rels_I}
    Αν $V$ είναι $G$-πρότυπο με ισοτυπική διάσπαση 
    \[
    V = V_1^{m_1} \oplus V_2^{m_2} \oplus \cdots \oplus V_n^{m_n},
    \]
    όπου $\{V_1, V_2, \dots, V_n\}$ είναι ένα σύνολο ανά δυο μη ισόμορφων $G$-προτύπων, τότε 
    \begin{itemize}
        \setlength{\itemsep}{4pt}
        \item[(1)] $\chi^V = m_1\chi^{V_1} + m_2\chi^{V_2} + \cdots + m_n\chi^{V_n}$,
        \item[(2)] $(\chi^V,\chi^{V_i}) = m_i$, για κάθε $1 \le i \le n$,
        \item[(3)] $(\chi^V,\chi^V) = m_1^2 + m_2^2 + \cdots + m_n^2$.
    \end{itemize}
    Ειδικότερα, το $V$ είναι ανάγωγο αν και μόνο αν $(\chi^V,\chi^V) = 1$.
\end{corollary}
\end{document}