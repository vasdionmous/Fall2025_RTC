\documentclass[12pt,a4paper,reqno]{amsart}

% section handling
\usepackage{subfiles} 

% language
\usepackage[greek,english]{babel}
\usepackage[utf8]{inputenc}
\usepackage{alphabeta}

% change default names to greek
\addto\captionsenglish{
    \renewcommand{\contentsname}{Περιεχόμενα}
    \renewcommand{\refname}{Βιβλιογραφία}
    \renewcommand{\datename}{Ημερομηνία:}
    \renewcommand{\urladdrname}{Ιστοσελίδα}
}

% math 
\usepackage{amsmath,amsthm,amssymb,amscd}

% font
\usepackage[cal=euler]{mathalfa}
\usepackage{libertinus-type1}
% \usepackage{txfonts} % for upright greek letters
\usepackage{bm} % for bold symbols
\usepackage{bbm} % for the simply-looking bb symbols

% miscellaneous 
\usepackage{changepage} %for indenting environments
\usepackage{csquotes} % example: \textcquote{}

% drawing
\usepackage{tikz,tikz-cd}
\usetikzlibrary{shapes.misc, patterns, matrix, calc, intersections,positioning}
\usepackage{graphics,graphicx}
\usepackage{float} % provides enhanced control and customization options for floating objects such as figures and tables

% colors
\usepackage{xcolor}
\definecolor{darkcandyapplered}{rgb}{0.64, 0.0, 0.0}
\definecolor{midnightblue}{rgb}{0.1, 0.1, 0.44}
\definecolor{mylightblue}{HTML}{336699}
\definecolor{burntorange}{rgb}{0.8, 0.33, 0.0}
\definecolor{iceberg}{rgb}{0.44, 0.65, 0.82}

% hrefs
\usepackage{hyperref}
\usepackage[noabbrev,capitalize]{cleveref}
\hypersetup{
    pdftoolbar=true,        
    pdfmenubar=true,        
    pdffitwindow=false,     
    pdfstartview={FitH},  % fits the width of the page to the window
    pdftitle={},
    pdfauthor={},
    pdfsubject={},
    pdfkeywords={},
    pdfnewwindow=true,  % links in new window
    colorlinks=true,  % false: boxed links; true: colored links
    linkcolor=darkcandyapplered,   % color of internal links
    citecolor=midnightblue,  % color of links to bibliography
    urlcolor=cyan,  % color of external links
    linktocpage=true  % changes the links from the section body to the page number
    }

% geometry
\textwidth=16cm 
\textheight=21cm 
\hoffset=-55pt 
\footskip=25pt

% thm envs (you might need to change the path)
% In this macro I define all the theorem environments

\theoremstyle{definition}
\newtheorem{theorem}{Θεώρημα}
\newtheorem{proposition}[theorem]{Πρόταση}
\newtheorem{lemma}[theorem]{Λήμμα}
\newtheorem{corollary}[theorem]{Πόρισμα}
\newtheorem{conjecture}[theorem]{Εικασία}
\newtheorem{problem}[theorem]{Πρόβλημα}
\newtheorem*{claim}{Ισχυρισμός}
\newtheorem{observation}[theorem]{Παρατήρηση}
\newtheorem{definition}[theorem]{Ορισμός}
\newtheorem{question}[theorem]{Ερώτηση}
\newtheorem{example}[theorem]{Παράδειγμα}
\newtheorem{exercise}{Άσκηση}

\theoremstyle{remark}
\newtheorem*{remark}{Παρατήρηση}

% fixes the correct numbering of environments
\numberwithin{theorem}{section}
\numberwithin{exercise}{section}
\numberwithin{equation}{section}

% math ops (you might need to change the path)
% In this macro I define all of my math operators

% fields
\newcommand{\NN}{\mathbbmss{N}} 
\newcommand{\ZZ}{\mathbbmss{Z}} 
\newcommand{\QQ}{\mathbbmss{Q}} 
\newcommand{\RR}{\mathbbmss{R}} 
\newcommand{\CC}{\mathbbmss{C}} 
\newcommand{\KK}{\mathbbmss{K}} 
\newcommand{\FF}{\mathbbmss{F}} 

% symmetric group
\newcommand{\fS}{\mathfrak{S}}  

% calligraphic 
\newcommand{\aA}{\mathcal{A}} 
\newcommand{\bB}{\mathcal{B}}
\newcommand{\cC}{\mathcal{C}}
\newcommand{\dD}{\mathcal{D}}
\newcommand{\eE}{\mathcal{E}}
\newcommand{\fF}{\mathcal{F}}
\newcommand{\hH}{\mathcal{H}}
\newcommand{\iI}{\mathcal{I}}
\newcommand{\lL}{\mathcal{L}}
\newcommand{\oO}{\mathcal{O}}
\newcommand{\pP}{\mathcal{P}}
\newcommand{\sS}{\mathcal{S}}
\newcommand{\mM}{\mathcal{M}}
\newcommand{\uU}{\mathcal{U}}

% bold
\newcommand{\bfa}{\mathbf{a}}
\newcommand{\bfe}{\mathbf{e}}
\newcommand{\bfF}{\pmb{F}}
\newcommand{\bfR}{\pmb{R}}
\newcommand{\bfv}{\mathbf{v}}
%\newcommand{\bfx}{\bm{x}}
%\newcommand{\bfx}{\mathbf{x}} 
\newcommand{\bfx}{\pmb{x}}
\newcommand{\bfX}{\pmb{X}}
\newcommand{\bfy}{\pmb{y}}
\newcommand{\bfz}{\pmb{z}}

% roman
\newcommand{\rmB}{\mathrm{B}}
\newcommand{\rmC}{\mathrm{C}}
\newcommand{\rmD}{\mathrm{D}} 
\newcommand{\rmI}{\mathrm{I}} 
\newcommand{\rmK}{\mathrm{K}}
\newcommand{\rmM}{\mathrm{M}}
\newcommand{\rmP}{\mathrm{P}}  
\newcommand{\rmQ}{\mathrm{Q}}  
\newcommand{\rmR}{\mathrm{R}}
\newcommand{\rmS}{\mathrm{S}}
\newcommand{\rmT}{\mathrm{T}}
\newcommand{\rmU}{\mathrm{U}}
\newcommand{\rmV}{\mathrm{V}}
\newcommand{\rmY}{\mathrm{Y}}
\newcommand{\rmZ}{\mathrm{Z}}

% greek letters
% I'm renewing some commands in order to appear in upright font
% If I want to change it later, I don't have to do it manually, I just change it from here.
% \newcommand{\uaa}{\alphaup}
% \renewcommand{\alpha}{\alphaup}
% \renewcommand{\beta}{\betaup}
% \renewcommand{\gamma}{\gammaup}
% \renewcommand{\delta}{\deltaup}
% \renewcommand{\epsilon}{\epsilonup}
% \newcommand{\ee}{\epsilon}
% \renewcommand{\varepsilon}{\varepsilonup}
% \renewcommand{\theta}{\thetaup}
% \renewcommand{\lambda}{\lambdaup}
% \newcommand{\ull}{\lambda}
% \renewcommand{\mu}{\muup}
% \renewcommand{\nu}{\nuup}
% \renewcommand{\pi}{\piup}
% \renewcommand{\rho}{\rhoup}
% \renewcommand{\varrho}{\varrhoup}
% \renewcommand{\sigma}{\sigmaup}
% \renewcommand{\tau}{\tauup} 
% \renewcommand{\phi}{\phiup}
% \renewcommand{\chi}{\chiup}
% \renewcommand{\psi}{\psiup}
% \renewcommand{\omega}{\omegaup}

% arrows and symbols 
\renewcommand{\to}{\rightarrow}
\newcommand{\toto}{\longrightarrow}
\newcommand{\mapstoto}{\longmapsto}
\newcommand{\then}{\Rightarrow}
\newcommand{\IFF}{\Leftrightarrow}
\newcommand{\tl}{\tilde}
\newcommand{\wtl}{\widetilde}
\newcommand{\ol}{\overline}
\newcommand{\ul}{\underline}
\newcommand{\oldemptyset}{\emptyset}
\renewcommand{\emptyset}{\varnothing}
\DeclareMathSymbol{\Arg}{\mathbin}{AMSa}{"39} % for arguments 
\newcommand{\onto}{\ensuremath{\twoheadrightarrow}}

% absolute value symbol
\usepackage{mathtools} 
\DeclarePairedDelimiter\abs{\lvert}{\rvert}%
\DeclarePairedDelimiter\norm{\lVert}{\rVert}%
\makeatletter
\let\oldabs\abs
\def\abs{\@ifstar{\oldabs}{\oldabs*}}

% tensor symbol
\newcommand{\tensor}[1]{%
  \mathbin{\mathop{\otimes}\limits_{#1}}%
}

% permutation cycle notation
\ExplSyntaxOn
\NewDocumentCommand{\cycle}{ O{\;} m }
 {
  (
  \alec_cycle:nn { #1 } { #2 }
  )
 }

\seq_new:N \l_alec_cycle_seq
\cs_new_protected:Npn \alec_cycle:nn #1 #2
 {
  \seq_set_split:Nnn \l_alec_cycle_seq { , } { #2 }
  \seq_use:Nn \l_alec_cycle_seq { #1 }
 }
\ExplSyntaxOff

% setminus symbol
\newcommand{\mysetminusD}{\hbox{\tikz{\draw[line width=0.6pt,line cap=round] (3pt,0) -- (0,6pt);}}}
\newcommand{\mysetminusT}{\mysetminusD}
\newcommand{\mysetminusS}{\hbox{\tikz{\draw[line width=0.45pt,line cap=round] (2pt,0) -- (0,4pt);}}}
\newcommand{\mysetminusSS}{\hbox{\tikz{\draw[line width=0.4pt,line cap=round] (1.5pt,0) -- (0,3pt);}}}
\newcommand{\sm}{\mathbin{\mathchoice{\mysetminusD}{\mysetminusT}{\mysetminusS}{\mysetminusSS}}}

% custom math operators
\newcommand{\Des}{\mathrm{Des}} 
\newcommand{\des}{\mathrm{des}} 
\newcommand{\Asc}{\mathrm{Asc}}
\newcommand{\asc}{\mathrm{asc}} 
\newcommand{\inv}{\mathrm{inv}}
\newcommand{\Inv}{\mathrm{Inv}}
\newcommand{\maj}{\mathrm{maj}} 
\newcommand{\comaj}{\mathrm{comaj}} 
\newcommand{\fix}{\mathrm{fix}} 
\newcommand{\Sym}{\mathrm{Sym}} 
\newcommand{\QSym}{\mathrm{QSym}}
\newcommand{\FQSym}{\mathrm{FQSym}} 
\newcommand{\End}{\mathrm{End}} 
\newcommand{\Rad}{\mathrm{Rad}} 
\newcommand{\rmMat}{\mathrm{Mat}} 
\newcommand{\rmdim}{\mathrm{dim}} 
\newcommand{\rmTop}{\mathrm{Top}} 
\newcommand{\rmCF}{\mathrm{CF}} 
\newcommand{\rmId}{\mathrm{Id}}
\newcommand{\rmid}{\mathrm{id}}
\newcommand{\rmtw}{\mathrm{tw}}
\newcommand{\trace}{\mathrm{tr}}
\newcommand{\Irr}{\mathrm{Irr}}
\newcommand{\Ind}{\mathrm{Ind}} % induction
\newcommand{\Res}{\mathrm{Res}} % restriction
\newcommand{\triv}{\mathrm{triv}} % trivial rep
\newcommand{\rmdef}{\mathrm{def}} % defining rep
\newcommand{\dom}{\triangleleft}
\newcommand{\domeq}{\trianglelefteq}
\newcommand{\lex}{\mathrm{lex}}
\newcommand{\sign}{\mathrm{sign}}
\newcommand{\SYT}{\mathrm{SYT}}
\renewcommand{\Im}{\mathrm{Im}}
\newcommand{\Ker}{\mathrm{Ker}}
\newcommand{\GL}{\mathrm{GL}}
\newcommand{\FL}{\mathrm{FL}}
\newcommand{\Span}{\mathrm{span}}
\newcommand{\pos}{\mathrm{pos}}
\newcommand{\Comp}{\mathrm{Comp}}
\newcommand{\Set}{\mathrm{Set}}
\newcommand{\std}{\mathrm{std}}
\newcommand{\cont}{\mathrm{cont}} %content of a SSYT
\newcommand{\SSYT}{\mathrm{SSYT}}
\newcommand{\rmz}{\mathrm{z}}
\newcommand{\ct}{\mathrm{ct}} % cycle type
\newcommand{\ch}{\mathrm{ch}} % Frobenius characteristic map
\newcommand{\height}{\mathrm{ht}}
\newcommand{\FPS}{\CC[\![\bfx]\!]} % formal power series
\newcommand{\FPSS}{\CC[\![\bfx,\bfy]\!]}
\newcommand{\reg}{\mathrm{reg}}
\newcommand{\hook}{\mathrm{h}}
\newcommand{\weight}{\mathrm{wt}}
\newcommand{\co}{\mathrm{co}}
\newcommand{\ps}{\mathrm{ps}}
\newcommand{\rmsum}{\mathrm{sum}}
\newcommand{\NSym}{\mathrm{NSym}}
\newcommand{\Hom}{\mathrm{Hom}}
\newcommand{\proj}{\mathrm{proj}}
\newcommand{\stat}{\mathrm{stat}}

% miscellaneous commands
\newcommand{\defn}[1]{{\color{mylightblue}{#1}}}
\newcommand{\toDo}{{\bf\color{red} TODO}}
\newcommand{\toCite}{{\bf\color{green} CITE}}

% 
\newenvironment{nouppercase}{%
  \let\uppercase\relax%
  \renewcommand{\uppercasenonmath}[1]{}}{}

% titlepage
\title{Θ2.04: Θεωρία Αναπαραστάσεων και Συνδυαστική}
\author[Β.~Δ. Μουστακας]{Βασίλης Διονύσης Μουστάκας \\ Πανεπιστήμιο Κρήτης}
\date{9 Οκτωβρίου 2025}
% \urladdr{\href{https://sites.google.com/view/vasmous}{https://sites.google.com/view/vasmous}}

\begin{document}

\begingroup
\def\uppercasenonmath#1{} % this disables uppercase title
\let\MakeUppercase\relax % this disables uppercase authors
\maketitle
\endgroup

\setcounter{section}{3}
\thispagestyle{empty}

\begin{center}
    \textbf{3. Λήμμα του Schur και εφαρμογές
}
\end{center}

Σε αυτή την παράγραφο υποθέτουμε ότι
\begin{itemize}
    \item $G$ είναι μια πεπερασμένη ομάδα,
    \item $\FF$ είναι ένα αυθαίρετο σώμα, και
    \item όλοι οι διανυσματικοί χώροι είναι πεπερασμένοι.
\end{itemize}  

\begin{definition}
    \label{def:representation_homomorphism}
    Έστω $(\rho, V)$ και $(\sigma, W)$ δυο αναπαραστάσεις της $G$. \defn{Ομομορφισμός αναπαραστάσεων} (ή \defn{$G$-ομομορφισμός}) ονομάζεται μια γραμμική απεικόνιση $\varphi : V \to W$ η οποία \emph{διατηρεί} την δράση της $G$, δηλαδή 
    %
    \begin{equation}
        \label{eq:representation_homomorphism}
        \varphi(\rho(g)(v)) = \sigma(g)(\phi(v))
    \end{equation}
    για κάθε $g \in G$ και $v \in V$, ή ισοδύναμα αν το διάγραμμα
    \[
    \begin{tikzcd}
    V \arrow[r, "\varphi"] & W \\
    V \arrow[r, "\varphi"'] \arrow[u, "\rho(g)"] & W \arrow[u, "\sigma(g)"', right]
    \end{tikzcd}
    \]
    είναι μεταθετικό. Επιπλέον, αν η $\varphi$ είναι γραμμικός ισομορφισμός, τότε ονομάζεται \defn{ισομορφισμός αναπαραστάσεων} (ή \defn{$G$-ισομορφισμός}) και στην περίπτωση αυτή γράφουμε $V \cong_G W$ (ή απλά $V\cong W$).
\end{definition}

Αν έχουμε έναν $G$-ισομορφισμό $\varphi : V \to W$ με πίνακα $T$ (ως προς κάποια βάση της $V$), τότε η Ταυτότητα~\eqref{eq:representation_homomorphism} γίνεται
\[
\rho(g) = T^{-1}\sigma(g)T.
\]
Με άλλα λόγια, δυο αναπαραστάσεις είναι ισόμορφες όταν \textquote{διαφέρουν} κατά μια αλλαγή βάσης. 

Στον τρέχον παράδειγμα, όπου αναπαριστούμε την συμμετρική ομάδα $\fS_3$ ως ομάδα συμμετρίας του ισόπλευρου τριγώνου $\Delta$ έχουμε δει διάφορες εκδοχές της ίδιας αναπαράστασης στις ακόλουθες βάσεις του $\RR^3$:
\begin{itemize}
    \item $\{\bfe_1, \bfe_2, \bfe_3\}$
    \item $\{\bfe_1+\bfe_2+\bfe_3, \bfe_2-\bfe_1, \bfe_3-\bfe_1\}$
    \item $\{\bfe_1+\bfe_2+\bfe_3, \bfe_2, \bfe_3\}$.
\end{itemize}

Γενικότερα, η δράση αυτή της $\fS_n$ στον $\RR^n$ δίνεται από 
\[
\pi \cdot (v_1, v_2, \dots, v_n) \coloneqq (v_{\pi_1^{-1}}, v_{\pi_2^{-1}}, \dots, v_{\pi_n^{-1}})
\]
για κάθε $\pi \in \fS_n$ και $(v_1, v_2, \dots, v_n) \in \RR^n$ (βλ. Άσκηση~(1.2)). Η αναπαράσταση αυτή είναι ισόμορφη με την αναπαράσταση καθορισμού της $\fS_n$, με τον ισομορφισμό να δίνεται από 
\[
\bfe_i \mapsto i,
\]
για κάθε $1 \le i \le n$ και ο αντίστοιχος πίνακας είναι ο ταυτοτικός.

Έστω $\rmC_2$ η κυκλική ομάδα τάξης 2, η οποία παράγεται από ένα στοιχείο $g$, δηλαδή $\rmC_2 = \{\epsilon, g\}$. Θεωρούμε τη δράση της $\rmC_2$ στον $\RR^2$ που ορίζεται ως εξής:
\[
\begin{tikzpicture}[>=stealth, thick, scale=1]
\begin{scope}[shift={(-1,0)}]
    % Axes
    \draw[->] (-1.5,0) -- (1.5,0);
    \draw[->] (0,-1.5) -- (0,1.5);

    % Vectors e1 and e2
    \draw[->, burntorange, very thick] (0,0) -- (1,0) node[below] {\textcolor{black}{$\bfe_1$}};
    \draw[->, iceberg, very thick] (0,0) -- (0,1) node[left] {\textcolor{black}{$\bfe_2$}};

    % Origin dot
    \fill (0,0) circle (1.5pt);
\end{scope}
\begin{scope}[shift={(6,0)}]
    % Axes
    \draw[->,burntorange!40] (-1.5,0) -- (1.5,0);
    \draw[->, iceberg!40] (0,-1.5) -- (0,1.5);

    % Vectors
    \draw[->, burntorange, very thick] (0,0) -- (1,0) node[above] {\textcolor{black}{$\bfe_1$}};
    \draw[->, iceberg, very thick] (0,0) -- (0,-1) node[right] {\textcolor{black}{$\bfe_2$}};

    % Origin dot
    \fill (0,0) circle (1.5pt);
\end{scope}

% --- Curved arrow between them ---
\draw[->, thick]
(1.7,0.3) .. controls (2,0.7) and (3,0.7) .. (3.3,0.3)
node[midway, above] {$g$};
\end{tikzpicture}
\]
Ισοδύναμα, έχουμε την αναπαράσταση $\sigma : \rmC_2 \to \GL(\RR^2)$ με 
\[
\sigma(\epsilon) = 
\begin{pmatrix}
    1 & 0 \\
    0 & 1
\end{pmatrix}
\quad
\text{και}
\quad
\sigma(g) = 
\begin{pmatrix}
    1 & 0 \\
    0 & -1
\end{pmatrix}.
\]
Αυτή είναι ισόμορφη με την απαράσταση $(\rho, \RR^2)$ της Παραγράφου~2. Πράγματι, τα ιδιοδιανύσματα του $\rho(g)$ είναι
\begin{itemize}
    \item $\left(\begin{smallmatrix}
        1 \\ 0
    \end{smallmatrix}\right)$, που αντιστοιχεί στην ιδιοτιμή 1, και 
    \item $\left(\begin{smallmatrix}
        1 \\ -2
    \end{smallmatrix}\right)$, που αντιστοιχεί στην ιδιοτιμή -1
\end{itemize}
(γιατί;) και γι' αυτό
\[
\begin{pmatrix}
    1 & 1 \\
    0 & -2
\end{pmatrix}^{-1} 
\rho(g) 
\begin{pmatrix}
    1 & 1 \\
    0 & -2
\end{pmatrix}
\ = \
\sigma(g).
\]

Στην καινούργια βάση, μας είναι πιο εύκολο να \textquote{ξεχωρίσουμε} τις ανάγωγες υποαναπαραστάσεις. Κοιτώντας τον πίνακα $\sigma(g)$, προκύπτει η διάσπαση
\[
\RR^2 = \RR[\bfe_1] \oplus \RR[\bfe_2],
\]
όπου 
\begin{align*}
    \sigma(g)\left(\bfe_1\right) &= \bfe_1 \\
    \sigma(g)\left(\bfe_2\right) &= -\bfe_2
\end{align*}
και κατά συνέπεια το $\RR[\bfe_1]$ είναι ισόμορφο με την τετριμμένη αναπαράσταση και το $\RR[-\bfe_2]$ είναι ισόμορφο με την αναπαράσταση προσήμου (γιατί;).

Δοθείσης μια γραμμικής απεικόνισης $\varphi : V \to W$, οι υπόχωροι 
%
\begin{align*}
    \Ker(\varphi) &\coloneqq \{v \in V : \varphi(v) = 0\} \\
    \Im(\varphi) &\coloneqq \{w \in W : \varphi(v) = w, \ \text{για κάποιο $v \in V$}\}
\end{align*}
%
ονομάζονται \defn{πυρήνας} και \defn{εικόνα} της $\varphi$. Αν η $\varphi$ είναι $G$-ομομορφισμός, τότε ο πυρήνας και η εικόνα είναι υποπρότυπα του $V$ και $W$, αντίστοιχα (γιατί;).
Το επόμενο αποτέλεσμα, γνωστό ως \emph{Λήμμα του Schur}, χαρακτηρίζει τους $G$-ομομορφισμούς μεταξύ ανάγωγων αναπαραστάσεων και (παρά την απλή του απόδειξη) έχει πολύ σημαντικές συνέπειες, όπως θα δούμε στη συνέχεια.

\begin{theorem}{\rm(I. Schur 1905)}
    \label{thm:Schur_lemma}
    Έστω $V$ και $W$ δυο ανάγωγα $G$-πρότυπα και $\varphi : V \to W$ είναι ένας $G$-ομομορφισμός. 
    \begin{itemize}
    \item[(1)] Ο $\varphi$ είναι είτε η μηδενική απεικόνιση, είτε είναι ισομορφισμός.
    \item[(2)] Αν το $\FF$ είναι αλγεβρικά κλειστό σώμα, τότε ο $\varphi$ είναι πολλαπλάσιο του ταυτοτικού ομομορφισμού.
    \end{itemize}
\end{theorem}

\begin{proof}[Απόδειξη]
    \leavevmode 
    \begin{itemize}
        \item[(1)] Αφού τα $V$ και $W$ είναι ανάγωγα, έπεται ότι το υποπρότυπο $\Ker(\varphi)$ (αντ. $\Im(\varphi)$) είναι είτε $\{0\}$, είτε $V$ (αντ. $W$).  Αν $\Ker(\varphi) = V$, τότε η $\varphi$ είναι η μηδενική απεικόνιση. 
        
        Διαφορετικά, έστω $\Ker(\varphi) = \{0\}$. Αν $\Im(\varphi) \neq \{0\}$, τότε $V = \{0\}$, τ' οποίο είναι αδύνατο (γιατί;). Συνεπώς, $\Im(\varphi) = W$ και γι' αυτό η $\varphi$ είναι ισομορφισμός.
        
        \item[(2)] Αφού το $\FF$ είναι αλγεβρικά κλειστό, ο $\varphi$ (ως γραμμική απεικόνιση) έχει κάποια ιδιοτιμή $c \in \FF$. Συνεπώς, ο ομομορφισμός 
        \[
        \varphi - c\, \rmid 
        \]
        όπου $\rmid$ είναι η ταυτοτική επεικόνιση, έχει μη-τετριμμένο πυρήνα. Άρα, δεν μπορεί να είναι (γραμμικός) ισομορφισμός και γι' αυτό από το (1) έπεται ότι είναι η μηδενική απεικόνιση. Με άλλα λόγια 
        \[
        \varphi = c\, \rmid 
        \] 
        που είναι το ζητούμενο.
    \end{itemize}
\end{proof}

Σε ότι ακολουθεί υποθέτουμε ότι το $\FF$ είναι αλγεβρικά κλειστό σώμα (για παράδειγμα, το $\CC$). Για δυο $G$-πρότυπα $V$ και $W$, θέτουμε\footnote{Ομομορφισμός μεταξύ διανυσματικών χώρων δεν είναι τίποτα άλλα παρά μια γραμμική απεικόνιση και μια γραμμική απεικόνιση του ίδιου χώρου ονομάζεται και ενδομορφισμός. Αυτό εξηγεί τα σύμβολα $\Hom$ και $\End$.} 
\begin{align*}
    \Hom(V,W) &\coloneqq \{\varphi : V \to W : \ \text{$\varphi$ είναι γραμμική}\} \\
    \Hom_G(V,W) &\coloneqq \{\varphi : V \to W : \ \text{$\varphi$ είναι $G$-ομομορφισμός}\} \\
    \End_G(V) &\coloneqq \Hom_G(V,V).
\end{align*}
Το $\Hom(V,W)$ έχει και αυτό τη δομή διανυσματικού χώρου. Στην Άσκηση~(1.4), βλέπουμε ότι υπάρχει μια δράση της $G$ η οποία του δίνει την δομή $G$-προτύπου. Σε αυτή την περίπτωση, το $\Hom_G(V,W)$ ταυτίζεται με το σύνολο των σταθερών σημείων αυτής της δράσης. Το Λήμμα του Schur μας πληροφορεί ότι 
\begin{equation}
    \label{eq:endV_schur}
    \End_G(V) \cong \FF,    
\end{equation}
ως διανυσματικοί χώροι.

\begin{corollary}
    \label{cor:schur_dimension_of_Hom_space}
    Αν $V$ και $W$ είναι δυο ανάγωγα $G$-πρότυπα, τότε 
    \[
    \dim\Hom_G(V,W)\ = \ 
    \begin{cases}
        1, &\ \text{αν $V \cong_G W$} \\
        0, &\ \text{διαφορετικά}.
    \end{cases}
    \]
\end{corollary}

\begin{proof}[Απόδειξη]
    Αν τα $V$ και $W$ δεν είναι ισόμορφα, τότε από το Λήμμα του Schur έπεται ότι $\Hom_G(V,W) = \{0\}$. Στην περίπτωση αυτή 
    \[
    \dim\Hom_G(V,W) = 0.
    \] 
    
    Διαφορετικά, έστω $\varphi, \vartheta : V \to W$ δυο μη-μηδενικοί $G$-ομομορφισμοί. Πάλι από το Λήμμα του Schur, έπεται ότι αυτοί είναι ισομορφισμοί και γι' αυτό μπορούμε να θεωρήσουμε το $\vartheta^{-1}\circ\varphi \in \End_G(V)$. Από την Ταυτότητα~\eqref{eq:endV_schur}, έπεται ότι υπάρχει $c \in \FF$ τέτοιο ώστε $\vartheta^{-1}\circ\varphi = c\, \rmid$ ή ισοδύναμα $\varphi = c\, \vartheta$. Με άλλα λόγια, κάθε δυο στοιχεία του $\Hom_G(V,W)$ είναι συγγραμμικά και γι΄αυτό 
    \[
    \dim\Hom_G(V,W) = 1.
    \] 
\end{proof}

\begin{corollary}
    \label{cor:schur_representations_of_abelian_groups}
    Κάθε ανάγωγη αναπαράσταση μιας αβελιανής ομάδας είναι έχει διάσταση 1.
\end{corollary}

\begin{proof}[Απόδειξη]
    Υποθέτουμε ότι η $G$ είναι αβελιανή και θεωρούμε μια ανάγωγη αναπαράστασή της $(\rho,V)$. Για κάθε $g \in G$, παρατηρούμε ότι $\rho(g) \in \End_G(V)$. Πράγματι, για κάθε $h \in G$ και $v \in V$, 
    \[
    \rho(g)\left(\rho(h)(v)\right) = 
    \rho(gh)(v) = 
    \rho(hg)(v) = 
    \rho(h)\left(\rho(g)(v)\right)
    \]
    όπου η πρώτη και τρίτη ισότητα έπονται από το ότι η $\rho$ είναι ομομορφισμός ομάδων και η δεύτερη από το ότι η $G$ είναι αβελιανή. 
    
    Επομένως, εφαρμόζωντας το Λήμμα του Schur σε κάθε $\rho(g)$ έπεται ότι η δράση κάθε στοιχείου της ομάδας $G$ δίνεται από κάποιο πολλαπλάσιο της ταυτοτικής και κατά συνέπεια όλοι οι υπόχωροι της $V$ θα είναι αναγκαστικά $G$-αναλλοίωτοι (γιατί;). Καθώς η $V$ είναι ανάγωγη, αυτό αφήνει μόνο μια περίπτωση για τη διάστασής της (γιατί;), δηλαδή $\dim(V) = 1$.
\end{proof}

Στην Άσκηση~(1.6) χρησιμοποιούμε το Πόρισμα~\ref{cor:schur_representations_of_abelian_groups} για να καθορίσουμε \emph{όλες} τις ανάγωγες αναπαραστάσεις μιας κυκλικής ομάδας όταν $\FF = \CC$. Θα ολοκληρώσουμε την παράγραφο αυτή με μια ακόμα σημαντική εφαρμογή του Λήμματος του Schur.

Στο τέλος της Παραγράφου~2, είδαμε ότι ένα πλήρως αναγωγικό $G$-πρότυπο $V$ μπορεί να γραφεί ως 
\begin{equation}
    \label{eq:isotypic_decomposition}
    V \cong_G V_1^{m_1} \oplus V_2^{m_2} \oplus \cdots \oplus V_n^{m_n},
\end{equation}
για κάποια συλλογή $V_1, V_2, \dots, V_n$ ανά δύο \emph{μη-ισόμορφων} αναπαραστάσεων της $G$ και μη αρνητικούς ακέραιους $m_1, m_2, \dots, m_n$.

Η Έκφραση~\eqref{eq:isotypic_decomposition} ονομάζεται \defn{ισοτυπική διάσπαση} της $V$ και κάθε $m_i$ ονομάζεται \defn{πολλαπλότητα} εμφάνισης του $V_i$ στο $V$. Το παρακάτω αποτέλεσμα ολοκληρώνει τον παραλληλισμό που κάναμε με το Θεμελιώδες Θεώρημα της Αριθμητικής.

\begin{corollary}
    \label{cor:Schur_uniqueness_of_isotypic_decomposition}
    Η ισοτυπική διάσπαση ενός πλήρως αναγωγικού προτύπου είναι μοναδική ως προς ισομορφισμούς και αναδιατάξεις των μερών της.
\end{corollary}

Η απόδειξη του Πορίσματος~\ref{cor:Schur_uniqueness_of_isotypic_decomposition} βασίζεται στην εξής παρατήρηση, η οποία με τη σειρά της είναι μια ακόμη εφαρμογή του Λήμματος του Schur.

\begin{lemma}
    \label{lem:Schur_uniqueness_of_isotypic_decomposition}
    Έστω $V$ ένα πλήρως αναγωγικό $G$-πρότυπο. Η πολλαπλότητα ενός ανάγωγου $G$-προτύπου $W$ στην ισοτυπική διάσπαση του $V$ ισούται με 
    \[
    \dim\Hom_G(W,V).
    \]
\end{lemma}

\begin{proof}[Απόδειξη]
    Έστω 
    \[
    V = W_1 \oplus W_2 \oplus \cdots \oplus W_k
    \]
    η ανάλυση σε ανάγωγα υποπρότυπα του $V$. Τότε 
    %
    \begin{align*}
    \Hom_G(W,V) 
        &= \Hom_G(W, W_1 \oplus W_2 \oplus \cdots \oplus W_k) \\
        &\cong \Hom_G(W, W_1) \oplus \Hom_G(W, W_2) \oplus \cdots \oplus \Hom_G(W, W_k) \\
        &\cong \FF^{\abs{\{1 \le i \le k : W_i \cong W\}}},
    \end{align*}
    %
    όπου ο τρίτος ισομορφισμός έπεται από το Πόρισμα~\ref{cor:schur_dimension_of_Hom_space}. Για τον δεύτερο ισομορφισμό δείτε την παρατή\-ρηση μετά το τέλος της απόδειξης. Συνεπώς, 
    \[
    \dim\Hom_G(W,V) \ = \ \abs{\{1 \le i \le k : W_i \cong W\}},
    \]
    και το ζητούμενο έπεται.
\end{proof}

\begin{remark}{{\rm(Γραμμικής Άλγεβρας)}}
Έστω $V, V_1, V_2, W, W_1$ και $W_2$ διανυσματικοί χώροι. Υπάρχουν \emph{φυσικοί} ισομορφισμοί 
\begin{align*}
    \Hom(V, W_1 \oplus W_2) &\cong \Hom(V,W_1) \oplus \Hom(V,W_2) \\ 
    \Hom(V_1\oplus V_2, W) &\cong \Hom(V_1,W) \oplus \Hom(V_2,W),
\end{align*}
οι οποίοι δίνονται από 
\begin{align*}
    \varphi &\mapsto (\proj_1\circ\varphi, \proj_2\circ\varphi) \\ 
    \varphi &\mapsto (\varphi\circ\iota_1, \varphi\circ\iota_2),
\end{align*}
αντίστοιχα, όπου $\proj_i : W_1 \oplus W_2 \to W_i$ είναι η φυσική προβολή και $\iota_i : V_i \to V_1 \oplus V_2$ είναι ο φυσικός εγκλεισμός (γιατί;). Ομοίως, για κάθε πεπερασμένο αριθμό προσθετέων. Αυτό εξηγεί τον δεύτερο ισομορφισμό στην απόδειξη του Πορίσματος~\ref{lem:Schur_uniqueness_of_isotypic_decomposition}.

Αυτό έχει ως συνέπεια, η \textquote{ίδια} απόδειξη να μας πληροφορεί ότι η πολλαπλότητα ενός ανάγωγου $G$-προτύπου $W$ στην ισοτυπική διάσπαση του $V$ ισούται και με 
\[
\dim\Hom_G(V,W)
\]
(γιατί;). Αυτή η συμμετρία θα εξηγεί σε επόμενες παραγράφους, όταν μιλήσουμε για χαρακτήρες ομάδων.
\end{remark}

Ουσιαστικά το Λήμμα~\ref{lem:Schur_uniqueness_of_isotypic_decomposition} μας πληροφορεί ότι η πολλαπλότητα της $W$ στην ισοτυπική διάσπαση της $V$ εξαρτάται \emph{μόνο} από τα πρότυπα $V$ και $W$ και \emph{όχι} από την εκάστοτε διάσπαση. Τώρα, μπορεί κανείς να αποδείξει το Πόρισμα~\ref{cor:Schur_uniqueness_of_isotypic_decomposition}, η απόδειξη του οποίου αφήνεται ως άσκηση.
\end{document}