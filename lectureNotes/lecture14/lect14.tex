\documentclass[12pt,a4paper,reqno]{amsart}

% language
\usepackage[greek,english]{babel}
\usepackage[utf8]{inputenc}
\usepackage{alphabeta}

% change default names to greek
\addto\captionsenglish{
    \renewcommand{\contentsname}{Περιεχόμενα}
    \renewcommand{\refname}{Βιβλιογραφία}
    \renewcommand{\datename}{Ημερομηνία:}
    \renewcommand{\urladdrname}{Ιστοσελίδα}
}

% math 
\usepackage{amsmath,amsthm,amssymb,amscd}

% font
\usepackage[cal=euler]{mathalfa}
\usepackage{libertinus-type1}
% \usepackage{txfonts} % for upright greek letters
\usepackage{bm} % for bold symbols
\usepackage{bbm} % for the simply-looking bb symbols

% miscellaneous 
\usepackage{changepage} %for indenting environments
\usepackage{csquotes} % example: \textcquote{}
\usepackage{blkarray}
\setcounter{MaxMatrixCols}{20} % default for pmatrix is 10!!
\usepackage{ytableau}

% drawing
\usepackage{tikz,tikz-cd}
\usetikzlibrary{shapes.misc, patterns, matrix, calc, intersections,positioning}
\usepackage{graphics,graphicx}
\usepackage{float} % provides enhanced control and customization options for floating objects such as figures and tables

% colors
\usepackage{xcolor}
\definecolor{darkcandyapplered}{rgb}{0.64, 0.0, 0.0}
\definecolor{midnightblue}{rgb}{0.1, 0.1, 0.44}
\definecolor{mylightblue}{HTML}{336699}
\definecolor{burntorange}{rgb}{0.8, 0.33, 0.0}
\definecolor{iceberg}{rgb}{0.44, 0.65, 0.82}
\definecolor{applegreen}{rgb}{0.55, 0.71, 0.0}
\definecolor{canaryyellow}{rgb}{1.0, 0.94, 0.0}

% hrefs
\usepackage{hyperref}
\usepackage[noabbrev,capitalize]{cleveref}
\hypersetup{
    pdftoolbar=true,        
    pdfmenubar=true,        
    pdffitwindow=false,     
    pdfstartview={FitH},  % fits the width of the page to the window
    pdftitle={},
    pdfauthor={},
    pdfsubject={},
    pdfkeywords={},
    pdfnewwindow=true,  % links in new window
    colorlinks=true,  % false: boxed links; true: colored links
    linkcolor=darkcandyapplered,   % color of internal links
    citecolor=midnightblue,  % color of links to bibliography
    urlcolor=cyan,  % color of external links
    linktocpage=true  % changes the links from the section body to the page number
    }

% geometry
\textwidth=16cm 
\textheight=21cm 
\hoffset=-55pt 
\footskip=25pt

% thm envs (you might need to change the path)
% In this macro I define all the theorem environments

\theoremstyle{definition}
\newtheorem{theorem}{Θεώρημα}
\newtheorem{proposition}[theorem]{Πρόταση}
\newtheorem{lemma}[theorem]{Λήμμα}
\newtheorem{corollary}[theorem]{Πόρισμα}
\newtheorem{conjecture}[theorem]{Εικασία}
\newtheorem{problem}[theorem]{Πρόβλημα}
\newtheorem*{claim}{Ισχυρισμός}
\newtheorem{observation}[theorem]{Παρατήρηση}
\newtheorem{definition}[theorem]{Ορισμός}
\newtheorem{question}[theorem]{Ερώτηση}
\newtheorem*{questions}{Ερωτήματα}
\newtheorem{example}[theorem]{Παράδειγμα}
\newtheorem{exercise}{Άσκηση}

\newtheorem*{combInterlude}{Ιντερλούδιο Συνδυαστικής}
\newtheorem*{example_cont}{Παράδειγμα~6.6}
\newtheorem*{digression_la}{Παρέκβαση Γραμμικής Άλγεβρας}
\newtheorem*{thm}{Θεώρημα}

\theoremstyle{remark}
\newtheorem*{remark}{Παρατήρηση}

% fixes the correct numbering of environments
\numberwithin{theorem}{section}
\numberwithin{exercise}{section}
\numberwithin{equation}{section}

% math ops (you might need to change the path)
% In this macro I define all of my math operators

% fields
\newcommand{\NN}{\mathbbmss{N}} 
\newcommand{\ZZ}{\mathbbmss{Z}} 
\newcommand{\QQ}{\mathbbmss{Q}} 
\newcommand{\RR}{\mathbbmss{R}} 
\newcommand{\CC}{\mathbbmss{C}} 
\newcommand{\KK}{\mathbbmss{K}} 
\newcommand{\FF}{\mathbbmss{F}} 

% symmetric group
\newcommand{\fS}{\mathfrak{S}}  

% calligraphic 
\newcommand{\aA}{\mathcal{A}} 
\newcommand{\bB}{\mathcal{B}}
\newcommand{\cC}{\mathcal{C}}
\newcommand{\dD}{\mathcal{D}}
\newcommand{\eE}{\mathcal{E}}
\newcommand{\fF}{\mathcal{F}}
\newcommand{\hH}{\mathcal{H}}
\newcommand{\iI}{\mathcal{I}}
\newcommand{\lL}{\mathcal{L}}
\newcommand{\oO}{\mathcal{O}}
\newcommand{\pP}{\mathcal{P}}
\newcommand{\sS}{\mathcal{S}}
\newcommand{\mM}{\mathcal{M}}
\newcommand{\uU}{\mathcal{U}}

% bold
\newcommand{\bfa}{\mathbf{a}}
\newcommand{\bfe}{\mathbf{e}}
\newcommand{\bfF}{\pmb{F}}
\newcommand{\bfR}{\pmb{R}}
\newcommand{\bfv}{\mathbf{v}}
%\newcommand{\bfx}{\bm{x}}
%\newcommand{\bfx}{\mathbf{x}} 
\newcommand{\bfx}{\pmb{x}}
\newcommand{\bfX}{\pmb{X}}
\newcommand{\bfy}{\pmb{y}}
\newcommand{\bfz}{\pmb{z}}

% roman
\newcommand{\rmA}{\mathrm{A}}
\newcommand{\rmB}{\mathrm{B}}
\newcommand{\rmC}{\mathrm{C}}
\newcommand{\rmD}{\mathrm{D}} 
\newcommand{\rmI}{\mathrm{I}} 
\newcommand{\rmK}{\mathrm{K}}
\newcommand{\rmM}{\mathrm{M}}
\newcommand{\rmP}{\mathrm{P}}  
\newcommand{\rmp}{\mathrm{p}}  
\newcommand{\rmQ}{\mathrm{Q}}  
\newcommand{\rmR}{\mathrm{R}}
\newcommand{\rmS}{\mathrm{S}}
\newcommand{\rmT}{\mathrm{T}}
\newcommand{\rmU}{\mathrm{U}}
\newcommand{\rmV}{\mathrm{V}}
\newcommand{\rmY}{\mathrm{Y}}
\newcommand{\rmZ}{\mathrm{Z}}
\newcommand{\rmz}{\mathrm{z}}

% greek letters
% I'm renewing some commands in order to appear in upright font
% If I want to change it later, I don't have to do it manually, I just change it from here.
% \newcommand{\uaa}{\alphaup}
% \renewcommand{\alpha}{\alphaup}
% \renewcommand{\beta}{\betaup}
% \renewcommand{\gamma}{\gammaup}
% \renewcommand{\delta}{\deltaup}
% \renewcommand{\epsilon}{\epsilonup}
% \newcommand{\ee}{\epsilon}
% \renewcommand{\varepsilon}{\varepsilonup}
% \renewcommand{\theta}{\thetaup}
% \renewcommand{\lambda}{\lambdaup}
% \newcommand{\ull}{\lambda}
% \renewcommand{\mu}{\muup}
% \renewcommand{\nu}{\nuup}
% \renewcommand{\pi}{\piup}
% \renewcommand{\rho}{\rhoup}
% \renewcommand{\varrho}{\varrhoup}
% \renewcommand{\sigma}{\sigmaup}
% \renewcommand{\tau}{\tauup} 
% \renewcommand{\phi}{\phiup}
% \renewcommand{\chi}{\chiup}
% \renewcommand{\psi}{\psiup}
% \renewcommand{\omega}{\omegaup}

% arrows and symbols 
\renewcommand{\to}{\rightarrow}
\newcommand{\toto}{\longrightarrow}
\newcommand{\mapstoto}{\longmapsto}
\newcommand{\then}{\Rightarrow}
\newcommand{\IFF}{\Leftrightarrow}
\newcommand{\tl}{\tilde}
\newcommand{\wtl}{\widetilde}
\newcommand{\ol}{\overline}
\newcommand{\ul}{\underline}
\newcommand{\oldemptyset}{\emptyset}
\renewcommand{\emptyset}{\varnothing}
\DeclareMathSymbol{\Arg}{\mathbin}{AMSa}{"39} % for arguments 
\newcommand{\onto}{\ensuremath{\twoheadrightarrow}}
\newcommand{\tle}{\trianglelefteq}
\newcommand{\tge}{\trianglerighteq}

% absolute value symbol
\usepackage{mathtools} 
\DeclarePairedDelimiter\abs{\lvert}{\rvert}%
\DeclarePairedDelimiter\norm{\lVert}{\rVert}%
\makeatletter
\let\oldabs\abs
\def\abs{\@ifstar{\oldabs}{\oldabs*}}

% tensor symbol
\newcommand{\tensor}[1]{%
  \mathbin{\mathop{\otimes}\limits_{#1}}%
}

% permutation cycle notation
\ExplSyntaxOn
\NewDocumentCommand{\cycle}{ O{\;} m }
 {
  (
  \alec_cycle:nn { #1 } { #2 }
  )
 }

\seq_new:N \l_alec_cycle_seq
\cs_new_protected:Npn \alec_cycle:nn #1 #2
 {
  \seq_set_split:Nnn \l_alec_cycle_seq { , } { #2 }
  \seq_use:Nn \l_alec_cycle_seq { #1 }
 }
\ExplSyntaxOff

% setminus symbol
\newcommand{\mysetminusD}{\hbox{\tikz{\draw[line width=0.6pt,line cap=round] (3pt,0) -- (0,6pt);}}}
\newcommand{\mysetminusT}{\mysetminusD}
\newcommand{\mysetminusS}{\hbox{\tikz{\draw[line width=0.45pt,line cap=round] (2pt,0) -- (0,4pt);}}}
\newcommand{\mysetminusSS}{\hbox{\tikz{\draw[line width=0.4pt,line cap=round] (1.5pt,0) -- (0,3pt);}}}
\newcommand{\sm}{\mathbin{\mathchoice{\mysetminusD}{\mysetminusT}{\mysetminusS}{\mysetminusSS}}}

% custom math operators
\newcommand{\Des}{\mathrm{Des}} 
\newcommand{\des}{\mathrm{des}} 
\newcommand{\Asc}{\mathrm{Asc}}
\newcommand{\asc}{\mathrm{asc}} 
\newcommand{\inv}{\mathrm{inv}}
\newcommand{\Inv}{\mathrm{Inv}}
\newcommand{\maj}{\mathrm{maj}} 
\newcommand{\comaj}{\mathrm{comaj}} 
\newcommand{\fix}{\mathrm{fix}} 
\newcommand{\Sym}{\mathrm{Sym}} 
\newcommand{\QSym}{\mathrm{QSym}}
\newcommand{\FQSym}{\mathrm{FQSym}} 
\newcommand{\End}{\mathrm{End}} 
\newcommand{\Rad}{\mathrm{Rad}} 
\newcommand{\rmMat}{\mathrm{Mat}} 
\newcommand{\rmdim}{\mathrm{dim}} 
\newcommand{\rmTop}{\mathrm{Top}} 
\newcommand{\rmCF}{\mathrm{CF}} 
\newcommand{\rmId}{\mathrm{Id}}
\newcommand{\rmid}{\mathrm{id}}
\newcommand{\rmtw}{\mathrm{tw}}
\newcommand{\trace}{\mathrm{tr}}
\newcommand{\Irr}{\mathrm{Irr}}
\newcommand{\Ind}{\mathrm{Ind}} % induction
\newcommand{\Res}{\mathrm{Res}} % restriction
\newcommand{\triv}{\mathrm{triv}} % trivial rep
\newcommand{\rmdef}{\mathrm{def}} % defining rep
\newcommand{\dom}{\triangleleft}
\newcommand{\domeq}{\trianglelefteq}
\newcommand{\lex}{\mathrm{lex}}
\newcommand{\sign}{\mathrm{sign}}
\newcommand{\SYT}{\mathrm{SYT}}
\renewcommand{\Im}{\mathrm{Im}}
\newcommand{\Ker}{\mathrm{Ker}}
\newcommand{\GL}{\mathrm{GL}}
\newcommand{\FL}{\mathrm{FL}}
\newcommand{\Span}{\mathrm{span}}
\newcommand{\pos}{\mathrm{pos}}
\newcommand{\Comp}{\mathrm{Comp}}
\newcommand{\Set}{\mathrm{Set}}
\newcommand{\std}{\mathrm{std}}
\newcommand{\cont}{\mathrm{cont}} %content of a SSYT
\newcommand{\SSYT}{\mathrm{SSYT}}
\newcommand{\ct}{\mathrm{ct}} % cycle type
\newcommand{\ch}{\mathrm{ch}} % Frobenius characteristic map
\newcommand{\height}{\mathrm{ht}}
\newcommand{\FPS}{\CC[\![\bfx]\!]} % formal power series
\newcommand{\FPSS}{\CC[\![\bfx,\bfy]\!]}
\newcommand{\reg}{\mathrm{reg}}
\newcommand{\hook}{\mathrm{h}}
\newcommand{\weight}{\mathrm{wt}}
\newcommand{\co}{\mathrm{co}}
\newcommand{\ps}{\mathrm{ps}}
\newcommand{\rmsum}{\mathrm{sum}}
\newcommand{\NSym}{\mathrm{NSym}}
\newcommand{\Hom}{\mathrm{Hom}}
\newcommand{\proj}{\mathrm{proj}}
\newcommand{\stat}{\mathrm{stat}}
\newcommand{\Par}{\mathrm{Par}}
\newcommand{\rmset}{\mathrm{set}}
\newcommand{\comp}{\mathrm{comp}}

% miscellaneous commands
\newcommand{\defn}[1]{{\color{mylightblue}{#1}}}
\newcommand{\toDo}{{\bf\color{red} TODO}}
\newcommand{\toCite}{{\bf\color{green} CITE}}
\newcommand*{\vertbar}{\rule[-1ex]{0.5pt}{2.5ex}} % for matrices with column vectors
\newcommand*{\horzbar}{\rule[.5ex]{2.5ex}{0.5pt}} % for matrices with row vectors
\newcommand{\myblue}[1]{{\color{iceberg}{#1}}}
\newcommand{\myorange}[1]{{\color{burntorange}{#1}}}
\newcommand{\mygreen}[1]{{\color{applegreen}{#1}}}
\newcommand{\myred}[1]{{\color{darkcandyapplered}{#1}}}

% ferrer's diagram
\newcommand{\fdiagram}[1]{
    \begin{tikzpicture}[scale=.7]
        \fill foreach \Z [count=\Y] in {#1}
        {foreach \X in {1,...,\Z} 
        {(\X,-\Y) circle[radius=3pt]}};
    \end{tikzpicture}
}

%
\newcommand{\tcbo}[1]{\textcolor{burntorange}{#1}}

% 
\newenvironment{nouppercase}{%
  \let\uppercase\relax%
  \renewcommand{\uppercasenonmath}[1]{}}{}

% titlepage
\title{Θ2.04: Θεωρία Αναπαραστάσεων και Συνδυαστική}
\author[Β.~Δ. Μουστακας]{Βασίλης Διονύσης Μουστάκας \\ Πανεπιστήμιο Κρήτης}
\date{20 Νοεμβρίου 2025}
% \urladdr{\href{https://sites.google.com/view/vasmous}{https://sites.google.com/view/vasmous}}

\begin{document}

\begingroup
\def\uppercasenonmath#1{} % this disables uppercase title
\let\MakeUppercase\relax % this disables uppercase authors
\maketitle
\endgroup

\setcounter{section}{11}
\setcounter{theorem}{2}
\begin{center}
    \textbf{11. Πρότυπα Specht
} (Συνέχεια)
\end{center}

\begin{example}
    \label{ex:specht_modules}
    \leavevmode
    \begin{itemize}
        \item[(1)] Έστω $\lambda = (2,1) \vdash 3$. Για κάθε ένα από τα έξι ταμπλώ σχήματος $\lambda$
        \[
        \begin{ytableau}
            1 & 2 \\
            3
        \end{ytableau}\,, \
        \begin{ytableau}
            2 & 1 \\
            3
        \end{ytableau}\,, \
        \begin{ytableau}
            1 & 3 \\
            2
        \end{ytableau}\,, \
        \begin{ytableau}
            3 & 1 \\
            2
        \end{ytableau}\,, \
        \begin{ytableau}
            2 & 3 \\
            1
        \end{ytableau}\,, \
        \begin{ytableau}
            3 & 2 \\
            1
        \end{ytableau}
        \]
        παίρνουμε ένα πολυταμπλοειδές. Για διαφορετικούς αριθμούς $a, b, c \in [3]$, έχουμε
        \[
        \ytableausetup{smalltableaux,notabloids}
        \bfe_{\ytableaushort{ab, c}} 
        = \nabla_{\ytableaushort{ab, c}}^- \
        \ytableausetup{nosmalltableaux,tabloids}
        \ytableaushort{ab, c}  
        = \left(\epsilon - \cycle{a,c}\right) \ \ytableaushort{ab, c}  
        = \ytableaushort{ab, c} - \ytableaushort{cb, a}\ .
        \]
        Χρησιμοποιώντας τον ισομορφισμό $\rmM^{(2,1)} \cong \CC[1,2,3]$ που είδαμε στην αρχή της παραγράφου προκύπτει
        \[
        \ytableausetup{smalltableaux,notabloids}
        \bfe_{\ytableaushort{ab, c}} \ \mapsto c - a. 
        \]
        Με άλλα λόγια, το πρότυπο Specht $\sS^{(2,1)}$ παράγεται από όλες τις διαφορές $c - a$ για κάθε $c \neq a$ στο $\rmM^{(2,1)}$. Όπως είδαμε στο Παράδειγμα 6.7, αυτό δεν είναι άλλο από την συνήθη αναπαράσταση της $\fS_3$ και γι αυτό
        \[
        \sS^{(2,1)} = \CC[\,\bfe_{\ytableaushort{12, 3}}\, , \bfe_{\ytableaushort{13, 2}} \ ].
        \]
        Γενικότερα, το $\sS^{(n-1,1)}$ είναι ένα ανάγωγο υποπρότυπο του $\rmM^{(n-1,1)}$ ισόμορφο με τη συνήθη αναπαράσταση της $\fS_n$.

        \item[(2)] Αν $\lambda = (n)$, τότε υπάρχει ακριβώς ένα ταμπλώ σχήματος $(n)$ και γι αυτό 
        \[
        \ytableausetup{smalltableaux,notabloids}
        \bfe_{\ytableaushort{12\cdots{n}}} = 
        \nabla_{\ytableaushort{12\cdots{n}}}^- \
        \ytableausetup{nosmalltableaux,tabloids,centertableaux}
        \ytableaushort{12\cdots{n}} = \ytableaushort{12\cdots{n}} \, .
        \]
        Με άλλα λόγια, το $\sS^{(n)} = \rmM^{(n)}$. 

        \item[(3)] Αν $\lambda = (1^n)$, τότε για κάθε ταμπλώ $T$ σχήματος $(1^n)$ έχουμε $\rmC(T) = \fS_n$ (γιατί;) και γι αυτό 
        \[
        \nabla_T^- = \sum_{\pi \in \rmC(T)} \sign(\pi)\pi = \sum_{\pi \in \fS_n} \sign(\pi)\pi \coloneqq \nabla_n^-.
        \]
        Συνεπώς, για ένα αυθαίρετο πολυταμπλοειδές 
        \begin{align*}
            \ytableausetup{smalltableaux,notabloids}
            \ytableausetup{boxsize=1.3em}
            \bfe_{\ytableaushort{{\pi_1},{\pi_2},\vdots,{\pi_n}}} 
            &= \sum_{\sigma \in \fS_n} \sign(\sigma)\sigma \ 
            \ytableausetup{nosmalltableaux,tabloids,centertableaux} 
            \ytableaushort{{\pi_1},{\pi_2},\vdots,{\pi_n}} \\
            &\mapsto \sum_{\sigma \in \fS_n} \sign(\sigma)\sigma\pi \\
            &= \sign(\pi^{-1}) \sum_{\sigma \in \fS_n} \sign(\sigma\pi)\sigma\pi \\
            &= \sign(\pi) \sum_{\tau \in \fS_n} \sign(\tau)\tau \\
            &= \sign(\pi)\nabla_n^-,
        \end{align*}
        όπου στη δεύτερη γραμμή χρησιμοποιήσαμε τον ισομορφισμό $M^{(1^n)} \cong \CC[\fS_n]$ και η δεύτερη και τρίτη ισότητα έπονται από την Άσκηση 3.6 (4). 

        Άρα, το πρότυπο Specht που αντιστοιχεί στην διαμέριση $(1^n)$ είναι ισόμορφο με το υποπρότυπο $\CC[\nabla_n^+]$ της κανονικής αναπαράστασης με τη δράση να δίνεται από το πρόσημο μιας μετάθεσης. Αυτό δεν είναι άλλο από το πρότυπο της αναπαράστασης προσήμου της $\fS_n$.
    \end{itemize}
\end{example}

Οι υπολογισμοί του Παραδείγματος \ref{ex:specht_modules} δείχνουν ότι \textquote{βαδίζουμε} στο σωστό μονοπάτι αναφορικά με τον στόχο μας να βρούμε τα ανάγωγα $\fS_n$-πρότυπα. Η επόμενη πρόταση εξερευνεί ορισμένες ιδιότητες των προτύπων Specht. Η απόδειξη, αν και τεχνική, δεν παρουσιάζει δυσκολίες και είναι μια καλή ευκαιρία να ελέγξει κανείς την κατανόηση των ορισμών.

\begin{lemma}
    \label{lem:specht_module}
    Αν $T$ είναι ταμπλώ περιεχομένου $[n]$ και $\pi \in \fS_n$, τότε ισχύουν τα εξής 
    \begin{itemize}
        \item[(1)] $\rmC(\pi{T}) = \pi\rmC(T)\pi^{-1}$
        \item[(2)] $\nabla_{\pi{T}}^- = \pi\nabla_{T}^-\pi^{-1}$
        \item[(3)] $\bfe_{\pi{T}} = \pi\bfe_T$
        \item[(4)] Αν επιπλεόν $\pi \in \rmC(T)$, τότε $\bfe_{\pi{T}} = \sign(\pi)\bfe_T$.
    \end{itemize}
\end{lemma}

\begin{proof}[Απόδειξη]
    Για το (1), έχουμε $\sigma \in \rmC(\pi{T})$ αν και μόνο αν για κάθε $i \in [n]$, τα $\sigma(i)$ και $i$ βρίσκονται στην ίδια στήλη του $\pi{T}$. Ισοδύναμα, αν και μόνο αν για κάθε $j \in [n]$, τα $\sigma(\pi(j))$ και $\pi(j)$ βρίσκονται στην ίδια στήλη του $\pi{T}$, καθώς η $\pi$ είναι μετάθεση. Με τη σειρά του, αυτό είναι ισόδυναμο με το να βρίσκονται τα $\pi^{-1}(\sigma(\pi(j)))$ και $j$ στην ίδια στήλη του $T$, για κάθε $j \in [n]$ (γιατί;), δηλαδή με το $\pi^{-1}\sigma\pi \in \rmC(T)$ και γι αυτό $\sigma \in \pi\rmC(T)\pi^{-1}$.

    Για το (2), υπολογίζουμε 
    \begin{align*}
        \pi\nabla_{T}^-\pi^{-1} 
        &= \sum_{\sigma \in \rmC(T)} \sign(\sigma) \pi\sigma\pi^{-1} \\
        &= \sum_{\sigma \in \rmC(T)} \sign(\pi\sigma\pi^{-1}) \pi\sigma\pi^{-1} \\
        &= \sum_{\tau \in \pi\rmC(T)\pi^{-1}} \sign(\tau) \tau \\
        &= \sum_{\tau \in \rmC(\pi{T})} \sign(\tau) \tau \\
        &= \nabla_{\pi{T}}^-,
    \end{align*}
    όπου η προτελευταία ισότητα έπεται από το (1).

    Για το (3), έχουμε 
    \begin{align*} 
        \bfe_{\pi{T}} 
        = \nabla_{\pi{T}}^-[\pi{T}] 
        = \pi\nabla_{T}^-\pi^{-1}[\pi{T}] 
        = \pi\nabla_{T}^-[T] 
        = \pi\bfe_T,
    \end{align*}
    όπου η δεύτερη ισότητα έπεται από το (2).

    Τέλος, για το (4), έχουμε 
    \begin{align*}
        \bfe_{\pi{T}} 
        &= \nabla_{\pi{T}}^-[\pi{T}] \\
        &= \sum_{\sigma \in \rmC(\pi{T})}\sign(\sigma)\sigma[\pi{T}] \\
        &= \sum_{\sigma \in \rmC(T)}\sign(\sigma)\sigma[\pi{T}] \\
        &= \sign(\pi^{-1})\sum_{\sigma \in \rmC(T)}\sign(\sigma\pi)\sigma\pi[T] \\
        &= \sign(\pi)\sum_{\tau \in \rmC(T)}\sign(\tau)\tau[T] \\
        &= \sign(\pi)\nabla_T^-[T] \\
        &= \sign(\pi)\bfe_T,
    \end{align*}
    όπου η τρίτη ισότητα έπεται από το (1) και το γεγονός ότι $\pi \in \rmC(T)$ και η πέμπτη ισότητα έπεται από το ότι η αντιστοιχία $\sigma \mapsto \sigma\pi$ είναι αμφιμονοσήμαντη στο $\rmC(T)$, καθώς $\pi \in \rmC(T)$.
\end{proof}

Μια σημαντική συνέπεια του Λήμματος \ref{lem:specht_module}~(3) είναι ότι το πρότυπο Specht είναι υποπρότυπο του προτύπου Young. Πράγματι, για κάθε $\lambda \vdash n$ αν $v \in \sS^\lambda$, τότε 
\[
v = \sum_{T} c_T \, \bfe_T
\]
όπου το άθροισμα διατρέχει όλα τα ταμπλώ σχήματος $\lambda$, και $c_T \in \CC$. Τότε, 
\[
\pi{v} = \pi\left(\sum_{T} c_T \, \bfe_T\right) = \sum_{T} c_T \, (\pi\bfe_T) = \sum_{T} c_T \, \bfe_{\pi{T}} \ \in \ \sS^\lambda,
\]
για κάθε $\pi \in \fS_n$.

Για την ακρίβεια, μπορούμε να πούμε κάτι παραπάνω. Τα πρότυπα Young και Specht έχουν την ιδιότητα ένα αυθαίρετο στοιχείο τους να έχει την μορφή 
\[
\left(\sum_{\pi \in \fS_n}c_\pi\pi\right)[T]
\quad
\text{και}
\quad
\left(\sum_{\pi \in \fS_n}c_\pi\pi\right)\bfe_T
\]
για οποιοδήποτε ταμπλοειδές $[T]$ και πολυταμπλοειδές $\bfe_T$, αντίστοιχα (γιατί;). Ορισμένες φορές πρότυπα με αυτή την ιδιότητα ονομάζονται \emph{κυκλικά}.

\begin{digression_a}
    Γενικότερα, αν $G$ είναι μια πεπερασμένη ομάδα, τότε η κανονική αναπαράσταση $\FF[G]$, εκτός από δομή διανυσματικού χώρου έχει και τη δομή \emph{δακτύλιου}, με τον πολλαπλασιασμό να κληρονομείται από την πράξη στην $G$. Πιο συγκεκριμένα, 
    \[
    \left(\sum_{g \in G} c_g{g}\right)
    \left(\sum_{x \in G} c_x{x}\right) \ \coloneqq \ 
    \sum_{h \in G} \left(\sum_{\substack{g, x \in G \\ h = gx}}c_gc_x\right){h}.
    \]
    Αυτό ονομάζεται \emph{ομαδοδακτύλιος} (group ring).
    
    Έχοντας αυτό κατά νου, αυτά που ονομάζουμε $G$-πρότυπα μέχρι στιγμής, δεν είναι τίποτα άλλο παρά (αριστερά) $\FF[G]$-πρότυπα. Διαισθητικά, ένα πρότυπο πάνω από έναν δακτύλιο $R$ είσαι σαν ένα \textquote{διανυσματικό χώρο}, μόνο που τώρα ο βαθμωτός πολλαπλασιασμός δίνεται από τα στοιχεία του $R$, αντί για τα στοιχεία κάποιου σώματος $\FF$.
    
    Πιο συγκεκριμένα, ένα (\emph{αριστερό}) \emph{$R$-πρότυπο} (left $R$-module) είναι μια αβελιανή ομάδα $M$ εφοδιασμένη με μια πράξη $R \times M \to M$, την οποία συμβολίζουμε με $(r,m) \mapsto r\cdot{m}$ τέτοια ώστε 
    \begin{align*}
        1\cdot{m} &= m \\
        r\cdot{(m+m')} &= r\cdot{m} + r\cdot{m'} \\
        (r+r')\cdot{m} &= r\cdot{m} + r'\cdot{m} \\
        (rr')\cdot{m} &= r\cdot(r'\cdot(m)) 
    \end{align*}
    για κάθε $r, r' \in R$ και $m, m' \in M$. Συγκρίνοντας αυτά τα αξιώματα με τα αντίστοιχα ενός $G$-προτύπου, τι παρατηρείτε; Ποια είναι τα $\FF$-πρότυπα; Ποιά ειναι τα $\ZZ$-πρότυπα;

    Σε αυτό το πλαίσιο, ένα $R$-πρότυπο $M$ ονομάζεται \emph{κυκλικό} αν $M = Rm$, δηλαδή 
    \[
    M = \{rm : r \in R\}
    \]
    για οποιοδήποτε (μη μηδενικό) $m \in M$. Η έννοια του κυκλικού $\FF[G]$-προτύπου δεν είναι ακριβώς ίδια με την παραπάνω που έχουν τα πρότυπα Young και Specht. Ειδικότερα, τα πρότυπα Young παράγονται από οποιοδήποτε ταμπλοειδές, αλλά όχι οποιοδήποτε \emph{γραμμικό συνδυασμό ταμπλειδών}, που είναι και ένα αυθαίρετο στοιχείο του $\rmM^\lambda$. Από την άλλη μεριά όμως, θα δείξουμε ότι τα πρότυπα Specht είναι ανάγωγα και ως τέτοια είναι κυκλικά \emph{και} με τις δυο έννοιες\footnote{Δεν είναι δύσκολο να δείξει κανείς ότι ένα μη-τετριμμένο $\FF[G]$-πρότυπο είναι ανάγωγο αν και μόνο αν είναι κυκλικό ως $\FF[G]$-πρότυπο.}. 
\end{digression_a}

Θα δείξουμε τώρα ότι τα πρότυπα Specht είναι ανάγωγα. Γι αυτό θεωρούμε το (μοναδικό) εσωτερικό γινόμενο $(\Arg, \Arg)$ στο $M^\lambda$ που ορίζεται κάνοντας την βάση των ταμπλοειδών σχήματος $\lambda$ ορθοκανονική, δηλαδή
\[
([T], [Q]) = 
\begin{cases}
    1, &\ \text{αν $[T] = [Q]$} \\
    0, &\ \text{διαφορετικά}. \\
\end{cases}
\]
Το $(\Arg,\Arg)$ είναι $\fS_n$-αναλλοίωτο (γιατί;).
\begin{theorem}{\rm(Submodule theorem, James 1976)}
    \label{thm:james}
    Έστω $\lambda \vdash n$. Αν $U$ είναι υποπρότυπο του $M^\lambda$, τότε είτε $U \supseteq \sS^\lambda$, είτε $U \subseteq (\sS^\lambda)^\perp$, ως προς το εσωτερικό γινόμενο $(\Arg, \Arg)$.
\end{theorem}

Αυτό έχει ως άμεση συνέπεια το παρακάτω.
\begin{corollary}
    \label{cor:specht_irreducible}
    Τα πρότυπα Specht είναι ανάγωγα.
\end{corollary}
\begin{proof}[Απόδειξη]
    Έστω $\lambda$ μια διαμέριση ακεραίου. Αν $U \neq \{0\}$ είναι υποπρότυπο του $\sS^\lambda$, τότε από το Θεώρημα \ref{thm:james} έπεται ότι 
    \begin{itemize}
    \item είτε $U \supseteq \sS^\lambda$ και γι αυτό $U = \sS^\lambda$,      
    \item είτε $U \subseteq (\sS^\lambda)^\perp$ και γι αυτό $U \subseteq \sS^\lambda \cap (\sS^\lambda)^\perp = \{0\}$, το οποίο είναι αδύνατο καθώς $U \neq \{0\}$. 
    \end{itemize}
    Με άλλα λόγια, το $\sS^\lambda$ δεν έχει γνήσια υποπρότυπα διαφορετικά του $\{0\}$ και κατά συνέπεια είναι ανάγωγο.
\end{proof}

Η απόδειξη του Θεωρήματος \ref{thm:james} βασίζεται στο παρακάτω τεχνικό λήμμα.
\begin{lemma}
    \label{lem:james}
    Έστω $\lambda \vdash n$ και $T, Q$ δύο ταμπλώ σχήματος $\lambda$. 
    \begin{itemize}
        \item[(1)] Για κάθε $v, u \in \rmM^\lambda$, ισχύει ότι 
        \[
        (\nabla_T^-v, u) = (v, \nabla_T^-u).
        \]
        \item[(2)] Αν $\cycle{i,j} \in \rmR(Q) \cap \rmC(T)$, τότε $\nabla_T^-[Q] = 0$.
        \item[(3)] Αν $\nabla_T^-[Q] \neq 0$, τότε $\nabla_T^-[Q] = \pm \bfe_T$.
        \item[(4)] Αν $v \in \rmM^\lambda$, τότε 
        \[
        \nabla_T^-v = c \, \bfe_T
        \]
        για κάποιο $c \in \CC$.
    \end{itemize}
\end{lemma}

\begin{proof}[Απόδειξη]
    Για το (1), υπολογίζουμε 
    \begin{align*}
        (\nabla_T^-v, u) 
        &= \left( \sum_{\pi \in \rmC(T)} \sign(\pi)\pi v, u \right) \\
        &= \sum_{\pi \in \rmC(T)} \sign(\pi)\left(\pi v, u\right) \\
        &= \sum_{\pi \in \rmC(T)} \sign(\pi)\left(v,\pi^{-1}u\right) \\
        &= \left(v,\sum_{\pi \in \rmC(T)} \sign(\pi)\pi^{-1}u\right) \\
        &= \left(v,\sum_{\pi \in \rmC(T)} \sign(\pi^{-1})\pi^{-1}u\right) \\
        &= \left(v,\nabla_T^-u\right),
    \end{align*}
    όπου η τρίτη ισότητα έπεται από το γεγονός ότι το $(\Arg,\Arg)$ είναι $\fS_n$-αναλλοίωτο (γιατί;).

    Για το (2), αρχικά παρατηρούμε ότι 
    \begin{equation}
        \label{eq:james_help}
        \nabla_T^- = x(\epsilon - \cycle{i,j}),
    \end{equation}
    για κάποιο $x \in \CC[\fS_n]$. Πράγματι, έστω $H$ η υποομάδα της $\rmC(T)$ που παράγεται από το $\cycle{i,j}$, το οποίο εξ υποθέσεως είναι στοιχείο της $\rmC(T)$, και $\{x_1, x_2, \dots, x_k\}$ ένα σύνολο αντιπροσώπων των αριστερών κλάσεων της $H$ στην $\rmC(T)$. Τότε 
    \[
    \rmC(T) = x_1H \uplus x_2H \uplus \cdots \uplus x_kH
    \]
    και γι αυτό 
    \begin{align*}
        \nabla_T^- 
        &= \sum_{\pi \in \rmC(T)} \sign(\pi) \pi \\
        &= \sum_{i=r}^k \left(\sign(x_r)x_r + \sign(x_r\cycle{i,j})x_r\cycle{i,j}\right) \\
        &= \left(\underbrace{\sum_{r=1}^k \sign(x_r)x_r}_{= \ x}\right)(\epsilon - \cycle{i,j}).
    \end{align*}
    Τώρα, εφαρμόζοντας της Ταυτότητα~\eqref{eq:james_help} παίρνουμε 
    \[
    \nabla_T^-[Q] 
    = x(\epsilon - \cycle{i,j})[Q]
    = x([Q] - \cycle{i,j}[Q]) 
    = x([Q]- [Q])
    =0,
    \]
    όπου η τρίτη ισότητα έπεται από την υπόθεση $\cycle{i,j} \in \rmR(Q)$.

    Για το (3), παρατηρούμε ότι κάθε δυο στοιχεία $i$ και $j$ τα οποία ανήκουν στην ίδια γραμμή του $Q$ \emph{δεν μπορούν} να ανήκουν στην ίδια στήλη του $T$, διότι διαφορετικά $\cycle{i,j} \in \rmR(Q) \cap \rmC(T)$ και από το (2) θα είχαμε $\nabla_T^-[Q] = 0$, το οποίο είναι αδύνατο. Αυτό έχει ως αποτέλεσμα την ύπαρξη μιας $\pi \in \rmC(T)$ τέτοια ώστε $[\pi{T}] = [Q]$.

    Για παράδειγμα, αν $\lambda = (4,3,3,1)$ και 
    \[
    T = 
    \ytableausetup{nosmalltableaux,notabloids}
    \ytableaushort{1234,567,89{10},{11}},
    \quad 
    \text{και}
    \quad
    Q = \ytableaushort{16{10}4,27{11},389,5}
    \]
    τότε \textquote{κοιτάμε} τα στοιχεία κάθε στήλης του $T$ ξεχωριστά για το αν βρίσκονται στην αντίστοιχη γραμμή του $Q$. Αν όχι, τότε τα μεταθέτουμε με ένα στοιχείο της $\rmC(T)$ για να βρεθούν στην \textquote{σωστή} τους θέση. Στο παράδειγμα, στην πρώτη στήλη 
    \[
    T = 
    \begin{ytableau}
        1 & 2 & 3 & 4 \\
        *(burntorange)5 & 6 & 7 \\
        8 & 9 & 10 \\
        *(burntorange)11
    \end{ytableau}
    \quad 
    \text{και}
    \quad
    Q = 
    \begin{ytableau}
        1 & 6 & 10 & 4 \\
        2 & 7 & *(burntorange)11 \\
        3 & 8 & 9 \\
        *(burntorange)5
    \end{ytableau}
    \]
    και γι αυτό δρούμε με την $\cycle{5, 11}$. Έπειτα, για τη δεύτερη στήλη 
    \[
    \cycle{5, 11}T = 
    \begin{ytableau}
        1  & *(iceberg)2 & 3 & 4 \\
        11 & *(iceberg)6 & 7 \\
        8  & 9 & 10 \\
        5
    \end{ytableau}
    \quad 
    \text{και}
    \quad
    Q = 
    \begin{ytableau}
        1 & *(iceberg)6 & 10 & 4 \\
        *(iceberg)2 & 7 & 11 \\
        3 & 8 & 9 \\
        5
    \end{ytableau}
    \]
    και γι αυτό δρούμε με την $\cycle{2,6}$. Τέλος, για την τρίτη στήλη 
    \[
    \cycle{2,6}\cycle{5, 11}T = 
    \begin{ytableau}
        1  & 6 & *(applegreen)3 & 4 \\
        11 & 2 & 7 \\
        8  & 9 & *(applegreen)10 \\
        5
    \end{ytableau}
    \quad 
    \text{και}
    \quad
    Q = 
    \begin{ytableau}
        1 & 6 & *(applegreen)10 & 4 \\
        2 & 7 & 11 \\
        *(applegreen)3 & 8 & 9 \\
        5
    \end{ytableau}
    \]
    και γι αυτό δρούμε με την $\cycle{3,10}$, για να πάρουμε 
    \[
    [\cycle{3,10}\cycle{2,6}\cycle{5, 11}T] = 
    \ytableausetup{tabloids}
    \begin{ytableau}
        1  & 6 & 10 & 4 \\
        11 & 2 & 7 \\
        8  & 9 & 3 \\
        5
    \end{ytableau}
    = \
    \begin{ytableau}
        1 & 6 & 10 & 4 \\
        2 & 7 & 11 \\
        3 & 8 & 9 \\
        5
    \end{ytableau}
    = [Q].
    \]

    Συνεπώς, 
    \begin{align*}
        \nabla_T^-[Q] 
        &= \nabla_T^-[\pi{T}] \\
        &= \sum_{\sigma \in \rmC(T)} \sign(\sigma) \sigma\pi[T] \\
        &= \sign(\pi^{-1})\sum_{\sigma \in \rmC(T)} \sign(\sigma\pi) \sigma\pi[T] \\
        &= \sign(\pi)\sum_{\tau \in \rmC(T)} \sign(\tau) \tau[T] \\
        &= \sign(\pi)\bfe_{T},
    \end{align*}
    όπου η τέταρτη ισότητα έπεται από το ότι η αντιστοιχία $\sigma \mapsto \sigma\pi$ είναι αμφιμονοσήμαντη στο $\rmC(T)$, διότι $\pi \in \rmC(T)$.

    Τέλος, αν $v \in \rmM^\lambda$, τότε 
    \[
    v = \sum_{[Q]} c_{[Q]}[Q],
    \]
    όπου στο άθροισμα το $[Q]$ διατρέχει όλα τα ταμπλοειδή σχήματος $\lambda$ και $c_{[Q]} \in \CC$. Υπολογίζουμε
    \begin{align*}
        \nabla_T^-v 
        &= \nabla_T^-\left(\sum_{[Q]} c_{[Q]}[Q]\right) \\
        &= \sum_{[Q]} c_{[Q]}\nabla_T^-[Q] \\
        &= \sum_{\substack{[Q] \\ \nabla_T^-[Q] \neq 0}} c_{[Q]}\nabla_T^-[Q] \\
        &= \sum_{\substack{[Q] \\ \nabla_T^-[Q] \neq 0}} c_{[Q]}\left(\pm \bfe_T\right)\\
        &= \left(\underbrace{\sum_{\substack{[Q] \\ \nabla_T^-[Q] \neq 0}} \pm c_{[Q]}}_{= \, c \ \in \ \CC}\right)\bfe_T,
    \end{align*}
    όπου η τέταρτη ισότητα έπεται από το (3) και έτσι ολοκληρώνεται η απόδειξη.
\end{proof}
\end{document}