\documentclass[12pt,a4paper,reqno]{amsart}

% section handling
\usepackage{subfiles} 

% language
\usepackage[greek,english]{babel}
\usepackage[utf8]{inputenc}
\usepackage{alphabeta}

% change default names to greek
\addto\captionsenglish{
    \renewcommand{\contentsname}{Περιεχόμενα}
    \renewcommand{\refname}{Βιβλιογραφία}
    \renewcommand{\datename}{Ημερομηνία:}
    \renewcommand{\urladdrname}{Ιστοσελίδα}
}

% math 
\usepackage{amsmath,amsthm,amssymb,amscd}

% font
\usepackage[cal=euler]{mathalfa}
\usepackage{libertinus-type1}
% \usepackage{txfonts} % for upright greek letters
\usepackage{bm} % for bold symbols
\usepackage{bbm} % for the simply-looking bb symbols

% miscellaneous 
\usepackage{changepage} %for indenting environments
\usepackage{csquotes} % example: \textcquote{}

% drawing
\usepackage{tikz,tikz-cd}
\usetikzlibrary{shapes.misc, patterns, matrix, calc, intersections,positioning}
\usepackage{graphics,graphicx}
\usepackage{float} % provides enhanced control and customization options for floating objects such as figures and tables

% colors
\usepackage{xcolor}
\definecolor{darkcandyapplered}{rgb}{0.64, 0.0, 0.0}
\definecolor{midnightblue}{rgb}{0.1, 0.1, 0.44}
\definecolor{mylightblue}{HTML}{336699}

% hrefs
\usepackage{hyperref}
\usepackage[noabbrev,capitalize]{cleveref}
\hypersetup{
    pdftoolbar=true,        
    pdfmenubar=true,        
    pdffitwindow=false,     
    pdfstartview={FitH},  % fits the width of the page to the window
    pdftitle={},
    pdfauthor={},
    pdfsubject={},
    pdfkeywords={},
    pdfnewwindow=true,  % links in new window
    colorlinks=true,  % false: boxed links; true: colored links
    linkcolor=darkcandyapplered,   % color of internal links
    citecolor=midnightblue,  % color of links to bibliography
    urlcolor=cyan,  % color of external links
    linktocpage=true  % changes the links from the section body to the page number
    }

% geometry
\textwidth=16cm 
\textheight=21cm 
\hoffset=-55pt 
\footskip=25pt

% thm envs (you might need to change the path)
% In this macro I define all the theorem environments

\theoremstyle{definition}
\newtheorem{theorem}{Θεώρημα}
\newtheorem{proposition}[theorem]{Πρόταση}
\newtheorem{lemma}[theorem]{Λήμμα}
\newtheorem{corollary}[theorem]{Πόρισμα}
\newtheorem{conjecture}[theorem]{Εικασία}
\newtheorem{problem}[theorem]{Πρόβλημα}
\newtheorem*{claim}{Ισχυρισμός}
\newtheorem{observation}[theorem]{Παρατήρηση}
\newtheorem{definition}[theorem]{Ορισμός}
\newtheorem{question}[theorem]{Ερώτηση}
\newtheorem{example}[theorem]{Παράδειγμα}
\newtheorem{exercise}{Άσκηση}

\theoremstyle{remark}
\newtheorem*{remark}{Παρατήρηση}

% fixes the correct numbering of environments
\numberwithin{theorem}{section}
\numberwithin{exercise}{section}
\numberwithin{equation}{section}

% math ops (you might need to change the path)
% In this macro I define all of my math operators

% fields
\newcommand{\NN}{\mathbbmss{N}} 
\newcommand{\ZZ}{\mathbbmss{Z}} 
\newcommand{\QQ}{\mathbbmss{Q}} 
\newcommand{\RR}{\mathbbmss{R}} 
\newcommand{\CC}{\mathbbmss{C}} 
\newcommand{\KK}{\mathbbmss{K}} 
\newcommand{\FF}{\mathbbmss{F}} 

% symmetric group
\newcommand{\fS}{\mathfrak{S}}  

% calligraphic 
\newcommand{\aA}{\mathcal{A}} 
\newcommand{\bB}{\mathcal{B}}
\newcommand{\cC}{\mathcal{C}}
\newcommand{\dD}{\mathcal{D}}
\newcommand{\eE}{\mathcal{E}}
\newcommand{\fF}{\mathcal{F}}
\newcommand{\hH}{\mathcal{H}}
\newcommand{\iI}{\mathcal{I}}
\newcommand{\lL}{\mathcal{L}}
\newcommand{\oO}{\mathcal{O}}
\newcommand{\pP}{\mathcal{P}}
\newcommand{\sS}{\mathcal{S}}
\newcommand{\mM}{\mathcal{M}}
\newcommand{\uU}{\mathcal{U}}

% bold
\newcommand{\bfa}{\mathbf{a}}
\newcommand{\bfe}{\mathbf{e}}
\newcommand{\bfF}{\pmb{F}}
\newcommand{\bfR}{\pmb{R}}
\newcommand{\bfv}{\mathbf{v}}
%\newcommand{\bfx}{\bm{x}}
%\newcommand{\bfx}{\mathbf{x}} 
\newcommand{\bfx}{\pmb{x}}
\newcommand{\bfX}{\pmb{X}}
\newcommand{\bfy}{\pmb{y}}
\newcommand{\bfz}{\pmb{z}}

% roman
\newcommand{\rmB}{\mathrm{B}}
\newcommand{\rmC}{\mathrm{C}}
\newcommand{\rmD}{\mathrm{D}} 
\newcommand{\rmI}{\mathrm{I}} 
\newcommand{\rmK}{\mathrm{K}}
\newcommand{\rmM}{\mathrm{M}}
\newcommand{\rmP}{\mathrm{P}}  
\newcommand{\rmQ}{\mathrm{Q}}  
\newcommand{\rmR}{\mathrm{R}}
\newcommand{\rmS}{\mathrm{S}}
\newcommand{\rmT}{\mathrm{T}}
\newcommand{\rmU}{\mathrm{U}}
\newcommand{\rmV}{\mathrm{V}}
\newcommand{\rmY}{\mathrm{Y}}
\newcommand{\rmZ}{\mathrm{Z}}

% greek letters
% I'm renewing some commands in order to appear in upright font
% If I want to change it later, I don't have to do it manually, I just change it from here.
% \newcommand{\uaa}{\alphaup}
% \renewcommand{\alpha}{\alphaup}
% \renewcommand{\beta}{\betaup}
% \renewcommand{\gamma}{\gammaup}
% \renewcommand{\delta}{\deltaup}
% \renewcommand{\epsilon}{\epsilonup}
% \newcommand{\ee}{\epsilon}
% \renewcommand{\varepsilon}{\varepsilonup}
% \renewcommand{\theta}{\thetaup}
% \renewcommand{\lambda}{\lambdaup}
% \newcommand{\ull}{\lambda}
% \renewcommand{\mu}{\muup}
% \renewcommand{\nu}{\nuup}
% \renewcommand{\pi}{\piup}
% \renewcommand{\rho}{\rhoup}
% \renewcommand{\varrho}{\varrhoup}
% \renewcommand{\sigma}{\sigmaup}
% \renewcommand{\tau}{\tauup} 
% \renewcommand{\phi}{\phiup}
% \renewcommand{\chi}{\chiup}
% \renewcommand{\psi}{\psiup}
% \renewcommand{\omega}{\omegaup}

% arrows and symbols 
\renewcommand{\to}{\rightarrow}
\newcommand{\toto}{\longrightarrow}
\newcommand{\mapstoto}{\longmapsto}
\newcommand{\then}{\Rightarrow}
\newcommand{\IFF}{\Leftrightarrow}
\newcommand{\tl}{\tilde}
\newcommand{\wtl}{\widetilde}
\newcommand{\ol}{\overline}
\newcommand{\ul}{\underline}
\newcommand{\oldemptyset}{\emptyset}
\renewcommand{\emptyset}{\varnothing}
\DeclareMathSymbol{\Arg}{\mathbin}{AMSa}{"39} % for arguments 
\newcommand{\onto}{\ensuremath{\twoheadrightarrow}}

% absolute value symbol
\usepackage{mathtools} 
\DeclarePairedDelimiter\abs{\lvert}{\rvert}%
\DeclarePairedDelimiter\norm{\lVert}{\rVert}%
\makeatletter
\let\oldabs\abs
\def\abs{\@ifstar{\oldabs}{\oldabs*}}

% tensor symbol
\newcommand{\tensor}[1]{%
  \mathbin{\mathop{\otimes}\limits_{#1}}%
}

% permutation cycle notation
\ExplSyntaxOn
\NewDocumentCommand{\cycle}{ O{\;} m }
 {
  (
  \alec_cycle:nn { #1 } { #2 }
  )
 }

\seq_new:N \l_alec_cycle_seq
\cs_new_protected:Npn \alec_cycle:nn #1 #2
 {
  \seq_set_split:Nnn \l_alec_cycle_seq { , } { #2 }
  \seq_use:Nn \l_alec_cycle_seq { #1 }
 }
\ExplSyntaxOff

% setminus symbol
\newcommand{\mysetminusD}{\hbox{\tikz{\draw[line width=0.6pt,line cap=round] (3pt,0) -- (0,6pt);}}}
\newcommand{\mysetminusT}{\mysetminusD}
\newcommand{\mysetminusS}{\hbox{\tikz{\draw[line width=0.45pt,line cap=round] (2pt,0) -- (0,4pt);}}}
\newcommand{\mysetminusSS}{\hbox{\tikz{\draw[line width=0.4pt,line cap=round] (1.5pt,0) -- (0,3pt);}}}
\newcommand{\sm}{\mathbin{\mathchoice{\mysetminusD}{\mysetminusT}{\mysetminusS}{\mysetminusSS}}}

% custom math operators
\newcommand{\Des}{\mathrm{Des}} 
\newcommand{\des}{\mathrm{des}} 
\newcommand{\Asc}{\mathrm{Asc}}
\newcommand{\asc}{\mathrm{asc}} 
\newcommand{\inv}{\mathrm{inv}}
\newcommand{\Inv}{\mathrm{Inv}}
\newcommand{\maj}{\mathrm{maj}} 
\newcommand{\comaj}{\mathrm{comaj}} 
\newcommand{\fix}{\mathrm{fix}} 
\newcommand{\Sym}{\mathrm{Sym}} 
\newcommand{\QSym}{\mathrm{QSym}}
\newcommand{\FQSym}{\mathrm{FQSym}} 
\newcommand{\End}{\mathrm{End}} 
\newcommand{\Rad}{\mathrm{Rad}} 
\newcommand{\rmMat}{\mathrm{Mat}} 
\newcommand{\rmdim}{\mathrm{dim}} 
\newcommand{\rmTop}{\mathrm{Top}} 
\newcommand{\rmCF}{\mathrm{CF}} 
\newcommand{\rmId}{\mathrm{Id}}
\newcommand{\rmid}{\mathrm{id}}
\newcommand{\rmtw}{\mathrm{tw}}
\newcommand{\trace}{\mathrm{tr}}
\newcommand{\Irr}{\mathrm{Irr}}
\newcommand{\Ind}{\mathrm{Ind}} % induction
\newcommand{\Res}{\mathrm{Res}} % restriction
\newcommand{\triv}{\mathrm{triv}} % trivial rep
\newcommand{\rmdef}{\mathrm{def}} % defining rep
\newcommand{\dom}{\triangleleft}
\newcommand{\domeq}{\trianglelefteq}
\newcommand{\lex}{\mathrm{lex}}
\newcommand{\sign}{\mathrm{sign}}
\newcommand{\SYT}{\mathrm{SYT}}
\renewcommand{\Im}{\mathrm{Im}}
\newcommand{\Ker}{\mathrm{Ker}}
\newcommand{\GL}{\mathrm{GL}}
\newcommand{\FL}{\mathrm{FL}}
\newcommand{\Span}{\mathrm{span}}
\newcommand{\pos}{\mathrm{pos}}
\newcommand{\Comp}{\mathrm{Comp}}
\newcommand{\Set}{\mathrm{Set}}
\newcommand{\std}{\mathrm{std}}
\newcommand{\cont}{\mathrm{cont}} %content of a SSYT
\newcommand{\SSYT}{\mathrm{SSYT}}
\newcommand{\rmz}{\mathrm{z}}
\newcommand{\ct}{\mathrm{ct}} % cycle type
\newcommand{\ch}{\mathrm{ch}} % Frobenius characteristic map
\newcommand{\height}{\mathrm{ht}}
\newcommand{\FPS}{\CC[\![\bfx]\!]} % formal power series
\newcommand{\FPSS}{\CC[\![\bfx,\bfy]\!]}
\newcommand{\reg}{\mathrm{reg}}
\newcommand{\hook}{\mathrm{h}}
\newcommand{\weight}{\mathrm{wt}}
\newcommand{\co}{\mathrm{co}}
\newcommand{\ps}{\mathrm{ps}}
\newcommand{\rmsum}{\mathrm{sum}}
\newcommand{\NSym}{\mathrm{NSym}}
\newcommand{\Hom}{\mathrm{Hom}}
\newcommand{\proj}{\mathrm{proj}}
\newcommand{\stat}{\mathrm{stat}}

% miscellaneous commands
\newcommand{\defn}[1]{{\color{mylightblue}{#1}}}
\newcommand{\toDo}{{\bf\color{red} TODO}}
\newcommand{\toCite}{{\bf\color{green} CITE}}

% 
\newenvironment{nouppercase}{%
  \let\uppercase\relax%
  \renewcommand{\uppercasenonmath}[1]{}}{}

% titlepage
\title{Θ2.04: Θεωρία Αναπαραστάσεων και Συνδυαστική}
\author[Β.~Δ. Μουστακας]{Βασίλης Διονύσης Μουστάκας \\ Πανεπιστήμιο Κρήτης}
\date{2 Οκτωβρίου 2025}
% \urladdr{\href{https://sites.google.com/view/vasmous}{https://sites.google.com/view/vasmous}}

\begin{document}

\begingroup
\def\uppercasenonmath#1{} % this disables uppercase title
\let\MakeUppercase\relax % this disables uppercase authors
\maketitle
\endgroup

\setcounter{section}{1}
\setcounter{theorem}{4} % according to where we stopped at lecture 1
\setcounter{equation}{1} % according to where we stopped at lecture 1
\thispagestyle{empty}

\begin{center}
    \textbf{1. Δράσεις ομάδων και αναπαραστάσεις} (συνέχεια)
\end{center}

Σε ότι ακολουθεί υποθέτουμε ότι $\FF$ είναι ένα αυθαίρετο σώμα. Ας δούμε μερικά σημαντικά παραδείγματα αναπαραστάσεων.
\begin{example}
    \label{ex:representations_examples}
    Έστω $G$ μια ομάδα.
    \leavevmode
    \begin{itemize}
        \item[(α)] Κάθε διανυσματικός χώρος $V$ διάστασης 1 γίνεται $G$-πρότυπο θέτοντας 
        \[
        gv = v
        \]
        για κάθε $g \in G$ και $v \in V$. Η αντίστοιχη αναπαράσταση ονομάζεται \defn{τετριμμένη αναπαράσταση} (trivial representation) της $G$ και την συμβολίζουμε με $(\rho^\triv, V)$.
        \item[(β)] Η αναπαράσταση μεταθέσεων που επάγεται από την δράση της $G$ στον εαυτό της με αριστερό πολλαπλασιασμό ονομάζεται \defn{κανονική αναπαράσταση} (regular representation) της $G$ και την συμβολίζουμε με $(\rho^\reg, \FF[G])$. Όπως θα δούμε σε επόμενη παράγραφο, η κανονική αναπαράσταση αποτελεί ένα από τα σημαντικότερα παραδείγματα αναπαραστάσεων.
        \item[(γ)] Έστω $H$ υποομάδα της $G$. Η αναπαράσταση μεταθέσεων που επάγεται από τη δράση της $G$ στο σύνολο των αριστερών συμπλόκων της $H$ ονομάζεται \defn{αναπαράσταση συμπλόκου} (coset representation). Στην περίπτωση όπου $H = \{\epsilon\}$ είναι η τετριμμένη υποομάδα, τότε η αναπαράσταση συμπλόκου εξειδικεύεται στην κανονική αναπαράσταση. Όπως θα δούμε σε επόμενη ενότητα, η αναπαράσταση συμπλόκου είναι ειδική περίπτωση μιας \emph{επαγόμενης αναπαράστασης} (induced representation) μεγάλου ενδιαφέροντος.
    \end{itemize}
    %
    Για τα υπόλοιπα παραδείγματα υποθέτουμε ότι $G = \fS_n$.
    %
    \begin{itemize}
        \item[(δ)] Ο πίνακας της μετάθεσης $\cycle{1,2}$ στην κανονική αναπαράσταση της $\fS_3$ ως προς την βάση $\{\cycle{1}\cycle{2}\cycle{3}, 
        \cycle{1,2}, 
        \cycle{2,3}, 
        \cycle{1,3}, 
        \cycle{1,2,3}, 
        \cycle{1,3,2}\}$ του $\FF[\fS_3]$ είναι   
        \[
        \begin{pmatrix}
            0 & 1 & 0 & 0 & 0 & 0 \\
            1 & 0 & 0 & 0 & 0 & 0 \\
            0 & 0 & 0 & 0 & 1 & 0 \\
            0 & 0 & 0 & 0 & 0 & 1 \\
            0 & 0 & 1 & 0 & 0 & 0 \\
            0 & 0 & 0 & 1 & 0 & 0
        \end{pmatrix}.
        \]
        Γιατί; Ποιοί είναι οι υπόλοιποι; Τι παρατηρείτε;
        \item[(στ)] Έστω $H$ η υποομάδα της $\fS_3$ που παράγεται από την μετάθεση $\cycle{2,3}$. Ο πίνακας της μετάθεσης $\cycle{1,2}$ στην αντίστοιχη αναπαράσταση συμπλόκου ως προς τη βάση $\{H, \cycle{1,2}H, \cycle{1,3}H\}$ είναι 
        \[
        \begin{pmatrix}
            0 & 1 & 0 \\
            1 & 0 & 0 \\
            0 & 0 & 1
        \end{pmatrix}.
        \]
        Γιατί; Ποιοί είναι οι υπόλοιποι; Τι παρατηρείτε;
        \item[(ζ)] Εκτός από την τετριμμένη αναπαράσταση, η συμμετρική ομάδα έχει μια ακόμη αναπαράσταση διάστασης 1, η οποία προκύπτει με φυσικό τρόπο. Ένας διανυσματικός χώρος $V$ διάστασης 1 γίνεται $\fS_n$-πρότυπο θέτοντας
        \[
        \pi v = \sign(\pi)v,
        \]
        για κάθε $\pi \in \fS_n$ και $v \in V$, όπου $\sign(\pi)$ είναι το \defn{πρόσημο}\footnote{Μια μετάθεση ονομάζεται \defn{άρτια} (αντ. \defn{περιττή}) αν μπορεί να γραφεί ως γινόμενο άρτιου (αντ. περιττού) πλήθους 2-κύκλων της μορφής $\cycle{i, i+1}$. Το πρόσημο μιας μετάθεσης ορίζεται να είναι 1 ή -1, ανάλογα με το αν είναι άρτια ή περιττή, αντίστοιχα. Θα δούμε περισσότερα για το πρόσημο μιας μετάθεσης σε επόμενη παράγραφο.} της μετάθεσης $\pi$. Η αναπαράσταση αυτή ονομάζεται \defn{αναπαράσταση προσήμου} (sign representation) και την συμβολίζουμε με $(\rho^\sign,V)$.
        \item[(η)] Η αναπαράσταση μεταθέσεων που επάγεται από την προφανή δράση της $\fS_n$ στο $[n]$ ονομάζεται \defn{αναπαράσταση καθορισμού} (defining representation) και την συμβολίζουμε με $\left(\rho^\rmdef,\FF[1,2,\dots,n]\right)$. Για $\pi \in \fS_n$, ως προς τη συνήθη βάση έχουμε   
        \[
       \left(\rho^\rmdef(\pi)\right)_{ij} = 
        \begin{cases}
            1, &\ \text{αν $\pi_j = i$} \\
            0, &\ \text{διαφορετικά}
        \end{cases}
        \]
        για κάθε $1 \le i, j \le n$ (γιατί;). Πίνακες αυτής της μορφής ονομάζονται \defn{πίνακες μετάθεσης} (permutation matrices). Ποιοί είναι οι πίνακες μετάθεσης για $n=3$; Τι παρατηρείτε;
    \end{itemize}
\end{example}

Επιστρέφοντας στο τρέχον παράδειγμα με την αναπαράσταση της $\fS_3$ ως ομάδα συμμετρίας του τριγώνου $\Delta$, ας \textquote{αλλάξουμε οπτική}, γράφοντας του πίνακες στην νέα βάση 
\[
\{\bfe_1 + \bfe_2 + \bfe_3, \bfe_2-\bfe_1, \bfe_3-\bfe_1\}
\]
του $\RR^3$. Ως προς αυτή την βάση έχουμε 
\[
\begin{array}{ccc}
    \cycle{1}\cycle{2}\cycle{3} \rightsquigarrow 
    \begin{pmatrix} 
        1 & 0 & 0 \\ 
        0 & 1 & 0 \\ 
        0 & 0 & 1 
    \end{pmatrix}  
    & \cycle{1,2,3} \rightsquigarrow 
    \begin{pmatrix} 
        1 & 0 & 0 \\ 
        0 & -1 & -1 \\ 
        0 & 1 & 0 
    \end{pmatrix}
    & \cycle{1,3,2} \rightsquigarrow 
    \begin{pmatrix} 
        1 & 0 & 0 \\ 
        0 & 0 & 1 \\ 
        0 & -1 & -1 
    \end{pmatrix} \\
     &  &  \\
    \cycle{1,2} \rightsquigarrow 
    \begin{pmatrix} 
        1 & 0 & 0 \\ 
        0 & -1 & -1 \\ 
        0 & 0 & 1 
    \end{pmatrix} 
    & \cycle{1,3} \rightsquigarrow 
    \begin{pmatrix} 
        1 & 0 & 0 \\ 
        0 & 1 & 0 \\ 
        0 & -1 & -1 
    \end{pmatrix} 
    & \cycle{2,3} \rightsquigarrow 
    \begin{pmatrix} 
        1 & 0 & 0 \\ 
        0 & 0 & 1 \\ 
        0 & 1 & 0 
    \end{pmatrix}.
\end{array}
\]

Συνεπώς, έχουμε μια διάσπαση
%
\begin{equation}
\label{eq:decomposition_R3}
\RR^3 = \RR[\bfe_1+\bfe_2+\bfe_3] \oplus \RR[\bfe_2-\bfe_1, \bfe_3-\bfe_1],
\end{equation}
%
η οποία \textquote{σέβεται} τη δράση της συμμετρικής ομάδας. 
Επιπλέον, παρατηρούμε ότι η $\fS_3$ δρα στον υπόχωρο $\RR[\bfe_1+\bfe_2+\bfe_3]$ τετριμμένα.

\begin{definition}
    \label{def:subrepresentation}
    Έστω $(\rho, V)$ αναπαράσταση μιας ομάδας $G$ και $W$ ένας υπόχωρος του $V$. Το ζεύγος $(\rho,W)$ ονομάζεται \defn{υποαναπαράσταση} (ή \defn{$G$-υποπρότυπο}, στην γλώσσα των προτύπων) της $V$ αν ο $W$ είναι \emph{$G$-αναλλοίωτος}, δηλαδή αν για κάθε $g \in G$ ισχύει ότι 
    \[
    \rho(g)(w) \in W
    \]
    για κάθε $w \in W$.
\end{definition}

Στην αναπαράσταση καθορισμού (όπως και στην περίπτωση της αναπαράστασης του τρέχοντος παραδείγματος) ο υπόχωρος 
\[
W = \FF[1 + 2 + \cdots + n] = \{c(1+2+ \cdots + n) : c \in \FF\}
\]
είναι $\fS_n$-αναλλοίωτος, διότι
\[
\pi \cdot (1+2+\cdots+n) = \pi_1 + \pi_2 + \cdots + \pi_n = 1 + 2 + \cdots + n \ \in \ W,
\]
για κάθε $\pi \in \fS_n$. Συνεπώς, βρήκαμε μια μονοδιάστατη υποαναπαράσταση της $\FF[1,2,\dots,n]$ η οποία είναι \textquote{αντίγραφο} της τετριμμένης αναπαράστασης. Γενικότερα, κάθε αναπαράσταση μεταθέσεων σ' ένα σύνολο $S = \{s_1, s_2, \dots, s_n\}$ περιέχει την μονοδιάστατη υποαναπαράσταση $\FF[s_1 + s_2 + \cdots + s_n]$.

\begin{definition}
    \label{def:irreducible_representation}
    Μια αναπαράσταση η οποία περιέχει μια γνήσια υποαναπαράσταση, που δεν είναι ο τετριμμένος υπόχωρος, ονομάζεται \defn{αναγωγική} (reducible). Διαφορετικά, την αποκαλούμε \defn{ανάγωγη αναπαράσταση} (irreducible representation).
\end{definition}

Κάθε μονοδιάσταση αναπαράσταση είναι κατ' ανάγκη ανάγωγη (γιατί;) και κάθε αναπαράσταση μεταθέσεων είναι αναγωγική.

\begin{proposition}
    \label{prop:irreducible_representation_criterion}
    Έστω $G$ μια ομάδα. Μια αναπαράσταση $(\rho,V)$ πεπερασμένης διάστασης της $G$ είναι αναγωγική αν και μόνο αν υπάρχει μια βάση του $V$ τέτοια ώστε ο πίνακας της $\rho(g)$ ως προς αυτή την βάση να είναι μπλόκ-άνω τριγωνικός, για κάθε $g \in G$.
\end{proposition}

\begin{proof}[Απόδειξη]
Έστω $\dim(V) = n$. Για την κατεύθυνση \textquote{$\then$}, υποθέτουμε ότι $W$ είναι υποαναπαράσταση του $V$ με βάση $\{w_1, w_2, \dots, w_k\}$. Συμπληρώνουμε την βάση αυτή με στοιχεία του $V$, για να πάρουμε μια βάση $\{w_1, w_2, \dots, w_k, v_{k+1}, \dots, v_n\}$ του $V$. Για κάθε $g \in G$, ο πίνακας $\rho(g)$ ως προς αυτή την βάση έχει την ζητούμενη μορφή, διότι 
\[
\rho(g)(w_i) \ \in W
\]
για κάθε $1 \le i \le k$.

Αντιστρόφως, υποθέτουμε ότι υπάρχει μια βάση του $V$ ως προς την οποία, για κάθε $g \in G$
\[
\rho(g) = 
\begin{pmatrix}
    A & \ast \\ 
    0 & B
\end{pmatrix}
\]
όπου $A, B$ είναι τριγωνικοί πίνακες. Αν ο $A$ είναι $k\times{k}$, τότε ο υπόχωρος που παράγεται από τα πρώτα $k$ στοιχεία της βάσης αυτής είναι $G$-αναλλοίωτος (γιατί;) και το ζητούμενο έπεται.
\end{proof}

Στο τρέχον παράδειγμα, είδαμε ότι ο υπόχωρος $\RR[\bfe_1+\bfe_2+\bfe_3]$ είναι μια υποαναπαράσταση. Μιμούμενοι την απόδειξη της Πρότασης~\ref{prop:irreducible_representation_criterion}, μπορούμε να επεκτείνουμε τη βάση με τον προφανή τρόπο
\[
\{\bfe_1+\bfe_2+\bfe_3, \bfe_2, \bfe_3\}
\]
και υπολογίζοντας τους πίνακες των στοιχείων της $\fS_3$ ως προς αυτή τη βάση βρίσκουμε 
\[
\begin{array}{ccc}
    \cycle{1}\cycle{2}\cycle{3} \rightsquigarrow 
    \begin{pmatrix} 
        1 & 0 & 0 \\ 
        0 & 1 & 0 \\ 
        0 & 0 & 1 
    \end{pmatrix}  
    & \cycle{1,2,3} \rightsquigarrow 
    \begin{pmatrix} 
        1 & 0 & 1 \\ 
        0 & 0 & -1 \\ 
        0 & 1 & -1 
    \end{pmatrix}
    & \cycle{1,3,2} \rightsquigarrow 
    \begin{pmatrix} 
        1 & 1  & 0 \\ 
        0 & -1 & 1 \\ 
        0 & -1 & 0
    \end{pmatrix} \\
     &  &  \\
    \cycle{1,2} \rightsquigarrow 
    \begin{pmatrix} 
        1 & 1  & 0 \\ 
        0 & -1 & 0 \\ 
        0 & 0  & 1 
    \end{pmatrix} 
    & \cycle{1,3} \rightsquigarrow 
    \begin{pmatrix} 
        1 & 0 & 0 \\ 
        0 & 1 & 1 \\ 
        0 & 0 & 0
    \end{pmatrix} 
    & \cycle{2,3} \rightsquigarrow 
    \begin{pmatrix} 
        1 & 0 & 1 \\ 
        0 & 0 & -1 \\ 
        0 & 1 & -1 
    \end{pmatrix}.
\end{array}
\]
όπως προέβλεψε η Πρόταση~\ref{prop:irreducible_representation_criterion}. Αλλά, ο υπόχωρος $\RR[\bfe_2,\bfe_3]$ \emph{δεν} είναι $\fS_3$-αναλλοίωτος, διότι 
\[
\cycle{1,2} \cdot \bfe_2 = \bfe_1 \ \notin \ \RR[\bfe_2, \bfe_3]
\]
και κατ' επέκταση δεν αποτελεί υποαναπαράσταση. Συνεπώς, η διάσπαση 
\[
\RR^3 = \RR[\bfe_1+\bfe_2+\bfe_3] \oplus \RR[\bfe_2,\bfe_3]
\]
δεν \textquote{σέβεται} της δράση της συμμετρικής ομάδας, σ' αντίθεση με αυτή της Διάσπασης~\eqref{eq:decomposition_R3}, γεγονός το οποίο εξηγεί γιατί οι αντίστοιχοι πίνακες της τελευταίας περίπτωσης είναι μπλοκ-διαγώνιοι.
\end{document}