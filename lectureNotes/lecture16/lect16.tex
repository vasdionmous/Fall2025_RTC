\documentclass[12pt,a4paper,reqno]{amsart}

% language
\usepackage[greek,english]{babel}
\usepackage[utf8]{inputenc}
\usepackage{alphabeta}

% change default names to greek
\addto\captionsenglish{
    \renewcommand{\contentsname}{Περιεχόμενα}
    \renewcommand{\refname}{Βιβλιογραφία}
    \renewcommand{\datename}{Ημερομηνία:}
    \renewcommand{\urladdrname}{Ιστοσελίδα}
}

% math 
\usepackage{amsmath,amsthm,amssymb,amscd}

% font
\usepackage[cal=euler]{mathalfa}
\usepackage{libertinus-type1}
% \usepackage{txfonts} % for upright greek letters
\usepackage{bm} % for bold symbols
\usepackage{bbm} % for the simply-looking bb symbols

% miscellaneous 
\usepackage{changepage} %for indenting environments
\usepackage{csquotes} % example: \textcquote{}
\usepackage{blkarray}
\setcounter{MaxMatrixCols}{20} % default for pmatrix is 10!!
\usepackage{ytableau}

% drawing
\usepackage{tikz,tikz-cd}
\usetikzlibrary{shapes.misc, patterns, matrix, calc, intersections,positioning}
\usepackage{graphics,graphicx}
\usepackage{float} % provides enhanced control and customization options for floating objects such as figures and tables

% colors
\usepackage{xcolor}
\definecolor{darkcandyapplered}{rgb}{0.64, 0.0, 0.0}
\definecolor{midnightblue}{rgb}{0.1, 0.1, 0.44}
\definecolor{mylightblue}{HTML}{336699}
\definecolor{burntorange}{rgb}{0.8, 0.33, 0.0}
\definecolor{iceberg}{rgb}{0.44, 0.65, 0.82}
\definecolor{applegreen}{rgb}{0.55, 0.71, 0.0}
\definecolor{canaryyellow}{rgb}{1.0, 0.94, 0.0}

% hrefs
\usepackage{hyperref}
\usepackage[noabbrev,capitalize]{cleveref}
\hypersetup{
    pdftoolbar=true,        
    pdfmenubar=true,        
    pdffitwindow=false,     
    pdfstartview={FitH},  % fits the width of the page to the window
    pdftitle={},
    pdfauthor={},
    pdfsubject={},
    pdfkeywords={},
    pdfnewwindow=true,  % links in new window
    colorlinks=true,  % false: boxed links; true: colored links
    linkcolor=darkcandyapplered,   % color of internal links
    citecolor=midnightblue,  % color of links to bibliography
    urlcolor=cyan,  % color of external links
    linktocpage=true  % changes the links from the section body to the page number
    }

% geometry
\textwidth=16cm 
\textheight=21cm 
\hoffset=-55pt 
\footskip=25pt

% thm envs (you might need to change the path)
% In this macro I define all the theorem environments

\theoremstyle{definition}
\newtheorem{theorem}{Θεώρημα}
\newtheorem{proposition}[theorem]{Πρόταση}
\newtheorem{lemma}[theorem]{Λήμμα}
\newtheorem{corollary}[theorem]{Πόρισμα}
\newtheorem{conjecture}[theorem]{Εικασία}
\newtheorem{problem}[theorem]{Πρόβλημα}
\newtheorem*{claim}{Ισχυρισμός}
\newtheorem{observation}[theorem]{Παρατήρηση}
\newtheorem{definition}[theorem]{Ορισμός}
\newtheorem{question}[theorem]{Ερώτηση}
\newtheorem*{questions}{Ερωτήματα}
\newtheorem{example}[theorem]{Παράδειγμα}
\newtheorem{exercise}{Άσκηση}

\newtheorem*{combInterlude}{Ιντερλούδιο Συνδυαστικής}
\newtheorem*{example_cont}{Παράδειγμα~6.6}
\newtheorem*{digression_la}{Παρέκβαση Γραμμικής Άλγεβρας}
\newtheorem*{thm}{Θεώρημα}

\theoremstyle{remark}
\newtheorem*{remark}{Παρατήρηση}

% fixes the correct numbering of environments
\numberwithin{theorem}{section}
\numberwithin{exercise}{section}
\numberwithin{equation}{section}

% math ops (you might need to change the path)
% In this macro I define all of my math operators

% fields
\newcommand{\NN}{\mathbbmss{N}} 
\newcommand{\ZZ}{\mathbbmss{Z}} 
\newcommand{\QQ}{\mathbbmss{Q}} 
\newcommand{\RR}{\mathbbmss{R}} 
\newcommand{\CC}{\mathbbmss{C}} 
\newcommand{\KK}{\mathbbmss{K}} 
\newcommand{\FF}{\mathbbmss{F}} 

% symmetric group
\newcommand{\fS}{\mathfrak{S}}  

% calligraphic 
\newcommand{\aA}{\mathcal{A}} 
\newcommand{\bB}{\mathcal{B}}
\newcommand{\cC}{\mathcal{C}}
\newcommand{\dD}{\mathcal{D}}
\newcommand{\eE}{\mathcal{E}}
\newcommand{\fF}{\mathcal{F}}
\newcommand{\hH}{\mathcal{H}}
\newcommand{\iI}{\mathcal{I}}
\newcommand{\lL}{\mathcal{L}}
\newcommand{\oO}{\mathcal{O}}
\newcommand{\pP}{\mathcal{P}}
\newcommand{\sS}{\mathcal{S}}
\newcommand{\mM}{\mathcal{M}}
\newcommand{\uU}{\mathcal{U}}

% bold
\newcommand{\bfa}{\mathbf{a}}
\newcommand{\bfe}{\mathbf{e}}
\newcommand{\bfF}{\pmb{F}}
\newcommand{\bfR}{\pmb{R}}
\newcommand{\bfv}{\mathbf{v}}
%\newcommand{\bfx}{\bm{x}}
%\newcommand{\bfx}{\mathbf{x}} 
\newcommand{\bfx}{\pmb{x}}
\newcommand{\bfX}{\pmb{X}}
\newcommand{\bfy}{\pmb{y}}
\newcommand{\bfz}{\pmb{z}}

% roman
\newcommand{\rmA}{\mathrm{A}}
\newcommand{\rmB}{\mathrm{B}}
\newcommand{\rmC}{\mathrm{C}}
\newcommand{\rmD}{\mathrm{D}} 
\newcommand{\rmI}{\mathrm{I}} 
\newcommand{\rmK}{\mathrm{K}}
\newcommand{\rmM}{\mathrm{M}}
\newcommand{\rmP}{\mathrm{P}}  
\newcommand{\rmp}{\mathrm{p}}  
\newcommand{\rmQ}{\mathrm{Q}}  
\newcommand{\rmR}{\mathrm{R}}
\newcommand{\rmS}{\mathrm{S}}
\newcommand{\rmT}{\mathrm{T}}
\newcommand{\rmU}{\mathrm{U}}
\newcommand{\rmV}{\mathrm{V}}
\newcommand{\rmY}{\mathrm{Y}}
\newcommand{\rmZ}{\mathrm{Z}}
\newcommand{\rmz}{\mathrm{z}}

% greek letters
% I'm renewing some commands in order to appear in upright font
% If I want to change it later, I don't have to do it manually, I just change it from here.
% \newcommand{\uaa}{\alphaup}
% \renewcommand{\alpha}{\alphaup}
% \renewcommand{\beta}{\betaup}
% \renewcommand{\gamma}{\gammaup}
% \renewcommand{\delta}{\deltaup}
% \renewcommand{\epsilon}{\epsilonup}
% \newcommand{\ee}{\epsilon}
% \renewcommand{\varepsilon}{\varepsilonup}
% \renewcommand{\theta}{\thetaup}
% \renewcommand{\lambda}{\lambdaup}
% \newcommand{\ull}{\lambda}
% \renewcommand{\mu}{\muup}
% \renewcommand{\nu}{\nuup}
% \renewcommand{\pi}{\piup}
% \renewcommand{\rho}{\rhoup}
% \renewcommand{\varrho}{\varrhoup}
% \renewcommand{\sigma}{\sigmaup}
% \renewcommand{\tau}{\tauup} 
% \renewcommand{\phi}{\phiup}
% \renewcommand{\chi}{\chiup}
% \renewcommand{\psi}{\psiup}
% \renewcommand{\omega}{\omegaup}

% arrows and symbols 
\renewcommand{\to}{\rightarrow}
\newcommand{\toto}{\longrightarrow}
\newcommand{\mapstoto}{\longmapsto}
\newcommand{\then}{\Rightarrow}
\newcommand{\IFF}{\Leftrightarrow}
\newcommand{\tl}{\tilde}
\newcommand{\wtl}{\widetilde}
\newcommand{\ol}{\overline}
\newcommand{\ul}{\underline}
\newcommand{\oldemptyset}{\emptyset}
\renewcommand{\emptyset}{\varnothing}
\DeclareMathSymbol{\Arg}{\mathbin}{AMSa}{"39} % for arguments 
\newcommand{\onto}{\ensuremath{\twoheadrightarrow}}
\newcommand{\tle}{\trianglelefteq}
\newcommand{\tge}{\trianglerighteq}

% absolute value symbol
\usepackage{mathtools} 
\DeclarePairedDelimiter\abs{\lvert}{\rvert}%
\DeclarePairedDelimiter\norm{\lVert}{\rVert}%
\makeatletter
\let\oldabs\abs
\def\abs{\@ifstar{\oldabs}{\oldabs*}}

% tensor symbol
\newcommand{\tensor}[1]{%
  \mathbin{\mathop{\otimes}\limits_{#1}}%
}

% permutation cycle notation
\ExplSyntaxOn
\NewDocumentCommand{\cycle}{ O{\;} m }
 {
  (
  \alec_cycle:nn { #1 } { #2 }
  )
 }

\seq_new:N \l_alec_cycle_seq
\cs_new_protected:Npn \alec_cycle:nn #1 #2
 {
  \seq_set_split:Nnn \l_alec_cycle_seq { , } { #2 }
  \seq_use:Nn \l_alec_cycle_seq { #1 }
 }
\ExplSyntaxOff

% setminus symbol
\newcommand{\mysetminusD}{\hbox{\tikz{\draw[line width=0.6pt,line cap=round] (3pt,0) -- (0,6pt);}}}
\newcommand{\mysetminusT}{\mysetminusD}
\newcommand{\mysetminusS}{\hbox{\tikz{\draw[line width=0.45pt,line cap=round] (2pt,0) -- (0,4pt);}}}
\newcommand{\mysetminusSS}{\hbox{\tikz{\draw[line width=0.4pt,line cap=round] (1.5pt,0) -- (0,3pt);}}}
\newcommand{\sm}{\mathbin{\mathchoice{\mysetminusD}{\mysetminusT}{\mysetminusS}{\mysetminusSS}}}

% custom math operators
\newcommand{\Des}{\mathrm{Des}} 
\newcommand{\des}{\mathrm{des}} 
\newcommand{\Asc}{\mathrm{Asc}}
\newcommand{\asc}{\mathrm{asc}} 
\newcommand{\inv}{\mathrm{inv}}
\newcommand{\Inv}{\mathrm{Inv}}
\newcommand{\maj}{\mathrm{maj}} 
\newcommand{\comaj}{\mathrm{comaj}} 
\newcommand{\fix}{\mathrm{fix}} 
\newcommand{\Sym}{\mathrm{Sym}} 
\newcommand{\QSym}{\mathrm{QSym}}
\newcommand{\FQSym}{\mathrm{FQSym}} 
\newcommand{\End}{\mathrm{End}} 
\newcommand{\Rad}{\mathrm{Rad}} 
\newcommand{\rmMat}{\mathrm{Mat}} 
\newcommand{\rmdim}{\mathrm{dim}} 
\newcommand{\rmTop}{\mathrm{Top}} 
\newcommand{\rmCF}{\mathrm{CF}} 
\newcommand{\rmId}{\mathrm{Id}}
\newcommand{\rmid}{\mathrm{id}}
\newcommand{\rmtw}{\mathrm{tw}}
\newcommand{\trace}{\mathrm{tr}}
\newcommand{\Irr}{\mathrm{Irr}}
\newcommand{\Ind}{\mathrm{Ind}} % induction
\newcommand{\Res}{\mathrm{Res}} % restriction
\newcommand{\triv}{\mathrm{triv}} % trivial rep
\newcommand{\rmdef}{\mathrm{def}} % defining rep
\newcommand{\dom}{\triangleleft}
\newcommand{\domeq}{\trianglelefteq}
\newcommand{\lex}{\mathrm{lex}}
\newcommand{\sign}{\mathrm{sign}}
\newcommand{\SYT}{\mathrm{SYT}}
\renewcommand{\Im}{\mathrm{Im}}
\newcommand{\Ker}{\mathrm{Ker}}
\newcommand{\GL}{\mathrm{GL}}
\newcommand{\FL}{\mathrm{FL}}
\newcommand{\Span}{\mathrm{span}}
\newcommand{\pos}{\mathrm{pos}}
\newcommand{\Comp}{\mathrm{Comp}}
\newcommand{\Set}{\mathrm{Set}}
\newcommand{\std}{\mathrm{std}}
\newcommand{\cont}{\mathrm{cont}} %content of a SSYT
\newcommand{\SSYT}{\mathrm{SSYT}}
\newcommand{\ct}{\mathrm{ct}} % cycle type
\newcommand{\ch}{\mathrm{ch}} % Frobenius characteristic map
\newcommand{\height}{\mathrm{ht}}
\newcommand{\FPS}{\CC[\![\bfx]\!]} % formal power series
\newcommand{\FPSS}{\CC[\![\bfx,\bfy]\!]}
\newcommand{\reg}{\mathrm{reg}}
\newcommand{\hook}{\mathrm{h}}
\newcommand{\weight}{\mathrm{wt}}
\newcommand{\co}{\mathrm{co}}
\newcommand{\ps}{\mathrm{ps}}
\newcommand{\rmsum}{\mathrm{sum}}
\newcommand{\NSym}{\mathrm{NSym}}
\newcommand{\Hom}{\mathrm{Hom}}
\newcommand{\proj}{\mathrm{proj}}
\newcommand{\stat}{\mathrm{stat}}
\newcommand{\Par}{\mathrm{Par}}
\newcommand{\rmset}{\mathrm{set}}
\newcommand{\comp}{\mathrm{comp}}

% miscellaneous commands
\newcommand{\defn}[1]{{\color{mylightblue}{#1}}}
\newcommand{\toDo}{{\bf\color{red} TODO}}
\newcommand{\toCite}{{\bf\color{green} CITE}}
\newcommand*{\vertbar}{\rule[-1ex]{0.5pt}{2.5ex}} % for matrices with column vectors
\newcommand*{\horzbar}{\rule[.5ex]{2.5ex}{0.5pt}} % for matrices with row vectors
\newcommand{\myblue}[1]{{\color{iceberg}{#1}}}
\newcommand{\myorange}[1]{{\color{burntorange}{#1}}}
\newcommand{\mygreen}[1]{{\color{applegreen}{#1}}}
\newcommand{\myred}[1]{{\color{darkcandyapplered}{#1}}}

% ferrer's diagram
\newcommand{\fdiagram}[1]{
    \begin{tikzpicture}[scale=.7]
        \fill foreach \Z [count=\Y] in {#1}
        {foreach \X in {1,...,\Z} 
        {(\X,-\Y) circle[radius=3pt]}};
    \end{tikzpicture}
}

%
\newcommand{\tcbo}[1]{\textcolor{burntorange}{#1}}

% 
\newenvironment{nouppercase}{%
  \let\uppercase\relax%
  \renewcommand{\uppercasenonmath}[1]{}}{}

% titlepage
\title{Θ2.04: Θεωρία Αναπαραστάσεων και Συνδυαστική}
\author[Β.~Δ. Μουστακας]{Βασίλης Διονύσης Μουστάκας \\ Πανεπιστήμιο Κρήτης}
\date{26 Νοεμβρίου 2025}
% \urladdr{\href{https://sites.google.com/view/vasmous}{https://sites.google.com/view/vasmous}}

\begin{document}

\begingroup
\def\uppercasenonmath#1{} % this disables uppercase title
\let\MakeUppercase\relax % this disables uppercase authors
\maketitle
\endgroup

\setcounter{section}{12}
\setcounter{theorem}{0}
\begin{center}
    \textbf{12. Συνήθη Young ταμπλώ και το θεώρημα βάσης
} 
\end{center}

Από τον Ορισμό 11.2 δεν μπορούμε να αποφανθούμε ποιά είναι η διάσταση ενός προτύπου Specht, καθώς εν γένει υπάρχουν γραμμικές σχέσεις που ικανοποιούνται από πολυταμπλοειδή ίδιου σχήματος. Συνεπώς, είναι φυσικό να αναρωτηθεί κανείς το εξής.
\begin{que}
    Ποιά πολυταμπλοειδή αποτελούν βάση ενός προτύπου Specht και πόσα είναι αυτά;
\end{que}

Για $\lambda = (2,1) \vdash 3$, στο Παράδειγμα 11.3 (1), είδαμε ότι το 
\[
\ytableausetup{smalltableaux,notabloids}
\sS^{(2,1)} = \CC[\bfe_{\ytableaushort{12, 3}}\, , \bfe_{\ytableaushort{13, 2}} \,]
\]
είναι ισόμορφο με την συνήθη αναπαράσταση της $\fS_3$ και γι αυτό $\dim(\sS^{(2,1)}) = 2$. Οι γραμμικές σχέσεις που ικανοποιούν τα υπόλοιπα τέσσερα πολυταμπλοειδή είναι 
\begin{align*}
    \bfe_{\ytableaushort{21, 3}} \ 
    &= \ \bfe_{\ytableaushort{12, 3}} - \bfe_{\ytableaushort{13, 2}} \\
    \bfe_{\ytableaushort{31, 2}} \ 
    &= \ \bfe_{\ytableaushort{13, 2}} - \bfe_{\ytableaushort{12, 3}} \\ 
    \bfe_{\ytableaushort{23, 1}} \
    &= \ - \bfe_{\ytableaushort{13, 2}} \\ 
    \bfe_{\ytableaushort{32, 1}} \
    &= \ - \bfe_{\ytableaushort{12, 3}} \\ 
\end{align*}
(γιατί;). 

\begin{definition}
    \label{def:SYT}
    Έστω $\lambda \vdash n.$ Ένα Young ταμπλώ σχήματος $\lambda$ ονομάζεται \defn{σύνηθες} (standard) αν 
    \begin{itemize}
        \item τα στοιχεία της κάθε γραμμής του αυξάνουν από αριστερά προς τα δεξιά\footnote{Στην περίπτωση αυτή λέμε ότι το ταμπλώ είναι \emph{row-standard}.}, και
        \item τα στοιχεία της κάθε στήλης του αυξάνουν από πάνω προς τα κάτω\footnote{Στην περίπτωση αυτή λέμε ότι το ταμπλώ είναι \emph{column-standard}.}.
    \end{itemize}
    Έστω $\SYT(\lambda)$ το σύνολο των συνήθων Young ταμπλώ σχήματος $\lambda$.
\end{definition}

Ένα παράδειγμα συνήθους ταμπλώ σχήματος $(4,2,2,1)$ είναι 
\[
\ytableausetup{nosmalltableaux,centertableaux}
\begin{ytableau}
1 & 3 & 7 & 8 \\
2 & 5 \\
4 & 9 \\
6 \\
\end{ytableau}.
\]

\begin{example}
Έστω $f^\lambda \coloneqq \abs{\SYT(\lambda)}$.
\begin{itemize}
    \item[(1)] Προφανώς, $f^{(n)} = f^{(1^n)} = 1$.
    \item[(2)] Έχουμε $f^{(n-1,1)} = n-1$, διότι από την πρώτη συνθήκη του Ορισμού \ref{def:SYT}, η επιλογή του στοιχείου της δεύτερης γραμμής καθορίζει ένα σύνηθες ταμπλώ σχήματος $(n-1,1)$ (γιατί;). Γενικότερα\footnote{Με $\binom{n}{k}$ συμβολίζουμε το πλήθος των υποσυνόλων του $[n]$ με $k$ στοιχεία. Οι ακέραιοι $\binom{n}{k}$ ονομάζονται \defn{διωνυμικοί συντελεστές}.} 
    \[
    f^{(n-k,1^k)} = \binom{n-1}{k}
    \] 
    (βλ. Φυλλάδιο Ασκήσεων 4).
    \item[(3)] Τα συνήθη ταμπλώ σχήματος $(3,3)$ είναι  
    \[
    \ytableaushort{123,456} \, , \
    \ytableaushort{124,356} \, , \ 
    \ytableaushort{125,346} \, , \ 
    \ytableaushort{134,256} \, , \ 
    \ytableaushort{135,246} \, 
    \]
    και γι αυτό $f^{(3,3)} = 5$. Γενικότερα\footnote{Ο αριθμός $\frac{1}{n+1}\binom{2n}{n}$ ονομάζεται $n$-οστός \defn{αριθμός Catalan}.},
    \[
    f^{(n,n)} = \frac{1}{n+1}\binom{2n}{n}.
    \]
\end{itemize}
\end{example}

\begin{theorem}{\rm(Θεώρημα βάσης)}
    \label{thm:basis_thm}
    Έστω $\lambda \vdash n$. Το σύνολο 
    \[
    \{\bfe_T : T \in \SYT(\lambda)\}
    \]
    αποτελεί βάση του προτύπου Specht $\sS^\lambda$. Ειδικότερα,
    \[
    \dim(\sS^\lambda) = f^\lambda.
    \]
\end{theorem}

Για να προετοιμαστούμε για την απόδειξη του Θεωρήματος \ref{thm:basis_thm} χρειαζόμαστε μια διάταξη στο σύνολο των ταμπλειδών ίδιου σχήματος.

\begin{definition}
    \label{def:young_last_letter_order}
    Έστω $T$ και $Q$ δυο ταμπλώ ίδιου σχήματος. Γράφουμε $[T] < [Q]$ αν 
    \begin{itemize}
        \item υπάρχει $i \in [n]$, τέτοιο ώστε $r_T(i) \neq r_Q(i)$
        \item για το μεγαλύτερο τέτοιο $i$ ισχύει ότι $r_T(i) < r_Q(i)$.
    \end{itemize}
\end{definition}

Με άλλα λόγια, $[T] < [Q]$ αν και μόνο αν το μεγαλύτερο στοιχέιο που εμφανίζεται σε διαφορετικές γραμμές των $T$ και $Q$ βρίσκεται βορειότερα στο $[T]$ από ότι στο $[Q]$. Για παράδειγμα, 
\[
\ytableausetup{tabloids}
\begin{ytableau}
2 & 3 & 7 \\
4 & *(burntorange)6 \\
1 & 5 
\end{ytableau} 
\ < \ 
\ytableausetup{tabloids}
\begin{ytableau}
3 & 5 & 7 \\
1 & 4 \\
2 & *(burntorange)6 
\end{ytableau} \ .
\]
Η σχέση $<$ αποτελεί ολική διάταξη στο σύνολο των ταμπλειδών ίδιου σχήματος. Το ταμπλοδειδες που βρίσκεται  χαμηλότερα σε αυτή τη διάταξη, έχει τα μεγαλύτερα στοιχεία του συγκεντρωμένα σε υψηλότερες γραμμές. Για παράδειγμα, για $\lambda = (2,2)$ έχουμε 
\[
\ytableausetup{nosmalltableaux}
\ytableaushort{34,12} \ < \
\ytableaushort{24,13} \ < \
\ytableaushort{14,23} \ < \
\ytableaushort{23,14} \ < \
\ytableaushort{13,24} \ < \
\ytableaushort{12,34} \ .
\]

\begin{lemma}
    \label{lem:basis_thm}
    Αν $T$ είναι ταμπλώ σχήματος  $\lambda \vdash n$, τέτοιο ώστε τα στοιχεία της κάθε στήλης του $T$ να αυξάνουν από πάνω προς τα κάτω, τότε
    \begin{itemize}
        \item[(1)] υπάρχει μια μετάθεση $\pi \in \rmC(T)$ με $\pi \neq \epsilon$ για την οποία $[\pi{T}] < [T]$, και
        \item[(2)] $\bfe_T = [T] +  \left(\text{γραμμικός συνδυασμός ταμπλοειδών $[Q]$ σχήματος $\lambda$ με $[Q] < [T]$}\right)$. 
    \end{itemize}
\end{lemma}

\begin{proof}[Απόδειξη]
    Το (2) έπεται άμεσα από το (1), διότι 
    \begin{align*}
        \bfe_T 
        = \nabla_T^-[T] 
        = \sum_{\pi \in \rmC(T)} \sign(\pi) [\pi{T}] 
        = [T] + \sum_{\substack{\pi \in \rmC(T) \\ \pi \neq \epsilon}} \sign(\pi) \underbrace{[\pi{T}]}_{< \ [T]}.
    \end{align*}

    Για το (1), έστω $\pi \in \rmC(T)$ με $\pi \neq \epsilon$. Θεωρούμε τον μεγαλύτερο ακέραιο $k \in [n]$ τέτοιο ώστε $\pi(k) \neq k$. Κάθε στοιχείο νότια του $k$ στο $T$ μένει σταθερό στο $\pi{T}$ και το $k$ μετατίθεται βορειότερα (γιατί;). Άρα, $[\pi{T}] < [T]$. 
    
    Για παράδειγμα, για
    \[
    \ytableausetup{notabloids}
    T = \
    \begin{ytableau}
        1 & 3 & 5 & 4 \\
        2 & *(burntorange)6 & 7 \\
        9 & *(iceberg)8 
    \end{ytableau} 
    \quad 
    \text{και}
    \quad
    \pi = \cycle{3,6}
    \]
    με $k=6$, έχουμε 
    \[
    \pi{T} = \
    \begin{ytableau}
        1 & *(burntorange)6 & 5 & 4 \\
        2 & 3 & 7 \\
        9 & *(iceberg)8 
    \end{ytableau}
    \]
    και γι αυτό 
    \[
    \ytableausetup{tabloids}
    [\pi{T}] = \ 
    \begin{ytableau}
        1 & *(burntorange)6 & 5 & 4 \\
        2 & 3 & 7 \\
        9 & 8 
    \end{ytableau}
    \ < \ 
    \begin{ytableau}
        1 & 3 & 5 & 4 \\
        2 & *(burntorange)6 & 7 \\
        9 & 8 
    \end{ytableau} 
    \ = [T].
    \]
\end{proof}

θα χρησιμοποιήσουμε το Λήμμα \ref{lem:basis_thm} για να αποδείξουμε το θεώρημα βάσης.

\begin{proof}[Απόδειξη του Θεωρήματος~\ref{thm:basis_thm}]
    Θα δείξουμε ότι το σύνολο $\{\bfe_T : T \in \SYT(\lambda)\}$ είναι γραμμικώς ανεξάρτητο υποσύνολο του $\rmM^\lambda$. Για τον σκοπό αυτό, υποθέτουμε ότι 
    \[
    c_{T_1} \bfe_{T_1} + c_{T_2} \bfe_{T_2} + \cdots + c_{T_k} \bfe_{T_k} = 0 \quad \text{στο $\rmM^\lambda$},
    \]
    για κάποια $c_{T_1}, c_{T_2}, \dots, c_{T_k} \neq 0$ και $T_1, T_2, \dots, T_k \in \SYT(\lambda)$. Έστω $T_{\max}$ το σύνηθες ταμπλώ εκ των $T_1, T_2, \dots, T_k$ του οποίου το αντίστοιχο ταμπλειδές είναι το μεγαλύτερο ως προς τη διάταξη του Ορισμού \ref{def:young_last_letter_order}. 
    \begin{claim}
        Για κάθε $1 \le i \le k$, αν 
        \[
        \bfe_{T_i} = \sum_{[Q]} a_{[Q]}^i [Q],
        \]
        όπου στο άθροισμα το $[Q]$ διατρέχει όλα τα ταμπλοειδή σχήματος $\lambda$, για κάποια $a_{[Q]}^i \in \CC$, τότε 
        \[
        a_{[T_{\max}]}^i = 
        \begin{cases}
            1, &\ \text{αν $T_i = T_{\max}$} \\
            0, &\ \text{διαφορετικά} \\
        \end{cases}. 
        \]
    \end{claim}
    O ισχυρισμός έπεται άμεσα από το Λήμμα \ref{lem:basis_thm}. Πράγματι, το (2) του λήμματος μας πληροφορεί ότι 
    \[
    \bfe_{T_i} = 
    [T_i] + \ \left(\text{γραμμικός συνδυασμός ταμπλοειδών $[Q]$ σχήματος $\lambda$ με $[Q] < [T_i]$}\right)
    \]
    Συνεπώς, το $[T_{\max}]$ θα εμφανιστεί στο ανάπτυγμα κάποιου $\bfe_{T_i}$ μόνο αν $T_i = T_{\max}$ και σε αυτή την περίπτωση με συντελεστή 1 (γιατί;).

    Από τον ισχυρισμό, ο συντελεστής του $[T_{\max}]$ στο ανάπτυγμα του 
    \[
    c_{T_1} \bfe_{T_1} + c_{T_2} \bfe_{T_2} + \cdots + c_{T_k} \bfe_{T_k} \ \in \ \rmM^\lambda
    \]
    στην βάση των ταμπλοδειδών σχήματος $\lambda$ ισούται με 
    \[
    \sum_{i=1}^k c_{T_i}a_{[T_{\max}]}^i = c_{T_{\max}} \neq 0.
    \]
    Αυτό όμως είναι αδύνατο, διότι 
    \[
    c_{T_1} \bfe_{T_1} + c_{T_2} \bfe_{T_2} + \cdots + c_{T_k} \bfe_{T_k} = 0.
    \]

    Για να ολοκληρωθεί η απόδειξη, μένει να δείξουμε ότι το σύνολο $\{\bfe_T : T \in \SYT(\lambda)\}$ παράγει το $\sS^\lambda$. Από τον τύπο διάστασης (βλ. Πόρισμα 4.2), έχουμε
    \[
    n! = \sum_{\lambda \vdash n} \dim(\sS^\lambda)^2 \ge \sum_{\lambda \vdash n} (f^\lambda)^2,
    \]
    όπου η τελευταία ανισότητα έπεται από την γραμμική ανεξαρτησία του $\{\bfe_T : T \in \SYT(\lambda)\}$. Το ζητούμενο, δηλαδή ότι $\dim(\sS^\lambda) = f^\lambda$, έπεται από το επόμενο αποτέλεσμα\footnote{Για μια κατασκευαστική απόδειξη, θα έπρεπε να ξεκινήσουμε με ένα πολυταμπλοειδές $\bfe_T$ για κάποιο ταμπλώ $T$ και να δείξουμε ότι μπορεί να γραφεί ως γραμμικός συνδυασμός πολυταμπλοειδών $\bfe_Q$ για συνήθη ταμπλώ $Q$. Μια τέτοια απόδειξη δόθηκε από τον Garnir το 1950, επινοώντας έναν αλγόριθμο για να το κάνει αυτό, ο οποίος σήμερα ονομάζεται \emph{straightening algorithm}.}.
\end{proof}

Η αντιστοιχία του ακόλουθου θεωρήματος ονομάζεται \emph{αντιστοιχία Robinson--Schensted} και δίνει μια συνδυαστική απόδειξη του τύπου διάστασης για την συμμετρική ομάδα (συγκρίνετε με την συζήτηση των τελευταίων παραγράφων της Παραγράφου 4).
\begin{theorem}{\rm(Robinson 1936, Schensted 1961)}
    \label{thm:RS_correspondence}
    Υπάρχει μια αμφιμονοσήμαντη αντιστοιχία μεταξύ των μεταθέσεων της $\fS_n$ και ζευγών συνήθων Young ταμπλώ ίδιου σχήματος και περιεχομένου $[n]$. Ειδικότερα,
    \[
    n! = \sum_{\lambda \vdash n} (f^\lambda)^2.
    \]
\end{theorem}

Ολοκληρώνουμε την παρούσα παράγραφο, διατυπώνοντας έναν τύπο για την διάσταση του προτύπου Specht.
\begin{definition}
    \label{def:hook_length}
    Έστω $\lambda \vdash n$. Για ένα τετράγωνο $x$ του $\rmY_\lambda$, το οποίο συμβολικά γράφουμε $x \in \rmY_\lambda$, το σύνολο $\rmH(x)$ των τετραγώνων που βρίσκονται ασθενώς ανατολικά ή ασθενώς νότια του $x$ στο $\rmY_\lambda$ ονομάζεται \defn{γάντζος} (hook) του $x$. Θέτουμε $\rmh(x) \coloneqq \abs{\rmH(x)}$. 
\end{definition}

Για παράδειγμα, για $\lambda = (4,2,2,1)$ ο γάντζος του τετραγώνου $x$  
\[
\ytableausetup{notabloids}
\rmY_{(4,2,2,1)} = 
\begin{ytableau}
    {} & x & {} & {} \\
    {} & {} \\
    {} & {} \\
    {}
\end{ytableau}
\]
που βρίσκεται στη θέση (1,2) είναι 
\[
\ydiagram{4,2,2,1}
*[*(burntorange)]{1+1,1+1,1+1}
*[*(burntorange)]{2+2}
\]
και γι αυτό $\rmh(x) = 5$.

Το επόμενο αποτέλεσμα, γνωστό ως \emph{hook length formula}, δίνει έναν αναπάντεχα απλό\footnote{Δεν υπάρχει κανένας προφανής λόγος που να εξηγεί το ότι ο αριθμός στο δεξί μέλος της Ταυτότητας \eqref{eq:hook_length_formula} είναι ακέραιος.} τύπο για το πλήθος των συνήθων ταμπλώ δοσμένου σχήματος, και κατ' επέκταση τη διάσταση του προτύπου Specht.
\begin{theorem}{\rm(Frame--Robinson--Thrall 1954)}
    \label{thm:hook_length_formula}
    Για $\lambda \vdash n$, 
    \begin{equation} 
        \label{eq:hook_length_formula}
        f^\lambda = 
        \frac{n!}{\prod_{x \in \rmY_\lambda} \rmh(x)}.
    \end{equation}
\end{theorem}

Για παράδειγμα, ας υπολογίσουμε το πλήθος των συνήθων ταμπλώ σχήματος $\lambda = (4,2,$ $2,1)$. Γεμίζουμε τα τετράγωνα του διαγράμματος Young της $\lambda$ με τα μήκη κάθε γάντζου του αντίστοιχου τετραγώνου ως εξής 
\[
\begin{ytableau}
    7 & 5 & 2 & 1 \\
    4 & 2 \\
    3 & 1 \\
    1
\end{ytableau}
\]
Τότε, η Ταυτότητα~\eqref{eq:hook_length_formula} μας πληροφορεί ότι 
\[
f^{(4,2,2,1)} = \frac{9!}{1^3\cdot2^3\cdot3\cdot4\cdot5\cdot7} = 108.
\]
\end{document}