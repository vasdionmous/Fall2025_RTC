\documentclass[12pt,a4paper,reqno]{amsart}

% language
\usepackage[greek,english]{babel}
\usepackage[utf8]{inputenc}
\usepackage{alphabeta}

% change default names to greek
\addto\captionsenglish{
    \renewcommand{\contentsname}{Περιεχόμενα}
    \renewcommand{\refname}{Βιβλιογραφία}
    \renewcommand{\datename}{Ημερομηνία:}
    \renewcommand{\urladdrname}{Ιστοσελίδα}
}

% math 
\usepackage{amsmath,amsthm,amssymb,amscd}

% font
\usepackage[cal=euler]{mathalfa}
\usepackage{libertinus-type1}
% \usepackage{txfonts} % for upright greek letters
\usepackage{bm} % for bold symbols
\usepackage{bbm} % for the simply-looking bb symbols

% miscellaneous 
\usepackage{changepage} %for indenting environments
\usepackage{csquotes} % example: \textcquote{}
\usepackage{blkarray}
\setcounter{MaxMatrixCols}{20} % default for pmatrix is 10!!
\usepackage{ytableau}

% drawing
\usepackage{tikz,tikz-cd}
\usetikzlibrary{shapes.misc, patterns, matrix, calc, intersections,positioning}
\usepackage{graphics,graphicx}
\usepackage{float} % provides enhanced control and customization options for floating objects such as figures and tables

% colors
\usepackage{xcolor}
\definecolor{darkcandyapplered}{rgb}{0.64, 0.0, 0.0}
\definecolor{midnightblue}{rgb}{0.1, 0.1, 0.44}
\definecolor{mylightblue}{HTML}{336699}
\definecolor{burntorange}{rgb}{0.8, 0.33, 0.0}
\definecolor{iceberg}{rgb}{0.44, 0.65, 0.82}
\definecolor{applegreen}{rgb}{0.55, 0.71, 0.0}
\definecolor{canaryyellow}{rgb}{1.0, 0.94, 0.0}

% hrefs
\usepackage{hyperref}
\usepackage[noabbrev,capitalize]{cleveref}
\hypersetup{
    pdftoolbar=true,        
    pdfmenubar=true,        
    pdffitwindow=false,     
    pdfstartview={FitH},  % fits the width of the page to the window
    pdftitle={},
    pdfauthor={},
    pdfsubject={},
    pdfkeywords={},
    pdfnewwindow=true,  % links in new window
    colorlinks=true,  % false: boxed links; true: colored links
    linkcolor=darkcandyapplered,   % color of internal links
    citecolor=midnightblue,  % color of links to bibliography
    urlcolor=cyan,  % color of external links
    linktocpage=true  % changes the links from the section body to the page number
    }

% geometry
\textwidth=16cm 
\textheight=21cm 
\hoffset=-55pt 
\footskip=25pt

% thm envs (you might need to change the path)
% In this macro I define all the theorem environments

\theoremstyle{definition}
\newtheorem{theorem}{Θεώρημα}
\newtheorem{proposition}[theorem]{Πρόταση}
\newtheorem{lemma}[theorem]{Λήμμα}
\newtheorem{corollary}[theorem]{Πόρισμα}
\newtheorem{conjecture}[theorem]{Εικασία}
\newtheorem{problem}[theorem]{Πρόβλημα}
\newtheorem*{claim}{Ισχυρισμός}
\newtheorem{observation}[theorem]{Παρατήρηση}
\newtheorem{definition}[theorem]{Ορισμός}
\newtheorem{question}[theorem]{Ερώτηση}
\newtheorem*{questions}{Ερωτήματα}
\newtheorem{example}[theorem]{Παράδειγμα}
\newtheorem{exercise}{Άσκηση}

\newtheorem*{combInterlude}{Ιντερλούδιο Συνδυαστικής}
\newtheorem*{example_cont}{Παράδειγμα~6.6}
\newtheorem*{digression_la}{Παρέκβαση Γραμμικής Άλγεβρας}
\newtheorem*{thm}{Θεώρημα}

\theoremstyle{remark}
\newtheorem*{remark}{Παρατήρηση}

% fixes the correct numbering of environments
\numberwithin{theorem}{section}
\numberwithin{exercise}{section}
\numberwithin{equation}{section}

% math ops (you might need to change the path)
% In this macro I define all of my math operators

% fields
\newcommand{\NN}{\mathbbmss{N}} 
\newcommand{\ZZ}{\mathbbmss{Z}} 
\newcommand{\QQ}{\mathbbmss{Q}} 
\newcommand{\RR}{\mathbbmss{R}} 
\newcommand{\CC}{\mathbbmss{C}} 
\newcommand{\KK}{\mathbbmss{K}} 
\newcommand{\FF}{\mathbbmss{F}} 

% symmetric group
\newcommand{\fS}{\mathfrak{S}}  

% calligraphic 
\newcommand{\aA}{\mathcal{A}} 
\newcommand{\bB}{\mathcal{B}}
\newcommand{\cC}{\mathcal{C}}
\newcommand{\dD}{\mathcal{D}}
\newcommand{\eE}{\mathcal{E}}
\newcommand{\fF}{\mathcal{F}}
\newcommand{\hH}{\mathcal{H}}
\newcommand{\iI}{\mathcal{I}}
\newcommand{\lL}{\mathcal{L}}
\newcommand{\oO}{\mathcal{O}}
\newcommand{\pP}{\mathcal{P}}
\newcommand{\sS}{\mathcal{S}}
\newcommand{\mM}{\mathcal{M}}
\newcommand{\uU}{\mathcal{U}}

% bold
\newcommand{\bfa}{\mathbf{a}}
\newcommand{\bfe}{\mathbf{e}}
\newcommand{\bfF}{\pmb{F}}
\newcommand{\bfR}{\pmb{R}}
\newcommand{\bfv}{\mathbf{v}}
%\newcommand{\bfx}{\bm{x}}
%\newcommand{\bfx}{\mathbf{x}} 
\newcommand{\bfx}{\pmb{x}}
\newcommand{\bfX}{\pmb{X}}
\newcommand{\bfy}{\pmb{y}}
\newcommand{\bfz}{\pmb{z}}

% roman
\newcommand{\rmA}{\mathrm{A}}
\newcommand{\rmB}{\mathrm{B}}
\newcommand{\rmC}{\mathrm{C}}
\newcommand{\rmD}{\mathrm{D}} 
\newcommand{\rmI}{\mathrm{I}} 
\newcommand{\rmK}{\mathrm{K}}
\newcommand{\rmM}{\mathrm{M}}
\newcommand{\rmP}{\mathrm{P}}  
\newcommand{\rmp}{\mathrm{p}}  
\newcommand{\rmQ}{\mathrm{Q}}  
\newcommand{\rmR}{\mathrm{R}}
\newcommand{\rmS}{\mathrm{S}}
\newcommand{\rmT}{\mathrm{T}}
\newcommand{\rmU}{\mathrm{U}}
\newcommand{\rmV}{\mathrm{V}}
\newcommand{\rmY}{\mathrm{Y}}
\newcommand{\rmZ}{\mathrm{Z}}
\newcommand{\rmz}{\mathrm{z}}

% greek letters
% I'm renewing some commands in order to appear in upright font
% If I want to change it later, I don't have to do it manually, I just change it from here.
% \newcommand{\uaa}{\alphaup}
% \renewcommand{\alpha}{\alphaup}
% \renewcommand{\beta}{\betaup}
% \renewcommand{\gamma}{\gammaup}
% \renewcommand{\delta}{\deltaup}
% \renewcommand{\epsilon}{\epsilonup}
% \newcommand{\ee}{\epsilon}
% \renewcommand{\varepsilon}{\varepsilonup}
% \renewcommand{\theta}{\thetaup}
% \renewcommand{\lambda}{\lambdaup}
% \newcommand{\ull}{\lambda}
% \renewcommand{\mu}{\muup}
% \renewcommand{\nu}{\nuup}
% \renewcommand{\pi}{\piup}
% \renewcommand{\rho}{\rhoup}
% \renewcommand{\varrho}{\varrhoup}
% \renewcommand{\sigma}{\sigmaup}
% \renewcommand{\tau}{\tauup} 
% \renewcommand{\phi}{\phiup}
% \renewcommand{\chi}{\chiup}
% \renewcommand{\psi}{\psiup}
% \renewcommand{\omega}{\omegaup}

% arrows and symbols 
\renewcommand{\to}{\rightarrow}
\newcommand{\toto}{\longrightarrow}
\newcommand{\mapstoto}{\longmapsto}
\newcommand{\then}{\Rightarrow}
\newcommand{\IFF}{\Leftrightarrow}
\newcommand{\tl}{\tilde}
\newcommand{\wtl}{\widetilde}
\newcommand{\ol}{\overline}
\newcommand{\ul}{\underline}
\newcommand{\oldemptyset}{\emptyset}
\renewcommand{\emptyset}{\varnothing}
\DeclareMathSymbol{\Arg}{\mathbin}{AMSa}{"39} % for arguments 
\newcommand{\onto}{\ensuremath{\twoheadrightarrow}}
\newcommand{\tle}{\trianglelefteq}
\newcommand{\tge}{\trianglerighteq}

% absolute value symbol
\usepackage{mathtools} 
\DeclarePairedDelimiter\abs{\lvert}{\rvert}%
\DeclarePairedDelimiter\norm{\lVert}{\rVert}%
\makeatletter
\let\oldabs\abs
\def\abs{\@ifstar{\oldabs}{\oldabs*}}

% tensor symbol
\newcommand{\tensor}[1]{%
  \mathbin{\mathop{\otimes}\limits_{#1}}%
}

% permutation cycle notation
\ExplSyntaxOn
\NewDocumentCommand{\cycle}{ O{\;} m }
 {
  (
  \alec_cycle:nn { #1 } { #2 }
  )
 }

\seq_new:N \l_alec_cycle_seq
\cs_new_protected:Npn \alec_cycle:nn #1 #2
 {
  \seq_set_split:Nnn \l_alec_cycle_seq { , } { #2 }
  \seq_use:Nn \l_alec_cycle_seq { #1 }
 }
\ExplSyntaxOff

% setminus symbol
\newcommand{\mysetminusD}{\hbox{\tikz{\draw[line width=0.6pt,line cap=round] (3pt,0) -- (0,6pt);}}}
\newcommand{\mysetminusT}{\mysetminusD}
\newcommand{\mysetminusS}{\hbox{\tikz{\draw[line width=0.45pt,line cap=round] (2pt,0) -- (0,4pt);}}}
\newcommand{\mysetminusSS}{\hbox{\tikz{\draw[line width=0.4pt,line cap=round] (1.5pt,0) -- (0,3pt);}}}
\newcommand{\sm}{\mathbin{\mathchoice{\mysetminusD}{\mysetminusT}{\mysetminusS}{\mysetminusSS}}}

% custom math operators
\newcommand{\Des}{\mathrm{Des}} 
\newcommand{\des}{\mathrm{des}} 
\newcommand{\Asc}{\mathrm{Asc}}
\newcommand{\asc}{\mathrm{asc}} 
\newcommand{\inv}{\mathrm{inv}}
\newcommand{\Inv}{\mathrm{Inv}}
\newcommand{\maj}{\mathrm{maj}} 
\newcommand{\comaj}{\mathrm{comaj}} 
\newcommand{\fix}{\mathrm{fix}} 
\newcommand{\Sym}{\mathrm{Sym}} 
\newcommand{\QSym}{\mathrm{QSym}}
\newcommand{\FQSym}{\mathrm{FQSym}} 
\newcommand{\End}{\mathrm{End}} 
\newcommand{\Rad}{\mathrm{Rad}} 
\newcommand{\rmMat}{\mathrm{Mat}} 
\newcommand{\rmdim}{\mathrm{dim}} 
\newcommand{\rmTop}{\mathrm{Top}} 
\newcommand{\rmCF}{\mathrm{CF}} 
\newcommand{\rmId}{\mathrm{Id}}
\newcommand{\rmid}{\mathrm{id}}
\newcommand{\rmtw}{\mathrm{tw}}
\newcommand{\trace}{\mathrm{tr}}
\newcommand{\Irr}{\mathrm{Irr}}
\newcommand{\Ind}{\mathrm{Ind}} % induction
\newcommand{\Res}{\mathrm{Res}} % restriction
\newcommand{\triv}{\mathrm{triv}} % trivial rep
\newcommand{\rmdef}{\mathrm{def}} % defining rep
\newcommand{\dom}{\triangleleft}
\newcommand{\domeq}{\trianglelefteq}
\newcommand{\lex}{\mathrm{lex}}
\newcommand{\sign}{\mathrm{sign}}
\newcommand{\SYT}{\mathrm{SYT}}
\renewcommand{\Im}{\mathrm{Im}}
\newcommand{\Ker}{\mathrm{Ker}}
\newcommand{\GL}{\mathrm{GL}}
\newcommand{\FL}{\mathrm{FL}}
\newcommand{\Span}{\mathrm{span}}
\newcommand{\pos}{\mathrm{pos}}
\newcommand{\Comp}{\mathrm{Comp}}
\newcommand{\Set}{\mathrm{Set}}
\newcommand{\std}{\mathrm{std}}
\newcommand{\cont}{\mathrm{cont}} %content of a SSYT
\newcommand{\SSYT}{\mathrm{SSYT}}
\newcommand{\ct}{\mathrm{ct}} % cycle type
\newcommand{\ch}{\mathrm{ch}} % Frobenius characteristic map
\newcommand{\height}{\mathrm{ht}}
\newcommand{\FPS}{\CC[\![\bfx]\!]} % formal power series
\newcommand{\FPSS}{\CC[\![\bfx,\bfy]\!]}
\newcommand{\reg}{\mathrm{reg}}
\newcommand{\hook}{\mathrm{h}}
\newcommand{\weight}{\mathrm{wt}}
\newcommand{\co}{\mathrm{co}}
\newcommand{\ps}{\mathrm{ps}}
\newcommand{\rmsum}{\mathrm{sum}}
\newcommand{\NSym}{\mathrm{NSym}}
\newcommand{\Hom}{\mathrm{Hom}}
\newcommand{\proj}{\mathrm{proj}}
\newcommand{\stat}{\mathrm{stat}}
\newcommand{\Par}{\mathrm{Par}}
\newcommand{\rmset}{\mathrm{set}}
\newcommand{\comp}{\mathrm{comp}}

% miscellaneous commands
\newcommand{\defn}[1]{{\color{mylightblue}{#1}}}
\newcommand{\toDo}{{\bf\color{red} TODO}}
\newcommand{\toCite}{{\bf\color{green} CITE}}
\newcommand*{\vertbar}{\rule[-1ex]{0.5pt}{2.5ex}} % for matrices with column vectors
\newcommand*{\horzbar}{\rule[.5ex]{2.5ex}{0.5pt}} % for matrices with row vectors
\newcommand{\myblue}[1]{{\color{iceberg}{#1}}}
\newcommand{\myorange}[1]{{\color{burntorange}{#1}}}
\newcommand{\mygreen}[1]{{\color{applegreen}{#1}}}
\newcommand{\myred}[1]{{\color{darkcandyapplered}{#1}}}

% ferrer's diagram
\newcommand{\fdiagram}[1]{
    \begin{tikzpicture}[scale=.7]
        \fill foreach \Z [count=\Y] in {#1}
        {foreach \X in {1,...,\Z} 
        {(\X,-\Y) circle[radius=3pt]}};
    \end{tikzpicture}
}

%
\newcommand{\tcbo}[1]{\textcolor{burntorange}{#1}}

% 
\newenvironment{nouppercase}{%
  \let\uppercase\relax%
  \renewcommand{\uppercasenonmath}[1]{}}{}

% titlepage
\title{Θ2.04: Θεωρία Αναπαραστάσεων και Συνδυαστική}
\author[Β.~Δ. Μουστακας]{Βασίλης Διονύσης Μουστάκας \\ Πανεπιστήμιο Κρήτης}
\date{27 Νοεμβρίου 2025}
% \urladdr{\href{https://sites.google.com/view/vasmous}{https://sites.google.com/view/vasmous}}

\begin{document}

\begingroup
\def\uppercasenonmath#1{} % this disables uppercase title
\let\MakeUppercase\relax % this disables uppercase authors
\maketitle
\endgroup

\setcounter{section}{13}
\setcounter{theorem}{0}
\begin{center}
    \textbf{13. Κανόνες διακλάδωσης
} 
\end{center}

Στο Παράδειγμα 8.7 είδαμε ότι 
\[
\sS^{(n-1)}\uparrow_{\fS_{n-1}}^{\fS_n} \cong_{\fS_n} V^\triv\uparrow_{\fS_{n-1}}^{\fS_n} \cong_{\fS_n} V^\rmdef \cong_{\fS_n} V^\triv \oplus V^\std \cong_{\fS_n} \sS^{(n)} \oplus \sS^{(n-1,1)},
\]
όπου o τελευταίος ισομορφισμός έπεται από το Παράδειγμα 11.3 (1) και με $V^\triv, V^\rmdef$ και $V^\std$ συμβολίζουμε το πρότυπο της τετριμμένης αναπαράστασης, της αναπαράστασης καθορισμού και της συνήθους αναπαράστασης, αντίστοιχα, για την κατάλληλη συμμετρική ομάδα.  Παρατηρούμε λοιπόν ότι η ισοτυπική διάσπαση της επαγωγής του προτύπου Specht που αντιστοιχεί στην διαμέριση $(n-1)$ στην $\fS_n$ προκύπτει \textquote{επεκτείνοντας} την $(n-1)$ σε διαμέριση του $n$ με όλους τους πιθανούς τρόπους. Συνεπώς, είναι φυσικό να αναρωτηθεί κανείς το εξής.
\begin{que}
    Για $\lambda \vdash n$, υπάρχει κάποιος απλός, κατά προτίμηση συνδυαστικός, κανόνας για να υπολογίσουμε τις ισοτυπικές διασπάσεις της επαγωγής και του περιορισμού του $\sS^\lambda$ στην $\fS_{n+1}$ και $\fS_{n-1}$, αντίστοιχα;
\end{que}

Για ακόμα μια φορά, η απάντηση είναι αναπάντεχα απλή. Διαισθητικά, η επαγωγή και ο περιορισμός ενός προτύπου Specht αντιστοιχούν σε προσθαφαίρεση τετραγώνων στο διάγραμμα Young της αντίστοιχης διαμέρισης.
\begin{definition}
    \label{def:inne_outer_corner}
    Για $\lambda \vdash n$, έστω $\pP^-(\lambda)$ (αντ. $\pP^+(\lambda)$) το σύνολο των διαμερίσεων του $n-1$ (αντ. $n+1$) των οποίων το διάγραμμα Young προκύπτει από το $\rmY_\lambda$ αφαιρώντας (αντ. προσθέτοντας) ακριβώς ένα τετράγωνο, το οποίο ονομάζεται \defn{εσωτερική} (αντ. \defn{εξωτερική}) \defn{γωνία} της $\lambda$.
\end{definition}

Για παράδειγμα, οι εσωτερικές και εξωτερικές γωνίες της διαμέρισης $(4,2,2,1)$ είναι
\[
\ytableausetup{centertableaux}
\ydiagram{3,2,1}
*[*(burntorange)]{3+1}
*[*(burntorange)]{3+0,2+0,1+1}
*[*(burntorange)]{3+0,2+0,1+0,1}
\quad \text{και} \quad 
\ydiagram{4,2,2,1}
*[*(iceberg)]{4+1}
*[*(iceberg)]{4+0,2+1}
*[*(iceberg)]{4+0,2+0,2+0,1+1}
*[*(iceberg)]{4+0,2+0,2+0,1+0,1}
\]
αντίστοιχα, και γι αυτό 
\begin{align*}
    \pP^-(4,2,2,1) &= \{(3,2,2,1), (4,2,1,1), (4,2,2)\} \ \subseteq \Par_8 \\
    \pP^+(4,2,2,1) &= \{(5,2,2,1), (4,3,2,1), (4,2,2,2), (4,2,2,1,1)\} \ \subseteq \Par_{10}. \\
\end{align*}

\begin{proposition}
    \label{prop:branching_rule}
    Για κάθε $\lambda \vdash n$, 
    \begin{equation}
        \label{eq:branching_rule_help}
        f^\lambda = \sum_{\mu \in \pP^-(\lambda)} f^\mu.
    \end{equation}
\end{proposition}
\begin{proof}[Απόδειξη]
    Παρατηρούμε ότι σε κάθε σύνηθες ταμπλώ περιεχομένου $[n]$, το στοιχείο $n$ καταλαμβάνει μια εσωτερική γωνία του $\lambda$ (γιατί;). Για παράδειγμα, για $\lambda = (4,2,2,1) \vdash 9 = n$, έχουμε 
    \[
    \begin{ytableau}
    \ast & \ast & \ast & 9 \\
    \ast & \ast  \\
    \ast & \ast  \\
    \ast
    \end{ytableau} 
    \quad \text{ή} \quad 
    \begin{ytableau}
    \ast & \ast & \ast & \ast \\
    \ast & \ast  \\
    \ast & 9  \\
    \ast
    \end{ytableau} 
    \quad \text{ή} \quad 
    \begin{ytableau}
    \ast & \ast & \ast & \ast \\
    \ast & \ast  \\
    \ast & \ast  \\
    9
    \end{ytableau} 
    \]
    Αν $r_1, r_2, \dots, r_k$ είναι οι πιθανές εσωτερικές γωνίες της $\lambda$, με τις αντίστοιχες διαμερίσεις του $\pP^-(\lambda)$ να είναι $\mu^1, \mu^2, \dots, \mu^k$, τότε αφαιρώντας κάθε εσωτερική γωνία που κατέχει το στοιχείο $n$ και απαριθμώντας τα συνήθη ταμπλώ περιεχομένου $[n-1]$ που προκύπτουν έπεται ότι 
    \[
    f^\lambda = f^{\mu^1} + f^{\mu^2} + \cdots + f^{\mu^k}
    \]
    το οποίο είναι το ζητούμενο. Στο παράδειγμα, 
    \[
    f^{(4,2,2,1)} = f^{(3,2,2,1)} + f^{(4,2,1,1)} + f^{(4,2,2)}.
    \]
\end{proof}

\begin{theorem}{\rm(Κανόνες διακλάδωσης)}
    \label{thm:branching_rules}
    Για κάθε $\lambda \vdash n$, 
    \begin{align}
        \label{eq:branching_restriction}
        \sS^\lambda\!\downarrow_{\fS_{n-1}}^{\fS_n} 
        &\cong_{\fS_{n-1}} \bigoplus_{\mu \in \pP^-(\lambda)} \sS^\mu \\
        \label{eq:branching_induction}
        \sS^\lambda\!\uparrow_{\fS_n}^{\fS_{n+1}} 
        &\cong_{\fS_{n+1}} \bigoplus_{\mu \in \pP^+(\lambda)} \sS^\mu. 
    \end{align}
\end{theorem}

Για την διαμέριση του τρέχοντος παραδείγματος, οι κανόνες διακλάδωσης μας πληροφορούν ότι 
\begin{align*}
    \sS^{(4,2,2,1)}\!\downarrow_{\fS_8}^{\fS_9} 
    &\cong_{\fS_8} \sS^{(3,2,2,1)} \oplus \sS^{(4,2,1,1)} \oplus \sS^{(4,2,2)}\\
    \sS^{(4,2,2,1)}\!\uparrow_{\fS_9}^{\fS_{10}}  
    &\cong_{\fS_{10}} \sS^{(5,2,2,1)} \oplus \sS^{(4,3,2,1)} \oplus \sS^{(4,2,2,2)} \oplus \sS^{(4,2,2,1,1)}.
\end{align*}
Η απόδειξη που θα δούμε έχει το πλεονέκτημα ότι είναι ανεξάρτητη από το σώμα πάνω από το οποίο δουλεύουμε.
\begin{proof}[Απόδειξη του Θεωρήματος~\ref{thm:branching_rules} \rm(Peel 1975)]    
    Αρχικά, ο νόμος αντιστροφής Frobenius και οι σχέσεις ορθογωνιότητας Ι μας επιτρέπουν να αποδείξουμε την μία από τις δύο ταυτότητες. Πράγματι, ας υποθέσουμε ότι ισχύει η Ταυτότητα~\eqref{eq:branching_restriction}. Αν $\chi^\lambda$ είναι ο χαρακτήρας του $\sS^\lambda$, τότε για κάθε $\nu \vdash n+1$
    \begin{align*}
        \left(\chi^\lambda\!\uparrow_{\fS_n}^{\fS_{n+1}}, \chi^\nu\right)_{\fS_{n+1}} 
        &= \left(\chi^\lambda, \chi^\nu\!\downarrow_{\fS_n}^{\fS_{n+1}}\right)_{\fS_n} \\
        &= \left(\chi^\lambda, \sum_{\mu \in \pP^-(\nu)} \chi^\mu\right)_{\fS_n} \\
        &= 
        \begin{cases}
            1, &\ \text{αν $\lambda \in \pP^-(\nu)$} \\
            0, &\ \text{διαφορετικά} 
        \end{cases} \\ 
        &= 
        \begin{cases}
            1, &\ \text{αν $\nu \in \pP^+(\lambda)$} \\
            0, &\ \text{διαφορετικά} 
        \end{cases} \\ 
    \end{align*}
    όπου η πρώτη ισότητα έπεται από το Θεώρημα 8.8, η δεύτερη ισότητα από την Ταυτότητα \eqref{eq:branching_restriction} και η τρίτη ισότητα από το Θεώρημα 7.4. Με άλλα λόγια, το ανάγωγο $\fS_{n+1}$-πρότυπο που αντιστοιχεί στην διαμέριση $\nu$ εμφανίζεται στην ισοτυπική διάσπαση του $\sS^\lambda\!\uparrow_{\fS_n}^{\fS_{n+1}}$ αν και μόνο αν $\nu \in \pP^+(\lambda)$ και στην περίπτωση αυτή με πολλαπλότητα 1. Το ζητούμενο έπεται από το Πόρισμα 7.5.

    Αρκεί λοιπόν να αποδείξουμε την Ταυτότητα \eqref{eq:branching_restriction}. Γι αυτό τον σκοπό υποθέτουμε ότι οι εσωτερικές γωνίες της $\lambda$ βρίσκονται στις γραμμές $r_1, r_2, \dots, r_k$ και έστω $\mu^1, \mu^2, \dots, \mu^k \in \pP^-(\lambda)$ οι αντίστοιχες διαμερίσεις του $n-1$ που προκύπτουν αφαιρώντας τες. Η βασική ιδέα είναι ότι αρκεί να βρούμε μια ακολουθία 
    \[
    \{0\}=V_0 \subseteq V_1 \subseteq V_2 \subseteq \cdots \subseteq V_k = \sS^\lambda
    \]
    υποπροτύπων του $\sS^\lambda$, τα οποία είναι $\fS_{n-1}$-πρότυπα τέτοια ώστε τα διαδοχικά πηλίκα τους να ικανοποιούν 
    \[
    V_i / V_{i-1} \ \cong_{\fS_{n-1}} \sS^{\mu^i},
    \]
    για κάθε $1 \le i \le k$. Πράγματι, σε αυτή την περίπτωση, από την Άσκηση 3.1 έπεται ότι 
    \begin{align*}
        \sS^\lambda 
        &\cong_{\fS_{n-1}} V_{k-1} \oplus \underbrace{\left(V_k/V_{k-1}\right)}_{\cong_{\fS_{n-1}} \, \sS^{\mu^k}} \\
        &\cong_{\fS_{n-1}} V_{k-2} \oplus \underbrace{\left(V_{k-1}/V_{k-2}\right)}_{\cong_{\fS_{n-1}} \, \sS^{\mu^{k-1}}} \oplus \ \sS^{\mu^k} \\
        &\quad\vdots \\
        &\cong_{\fS_{n-1}} \sS^{\mu^1}  \oplus \cdots \oplus \sS^{\mu^{k-1}} \oplus \sS^{\mu^k},
    \end{align*}
    το οποίο είναι το ζητούμενο.
\end{proof}

Προς αυτή την κατεύθυνση, για κάθε $1 \le i \le k$, θεωρούμε τον υπόχωρο $V_i$ του $\sS^\lambda$ που παράγεται από τα πολυταμπλοειδή $\bfe_T$ για κάθε σύνηθες ταμπλώ σχήματος $\lambda$ με $r_T(n) \in \{r_1, r_2, \dots, r_i\}$. Προφανώς,
\begin{itemize}
    \item $\{0\}=V_0 \subseteq V_1 \subseteq V_2 \subseteq \cdots \subseteq V_k = \sS^\lambda$, και 
    \item κάθε $V_i$ είναι $\fS_{n-1}$-αναλλοίωτο.
\end{itemize}

Για κάθε $1 \le i \le k$, θεωρούμε την απεικόνιση $\vartheta_i : \rmM^\lambda \to \rmM^{\mu^i}$ η οποία ορίζεται θέτοντας 
\[
\vartheta_i([T]) = 
\begin{cases}
    [T]\sm{n}, &\ \text{αν $r_T(n) = r_i$} \\
    0, &\ \text{διαφορετικά}
\end{cases},
\]
όπου με $[T]\sm{n}$ συμβολίζουμε το ταμπλοειδές σχήματος $\mu^i$ που προκύπτει από το $[T]$ διαγράφοντας το $n$. Επεκτείνοντας γραμμικά κάθε $\vartheta_i$ προκύπτει ένας καλά ορισμένος $\fS_{n-1}$-ομομορφισμός, για κάθε $1 \le i \le k$. Διαισθητικά, η $\vartheta_i$ μετατρέπει κάθε ταμπλοειδές σχήματος $\lambda$ σε ένα ταμπλοειδές σχήματος $\mu^i$ \textquote{κοιτώντας} αν το $n$ \textquote{κάθεται} στην γραμμή $r_i$, ενώ διαφορετικά το \textquote{σκοτώνει}.

\begin{claim}
    Για κάθε $1 \le i \le k$ και κάθε $T \in \SYT(\lambda)$ με $r_T(n) = r_j$ έχουμε 
    \[
    \vartheta_i(\bfe_{T}) = 
    \begin{cases}
        \bfe_{T\sm{n}}, &\ \text{αν $i = j$} \\
        0, &\ \text{αν $i < j$}
    \end{cases},
    \]
    όπου με $T\sm{n}$ συμβολίζουμε το σύνηθες ταμπλώ σχήματος $\mu^j$ που προκύπτει από το $T$ διαγράφοντας το $n$. 
\end{claim}

Ο ισχυρισμός έπεται άμεσα από τον ορισμό της $\vartheta_i$ και το Λήμμα 12.5, το οποίο μας πληροφορεί ότι στα πολυταμπλοειδή του αναπτύγματος του $\bfe_T$ στην βάση των ταμπλοειδών του $\rmM^\lambda$, το $n$ βρίσκεται στην ίδια γραμμή ή βορειότερα απ' ότι στο $[T]$.

Άμεση συνέπεια του ισχυρισμού είναι ότι 
\begin{itemize}
    \item[(1)] $\vartheta_i(V_i) = \sS^{\mu^i}$
    \item[(2)] $\vartheta_i(V_{i-1}) = \{0\}$ και γι αυτό\footnote{Αυτό περιγράφει με σύμβολα αυτό που είπαμε με λόγια στη συζήτηση πριν την διατύπωση του ισχυρισμού.} $V_{i-1} \subseteq \Ker(\vartheta_i)$
\end{itemize}
για κάθε $1 \le i \le k$. Συνεπώς, από το πρώτο θεώρημα ισομορφισμού και το (1) έπεται ότι 
\begin{equation} 
    \label{eq:branching_rule_help}
\sS^{\mu^i} \cong_{\fS_{n-1}} V_i / \left(V_i \cap \Ker(\vartheta_i)\right) 
\end{equation}
για κάθε $1 \le i \le k$. Από το (2) έπεται ότι μπορούμε να εκλεπτύνουμε την ακολουθία υποπροτύπων μας, προσθέτοντας του όρους $V_i \cap \Ker(\vartheta_i)$, για κάθε $1 \le i \le k$, δηλαδή 
\[
\{0\} = V_0 \subseteq V_1 \cap \Ker(\vartheta_1) \subseteq V_1 \subseteq V_2 \cap \Ker(\vartheta_2) \subseteq \cdots \subseteq V_{k-1} \cap \Ker(\vartheta_{k-1}) \subseteq V_k = \sS^\lambda.
\]

\begin{claim}
    Για κάθε $1 \le i \le k$, 
    \[
    V_i \cap \Ker(\vartheta_i) = V_{i-1}.
    \]
\end{claim}

Πράγματι, από την Πρόταση \ref{prop:branching_rule} έπεται ότι 
\begin{align*}
    \dim(\sS^\lambda) 
    &= f^\lambda \\ 
    &= \sum_{i=1}^k f^{\mu^i} \\ 
    &= \sum_{i=1}^k \dim(\sS^{\mu^i}) \\ 
    &= \sum_{i=1}^k \dim\left(V_i/\left(V_i \cap \Ker(\vartheta_i)\right)\right) \\ 
    &= \sum_{i=1}^k \left(\dim(V_i) - \dim\left(V_i \cap \Ker(\vartheta_i)\right)\right).
\end{align*}
Επομένως, το \textquote{βήμα} από το $V_i \cap \Ker(\vartheta_i)$ στο $V_i$ \textquote{καλύπτει} όλη τη διαθέσιμη διάσταση αναγκάζοντας 
\[
\dim(V_{i-1}) - \dim\left(V_i \cap \Ker(\vartheta_i)\right) = 0
\]
και γι αυτό $V_{i-1} = V_i \cap \Ker(\vartheta_i)$, για κάθε $1 \le i \le k$.

Από τον ισχυρισμό, η Ταυτότητα \eqref{eq:branching_rule_help} γίνεται 
\[
\sS^{\mu^i} \cong_{\fS_{n-1}} V_i / \left(V_i \cap \Ker(\vartheta_i)\right) = V_i / V_{i-1}
\]
για κάθε $1 \le i \le k$ και η απόδειξη ολοκληρώνεται.
\end{document}