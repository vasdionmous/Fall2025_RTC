\documentclass[12pt,a4paper,reqno]{amsart}

% language
\usepackage[greek,english]{babel}
\usepackage[utf8]{inputenc}
\usepackage{alphabeta}

% change default names to greek
\addto\captionsenglish{
    \renewcommand{\contentsname}{Περιεχόμενα}
    \renewcommand{\refname}{Βιβλιογραφία}
    \renewcommand{\datename}{Ημερομηνία:}
    \renewcommand{\urladdrname}{Ιστοσελίδα}
}

% math 
\usepackage{amsmath,amsthm,amssymb,amscd}

% font
\usepackage[cal=euler]{mathalfa}
\usepackage{libertinus-type1}
% \usepackage{txfonts} % for upright greek letters
\usepackage{bm} % for bold symbols
\usepackage{bbm} % for the simply-looking bb symbols

% miscellaneous 
\usepackage{changepage} %for indenting environments
\usepackage{csquotes} % example: \textcquote{}
\usepackage{blkarray}
\setcounter{MaxMatrixCols}{20} % default for pmatrix is 10!!
\usepackage{ytableau}
\usepackage{array} %needed to increase the vertical length in a tabular

% drawing
\usepackage{tikz,tikz-cd}
\usetikzlibrary{shapes.misc, patterns, matrix, calc, intersections,positioning}
\usepackage{graphics,graphicx}
\usepackage{float} % provides enhanced control and customization options for floating objects such as figures and tables

% colors
\usepackage{xcolor}
\definecolor{darkcandyapplered}{rgb}{0.64, 0.0, 0.0}
\definecolor{midnightblue}{rgb}{0.1, 0.1, 0.44}
\definecolor{mylightblue}{HTML}{336699}
\definecolor{burntorange}{rgb}{0.8, 0.33, 0.0}
\definecolor{iceberg}{rgb}{0.44, 0.65, 0.82}
\definecolor{applegreen}{rgb}{0.55, 0.71, 0.0}
\definecolor{canaryyellow}{rgb}{1.0, 0.94, 0.0}

% hrefs
\usepackage{hyperref}
\usepackage[noabbrev,capitalize]{cleveref}
\hypersetup{
    pdftoolbar=true,        
    pdfmenubar=true,        
    pdffitwindow=false,     
    pdfstartview={FitH},  % fits the width of the page to the window
    pdftitle={},
    pdfauthor={},
    pdfsubject={},
    pdfkeywords={},
    pdfnewwindow=true,  % links in new window
    colorlinks=true,  % false: boxed links; true: colored links
    linkcolor=darkcandyapplered,   % color of internal links
    citecolor=midnightblue,  % color of links to bibliography
    urlcolor=cyan,  % color of external links
    linktocpage=true  % changes the links from the section body to the page number
    }

% geometry
\textwidth=16cm 
\textheight=21cm 
\hoffset=-55pt 
\footskip=25pt

% thm envs (you might need to change the path)
% In this macro I define all the theorem environments

\theoremstyle{definition}
\newtheorem{theorem}{Θεώρημα}
\newtheorem{proposition}[theorem]{Πρόταση}
\newtheorem{lemma}[theorem]{Λήμμα}
\newtheorem{corollary}[theorem]{Πόρισμα}
\newtheorem{conjecture}[theorem]{Εικασία}
\newtheorem{problem}[theorem]{Πρόβλημα}
\newtheorem*{claim}{Ισχυρισμός}
\newtheorem{observation}[theorem]{Παρατήρηση}
\newtheorem{definition}[theorem]{Ορισμός}
\newtheorem{question}[theorem]{Ερώτηση}
\newtheorem*{questions}{Ερωτήματα}
\newtheorem{example}[theorem]{Παράδειγμα}
\newtheorem{exercise}{Άσκηση}

\newtheorem*{combInterlude}{Ιντερλούδιο Συνδυαστικής}
\newtheorem*{example_cont}{Παράδειγμα~6.6}
\newtheorem*{digression_la}{Παρέκβαση Γραμμικής Άλγεβρας}
\newtheorem*{thm}{Θεώρημα}

\theoremstyle{remark}
\newtheorem*{remark}{Παρατήρηση}

% fixes the correct numbering of environments
\numberwithin{theorem}{section}
\numberwithin{exercise}{section}
\numberwithin{equation}{section}

% math ops (you might need to change the path)
% In this macro I define all of my math operators

% fields
\newcommand{\NN}{\mathbbmss{N}} 
\newcommand{\ZZ}{\mathbbmss{Z}} 
\newcommand{\QQ}{\mathbbmss{Q}} 
\newcommand{\RR}{\mathbbmss{R}} 
\newcommand{\CC}{\mathbbmss{C}} 
\newcommand{\KK}{\mathbbmss{K}} 
\newcommand{\FF}{\mathbbmss{F}} 

% symmetric group
\newcommand{\fS}{\mathfrak{S}}  

% calligraphic 
\newcommand{\aA}{\mathcal{A}} 
\newcommand{\bB}{\mathcal{B}}
\newcommand{\cC}{\mathcal{C}}
\newcommand{\dD}{\mathcal{D}}
\newcommand{\eE}{\mathcal{E}}
\newcommand{\fF}{\mathcal{F}}
\newcommand{\hH}{\mathcal{H}}
\newcommand{\iI}{\mathcal{I}}
\newcommand{\lL}{\mathcal{L}}
\newcommand{\oO}{\mathcal{O}}
\newcommand{\pP}{\mathcal{P}}
\newcommand{\sS}{\mathcal{S}}
\newcommand{\mM}{\mathcal{M}}
\newcommand{\uU}{\mathcal{U}}

% bold
\newcommand{\bfa}{\mathbf{a}}
\newcommand{\bfe}{\mathbf{e}}
\newcommand{\bfF}{\pmb{F}}
\newcommand{\bfR}{\pmb{R}}
\newcommand{\bfv}{\mathbf{v}}
%\newcommand{\bfx}{\bm{x}}
%\newcommand{\bfx}{\mathbf{x}} 
\newcommand{\bfx}{\pmb{x}}
\newcommand{\bfX}{\pmb{X}}
\newcommand{\bfy}{\pmb{y}}
\newcommand{\bfz}{\pmb{z}}

% roman
\newcommand{\rmA}{\mathrm{A}}
\newcommand{\rmB}{\mathrm{B}}
\newcommand{\rmC}{\mathrm{C}}
\newcommand{\rmD}{\mathrm{D}} 
\newcommand{\rmI}{\mathrm{I}} 
\newcommand{\rmK}{\mathrm{K}}
\newcommand{\rmM}{\mathrm{M}}
\newcommand{\rmP}{\mathrm{P}}  
\newcommand{\rmp}{\mathrm{p}}  
\newcommand{\rmQ}{\mathrm{Q}}  
\newcommand{\rmR}{\mathrm{R}}
\newcommand{\rmS}{\mathrm{S}}
\newcommand{\rmT}{\mathrm{T}}
\newcommand{\rmU}{\mathrm{U}}
\newcommand{\rmV}{\mathrm{V}}
\newcommand{\rmY}{\mathrm{Y}}
\newcommand{\rmZ}{\mathrm{Z}}
\newcommand{\rmz}{\mathrm{z}}

% greek letters
% I'm renewing some commands in order to appear in upright font
% If I want to change it later, I don't have to do it manually, I just change it from here.
% \newcommand{\uaa}{\alphaup}
% \renewcommand{\alpha}{\alphaup}
% \renewcommand{\beta}{\betaup}
% \renewcommand{\gamma}{\gammaup}
% \renewcommand{\delta}{\deltaup}
% \renewcommand{\epsilon}{\epsilonup}
% \newcommand{\ee}{\epsilon}
% \renewcommand{\varepsilon}{\varepsilonup}
% \renewcommand{\theta}{\thetaup}
% \renewcommand{\lambda}{\lambdaup}
% \newcommand{\ull}{\lambda}
% \renewcommand{\mu}{\muup}
% \renewcommand{\nu}{\nuup}
% \renewcommand{\pi}{\piup}
% \renewcommand{\rho}{\rhoup}
% \renewcommand{\varrho}{\varrhoup}
% \renewcommand{\sigma}{\sigmaup}
% \renewcommand{\tau}{\tauup} 
% \renewcommand{\phi}{\phiup}
% \renewcommand{\chi}{\chiup}
% \renewcommand{\psi}{\psiup}
% \renewcommand{\omega}{\omegaup}

% arrows and symbols 
\renewcommand{\to}{\rightarrow}
\newcommand{\toto}{\longrightarrow}
\newcommand{\mapstoto}{\longmapsto}
\newcommand{\then}{\Rightarrow}
\newcommand{\IFF}{\Leftrightarrow}
\newcommand{\tl}{\tilde}
\newcommand{\wtl}{\widetilde}
\newcommand{\ol}{\overline}
\newcommand{\ul}{\underline}
\newcommand{\oldemptyset}{\emptyset}
\renewcommand{\emptyset}{\varnothing}
\DeclareMathSymbol{\Arg}{\mathbin}{AMSa}{"39} % for arguments 
\newcommand{\onto}{\ensuremath{\twoheadrightarrow}}
\newcommand{\tle}{\trianglelefteq}
\newcommand{\tge}{\trianglerighteq}

% absolute value symbol
\usepackage{mathtools} 
\DeclarePairedDelimiter\abs{\lvert}{\rvert}%
\DeclarePairedDelimiter\norm{\lVert}{\rVert}%
\makeatletter
\let\oldabs\abs
\def\abs{\@ifstar{\oldabs}{\oldabs*}}

% tensor symbol
\newcommand{\tensor}[1]{%
  \mathbin{\mathop{\otimes}\limits_{#1}}%
}

% permutation cycle notation
\ExplSyntaxOn
\NewDocumentCommand{\cycle}{ O{\;} m }
 {
  (
  \alec_cycle:nn { #1 } { #2 }
  )
 }

\seq_new:N \l_alec_cycle_seq
\cs_new_protected:Npn \alec_cycle:nn #1 #2
 {
  \seq_set_split:Nnn \l_alec_cycle_seq { , } { #2 }
  \seq_use:Nn \l_alec_cycle_seq { #1 }
 }
\ExplSyntaxOff

% setminus symbol
\newcommand{\mysetminusD}{\hbox{\tikz{\draw[line width=0.6pt,line cap=round] (3pt,0) -- (0,6pt);}}}
\newcommand{\mysetminusT}{\mysetminusD}
\newcommand{\mysetminusS}{\hbox{\tikz{\draw[line width=0.45pt,line cap=round] (2pt,0) -- (0,4pt);}}}
\newcommand{\mysetminusSS}{\hbox{\tikz{\draw[line width=0.4pt,line cap=round] (1.5pt,0) -- (0,3pt);}}}
\newcommand{\sm}{\mathbin{\mathchoice{\mysetminusD}{\mysetminusT}{\mysetminusS}{\mysetminusSS}}}

% custom math operators
\newcommand{\Des}{\mathrm{Des}} 
\newcommand{\des}{\mathrm{des}} 
\newcommand{\Asc}{\mathrm{Asc}}
\newcommand{\asc}{\mathrm{asc}} 
\newcommand{\inv}{\mathrm{inv}}
\newcommand{\Inv}{\mathrm{Inv}}
\newcommand{\maj}{\mathrm{maj}} 
\newcommand{\comaj}{\mathrm{comaj}} 
\newcommand{\fix}{\mathrm{fix}} 
\newcommand{\Sym}{\mathrm{Sym}} 
\newcommand{\QSym}{\mathrm{QSym}}
\newcommand{\FQSym}{\mathrm{FQSym}} 
\newcommand{\End}{\mathrm{End}} 
\newcommand{\Rad}{\mathrm{Rad}} 
\newcommand{\rmMat}{\mathrm{Mat}} 
\newcommand{\rmdim}{\mathrm{dim}} 
\newcommand{\rmTop}{\mathrm{Top}} 
\newcommand{\rmCF}{\mathrm{CF}} 
\newcommand{\rmId}{\mathrm{Id}}
\newcommand{\rmid}{\mathrm{id}}
\newcommand{\rmtw}{\mathrm{tw}}
\newcommand{\trace}{\mathrm{tr}}
\newcommand{\Irr}{\mathrm{Irr}}
\newcommand{\Ind}{\mathrm{Ind}} % induction
\newcommand{\Res}{\mathrm{Res}} % restriction
\newcommand{\triv}{\mathrm{triv}} % trivial rep
\newcommand{\rmdef}{\mathrm{def}} % defining rep
\newcommand{\dom}{\triangleleft}
\newcommand{\domeq}{\trianglelefteq}
\newcommand{\lex}{\mathrm{lex}}
\newcommand{\sign}{\mathrm{sign}}
\newcommand{\SYT}{\mathrm{SYT}}
\renewcommand{\Im}{\mathrm{Im}}
\newcommand{\Ker}{\mathrm{Ker}}
\newcommand{\GL}{\mathrm{GL}}
\newcommand{\FL}{\mathrm{FL}}
\newcommand{\Span}{\mathrm{span}}
\newcommand{\pos}{\mathrm{pos}}
\newcommand{\Comp}{\mathrm{Comp}}
\newcommand{\Set}{\mathrm{Set}}
\newcommand{\std}{\mathrm{std}}
\newcommand{\cont}{\mathrm{cont}} %content of a SSYT
\newcommand{\SSYT}{\mathrm{SSYT}}
\newcommand{\ct}{\mathrm{ct}} % cycle type
\newcommand{\ch}{\mathrm{ch}} % Frobenius characteristic map
\newcommand{\height}{\mathrm{ht}}
\newcommand{\FPS}{\CC[\![\bfx]\!]} % formal power series
\newcommand{\FPSS}{\CC[\![\bfx,\bfy]\!]}
\newcommand{\reg}{\mathrm{reg}}
\newcommand{\hook}{\mathrm{h}}
\newcommand{\weight}{\mathrm{wt}}
\newcommand{\co}{\mathrm{co}}
\newcommand{\ps}{\mathrm{ps}}
\newcommand{\rmsum}{\mathrm{sum}}
\newcommand{\NSym}{\mathrm{NSym}}
\newcommand{\Hom}{\mathrm{Hom}}
\newcommand{\proj}{\mathrm{proj}}
\newcommand{\stat}{\mathrm{stat}}
\newcommand{\Par}{\mathrm{Par}}
\newcommand{\rmset}{\mathrm{set}}
\newcommand{\comp}{\mathrm{comp}}

% miscellaneous commands
\newcommand{\defn}[1]{{\color{mylightblue}{#1}}}
\newcommand{\toDo}{{\bf\color{red} TODO}}
\newcommand{\toCite}{{\bf\color{green} CITE}}
\newcommand*{\vertbar}{\rule[-1ex]{0.5pt}{2.5ex}} % for matrices with column vectors
\newcommand*{\horzbar}{\rule[.5ex]{2.5ex}{0.5pt}} % for matrices with row vectors
\newcommand{\myblue}[1]{{\color{iceberg}{#1}}}
\newcommand{\myorange}[1]{{\color{burntorange}{#1}}}
\newcommand{\mygreen}[1]{{\color{applegreen}{#1}}}
\newcommand{\myred}[1]{{\color{darkcandyapplered}{#1}}}

% ferrer's diagram
\newcommand{\fdiagram}[1]{
    \begin{tikzpicture}[scale=.7]
        \fill foreach \Z [count=\Y] in {#1}
        {foreach \X in {1,...,\Z} 
        {(\X,-\Y) circle[radius=3pt]}};
    \end{tikzpicture}
}

%
\newcommand{\tcbo}[1]{\textcolor{burntorange}{#1}}

% 
\newenvironment{nouppercase}{%
  \let\uppercase\relax%
  \renewcommand{\uppercasenonmath}[1]{}}{}

% titlepage
\title{Θ2.04: Θεωρία Αναπαραστάσεων και Συνδυαστική}
\author[Β.~Δ. Μουστακας]{Βασίλης Διονύσης Μουστάκας \\ Πανεπιστήμιο Κρήτης}
\date{16 Δεκεμβρίου 2025}
% \urladdr{\href{https://sites.google.com/view/vasmous}{https://sites.google.com/view/vasmous}}

\begin{document}

\begingroup
\def\uppercasenonmath#1{} % this disables uppercase title
\let\MakeUppercase\relax % this disables uppercase authors
\maketitle
\endgroup


\setcounter{section}{17}
\setcounter{theorem}{4}
\setcounter{equation}{2}
\begin{center}
    \textbf{17. Η χαρακτηριστική απεικόνιση Frobenius
} (Συνέχεια)
\end{center}

Ας δούμε μια σειρά από συνέπειες του Θεωρήματος 17.3.
\begin{corollary}
    \label{cor:young_module_char}
    Για κάθε $\lambda \vdash n$, 
    \begin{equation}
        \label{eq:young_character_ch}
        \ch_n(\varphi^\lambda) = h_\lambda.
    \end{equation}
    Ειδικότερα, 
    \begin{equation}    
    \label{eq:h_to_p}
        h_\lambda = \sum_{\mu \vdash n} \frac{1}{\rmz_\mu} \varphi^\lambda(\mu) p_\mu.
    \end{equation}
\end{corollary}

\begin{proof}[Απόδειξη]
    Η Ταυτότητα~\eqref{eq:h_to_p} έπεται άμεσα από την \eqref{eq:young_character_ch}. Για την τελευταία, αν $\lambda = (\lambda_1, \lambda_2, \dots$ $, \lambda_k)$, τότε υπολογίζουμε 
    \begin{align*}
        \ch_n(\varphi^\lambda) 
        &= \ch_n\left(
            \chi^\triv\uparrow_{\fS_\lambda}^{\fS_n} 
        \right) \\
        &= \ch_n\left(
            \left(
            \chi^{(\lambda_1)}\times\chi^{(\lambda_2)}\times\cdots\times \chi^{(\lambda_k)}
            \right)\uparrow_{\fS_\lambda}^{\fS_n} 
        \right) \\
        &= \ch_n\left(
            \chi^{(\lambda_1)}\circ\chi^{(\lambda_2)}\circ\cdots\circ \chi^{(\lambda_k)}
            \right) \\
        &= \ch_{\lambda_1}\left( \chi^{(\lambda_1)}\right)\ch_{\lambda_2}\left( \chi^{(\lambda_2)}\right)\cdots \ch_{\lambda_k}\left( \chi^{(\lambda_k)}\right) \\ 
        &= s_{(\lambda_1)}s_{(\lambda_2)}\cdots s_{(\lambda_k)} \\ 
        &= h_{\lambda_1}h_{\lambda_2}\cdots h_{\lambda_k} \\ 
        &= h_\lambda,
    \end{align*}
    όπου η πρώτη ισότητα έπεται από την Ταυτότητα~(10.1), η τέταρτη ισότητα έπεται από το Θεώρημα 17.3 και η πέμπτη ισότητα από την Ταυτότητα (17.2).
\end{proof}

\begin{corollary}
    \label{cor:h_to_schur}
    Για κάθε $\mu \vdash n$, 
    \begin{equation}
        \label{eq:h_to_s}
        h_\mu = \sum_{\lambda \vdash n} \rmK_{\lambda\mu} s_\lambda.
    \end{equation}
    Ειδικότερα,
    \[
    (s_\lambda, h_\mu) = \rmK_{\lambda\mu}
    \]
    για διαμερίσεις $\lambda$ και $\mu$ του ίδιου ακεραίου.
\end{corollary}

\begin{proof}[Απόδειξη]
    Ο δεύτερος ισχυρισμός έπεται άμεσα από την Ταυτότητα~\eqref{eq:h_to_s}, η οποία με τη σειρά της έπεται από την Ταυτότητα~\eqref{eq:young_character_ch} και τον κανόνα του Young.
\end{proof}

Το επόμενο πόρισμα μας πληροφορεί ότι οι βάσεις των μονωνυμικών και των πλήρως ομογενών συμμετρικών συναρτήσεων είναι δυϊκές ως προς το εσωτερικό γινόμενο Hall\footnote{Συχνά, το εσωτερικό γινόμενο Hall ορίζεται με αυτόν τον τρόπο.}. 

\begin{corollary}
    \label{cor:mh_scalar}
    Για διαμερίσεις $\lambda$ και $\mu$, 
    \[
    \left(m_\lambda, h_\mu\right) 
    = \begin{cases}
            1, &\ \text{αν $\lambda = \mu$} \\ 
            0, &\ \text{διαφορετικά}.
        \end{cases}
    \]
\end{corollary}

\begin{proof}[Απόδειξη]
    Από τα Πορίσματα 16.6 και \ref{cor:h_to_schur} έπεται ότι 
    \[
    \rmK_{\lambda\mu} 
    = \left(s_\lambda, h_\mu\right) 
    = \sum_{\nu \tle \lambda} \rmK_{\lambda\nu} \left(m_\nu, h_\mu\right).
    \]
    Το ζητούμενο έπεται από το γεγονός ότι ο πίνακας με στοιχεία τους αριθμούς Kostka είναι αντιστρέψιμος (γιατί;).
\end{proof}

Για $\lambda \vdash n$, έστω $\psi^\lambda \coloneqq \chi^\sign\uparrow_{\fS_\lambda}^{\fS_n}$. Το επόμενο πόρισμα αποδεικνύεται με παρόμοιο τρόπο με το Πόρισμα~\ref{cor:young_module_char}.
\begin{corollary}
    \label{cor:induction_of_sign_char}
    Για κάθε $\lambda \vdash n$,
    \begin{equation}
        \label{eq:induction_of_sign_char}
        \ch_n(\psi^\lambda) = e_\lambda.
    \end{equation}
    Ειδικότερα,
    \begin{equation}
        e_\lambda = \sum_{\mu\vdash n} \frac{\sign(\mu)}{\rmz_\mu} \phi^\lambda(\mu) p_\mu.
    \end{equation}
\end{corollary}

Τα επόμενα αποτελέσματα δεν αποτελούν άμεσα πορίσματα του Θεωρήματος~17.3, αλλά συμπεριλαμβάνονται σε αυτή την ενότητα λόγω της σημασίας τους, καθώς ολοκληρώνουν την εικόνα που αναπτύσσουμε για τις σχέσεις που έχουν οι διάφορες βάσεις του χώρου των συμμετρικών συναρτήσεων μεταξύ τους.
\begin{theorem}{\rm(Ορίζουσες Jacobi--Trudi)}
    Αν $\lambda = (\lambda_1, \lambda_2, \dots, \lambda_k)$, τότε
    \begin{align}
        \label{eq:jacobi-trudi_main}
        s_\lambda 
        &= \det\left(
            \left(
                h_{\lambda_i - i + j}
            \right)_{i,j=1}^k 
        \right)
        = \begin{vmatrix}
            h_{\lambda_1}     & h_{\lambda_1+1}   & \cdots & h_{\lambda_1+k-1} \\
            h_{\lambda_2-1}   & h_{\lambda_2}     & \cdots & h_{\lambda_2+k-2} \\
            \vdots            & \vdots            &        & \vdots \\ 
            h_{\lambda_k-k+1} & h_{\lambda_k-k+2} & \cdots & h_{\lambda_k}
        \end{vmatrix} \\ 
        \label{eq:jacobi-trudi_conjugate}
        s_{\lambda^\top} 
        &= \det\left(
            \left(
                e_{\lambda_i - i + j}
            \right)_{i,j=1}^k 
        \right)
        = \begin{vmatrix}
            e_{\lambda_1}     & e_{\lambda_1+1}   & \cdots & e_{\lambda_1+k-1} \\
            e_{\lambda_2-1}   & e_{\lambda_2}     & \cdots & e_{\lambda_2+k-2} \\
            \vdots            & \vdots            &        & \vdots \\ 
            e_{\lambda_k-k+1} & e_{\lambda_k-k+2} & \cdots & e_{\lambda_k}
        \end{vmatrix},
    \end{align}
    όπου $h_0 = e_0 = 1$ και $h_i = e_i = 0$ για κάθε $i < 0$.
\end{theorem}

Για παράδειγμα, για $\lambda = (4,2,2,1)$ η Ταυτότητα~\eqref{eq:jacobi-trudi_main} γίνεται\footnote{Χάριν εξοικονόμισης χώρου, στους υποδείκτες δεν γράφουμε τις παρενθέσεις και τα κώματα.} 
\begin{align*}
    s_{4221} 
    &= \begin{vmatrix}
        h_4 & h_5 & h_6 & h_7 \\
        h_1 & h_2 & h_3 & h_4 \\
        1   & h_1 & h_2 & h_3 \\
        0   &  0  & 1   & h_1 
    \end{vmatrix} \\
    &= h_{4421} - h_{443} - h_{5321} - h_{533} - h_{5411} + h_{551} + h_{6311} - h_{65} - h_{731} + h_{74}.
\end{align*}

Επίσης, παρατηρούμε ότι για $\lambda = (1^n)$ η Ταυτότητα~\eqref{eq:jacobi-trudi_main} παίρνει την μορφή 
\[
e_n = s_{(1^n)} = 
\begin{vmatrix}
    h_1    & h_2    & h_3    & \cdots & h_{n-1} & h_n \\
    1      & h_1    & h_2    & \cdots & h_{n-2} & h_{n-1} \\
    0      & 1      & h_1    & \cdots & h_{n-3} & h_{n-2} \\
    \vdots & \vdots & \vdots &        & \vdots  & \vdots \\
    0      & 0      & 0      & \cdots & 1       & h_1.
\end{vmatrix}
\]
Υπολογίζοντας την τελευταία ορίζουσα (αναπτύσσοντας, για παράδειγμα, ως προς την πρώτη στήλη) προκύπτει η Ταυτότητα~(15.4), γεγονός που δικαιολογεί την ονομασία της.

Εφαρμόζωντας την απεικόνιση $\omega$ στις ορίζουσες Jacobi--Trudi προκύπτει άμεσα ότι 
\begin{equation}
    \label{eq:omega_schur}
    \omega(s_\lambda) = s_{\lambda^\top}
\end{equation}
για κάθε διαμέριση $\lambda$, το οποίο με τη σειρά του έχει ως αποτέλεσμα το εξής.
\begin{corollary}
    \label{cor:omega_isometry}
    Η απεικόνιση $\omega$ είναι ισομετρία ως προς το εσωτερικό γινόμενο Hall.
\end{corollary}
\begin{proof}[Απόδειξη]
    Αρκεί να το αποδείξουμε για τις συναρτήσεις Schur (γιατί;). Πράγματι, 
    \begin{align*}
    \left(\omega(s_\lambda), \omega(s_\mu)\right) &= 
    \left(
        s_{\lambda^\top},s_{\mu^\top}
    \right)\\ &=
    \begin{cases}
        1, &\ \text{αν $\lambda^\top = \mu^\top$} \\
        0, &\ \text{διαφορετικά} \\
    \end{cases} \\
    &=
    \begin{cases}
        1, &\ \text{αν $\lambda = \mu$} \\
        0, &\ \text{διαφορετικά} \\
    \end{cases} \\ &= (s_\lambda, s_\mu),
    \end{align*}
    όπου η πρώτη ισότητα έπεται από την Ταυτότητα~\eqref{eq:omega_schur}.
\end{proof}

Η επόμενη πρόταση μας πληροφορεί ότι ο πίνακας αλλαγής βάσης από την βάση των power sum συμμετρικών συναρτήσεων στην βάση των συναρτήσεων Schur είναι ακριβώς ο πίνακας χαρακτήρων της $\fS_n$. 
\begin{proposition}{\rm(Τύπος Frobenius)}
    \label{prop:frobenius_formula}
    Για κάθε $\mu \vdash n$, 
    \begin{equation}
        \label{eq:frobenius_formula}
        p_\mu = \sum_{\lambda\vdash n}\chi^\lambda(\mu) s_\lambda.
    \end{equation}
    Ειδικότερα,
    \[
    (x_1 + x_2 + \cdots )^n = \sum_{\lambda \vdash n} f^\lambda s_\lambda.
    \]
\end{proposition}
\begin{proof}[Απόδειξη]
    Ο δεύτερος ισχυρισμός έπεται από την Ταυτότητα~\eqref{eq:frobenius_formula} για $\mu = (n)$. Για την τελευταία, υπολογίζουμε 
    \[
    (s_\lambda, p_\mu) = \sum_{\nu \vdash n} \frac{1}{\rmz_\nu} \chi^\lambda(\nu) (p_\nu,p_\mu) = \chi^\lambda(\mu),
    \]
    όπου η πρώτη ισότητα έπεται από την Ταυτότητα (16.4) και η τελευταία έπεται από το Παράδειγμα~17.2.
\end{proof}

Παρατηρούμε ότι εξισώνοντας τον συντελεστή του $x_1x_2\cdots x_n$ και στα δύο μέλη της ταυτότητας του δεύτερου ισχυρισμού προκύπτει για ακόμη μία φορά ο τύπος διάστασης της $\fS_n$ (γιατί;). 

\newpage

\setcounter{section}{18}
\setcounter{theorem}{0}
\setcounter{equation}{0}
\begin{center}
    \textbf{18. Ο κανόνας Murnaghan--Nakayama
} 
\end{center}

Στην προηγούμενη παράγραφο είδαμε ότι η χαρακτηριστική απεικόνιση Frobenius προσφέρει μια \textquote{γέφυρα} μεταξύ της θεωρίας αναπαραστάσεων της συμμετρικής ομάδας και της θεωρίας των συμμετρικών συναρτήσεων, μετατρέποντας το πρόβλημα υπολογισμού διάφορων χαρακτήρων σε πρόβλημα εξαγωγής συντελεστών από ταυτότητες που αφορούν συμμετρικές συναρτήσεις. Τίθεται λοιπόν το ακόλουθο φυσικό ερώτημα.

\begin{que}
Μπορούμε να βρούμε έναν \textquote{απλό}, κατά προτίμηση συνδυαστικό, τύπο για την τιμή του χαρακτήρα του προτύπου Specht;
\end{que}
 
Στην παράγραφο αυτή θα δώσουμε μια απάντηση σε αυτό το ερώτημα, η οποία θα προ\-κύψει από έναν κανόνα για το πως να εκφράσουμε το γινόμενο μιας power sum συμμετρικής συνά\-ρτησης και μιας συνάρτησης Schur στην βάση των συναρτήσεων Schur. Προς αυτή την κατεύθυνση, χρειαζόμαστε μια γενίκευση της έννοιας της διαμέρισης ενός ακεραίου.

Έστω $\lambda$ και $\mu$ διαμερίσεις ακεραίων τέτοιες ώστε $\mu \subseteq \lambda$. Το ζεύγος $(\lambda, \mu)$ ονομάζεται \defn{λοξή διαμέριση} (skew shape) και συμβολίζεται με $\lambda/\mu$. Με $\rmY_{\lambda/\mu}$ συμβολίζουμε το σύνολο των τετραγώνων του διαγράμματος Young της $\lambda$ που δεν ανήκουν σε εκείνο της $\mu$. Για παράδειγμα, για $\lambda = (6,5,3,3)$ και $\mu = (3,1)$, έχουμε
\[
\ytableausetup{centertableaux}
\rmY_{6533/31} \ = \ 
\ydiagram{3+3,1+4,3,3} \ .
\]

Μια λοξή διαμέριση ονομάζεται \defn{συνεκτική} αν το (τοπολογικό) εσωτερικό της ένωσης των τετραγώνων του διαγράμματος Young αυτής είναι συνεκτικό υποσύνολο του $\RR^2$. Η λοξή διαμέριση του προηγούμενου παραδείγματος είναι συνεκτική, ενώ οι 
\[
\rmY_{21/1} \ = \ 
\ydiagram{1+1,1}  \quad \text{και} \quad
\rmY_{6531/422} \ = \ 
\ydiagram{4+2, 2+3, 2+1, 0+1}
\]
δεν είναι. Με άλλα λόγια, μια λοξή διαμέριση είναι συνεκτική αν κάθε δύο τετράγωνα του διαγράμματος Young αυτής μπορούν να ενωθούν με ένα \textquote{μονοπάτι} τετραγώνων που μοιράζονται μια κοινή πλευρά.

\begin{definition}
    \label{def:ribbon}
    Μια συνεκτική λοξή διαμέριση $\lambda/\mu$ ονομάζεται \defn{λωρίδα} (ribbon\footnote{Υπάρχουν διάφορες ονομασίες για τις λωρίδες στην βιβλιογραφία, όπως border strip, rim hook, zigzag shape.}) αν το $\rmY_{\lambda\mu}$ δεν περιέχει κανένα ($2\times2$) τετράγωνο. Το \defn{ύψος} μια λωρίδας ορίζεται ως 
    \[
    \height(\lambda/\mu) \coloneqq \ell(\lambda/\mu) -1.
    \]
\end{definition}

Για παράδειγμα, η λοξή διαμέριση $(6,5,3,3)/(3,1)$ δεν είναι λωρίδα, ενώ η 
\[
\rmY_{63311/22} \ = \
\ydiagram{2+4,2+1,3,1,1}
\]
είναι λωρίδα ύψους 4. 

Παρατηρούμε ότι υπάρχουν οκτώ λωρίδες με 4 τετράγωνα
\[
\ytableausetup{smalltableaux}
\ydiagram{4} \ , \quad 
\ydiagram{3,1} \ , \quad 
\ydiagram{2+1,3} \ , \quad 
\ydiagram{1+2,2} \ , \quad 
\ydiagram{1+1,1+1,2} \ , \quad 
\ydiagram{1+1,2,1} \ , \quad 
\ydiagram{2,1,1} \ , \quad 
\ydiagram{1,1,1,1} \ .
\]
Γενικότερα, πόσες λωρίδες με $n$ τετράγωνα υπάρχουν;

\begin{definition}
    \label{def:ribbon_tableau}
    Έστω $\lambda/\mu$ μια λοξή διαμέριση. \defn{Ταμπλώ λωρίδων} (ribbon tableau) ονομάζεται μια αμφιμονοσήμαντη αντιστοιχία μεταξύ του συνόλου των τετραγώνων του $\rmY_{\lambda\mu}$ και του $\ZZ_{>0}$ τέτοια ώστε 
    \begin{itemize}
        \item τα στοιχεία κάθε γραμμής (αντ. στήλης) αυξάνουν ασθενώς από τα αριστερά προς τα δεξιά (αντ. από πάνω προς τα κάτω)
        \item για κάθε $i \ge 1$, το σύνολο των τετραγώνων που περιέχουν το $i$ αποτελεί (πιθανώς κενή) λωρίδα.
    \end{itemize}
    Το σύνολο όλων των ταμπλώ λωρίδων σχήματος $\lambda/\mu$ συμβολίζεται με $\RT(\lambda/\mu)$. Αν $T \in \RT(\lambda/\mu)$ περιέχει τις λωρίδες $R_1, R_2, \dots$, τότε το ύψος του $T$ ορίζεται ως 
    \[
    \height(T) \coloneqq \height(R_1) + \height(R_2) + \cdots.
    \]
\end{definition}

Ο τύπος ενός ταμπλώ λωρίδων ορίζεται με τον ίδιο τρόπο όπως ενός ημισυνήθους ταμπλώ. Για παράδειγμα, το 
\[
\ytableausetup{nosmalltableaux}
\begin{ytableau}
\none & \none & \none & *(burntorange) 2 & *(burntorange) 2 & *(burntorange) 2 & *(iceberg) 3 & *(canaryyellow) 7\\
\none & *(burntorange) 2 & *(burntorange) 2 & *(burntorange) 2 & *(iceberg) 3 & *(iceberg) 3 & *(iceberg) 3 & *(canaryyellow) 7 \\
*(burntorange) 2 & *(burntorange) 2 & *(iceberg) 3 & *(iceberg) 3 & *(iceberg) 3 & *(applegreen) 5 & *(applegreen) 5\\
*(burntorange) 2 & *(iceberg) 3 & *(iceberg) 3 & *(applegreen) 5 & *(applegreen) 5 & *(applegreen) 5\\
*(iceberg) 3 & *(iceberg) 3 & *(applegreen) 5 & *(applegreen) 5 \\ 
*(applegreen) 5 & *(applegreen) 5 & *(applegreen) 5 \\
*(applegreen) 5
\end{ytableau}
\ \in \ \RT(8876431/31)
\]
έχει τύπο $(0,9,11,0,11,0,2,0,0,\dots)$ και ύψος $\textcolor{burntorange}{3} + \textcolor{iceberg}{4} + \textcolor{applegreen}{4} + \textcolor{canaryyellow}{1} = 12$.

\begin{theorem}{\rm(Κανόνας Murnaghan--Nakayama)}
    \label{thm:MN_rule}
    Για κάθε $\lambda, \mu \vdash n$, 
    \begin{equation}
        \label{eq:MN_rule}
        \chi^\lambda(\mu) = 
        \sum_{\substack{T \in \RT(\lambda) \\ \cont(T) = \mu}} (-1)^{\height(T)}.
    \end{equation}
    Ειδικότερα, οι ανάγωγοι χαρακτήρες της συμμετρικής ομάδας παίρνουν ακέραιες τιμές.
\end{theorem}

Για παράδειγμα, για $n=9, \lambda = (5,3,1)$ και $\mu = (4,4,1)$ έχουμε δυο ταμπλώ λεωρίδων  
\[
\begin{ytableau}
    *(burntorange) 1 & *(burntorange) 1 & *(iceberg) 2 & *(iceberg) 2 & *(applegreen) 3 \\
    *(burntorange) 1 & *(iceberg) 2     & *(iceberg) 2 \\
    *(burntorange) 1
\end{ytableau} \qquad \text{και} \qquad
\begin{ytableau}
    *(burntorange) 1 & *(burntorange) 1 & *(burntorange) 1 & *(burntorange) 1 & *(applegreen) 3 \\
    *(iceberg) 2 & *(iceberg) 2     & *(iceberg) 2 \\
    *(iceberg) 2
\end{ytableau} 
\]
σχήματος $(5,3,1)$ και τύπου $(4,4,1)$ και γι αυτό ο κανόνας Murnaghan--Nakayama μας πληροφορεί ότι 
\[
\chi^{(5,3,1)}(4,4,1) = (-1)^{\textcolor{burntorange}{2} + \textcolor{iceberg}{1} + \textcolor{applegreen}{0}} + (-1)^{\textcolor{burntorange}{0} + \textcolor{iceberg}{1} + \textcolor{applegreen}{0}} = (-1)^3 + (-1)^1 = -2.
\]
\end{document}