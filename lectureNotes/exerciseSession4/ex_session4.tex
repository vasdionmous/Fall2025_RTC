\documentclass[12pt,a4paper,reqno]{amsart}

% language
\usepackage[greek,english]{babel}
\usepackage[utf8]{inputenc}
\usepackage{alphabeta}

% change default names to greek
\addto\captionsenglish{
    \renewcommand{\contentsname}{Περιεχόμενα}
    \renewcommand{\refname}{Βιβλιογραφία}
    \renewcommand{\datename}{Ημερομηνία:}
    \renewcommand{\urladdrname}{Ιστοσελίδα}
}

% math 
\usepackage{amsmath,amsthm,amssymb,amscd}

% font
\usepackage[cal=euler]{mathalfa}
\usepackage{libertinus-type1}
% \usepackage{txfonts} % for upright greek letters
\usepackage{bm} % for bold symbols
\usepackage{bbm} % for the simply-looking bb symbols

% miscellaneous 
\usepackage{changepage} %for indenting environments
\usepackage{csquotes} % example: \textcquote{}
\usepackage{stmaryrd} % needed for \mapsfrom
\usepackage{multirow}


% drawing
\usepackage{tikz,tikz-cd}
\usetikzlibrary{shapes.misc, patterns, matrix, calc, intersections,positioning}
\usepackage{graphics,graphicx}
\usepackage{float} % provides enhanced control and customization options for floating objects such as figures and tables

% colors
\usepackage{xcolor}
\definecolor{darkcandyapplered}{rgb}{0.64, 0.0, 0.0}
\definecolor{midnightblue}{rgb}{0.1, 0.1, 0.44}
\definecolor{mylightblue}{HTML}{336699}
\definecolor{burntorange}{rgb}{0.8, 0.33, 0.0}
\definecolor{iceberg}{rgb}{0.44, 0.65, 0.82}
\definecolor{applegreen}{rgb}{0.55, 0.71, 0.0}
\definecolor{canaryyellow}{rgb}{1.0, 0.94, 0.0}

% hrefs
\usepackage{hyperref}
\usepackage[noabbrev,capitalize]{cleveref}
\hypersetup{
    pdftoolbar=true,        
    pdfmenubar=true,        
    pdffitwindow=false,     
    pdfstartview={FitH},  % fits the width of the page to the window
    pdftitle={},
    pdfauthor={},
    pdfsubject={},
    pdfkeywords={},
    pdfnewwindow=true,  % links in new window
    colorlinks=true,  % false: boxed links; true: colored links
    linkcolor=darkcandyapplered,   % color of internal links
    citecolor=midnightblue,  % color of links to bibliography
    urlcolor=cyan,  % color of external links
    linktocpage=true  % changes the links from the section body to the page number
    }

% geometry
\textwidth=16cm 
\textheight=21cm 
\hoffset=-55pt 
\footskip=25pt

% thm envs
\theoremstyle{definition}
\newtheorem*{example}{Παράδειγμα}
\newtheorem*{exercise}{Άσκηση}
\newtheorem*{solution}{Λύση}
\newtheorem*{gp_theory}{Παρέκβαση Θεωρίας Ομάδων}

% math ops (you might need to change the path)
% In this macro I define all of my math operators

% fields
\newcommand{\NN}{\mathbbmss{N}} 
\newcommand{\ZZ}{\mathbbmss{Z}} 
\newcommand{\QQ}{\mathbbmss{Q}} 
\newcommand{\RR}{\mathbbmss{R}} 
\newcommand{\CC}{\mathbbmss{C}} 
\newcommand{\KK}{\mathbbmss{K}} 
\newcommand{\FF}{\mathbbmss{F}} 

% symmetric group
\newcommand{\fS}{\mathfrak{S}}  

% calligraphic 
\newcommand{\aA}{\mathcal{A}} 
\newcommand{\bB}{\mathcal{B}}
\newcommand{\cC}{\mathcal{C}}
\newcommand{\dD}{\mathcal{D}}
\newcommand{\eE}{\mathcal{E}}
\newcommand{\fF}{\mathcal{F}}
\newcommand{\hH}{\mathcal{H}}
\newcommand{\iI}{\mathcal{I}}
\newcommand{\lL}{\mathcal{L}}
\newcommand{\oO}{\mathcal{O}}
\newcommand{\pP}{\mathcal{P}}
\newcommand{\sS}{\mathcal{S}}
\newcommand{\mM}{\mathcal{M}}
\newcommand{\uU}{\mathcal{U}}

% bold
\newcommand{\bfa}{\mathbf{a}}
\newcommand{\bfe}{\mathbf{e}}
\newcommand{\bfF}{\pmb{F}}
\newcommand{\bfR}{\pmb{R}}
\newcommand{\bfv}{\mathbf{v}}
%\newcommand{\bfx}{\bm{x}}
%\newcommand{\bfx}{\mathbf{x}} 
\newcommand{\bfx}{\pmb{x}}
\newcommand{\bfX}{\pmb{X}}
\newcommand{\bfy}{\pmb{y}}
\newcommand{\bfz}{\pmb{z}}

% roman
\newcommand{\rmA}{\mathrm{A}}
\newcommand{\rmB}{\mathrm{B}}
\newcommand{\rmC}{\mathrm{C}}
\newcommand{\rmD}{\mathrm{D}} 
\newcommand{\rmI}{\mathrm{I}} 
\newcommand{\rmK}{\mathrm{K}}
\newcommand{\rmM}{\mathrm{M}}
\newcommand{\rmP}{\mathrm{P}}  
\newcommand{\rmp}{\mathrm{p}}  
\newcommand{\rmQ}{\mathrm{Q}}  
\newcommand{\rmR}{\mathrm{R}}
\newcommand{\rmS}{\mathrm{S}}
\newcommand{\rmT}{\mathrm{T}}
\newcommand{\rmU}{\mathrm{U}}
\newcommand{\rmV}{\mathrm{V}}
\newcommand{\rmY}{\mathrm{Y}}
\newcommand{\rmZ}{\mathrm{Z}}
\newcommand{\rmz}{\mathrm{z}}

% greek letters
% I'm renewing some commands in order to appear in upright font
% If I want to change it later, I don't have to do it manually, I just change it from here.
% \newcommand{\uaa}{\alphaup}
% \renewcommand{\alpha}{\alphaup}
% \renewcommand{\beta}{\betaup}
% \renewcommand{\gamma}{\gammaup}
% \renewcommand{\delta}{\deltaup}
% \renewcommand{\epsilon}{\epsilonup}
% \newcommand{\ee}{\epsilon}
% \renewcommand{\varepsilon}{\varepsilonup}
% \renewcommand{\theta}{\thetaup}
% \renewcommand{\lambda}{\lambdaup}
% \newcommand{\ull}{\lambda}
% \renewcommand{\mu}{\muup}
% \renewcommand{\nu}{\nuup}
% \renewcommand{\pi}{\piup}
% \renewcommand{\rho}{\rhoup}
% \renewcommand{\varrho}{\varrhoup}
% \renewcommand{\sigma}{\sigmaup}
% \renewcommand{\tau}{\tauup} 
% \renewcommand{\phi}{\phiup}
% \renewcommand{\chi}{\chiup}
% \renewcommand{\psi}{\psiup}
% \renewcommand{\omega}{\omegaup}

% arrows and symbols 
\renewcommand{\to}{\rightarrow}
\newcommand{\toto}{\longrightarrow}
\newcommand{\mapstoto}{\longmapsto}
\newcommand{\then}{\Rightarrow}
\newcommand{\IFF}{\Leftrightarrow}
\newcommand{\tl}{\tilde}
\newcommand{\wtl}{\widetilde}
\newcommand{\ol}{\overline}
\newcommand{\ul}{\underline}
\newcommand{\oldemptyset}{\emptyset}
\renewcommand{\emptyset}{\varnothing}
\DeclareMathSymbol{\Arg}{\mathbin}{AMSa}{"39} % for arguments 
\newcommand{\onto}{\ensuremath{\twoheadrightarrow}}
\newcommand{\tle}{\trianglelefteq}
\newcommand{\tge}{\trianglerighteq}

% absolute value symbol
\usepackage{mathtools} 
\DeclarePairedDelimiter\abs{\lvert}{\rvert}%
\DeclarePairedDelimiter\norm{\lVert}{\rVert}%
\makeatletter
\let\oldabs\abs
\def\abs{\@ifstar{\oldabs}{\oldabs*}}

% tensor symbol
\newcommand{\tensor}[1]{%
  \mathbin{\mathop{\otimes}\limits_{#1}}%
}

% permutation cycle notation
\ExplSyntaxOn
\NewDocumentCommand{\cycle}{ O{\;} m }
 {
  (
  \alec_cycle:nn { #1 } { #2 }
  )
 }

\seq_new:N \l_alec_cycle_seq
\cs_new_protected:Npn \alec_cycle:nn #1 #2
 {
  \seq_set_split:Nnn \l_alec_cycle_seq { , } { #2 }
  \seq_use:Nn \l_alec_cycle_seq { #1 }
 }
\ExplSyntaxOff

% setminus symbol
\newcommand{\mysetminusD}{\hbox{\tikz{\draw[line width=0.6pt,line cap=round] (3pt,0) -- (0,6pt);}}}
\newcommand{\mysetminusT}{\mysetminusD}
\newcommand{\mysetminusS}{\hbox{\tikz{\draw[line width=0.45pt,line cap=round] (2pt,0) -- (0,4pt);}}}
\newcommand{\mysetminusSS}{\hbox{\tikz{\draw[line width=0.4pt,line cap=round] (1.5pt,0) -- (0,3pt);}}}
\newcommand{\sm}{\mathbin{\mathchoice{\mysetminusD}{\mysetminusT}{\mysetminusS}{\mysetminusSS}}}

% custom math operators
\newcommand{\Des}{\mathrm{Des}} 
\newcommand{\des}{\mathrm{des}} 
\newcommand{\Asc}{\mathrm{Asc}}
\newcommand{\asc}{\mathrm{asc}} 
\newcommand{\inv}{\mathrm{inv}}
\newcommand{\Inv}{\mathrm{Inv}}
\newcommand{\maj}{\mathrm{maj}} 
\newcommand{\comaj}{\mathrm{comaj}} 
\newcommand{\fix}{\mathrm{fix}} 
\newcommand{\Sym}{\mathrm{Sym}} 
\newcommand{\QSym}{\mathrm{QSym}}
\newcommand{\FQSym}{\mathrm{FQSym}} 
\newcommand{\End}{\mathrm{End}} 
\newcommand{\Rad}{\mathrm{Rad}} 
\newcommand{\rmMat}{\mathrm{Mat}} 
\newcommand{\rmdim}{\mathrm{dim}} 
\newcommand{\rmTop}{\mathrm{Top}} 
\newcommand{\rmCF}{\mathrm{CF}} 
\newcommand{\rmId}{\mathrm{Id}}
\newcommand{\rmid}{\mathrm{id}}
\newcommand{\rmtw}{\mathrm{tw}}
\newcommand{\trace}{\mathrm{tr}}
\newcommand{\Irr}{\mathrm{Irr}}
\newcommand{\Ind}{\mathrm{Ind}} % induction
\newcommand{\Res}{\mathrm{Res}} % restriction
\newcommand{\triv}{\mathrm{triv}} % trivial rep
\newcommand{\rmdef}{\mathrm{def}} % defining rep
\newcommand{\dom}{\triangleleft}
\newcommand{\domeq}{\trianglelefteq}
\newcommand{\lex}{\mathrm{lex}}
\newcommand{\sign}{\mathrm{sign}}
\newcommand{\SYT}{\mathrm{SYT}}
\renewcommand{\Im}{\mathrm{Im}}
\newcommand{\Ker}{\mathrm{Ker}}
\newcommand{\GL}{\mathrm{GL}}
\newcommand{\FL}{\mathrm{FL}}
\newcommand{\Span}{\mathrm{span}}
\newcommand{\pos}{\mathrm{pos}}
\newcommand{\Comp}{\mathrm{Comp}}
\newcommand{\Set}{\mathrm{Set}}
\newcommand{\std}{\mathrm{std}}
\newcommand{\cont}{\mathrm{cont}} %content of a SSYT
\newcommand{\SSYT}{\mathrm{SSYT}}
\newcommand{\ct}{\mathrm{ct}} % cycle type
\newcommand{\ch}{\mathrm{ch}} % Frobenius characteristic map
\newcommand{\height}{\mathrm{ht}}
\newcommand{\FPS}{\CC[\![\bfx]\!]} % formal power series
\newcommand{\FPSS}{\CC[\![\bfx,\bfy]\!]}
\newcommand{\reg}{\mathrm{reg}}
\newcommand{\hook}{\mathrm{h}}
\newcommand{\weight}{\mathrm{wt}}
\newcommand{\co}{\mathrm{co}}
\newcommand{\ps}{\mathrm{ps}}
\newcommand{\rmsum}{\mathrm{sum}}
\newcommand{\NSym}{\mathrm{NSym}}
\newcommand{\Hom}{\mathrm{Hom}}
\newcommand{\proj}{\mathrm{proj}}
\newcommand{\stat}{\mathrm{stat}}
\newcommand{\Par}{\mathrm{Par}}
\newcommand{\rmset}{\mathrm{set}}
\newcommand{\comp}{\mathrm{comp}}

% miscellaneous commands
\newcommand{\defn}[1]{{\color{mylightblue}{#1}}}
\newcommand{\toDo}{{\bf\color{red} TODO}}
\newcommand{\toCite}{{\bf\color{green} CITE}}

% 
\newenvironment{nouppercase}{%
  \let\uppercase\relax%
  \renewcommand{\uppercasenonmath}[1]{}}{}

\newcommand{\tcbo}[1]{\textcolor{burntorange}{#1}}

% titlepage
\title{Θ2.04: Θεωρία Αναπαραστάσεων και Συνδυαστική}
\author[Β.~Δ. Μουστακας]{Βασίλης Διονύσης Μουστάκας \\ Πανεπιστήμιο Κρήτης}
\date{29 Οκτωβρίου 2025}
% \urladdr{\href{https://sites.google.com/view/vasmous}{https://sites.google.com/view/vasmous}}

\begin{document}

\begingroup
\def\uppercasenonmath#1{} % this disables uppercase title
\let\MakeUppercase\relax % this disables uppercase authors
\maketitle
\endgroup

% \setcounter{section}{}
\thispagestyle{empty}

Στην παράγραφο 7, λίγο αργότερα, θα αποδείξουμε τα εξής:
\begin{itemize}
    \item Δυο αναπαραστάσεις είναι ισόμορφες αν και μόνο αν έχουν τον ίδιο χαρακτήρα.
    \item Το πλήθος των ανάγωγων χαρακτήρων μιας ομάδας ισούται με το πλήθος των κλά\-σεων συζυγίας της.
\end{itemize}
Ας υπολογίσουμε τον πίνακα χαρακτήρων της $\fS_5$, χρησιμοποιώντας τα εργαλεία που έχουμε αναπτύξει έως τώρα.

\begin{example}{\rm(Πίνακας χαρακτήρων της $\fS_5$)} \\
Η $\fS_5$ έχει επτά κλάσεις συζυγίας 
    \[
    \renewcommand{\arraystretch}{1.2} 
    \begin{array}{c|c|c|c|c|c|c|c}
                        & \rmK_{11111}  & \rmK_{2111}  & \rmK_{221}              & \rmK_{311}     & \rmK_{32}  & \rmK_{41} & \rmK_5        \\ \hline
    \text{αντιπρόσωπος} & \epsilon     & \cycle{1,2} & \cycle{1,2}\cycle{3,4} & \cycle{1,2,3} &  \cycle{1,2,3}\cycle{4,5} & \cycle{1,2,3,4} & \cycle{1,2,3,4,5} \\ \hline
    \text{πληθάριθμος} & 1            & 10           & 15                      & 20             &  20 & 30 & 24        
    \end{array}\ .
    \]
    Στην Παράγραφο 9, θα βρούμε έναν τύπο για το πληθάριθμο μιας αυθαίρετης κλάσης συζυ\-γίας της $\fS_n$.

    Έχουμε συναντήσει τρεις ανάγωγους χαρακτήρες της $\fS_5$, τους χαρακτήρες της τετριμμένης αναπαράστασης, της αναπαράστασης προσήμου και της συνήθους αναπαράστασης. Στον υπολογισμό του πίνακα χαρακτήρων της $\fS_4$, είδαμε ότι και το γινόμενο των δυο τελευταίων χαρακτήρων είναι και αυτός ανάγωγος χαρακτήρας. Οπότε
    \[
    \renewcommand{\arraystretch}{1.2} 
    \begin{array}{c|c|c|c|c|c|c|c}
                         & \rmK_{11111}  & \rmK_{2111}  & \rmK_{221} & \rmK_{311} & \rmK_{32} & \rmK_{41} & \rmK_5  \\ \hline
    \chi^\triv           & 1             & 1            & 1          & 1          & 1         & 1         & 1        \\ \hline 
    \chi^\sign           & 1             & -1           & 1          & 1          & -1        & -1        & 1        \\ \hline 
    \chi^\std            & 4             & 2            & 0          & 1          & -1        & 0         & -1        \\ \hline 
    \chi^\std\chi^\sign  & 4             & -2           & 0          & 1          & 1         & 0         & -1  
    \end{array}\ .
    \]

    Ψάχνουμε τρεις ακόμα, έστω $\chi_5, \chi_6$ και $\chi_7$. Αν $a, b$ και $c$ είναι οι διαστάσεις τους, ο τύπος διάστασης
    \[
    120 = 1^2 + 1^2 + 4^2 + 4^2 + a^2 + b^2 + c^2 \ \then a^2 + b^2 + c^2 = 86.
    \]
    μας πληροφορεί ότι υπάρχουν τρεις επιλογές $(1,2,9), (1,6,7)$ και $(5,5,6)$. Μπορεί η $\fS_5$ να έχει και άλλη ανάγωγη αναπαράσταση διάστασης 1;
\end{example}

Για να απαντήσουμε σε αυτό το ερώτημα σκεφτόμαστε ως εξής. Ένας χαρακτήρας διάστασης 1 είναι ουσιαστικά ίδιος με την αντίστοιχη αναπαράσταση, καθώς $\GL_1(\CC) \cong \CC\sm\{0\}$. Αν $\chi$ είναι ένα χαρακτήρας διάστασης 1, τότε 
\[
\chi(gx) = \chi(g)\chi(x) = \chi(x)\chi(g) = \chi(xg),
\]
για κάθε $x, g \in G$, όπου η δεύτερη ισότητα έπεται από την μεταθετικότητα στον $\CC$. Με άλλα λόγια, ο $\chi$ δεν \textquote{βλέπει} την μη μεταθετικότητα στην $G$. Αν λοιπόν $G^\ab$ ήταν η ομάδα που προκύπτει από την $G$ εξαναγκάζοντας \emph{όλα} τα στοιχεία της να μετατίθενται, δηλαδή
\[
gx \equiv xg 
\]
για κάθε $x, g \in G$, τότε έχουμε μια αμφιμονοσήμαντη απεικόνιση 
\[
\{\text{χαρακτήρες διάστασης 1 της $G^{ab}$}\}\ \toto \
\{\text{χαρακτήρες διάστασης 1 της $G$}\}.
\]

\begin{gp_theory}
    Πιο συγκεκριμένα, έστω $[G,G]$ η υποομάδα της $G$ που παράγεται από όλα τα στοιχεία της μορφής 
    \[
    gxg^{-1}x^{-1},
    \]
    για κάθε $g, x \in G$. Η $[G,G]$ ονομάζεται \emph{υποομάδα μεταθετών} (commutator subgroup) της $G$ και είναι παράδειγμα \emph{κανονικής υποομάδας}\footnote{Μεταξύ διάφορων ισοδύναμων διατυπώσεων, μια υποομάδα της $G$ ονομάζεται κανονική αν προκύπτει ως ο πυρήνας κάποιου ομομορφισμού.} της $G$. Η ομάδα $G^\ab$ ορίζεται να είναι η \emph{ομάδα πηλίκο}\footnote{Αν $N$ είναι μια κανονική υποομάδα της $G$, τότε το σύνολο $G/N$ των αριστερών κλάσεων της $N$ στην $G$ με πράξη $xN \ast yN = xyN$ αποτελεί ομάδα, η οποία ονομάζεται ομάδα πηλίκο.} $G/[G,G]$ και ονομάζεται \emph{αβελιανοποίηση} (abelianization) της $G$. Η ζητούμενη αμφιμονοσήμαντη αντιστοιχία δίνεται από 
    \begin{align*}
    \Hom(G/[G,G], \CC\sm\{0\}) &\to \Hom(G,\CC\sm\{0\}) \\
    f &\mapsto f\circ \proj,
    \end{align*}
    όπου $\proj : G \to G/[G,G]$ είναι η (κανονική) προβολή\footnote{Αν $N$ είναι μια κανονική υποομάδα της $G$, τότε η απεικόνιση $\proj : G \to G/N$ με $\proj(g) = gN$ έχει πυρήνα ακριβώς το $N$.} της $G$ στο $G/[G,G]$.
\end{gp_theory}

\begin{example}{\rm(Πίνακας χαρακτήρων της $\fS_5$, συνέχεια)} \\
    Σύμφωνα με τη παραπάνω συζήτηση, για να βρούμε το πλήθος των αναπαραστάσεων διάστασης 1 της $\fS_5$ αρκεί να καθορίσουμε την $\fS^{\ab}$. Στην Παράγραφο 9, θα δούμε ότι κάθε μετάθεση της $\fS_n$ μπορεί να γραφεί ως γινόμενο αντιμεταθέσεων της μορφής 
    \[
    \cycle{1,2}, \ \cycle{2,3}, \ \dots, \ \cycle{n-1,n}.
    \]
    Αρκεί λοιπόν να καταλάβουμε τι συμβαίνει αν απαιτήσουμε αυτές να μετατίθενται. 
    
    Στην $\fS_5$, οι $\cycle{1,2}$ και $\cycle{2,3}$ ικανοποιούν την σχέση
    \[
    \cycle{1,2}\cycle{2,3}\cycle{1,2} = \cycle{2,3}\cycle{1,2}\cycle{2,3}.
    \]
    Αν $\cycle{1,2}\cycle{2,3} \equiv \cycle{2,3}\cycle{1,2}$, τότε η παραπάνω σχέση γίνεται 
    \[
    \cycle{1,2}^2\cycle{2,3} \equiv \cycle{1,2}\cycle{2,3}^2 \ \then \cycle{1,2} \equiv \cycle{2,3}.
    \]
    Ομοίως, βλέπουμε ότι 
    \[
    \cycle{1,2} \equiv \cycle{2,3} \equiv \cycle{3,4} \equiv \cycle{4,5}
    \]
    στην $\fS^\ab$. Με άλλα λόγια, παιρνόντας από την  $\fS_5$ στην $\fS_5^\ab$ από τους τέσσερις γεννήτορες μένει μόνο ένας, έστω $t$, ο οποίος ικανοποιεί τη συνθήκη $t^2 = \epsilon$. Άρα, 
    \[
    \fS_5 \cong \rmC_2.
    \]
    Όμως, η κυκλική ομάδα τάξης 2 έχει δύο διαφορετικές ανάγωγες αναπαραστάσεις διάστασης 1 και κατά συνέπεια, η μοναδική τριάδα που μπορεί να υπάρξει για τις διαστάσεις των $\chi_5, \chi_6$ και $\chi_7$ είναι η $(5,5,6)$.
\end{example}

\begin{gp_theory}{\rm(Συνέχεια)}
Το ίδιο επιχείρημα δουλεύει για κάθε $n$ και γι αυτό η αβελιανοποίηση της συμμετρικής ομάδας είναι ισόμορφη με την κυκλική ομάδα τάξης 2. Αν αντί για $t$ χρησιμοποιού\-σαμε το σύμβολο $-1$ και κατά συνέπεια κάθε αντιμετάθεση ήταν ισοδύναμη με το $-1$ στην $\fS_n^\ab$, τότε η προβολή $\fS_n \to \fS_n^\ab$ της προηγούμενης συζήτησης δεν είναι άλλη από τον ομομορφισμό \emph{πρόσημο}
\begin{align*}
    \sign: \fS_n &\to \{\pm1\} \\ 
    \pi &\mapsto \sign(\pi).
\end{align*}
Ο πυρήνας αυτού του ομομορφισμού 
\[
\ker(\sign) = \{\pi \in \fS_n : \sign(\pi) = 1\} \coloneqq \rmA_n
\]
αποτελείται από όλες τις \emph{άρτιες} μεταθέσεις\footnote{Στην $\fS_n^\ab$ οι άρτιες μεταθέσεις είναι \emph{όλες} ισοδύναμες με το +1, ενώ οι περιττές με το -1. Αυτό είναι που \textquote{ξεχωρίζει} ο ομομορφισμός $\sign$.}, δηλαδή τις μεταθέσεις που γράφονται ως γινόμενο άρτιου πλήθους δυο αντιμεταθέσεων, ονομάζεται \emph{εναλλάσσουσα} (alternating) υποομάδα της $\fS_n$. 
\end{gp_theory}

\begin{example}{\rm(Πίνακας χαρακτήρων της $\fS_5$, συνέχεια)} \\
    Ψάχνουμε λοιπόν έναν ανάγωγο χαρακτήρα διάστασης 5. Ας θεωρήσουμε τη δράση της $\fS_5$ στο σύνολο  
    \[
    \binom{[5]}{2} \coloneqq \{12, 13, 14, 15, 23, 24, 25, 34, 35, 45\}
    \]
    όλων των υποσυνόλων του $[5]$ με δυο στοιχεία, που επάγεται από τη δράση καθορισμού της $\fS_5$ στο $[5]$. Αν $\chi$ είναι ο χαρακτήρας της αντίστοιχης αναπαράστασης μεταθέσεων, τότε από την Άσκηση 2.3 (1) υπολογίζουμε 
    \[
    \renewcommand{\arraystretch}{1.4} 
    \begin{array}{c|c|c|c|c|c|c|c}
        \pi                & \epsilon     & \cycle{1,2} & \cycle{1,2}\cycle{3,4} & \cycle{1,2,3} &  \cycle{1,2,3}\cycle{4,5} & \cycle{1,2,3,4} & \cycle{1,2,3,4,5} \\ \hline
        \binom{[5]}{2}^\pi & \binom{[5]}{2} & \{12,34,35,45\} & \{12,34\} & \{45\} & \{45\} & \emptyset & \emptyset \\ \hline 
        \chi               & 10             & 4              & 2         & 1      & 1       & 0         & 0
    \end{array}
    \]
    και γι αυτό 
    \[
    (\chi, \chi) = \frac{1}{120}(10^2 + 10\cdot4^2 + 15\cdot2^2 + 20\cdot1^2 + 20\cdot1^2 + 30\cdot0 + 24\cdot0) = 3.
    \]
    Άρα ο $\chi$ δεν είναι ανάγωγος, πράγμα το οποίο γνωρίζαμε ήδη (γιατί;), αλλά ο υπολογισμός αυτός σε συνδυασμό με το Πόρισμα 7.5 μας πληροφορεί ότι η ισοτυπική του διάσπαση περιέχει ακριβώς τρεις ανάγωγους χαρακτήρες με πολλαπλότητα ένα ο καθένας. Παρατηρούμε ότι 
    \[
        (\chi, \chi^\triv) =
        (\chi, \chi^\std) = 1.
    \]
    Συνεπώς, αν $\psi$ είναι ο τρίτος ανάγωγος, τότε
    \[
    \chi = \chi^\triv + \chi^\std + \psi \ \then \ \psi = \chi - \chi^\triv - \chi^\std
    \]
    όπου η τελευταία ισότητα έπεται από την Πρόταση 6.6. Υπολογίζουμε 
    \[
    \renewcommand{\arraystretch}{1.2} 
    \begin{array}{c|c|c|c|c|c|c|c}
         & \rmK_{11111} & \rmK_{2111} & \rmK_{221} & \rmK_{311} & \rmK_{32}  & \rmK_{41} & \rmK_5        \\ \hline
    \psi & 5            & 1           & 1          & -1         & 1          & -1        & 0 
    \end{array}
    \]
    και γι αυτό 
    \[
    (\psi, \psi) = \frac{1}{120}(5^2 + 10\cdot1^2 + 15\cdot1^2 + 20\cdot(-1)^2 + 20\cdot1^2 + 30\cdot(-1)^2 + 24\cdot0) = 1.
    \]
    Επομένως, ο $\psi$ είναι ο ένας από τους δυο ανάγωγους χαρακτήρες διάστασης 5 που ψάχναμε. Ας υποθέσουμε, χωρίς βλάβη της γενικότητας, ότι είναι ο $\chi_5$. Τότε ο $\chi_5\chi^\sign$ είναι ο $\chi_6$ (γιατί;) και κατά συνέπεια μπορούμε να προσθέσουμε δυο ακόμα γραμμές στον πίνακα χαρακτήρων 
    \[
    \renewcommand{\arraystretch}{1.2} 
    \begin{array}{c|c|c|c|c|c|c|c}
           & \rmK_{11111} & \rmK_{2111} & \rmK_{221} & \rmK_{311} & \rmK_{32}  & \rmK_{41} & \rmK_5        \\ \hline
    \chi_5 & 5            & 1           & 1          & -1         & 1          & -1        & 0  \\ \hline
    \chi_6 & 5            & -1          & 1          & -1         & -1         & 1         & 0  
    \end{array} \ .
    \]

    Τέλος, όπως και στην περίπτωση της $\fS_4$, έτσι και εδώ για να βρούμε τον τελευταίο χαρακτήρα μπορούμε να κοιτάξουμε στον χαρακτήρα της κανονικής αναπαράστασης της $\fS_5$
    \[
    \renewcommand{\arraystretch}{1.2} 
    \begin{array}{c|c|c|c|c|c|c|c}
           & \rmK_{11111} & \rmK_{2111} & \rmK_{221} & \rmK_{311} & \rmK_{32} & \rmK_{41} & \rmK_5        \\ \hline
    \chi_5 & 120          & 0           & 0          & 0          & 0         & 0         & 0  
    \end{array} \ .
    \]
    Από το Πόρισμα 7.5, έπεται ότι 
    \begin{align*}
    \chi^\reg &= \chi^\triv + \chi^\sign + 4\chi^\std + 4\chi^\std\chi^\sign + 5\chi_5 + 5\chi_6 + 6\chi_7 \ \then \\ 
    \chi_7 &= \left(\chi^\reg - \chi^\triv - \chi^\sign - 4\chi^\std - 4\chi^\std\chi^\sign - 5\chi_5 - 5\chi_6\right)/6
    \end{align*}
    και γι αυτό 
    \[
    \renewcommand{\arraystretch}{1.2} 
    \begin{array}{c|c|c|c|c|c|c|c}
           & \rmK_{11111} & \rmK_{2111} & \rmK_{221} & \rmK_{311} & \rmK_{32} & \rmK_{41} & \rmK_5        \\ \hline
    \chi_7 & 6            & 0           & -2         & 0          & 0         & 0         & 1  
    \end{array} \ .
    \]

    Συμπερασματικά, ο πίνακας χαρακτήρων της $\fS_5$ είναι 
    \[
    \renewcommand{\arraystretch}{1.2} 
    \begin{array}{c|c|c|c|c|c|c|c}
                         & \rmK_{11111} & \rmK_{2111} & \rmK_{221} & \rmK_{311} & \rmK_{32} & \rmK_{41} & \rmK_5  \\ \hline
    \chi^\triv           & 1            & 1           & 1          & 1          & 1         & 1         & 1        \\ \hline 
    \chi^\sign           & 1            & -1          & 1          & 1          & -1        & -1        & 1        \\ \hline 
    \chi^\std            & 4            & 2           & 0          & 1          & -1        & 0         & -1        \\ \hline 
    \chi^\std\chi^\sign  & 4            & -2          & 0          & 1          & 1         & 0         & -1       \\\hline
    \chi_5               & 5            & 1           & 1          & -1         & 1         & -1        & 0       \\ \hline
    \chi_6               & 5            & -1          & 1          & -1         & -1        & 1         & 0   \\ \hline
    \chi_7               & 6            & 0           & -2         & 0          & 0         & 0         & 1
    \end{array}\ .
    \]
    Τι παρατηρείτε;
\end{example}
\end{document}