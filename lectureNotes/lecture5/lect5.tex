\documentclass[12pt,a4paper,reqno]{amsart}

% section handling
\usepackage{subfiles} 

% language
\usepackage[greek,english]{babel}
\usepackage[utf8]{inputenc}
\usepackage{alphabeta}

% change default names to greek
\addto\captionsenglish{
    \renewcommand{\contentsname}{Περιεχόμενα}
    \renewcommand{\refname}{Βιβλιογραφία}
    \renewcommand{\datename}{Ημερομηνία:}
    \renewcommand{\urladdrname}{Ιστοσελίδα}
}

% math 
\usepackage{amsmath,amsthm,amssymb,amscd}

% font
\usepackage[cal=euler]{mathalfa}
\usepackage{libertinus-type1}
% \usepackage{txfonts} % for upright greek letters
\usepackage{bm} % for bold symbols
\usepackage{bbm} % for the simply-looking bb symbols

% miscellaneous 
\usepackage{changepage} %for indenting environments
\usepackage{csquotes} % example: \textcquote{}

% drawing
\usepackage{tikz,tikz-cd}
\usetikzlibrary{shapes.misc, patterns, matrix, calc, intersections,positioning}
\usepackage{graphics,graphicx}
\usepackage{float} % provides enhanced control and customization options for floating objects such as figures and tables

% colors
\usepackage{xcolor}
\definecolor{darkcandyapplered}{rgb}{0.64, 0.0, 0.0}
\definecolor{midnightblue}{rgb}{0.1, 0.1, 0.44}
\definecolor{mylightblue}{HTML}{336699}
\definecolor{burntorange}{rgb}{0.8, 0.33, 0.0}
\definecolor{iceberg}{rgb}{0.44, 0.65, 0.82}

% hrefs
\usepackage{hyperref}
\usepackage[noabbrev,capitalize]{cleveref}
\hypersetup{
    pdftoolbar=true,        
    pdfmenubar=true,        
    pdffitwindow=false,     
    pdfstartview={FitH},  % fits the width of the page to the window
    pdftitle={},
    pdfauthor={},
    pdfsubject={},
    pdfkeywords={},
    pdfnewwindow=true,  % links in new window
    colorlinks=true,  % false: boxed links; true: colored links
    linkcolor=darkcandyapplered,   % color of internal links
    citecolor=midnightblue,  % color of links to bibliography
    urlcolor=cyan,  % color of external links
    linktocpage=true  % changes the links from the section body to the page number
    }

% geometry
\textwidth=16cm 
\textheight=21cm 
\hoffset=-55pt 
\footskip=25pt

% thm envs (you might need to change the path)
% In this macro I define all the theorem environments

\theoremstyle{definition}
\newtheorem{theorem}{Θεώρημα}
\newtheorem{proposition}[theorem]{Πρόταση}
\newtheorem{lemma}[theorem]{Λήμμα}
\newtheorem{corollary}[theorem]{Πόρισμα}
\newtheorem{conjecture}[theorem]{Εικασία}
\newtheorem{problem}[theorem]{Πρόβλημα}
\newtheorem*{claim}{Ισχυρισμός}
\newtheorem{observation}[theorem]{Παρατήρηση}
\newtheorem{definition}[theorem]{Ορισμός}
\newtheorem{question}[theorem]{Ερώτηση}
\newtheorem{example}[theorem]{Παράδειγμα}
\newtheorem{exercise}{Άσκηση}

\theoremstyle{remark}
\newtheorem*{remark}{Παρατήρηση}

% fixes the correct numbering of environments
\numberwithin{theorem}{section}
\numberwithin{exercise}{section}
\numberwithin{equation}{section}

% math ops (you might need to change the path)
% In this macro I define all of my math operators

% fields
\newcommand{\NN}{\mathbbmss{N}} 
\newcommand{\ZZ}{\mathbbmss{Z}} 
\newcommand{\QQ}{\mathbbmss{Q}} 
\newcommand{\RR}{\mathbbmss{R}} 
\newcommand{\CC}{\mathbbmss{C}} 
\newcommand{\KK}{\mathbbmss{K}} 
\newcommand{\FF}{\mathbbmss{F}} 

% symmetric group
\newcommand{\fS}{\mathfrak{S}}  

% calligraphic 
\newcommand{\aA}{\mathcal{A}} 
\newcommand{\bB}{\mathcal{B}}
\newcommand{\cC}{\mathcal{C}}
\newcommand{\dD}{\mathcal{D}}
\newcommand{\eE}{\mathcal{E}}
\newcommand{\fF}{\mathcal{F}}
\newcommand{\hH}{\mathcal{H}}
\newcommand{\iI}{\mathcal{I}}
\newcommand{\lL}{\mathcal{L}}
\newcommand{\oO}{\mathcal{O}}
\newcommand{\pP}{\mathcal{P}}
\newcommand{\sS}{\mathcal{S}}
\newcommand{\mM}{\mathcal{M}}
\newcommand{\uU}{\mathcal{U}}

% bold
\newcommand{\bfa}{\mathbf{a}}
\newcommand{\bfe}{\mathbf{e}}
\newcommand{\bfF}{\pmb{F}}
\newcommand{\bfR}{\pmb{R}}
\newcommand{\bfv}{\mathbf{v}}
%\newcommand{\bfx}{\bm{x}}
%\newcommand{\bfx}{\mathbf{x}} 
\newcommand{\bfx}{\pmb{x}}
\newcommand{\bfX}{\pmb{X}}
\newcommand{\bfy}{\pmb{y}}
\newcommand{\bfz}{\pmb{z}}

% roman
\newcommand{\rmB}{\mathrm{B}}
\newcommand{\rmC}{\mathrm{C}}
\newcommand{\rmD}{\mathrm{D}} 
\newcommand{\rmI}{\mathrm{I}} 
\newcommand{\rmK}{\mathrm{K}}
\newcommand{\rmM}{\mathrm{M}}
\newcommand{\rmP}{\mathrm{P}}  
\newcommand{\rmQ}{\mathrm{Q}}  
\newcommand{\rmR}{\mathrm{R}}
\newcommand{\rmS}{\mathrm{S}}
\newcommand{\rmT}{\mathrm{T}}
\newcommand{\rmU}{\mathrm{U}}
\newcommand{\rmV}{\mathrm{V}}
\newcommand{\rmY}{\mathrm{Y}}
\newcommand{\rmZ}{\mathrm{Z}}

% greek letters
% I'm renewing some commands in order to appear in upright font
% If I want to change it later, I don't have to do it manually, I just change it from here.
% \newcommand{\uaa}{\alphaup}
% \renewcommand{\alpha}{\alphaup}
% \renewcommand{\beta}{\betaup}
% \renewcommand{\gamma}{\gammaup}
% \renewcommand{\delta}{\deltaup}
% \renewcommand{\epsilon}{\epsilonup}
% \newcommand{\ee}{\epsilon}
% \renewcommand{\varepsilon}{\varepsilonup}
% \renewcommand{\theta}{\thetaup}
% \renewcommand{\lambda}{\lambdaup}
% \newcommand{\ull}{\lambda}
% \renewcommand{\mu}{\muup}
% \renewcommand{\nu}{\nuup}
% \renewcommand{\pi}{\piup}
% \renewcommand{\rho}{\rhoup}
% \renewcommand{\varrho}{\varrhoup}
% \renewcommand{\sigma}{\sigmaup}
% \renewcommand{\tau}{\tauup} 
% \renewcommand{\phi}{\phiup}
% \renewcommand{\chi}{\chiup}
% \renewcommand{\psi}{\psiup}
% \renewcommand{\omega}{\omegaup}

% arrows and symbols 
\renewcommand{\to}{\rightarrow}
\newcommand{\toto}{\longrightarrow}
\newcommand{\mapstoto}{\longmapsto}
\newcommand{\then}{\Rightarrow}
\newcommand{\IFF}{\Leftrightarrow}
\newcommand{\tl}{\tilde}
\newcommand{\wtl}{\widetilde}
\newcommand{\ol}{\overline}
\newcommand{\ul}{\underline}
\newcommand{\oldemptyset}{\emptyset}
\renewcommand{\emptyset}{\varnothing}
\DeclareMathSymbol{\Arg}{\mathbin}{AMSa}{"39} % for arguments 
\newcommand{\onto}{\ensuremath{\twoheadrightarrow}}

% absolute value symbol
\usepackage{mathtools} 
\DeclarePairedDelimiter\abs{\lvert}{\rvert}%
\DeclarePairedDelimiter\norm{\lVert}{\rVert}%
\makeatletter
\let\oldabs\abs
\def\abs{\@ifstar{\oldabs}{\oldabs*}}

% tensor symbol
\newcommand{\tensor}[1]{%
  \mathbin{\mathop{\otimes}\limits_{#1}}%
}

% permutation cycle notation
\ExplSyntaxOn
\NewDocumentCommand{\cycle}{ O{\;} m }
 {
  (
  \alec_cycle:nn { #1 } { #2 }
  )
 }

\seq_new:N \l_alec_cycle_seq
\cs_new_protected:Npn \alec_cycle:nn #1 #2
 {
  \seq_set_split:Nnn \l_alec_cycle_seq { , } { #2 }
  \seq_use:Nn \l_alec_cycle_seq { #1 }
 }
\ExplSyntaxOff

% setminus symbol
\newcommand{\mysetminusD}{\hbox{\tikz{\draw[line width=0.6pt,line cap=round] (3pt,0) -- (0,6pt);}}}
\newcommand{\mysetminusT}{\mysetminusD}
\newcommand{\mysetminusS}{\hbox{\tikz{\draw[line width=0.45pt,line cap=round] (2pt,0) -- (0,4pt);}}}
\newcommand{\mysetminusSS}{\hbox{\tikz{\draw[line width=0.4pt,line cap=round] (1.5pt,0) -- (0,3pt);}}}
\newcommand{\sm}{\mathbin{\mathchoice{\mysetminusD}{\mysetminusT}{\mysetminusS}{\mysetminusSS}}}

% custom math operators
\newcommand{\Des}{\mathrm{Des}} 
\newcommand{\des}{\mathrm{des}} 
\newcommand{\Asc}{\mathrm{Asc}}
\newcommand{\asc}{\mathrm{asc}} 
\newcommand{\inv}{\mathrm{inv}}
\newcommand{\Inv}{\mathrm{Inv}}
\newcommand{\maj}{\mathrm{maj}} 
\newcommand{\comaj}{\mathrm{comaj}} 
\newcommand{\fix}{\mathrm{fix}} 
\newcommand{\Sym}{\mathrm{Sym}} 
\newcommand{\QSym}{\mathrm{QSym}}
\newcommand{\FQSym}{\mathrm{FQSym}} 
\newcommand{\End}{\mathrm{End}} 
\newcommand{\Rad}{\mathrm{Rad}} 
\newcommand{\rmMat}{\mathrm{Mat}} 
\newcommand{\rmdim}{\mathrm{dim}} 
\newcommand{\rmTop}{\mathrm{Top}} 
\newcommand{\rmCF}{\mathrm{CF}} 
\newcommand{\rmId}{\mathrm{Id}}
\newcommand{\rmid}{\mathrm{id}}
\newcommand{\rmtw}{\mathrm{tw}}
\newcommand{\trace}{\mathrm{tr}}
\newcommand{\Irr}{\mathrm{Irr}}
\newcommand{\Ind}{\mathrm{Ind}} % induction
\newcommand{\Res}{\mathrm{Res}} % restriction
\newcommand{\triv}{\mathrm{triv}} % trivial rep
\newcommand{\rmdef}{\mathrm{def}} % defining rep
\newcommand{\dom}{\triangleleft}
\newcommand{\domeq}{\trianglelefteq}
\newcommand{\lex}{\mathrm{lex}}
\newcommand{\sign}{\mathrm{sign}}
\newcommand{\SYT}{\mathrm{SYT}}
\renewcommand{\Im}{\mathrm{Im}}
\newcommand{\Ker}{\mathrm{Ker}}
\newcommand{\GL}{\mathrm{GL}}
\newcommand{\FL}{\mathrm{FL}}
\newcommand{\Span}{\mathrm{span}}
\newcommand{\pos}{\mathrm{pos}}
\newcommand{\Comp}{\mathrm{Comp}}
\newcommand{\Set}{\mathrm{Set}}
\newcommand{\std}{\mathrm{std}}
\newcommand{\cont}{\mathrm{cont}} %content of a SSYT
\newcommand{\SSYT}{\mathrm{SSYT}}
\newcommand{\rmz}{\mathrm{z}}
\newcommand{\ct}{\mathrm{ct}} % cycle type
\newcommand{\ch}{\mathrm{ch}} % Frobenius characteristic map
\newcommand{\height}{\mathrm{ht}}
\newcommand{\FPS}{\CC[\![\bfx]\!]} % formal power series
\newcommand{\FPSS}{\CC[\![\bfx,\bfy]\!]}
\newcommand{\reg}{\mathrm{reg}}
\newcommand{\hook}{\mathrm{h}}
\newcommand{\weight}{\mathrm{wt}}
\newcommand{\co}{\mathrm{co}}
\newcommand{\ps}{\mathrm{ps}}
\newcommand{\rmsum}{\mathrm{sum}}
\newcommand{\NSym}{\mathrm{NSym}}
\newcommand{\Hom}{\mathrm{Hom}}
\newcommand{\proj}{\mathrm{proj}}
\newcommand{\stat}{\mathrm{stat}}

% miscellaneous commands
\newcommand{\defn}[1]{{\color{mylightblue}{#1}}}
\newcommand{\toDo}{{\bf\color{red} TODO}}
\newcommand{\toCite}{{\bf\color{green} CITE}}

% 
\newenvironment{nouppercase}{%
  \let\uppercase\relax%
  \renewcommand{\uppercasenonmath}[1]{}}{}

% titlepage
\title{Θ2.04: Θεωρία Αναπαραστάσεων και Συνδυαστική}
\author[Β.~Δ. Μουστακας]{Βασίλης Διονύσης Μουστάκας \\ Πανεπιστήμιο Κρήτης}
\date{13 Οκτωβρίου 2025}
% \urladdr{\href{https://sites.google.com/view/vasmous}{https://sites.google.com/view/vasmous}}

\begin{document}

\begingroup
\def\uppercasenonmath#1{} % this disables uppercase title
\let\MakeUppercase\relax % this disables uppercase authors
\maketitle
\endgroup

\setcounter{section}{4}
\thispagestyle{empty}

\begin{center}
    \textbf{4. Η κανονική αναπαράσταση και ο τύπος διάστασης
}
\end{center}

Στην παράγραφο αυτή υποθέτουμε ότι $G$ είναι μια πεπερασμένη ομάδα και όλοι οι διανυσματικοί χώροι είναι πεπερασμένης διάστασης.

Το Θεώρημα του Maschke έστρεψε την προσοχή μας στα ανάγωγα $G$-πρότυπα και το Λήμμα του Schur μας είπε ότι μπορούμε να χρησιμοποιήσουμε τον χώρο $\Hom_G(\Arg,\Arg)$ για να \textquote{μετρήσουμε} πόσο απέχουν δυο $G$-πρότυπα από το να είναι ισόμορφα. Οι παρατηρήσεις αυτές εγείρουν τα εξής ερωτήματα.
\begin{questions}
\leavevmode
\begin{itemize}
    \item[(1)] Πόσα διαφορετικά ανάγωγα $G$-πρότυπα υπάρχουν ως προς ισομορφισμό?
    \item[(2)] Πως μπορούμε να υπολογίσουμε όλα τα ανάγωγα $G$-πρότυπα?
    \item[(3)] Υπάρχει εύκολος τρόπος να καταλάβουμε αν δυο ανάγωγα $G$-πρότυπα είναι ισόμορφα?
\end{itemize}
\end{questions}

Ας δούμε την ισοτυπική διάσπαση της κανονικής αναπαράστασης $\rho^\reg$ της $\rmC_4$. Ως προς τη συνήθη βάση $\{\epsilon, g, g^2, g^3\}$, 
\[
\rho^\reg(g) = 
\begin{pmatrix}
    0 & 0 & 0 & 1 \\
    1 & 0 & 0 & 0 \\
    0 & 1 & 0 & 0 \\
    0 & 0 & 1 & 0 \\
\end{pmatrix}.
\]
Ο πίνακας αυτός έχει τέσσερις ιδιοτιμές: $1, -1, i, -i$, με αντίστοιχα ιδιοδιανύσματα:
\[
\begin{pmatrix}
    1 \\ 
    1 \\ 
    1 \\
    1
\end{pmatrix}, \ 
\begin{pmatrix}
    1 \\ 
    -1 \\
    1 \\ 
    -1 
\end{pmatrix}, \ 
\begin{pmatrix}
    1 \\ 
    i \\ 
    -1 \\ 
    -i 
\end{pmatrix}, \ \text{και} \
\begin{pmatrix}
    1 \\ 
    -i \\
    -1 \\
    i
\end{pmatrix}.
\]
Συνεπώς, κάνοντας αλλαγή βάσης στην 
\[
\{
    \epsilon + g  + g^2 + g^3, \, 
    \epsilon - g  + g^2 - g^3, \, 
    \epsilon + ig - g^2 - ig^3, \, 
    \epsilon - ig - g^2 + ig^3
\}
\]
του $\CC[\rmC_4]$ έχουμε 
\[
\rho^\reg(g) = 
\begin{pmatrix}
    1 & 0 & 0 & 0 \\
    0 & -1 & 0 & 0 \\
    0 & 0 & i & 0 \\
    0 & 0 & 0 & -i 
\end{pmatrix}.
\]
Με άλλα λόγια, έχουμε τη διάσπαση 
\[
\CC[\rmC_4] = 
\CC[\epsilon + g  + g^2 + g^3] \oplus 
\CC[\epsilon - g  + g^2 - g^3] \oplus 
\CC[\epsilon + ig - g^2 - ig^3] \oplus 
\CC[\epsilon - ig - g^2 + ig^3],  
\]
όπου το $g$ δρα σε καθένα από τους προσθεταίους ως πολλαπλασιασμός με $1, -1, i$ και $-i$, αντίστοιχα (γιατί;).

Συγκρίνοντας με την Άσκηση~1.6 παρατηρούμε ότι κάθε ανάγωγη αναπαράσταση της $\rmC_4$ εμφανίζεται στην ισοτυπική διάσπαση της κανονικής αναπαράστασης της $\rmC_4$ με πολλαπλότητα ίση με τη διάστασή της. Αυτό ισχύει για κάθε ομάδα!

\begin{proposition}
    \label{prop:regularRep_irreducible_multiplicity}
    Η πολλαπλότητα εμφάνισης ενός ανάγωγου $G$-προτύπου στην ισοτυπική διάσπαση της κανονική αναπαράσταση της $G$ ισούται με τη διάστασή του.
\end{proposition}

\begin{proof}[Απόδειξη]
    Έστω $W$ ένα ανάγωγο $G$-πρότυπο. Από το Λήμμα~3.6, αρκεί να δείξουμε ότι 
    \[
    \dim\left(\Hom_G\left(\CC[G],W\right)\right) = \dim(W).
    \]
    Έστω $\{w_1, w_2, \dots, w_k\}$ μια βάση του $W$. Θεωρούμε τις απεικονίσεις 
    \begin{align*}
        \varphi_i : \CC[G] &\to W \\
                    g   &\mapsto gw_i
    \end{align*}
    για κάθε $1 \le i \le k$. Αυτοί είναι $G$-ομομορφισμοί (γιατί;) και το σύνολο τους αποτελεί βάση του $\Hom_G\left(\CC[G],W\right)$. Το τελευταίο αφήνεται ως άσκηση.
\end{proof}

Μια άμεση συνέπεια της Πρότασης~\ref{prop:regularRep_irreducible_multiplicity} είναι ότι το πλήθος των κλάσεων ισομορφισμού των ανάγωγων $G$-προτύπων είναι πεπερασμένο, απαντώντας μερικώς στο Ερώτημα~(α). Επιπλέον, υπολογίζοντας τις διαστάσεις στην ισοτυπική διάσπαση της κανονικής αναπαράστασης προκύπτει ο παρακάτω τύπος.

\begin{corollary}{\rm(Τύπος Διάστασης)}
    \label{cor:dimension_formula}
    Αν $\{V_1, V_2, \dots, V_n\}$ είναι ένα σύνολο ανά δυο μη ισόμορφων $G$-προτύπων, τότε 
    \[
    \CC[G] \cong_G V_1^{\dim(V_1)} \oplus V_2^{\dim(V_2)} \oplus \cdots \oplus V_n^{\dim(V_n)}.
    \]
    Ειδικότερα, 
    \begin{equation}
        \label{eq:dimension_formula}
        \abs{G} = \sum_{i=1}^n \left(\dim(V_i)\right)^2.
    \end{equation}
\end{corollary}

Η Ταυτότητα~\eqref{eq:dimension_formula} επιβάλει ισχυρούς περιορισμούς στο πλήθος των ανάγωγων $G$-προτύπων. Για παράδειγμα, μια άμεση συνέπεια είναι ότι το πλήθος τους είναι μικρότερο ή ίσο με την τάξη της $G$. Επιπλέον, για 
\begin{itemize}
    \item $\abs{G}= 2$ έχουμε 
    \[
    2 = 1^2 + 1^2
    \] 
    και γι' αυτό υπάρχουν δυο ανάγωγα $\rmC_2$-πρότυπα,
    \item $\abs{G} = 3$ έχουμε
    \[
    3 = 1^2 + 1^2 + 1^3
    \] 
    και γι' αυτό υπάρχει τρία ανάγωγα $\rmC_3$-πρότυπα,
    \item $\abs{G} =4 $ έχουμε 
    \[
    4 = 1^2 + 1^2 + 1^2 + 1^2
    \] 
    και γι' αυτό υπάρχουν τέσσερα ανάγωγα $\rmC_4$- ή $\rmV_4$-πρότυπα (ποιά είναι;), 
    \item $\abs{G} = 5$ έχουμε 
    \[
    5 = 1^2 + 1^2 + 1^2 + 1^2 + 1^2
    \]
    και γι' αυτό υπάρχουν πέντε ανάγωγα $\rmC_5$-πρότυπα,
    \item $\abs{G} = 6$ έχουμε 
    \[
    6 = 1^2 + 1^2 + 1^2 + 1^2 + 1^2 + 1^2 = 1^2 + 1^2 + 2^2,
    \]
    όπου η πρώτη περίπτωση αντιστοιχεί στην $\rmC_6$ η οποία έχει έξι ανάγωγα πρότυπα και η δεύτερη στην $\fS_3$ η οποία εμφανίζει το πρώτο ανάγωγο πρότυπο διάστασης 2 (ποιό είναι;),
    \item $\abs{G} = 7$ έχουμε 
    \[
    7 = 1^2 + 1^2 + 1^2 + 1^2 + 1^2 + 1^2 + 1^2,
    \]
    και γι' αυτό υπάρχουν επτά ανάγωγα $\rmC_7$-πρότυπα,
    \item $\abs{G} = 8$ έχουμε
    \begin{align*}
    8 &= 1^2 + 1^2 + 1^2 + 1^2 + 1^2 + 1^2 + 1^2 + 1^2 \\ 
    &=  1^2 + 1^2 + 1^2 + 1^2 + 2^2,
    \end{align*}
    όπου η πρώτη περίπτωση αντιστοιχεί στις αβελιανές ομάδες $\rmC_8 , \rmC_4\times\rmC_2$ και $\rmC_2\times\rmC_2\times\rmC_2$ οι οποίες έχουν οκτώ ανάγωγα πρότυπα και η δεύτερη περίπτωση στις μη-αβελιανές ομάδες $\rmD_8$ και $\rmQ_8$\footnote{Η $\rmQ_8$ είναι η  ομάδα των \emph{quaternions}. Μια παράστασή της είναι $\langle i, j \ \vert \ i^4=1, i^2 = j^2, i^{-1}ji=j^{-1}\rangle$. Για μια γεωμετρική εισαγωγή στην ομάδα αυτή δείτε \href{https://youtu.be/d4EgbgTm0Bg?si=oWdYRfOO3EwByIaq}{εδώ}.}, οι οποίες έχουν πέντε ανάγωγα πρότυπα. Στην Άσκηση~1.1 βλέπουμε την ανάγωγη αναπαράσταση διάστασης 2 της διεδρικής ομάδας (ποιές είναι οι υπόλοιπες;).
\end{itemize}

Η απόδειξη του Πορίσματος~\ref{cor:dimension_formula} δεν μας δίνει πληροφορίες για το πως να κατασκευάσουμε τα ανάγωγα $G$-πρότυπα. Ιδανικά, θα θέλαμε να βρούμε μια οικογένεια $\cC$ \emph{συνδυαστικών} αντικειμένων (για παράδειγμα, μεταθέσεις, ταμπλώ, γραφήματα, μερικές διατάξεις κ.ο.κ.) ώστε για κάθε στοιχείο του $c \in \cC$ να ορίσουμε ένα ανάγωγο $G$-πρότυπο $V_c$, το οποίο και αυτό ενδεχομένως θα δίνεται από μια άλλη οικογένεια συνδυαστικών αντικειμένων. Η ισοτυπική διάσπαση της κανονικής αναπαράστασης παίρνει έτσι την μορφή 
\[
\CC[G] \cong_G \bigoplus_{c \in \cC} V_c^{\dim(V_c)}.
\]

Στην περίπτωση αυτή, θα μπορούσαμε να επανααποδείξουμε τον τύπο διάστασης βρίσκοντας μια αμφιμονοσήμαντη απεικόνιση που εμπλέκει τα στοιχεία της $G$ με ζεύγη στοιχείων του $\cC$ και κάποιων άλλων συνδυαστικών αντικειμένων. Αυτό εν γένει είναι αρκετά δύσκολο. Θα δούμε παρακάτω πως να το κάνουμε στην περίπτωση της συμμετρικής ομάδας. Προβλήματα τέτοιου τύπου αποτελούν μέρος της \emph{συνδυαστικής θεωρίας αναπαραστάσεων}.

\newpage 

\setcounter{section}{5}
\setcounter{theorem}{0}
\begin{center}
    \textbf{5. Χαρακτήρες ομάδων: Εισαγωγή
}
\end{center}

Σε ότι ακολουθεί, υποθέτουμε ότι $G$ είναι μια πεπερασμένη ομάδα με ταυτοτικό στοιχείο $\epsilon$ και όλοι οι διανυσματικοί χώροι είναι πεπερασμένης διάστασης.

Όσο μεγαλώνει η διάσταση μιας αναπαραστάσης, τόσο πιο δύσκλο γίνεται να κάνει κανείς υπολογισμούς. Σ' αυτή την περίπτωση, και σκεπτόμενοι το Ερώτημα~(3), μια ιδέα θα ήταν να βρούμε μια \emph{αναλλοίωτη} μιας αναπαράστασης, δηλαδή μια \textquote{ποσότητα} που να την χαρακτηρίζει.

Μια τέτοια αναλλοίωτη, πρέπει να είναι αναλλοίωτη και από την αλλαγή βάσης, διότι όπως είδαμε στην Παράγραφο~3, δυο αναπαραστάσεις είναι ισόμορφες αν διαφέρουν κατά μια αλλαγή βάσης. Δυο προφανείς υποψήφιοι είναι η \emph{ορίζουσα} και το \emph{ίχνος} μιας γραμμικής απεικόνισης.

Η ορίζουσα δεν μπορεί να είναι η ζητούμενη αναλλοίωτη. Για παράδειγμα, ας θεωρήσουμε τις εξής αναπαραστάσεις της $\rmC_2 = \{\epsilon, g\}$
\[
    \rho(g) = 
        \begin{pmatrix}
            1 & 0 \\
            0 & 1
        \end{pmatrix} 
    \quad 
    \text{και}
    \quad
    \sigma(g) = 
        \begin{pmatrix}
            -1 & 0 \\
            0 & -1
        \end{pmatrix}.
\]
Οι αναπαραστάσεις $(\rho, \CC^2)$ και $(\sigma, \CC^2)$ δεν είναι ισόμορφές (γιατί;), αλλά 
\[
\det\left(\rho(g)\right) = 1 = \det\left(\sigma(g)\right).
\]
Από την άλλη μεριά,
\[
\trace\left(\rho(g)\right) = 2 \neq -2 = \trace\left(\sigma(g)\right).
\]

\begin{definition}
    \label{def:character}
    Έστω $(\rho,V)$ μια αναπαράσταση της $G$. Η απεικόνιση $\chi^{\rho,V} : G \to \CC$ (ή πιο απλά $\chi^V$), που ορίζεται θέτοντας 
    \[
    \chi^{\rho,V}(g) \coloneqq \trace\left(\rho(g)\right)
    \]
    ονομάζεται \defn{χαρακτήρας}\footnote{Η ορολογία που αναπτύσσουμε για τις αναπαραστάσεις θα χρησιμοποιείτε ελεύθερα για τους αντίστοιχους χαρακτήρες, όπου βγάζει νόημα. Για παράδειγμα, ένας χαρακτήρας θα λέγεται ανάγωγος, αν είναι ο χαρακτήρας μια ανάγωγης αναπαραστάασης.} της αναπαράστασης $(\rho,V)$.
\end{definition}

Στο τρέχον παράδειγμα, της αναπαράστασης της $\fS_3$ ως ομάδας συμμετρίας ενός ισόπλευρου τριγώνου, έχουμε
\begin{align*}
    \underbrace{\begin{pmatrix} 
        1 & 0 & 0 \\ 
        0 & 1 & 0 \\ 
        0 & 0 & 1 
    \end{pmatrix}
     }_{\cycle{1}\cycle{2}\cycle{3}} 
     \quad &\stackrel{\trace(\Arg)}{\mapstoto} \quad 3 \\
     \underbrace{\begin{pmatrix} 
        0 & 0 & 1 \\ 
        1 & 0 & 0 \\ 
        0 & 1 & 0
    \end{pmatrix}
     }_{\cycle{1,2,3}}, \ 
     \underbrace{\begin{pmatrix} 
        0 & 1 & 0 \\ 
        0 & 0 & 1 \\ 
        1 & 0 & 0
    \end{pmatrix}
     }_{\cycle{1,3,2}} 
     \quad &\stackrel{\trace(\Arg)}{\mapstoto} \quad 0 \\
     \underbrace{\begin{pmatrix} 
        0 & 1 & 0 \\ 
        1 & 0 & 0 \\ 
        0 & 0 & 1
    \end{pmatrix}
     }_{\cycle{1,2}\cycle{3}}, \ 
     \underbrace{\begin{pmatrix} 
        0 & 0 & 1 \\ 
        0 & 1 & 0 \\ 
        1 & 0 & 0
    \end{pmatrix}
     }_{\cycle{1,3}\cycle{2}}, \
     \underbrace{\begin{pmatrix} 
        1 & 0 & 0 \\ 
        0 & 0 & 1 \\ 
        0 & 1 & 0
    \end{pmatrix}
     }_{\cycle{2,3}\cycle{1}} 
     \quad &\stackrel{\trace(\Arg)}{\mapstoto} \quad 1. \\
\end{align*}

Αν $\chi^\rmdef$ είναι ο χαρακτήρας της αναπαράστασης καθορισμού της $\fS_n$, τότε 
\begin{align*}
\chi^\rmdef(\pi) &= \ \text{πλήθος των 1 στη διαγώνιο του $\rho^\rmdef(\pi)$} \\ 
&= \abs{\{i \in [n] : \pi_i = i\}} \\ 
&\coloneqq \fix(\pi)
\end{align*}
για κάθε $\pi \in \fS_n$. Με άλλα λόγια, ο χαρακτήρας της αναπαράστασης καθορισμού μιας μετάθεσης ισούται με το πλήθος των στοιχείων που μένουν σταθερά από την δράση της. 

Αυτό ισχύει γενικότερα, για κάθε αναπαράσταση μεταθέσεων, όπως θα δούμε στο δεύτερο φυλλάδιο ασκήσεων. Ειδικότερα, για τον χαρακτήρα $\chi^\reg$ της κανονικής αναπαράστασης της $G$ έχουμε
\[
\chi^\reg(g) = 
\begin{cases}
    \abs{G}, &\ \text{αν $g = \epsilon$} \\
    0, &\ \text{διαφορετικά}
\end{cases}
\]
(γιατί;).

\begin{combInterlude}
    Η απεικόνιση $\fix : \fS_n \to \NN$ αποτελεί παράδειγμα \emph{στατιστικ΄ής μεταθέσεων}. Γενικά, \defn{στατιστική} ενός συνόλου $\cC$ συνδυαστικών αντικειμένων ονομάζεται μια απεικόνιση 
    \[
    \stat : \cC \to \NN.
    \]
    Η $\stat$ εκλεπτύνει την απαρίθμηση στο $\cC$.

    Για παράδειγμα, αν $\cC = 2^{[n]}$ είναι το σύνολο όλων των υποσυνόλων του $[n]$, τότε ο πληθάριθμος $\abs{\Arg} : 2^{[n]} \to \NN$ είναι μια στατιστική η οποία εκλεπτύνει το πλήθος των στοιχείων του $2^{[n]}$ με την εξής έννοια 
    \[
    2^n = \sum_{S \ \in \ 2^{[n]}} \abs{S} = \sum_{k=0}^n \binom{n}{k},
    \]
    όπου με $\binom{n}{k}$ συμβολίζουμε τον \emph{δυωνυμικό συντελεστή}, δηλαδή το πλήθος των υποσυνόλων του $[n]$ με $k$ στοιχεία.

    Οι στατιστικές μεταθέσεων αποτελούν βασικό αντικείμενο μελέτης της \emph{απαριθμητικής} και \emph{αλγεβρικής συνδυαστικής}. Παρακάτω θα δούμε και άλλα παραδείγματα στατιστικών μεταθέσεων.
\end{combInterlude}

\begin{proposition}
    \label{prop:character_properties}
    Έστω $(\rho,V)$ μια αναπαράσταση της $G$.
    \begin{itemize}
        \item[(1)] Ισχύει ότι
        $
        \chi^{\rho,V}(\epsilon) = \dim(V).
        $
        \item[(2)] Ο χαρακτήρας μια αναπαράστασης της $G$ έχει σταθερή τιμή στις κλάσεις συζυγίας της $G$.
        \item[(3)] Δυο ισόμορφες αναπαραστάσεις έχουν τον ίδιο χαρακτήρα.
    \end{itemize}
\end{proposition}

\begin{proof}[Απόδειξη]
    To (1) έπεται από το ότι ο πίνακας του $\rho(\epsilon)$ είναι ο ταυτοτικός πίνακας διάστασης $\dim(V)$. Για το (2), θεωρούμε δυο συζυγή $g, h \in G$, δηλαδή $g = xhx^{-1}$ για κάποιο $x \in G$. Τότε,
    \begin{align*}
    \chi^{\rho,V}(g) 
    &= \trace\left(\rho(g)\right) \\
    &= \trace\left(\rho(xhx^{-1})\right) \\
    &= \trace\left(\rho(x)\rho(h)\left(\rho(x)\right)^{-1}\right) \\
    &= \trace\left(\rho(h)\right) \\
    &= \chi^{\rho,V}(h),
    \end{align*}
    όπου η τρίτη ισότητα έπεται από το ότι ο $\rho$ είναι ομομορφισμός ομάδων και η τέταρτη από το ότι το ίχνος είναι αναλλοίωτο για όμοιους πίνακες\footnote{Δυο τετραγωνικοί πίνακες $A,B$ ίδιας διάστασης ονομάζονται \defn{όμοιοι} αν υπάρχει αντιστρέψιμος πίνακας $P$ τέτοιος ώστε $B = PAP^{-1}$. Με άλλα λόγια, όμοιοι είναι οι συζυγείς πίνακες στην γενική γραμμική ομάδα.}. Όμοια έπεται και το (3).
\end{proof}
\end{document}