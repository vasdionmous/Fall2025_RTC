\documentclass[12pt,a4paper,reqno]{amsart}

% language
\usepackage[greek,english]{babel}
\usepackage[utf8]{inputenc}
\usepackage{alphabeta}

% change default names to greek
\addto\captionsenglish{
    \renewcommand{\contentsname}{Περιεχόμενα}
    \renewcommand{\refname}{Βιβλιογραφία}
    \renewcommand{\datename}{Ημερομηνία:}
    \renewcommand{\urladdrname}{Ιστοσελίδα}
}

% math 
\usepackage{amsmath,amsthm,amssymb,amscd}

% font
\usepackage[cal=euler]{mathalfa}
\usepackage{libertinus-type1}
% \usepackage{txfonts} % for upright greek letters
\usepackage{bm} % for bold symbols
\usepackage{bbm} % for the simply-looking bb symbols

% miscellaneous 
\usepackage{changepage} %for indenting environments
\usepackage{csquotes} % example: \textcquote{}
\usepackage{blkarray}

% drawing
\usepackage{tikz,tikz-cd}
\usetikzlibrary{shapes.misc, patterns, matrix, calc, intersections,positioning}
\usepackage{graphics,graphicx}
\usepackage{float} % provides enhanced control and customization options for floating objects such as figures and tables

% colors
\usepackage{xcolor}
\definecolor{darkcandyapplered}{rgb}{0.64, 0.0, 0.0}
\definecolor{midnightblue}{rgb}{0.1, 0.1, 0.44}
\definecolor{mylightblue}{HTML}{336699}
\definecolor{burntorange}{rgb}{0.8, 0.33, 0.0}
\definecolor{iceberg}{rgb}{0.44, 0.65, 0.82}
\definecolor{applegreen}{rgb}{0.55, 0.71, 0.0}
\definecolor{canaryyellow}{rgb}{1.0, 0.94, 0.0}

% hrefs
\usepackage{hyperref}
\usepackage[noabbrev,capitalize]{cleveref}
\hypersetup{
    pdftoolbar=true,        
    pdfmenubar=true,        
    pdffitwindow=false,     
    pdfstartview={FitH},  % fits the width of the page to the window
    pdftitle={},
    pdfauthor={},
    pdfsubject={},
    pdfkeywords={},
    pdfnewwindow=true,  % links in new window
    colorlinks=true,  % false: boxed links; true: colored links
    linkcolor=darkcandyapplered,   % color of internal links
    citecolor=midnightblue,  % color of links to bibliography
    urlcolor=cyan,  % color of external links
    linktocpage=true  % changes the links from the section body to the page number
    }

% geometry
\textwidth=16cm 
\textheight=21cm 
\hoffset=-55pt 
\footskip=25pt

% thm envs (you might need to change the path)
% In this macro I define all the theorem environments

\theoremstyle{definition}
\newtheorem{theorem}{Θεώρημα}
\newtheorem{proposition}[theorem]{Πρόταση}
\newtheorem{lemma}[theorem]{Λήμμα}
\newtheorem{corollary}[theorem]{Πόρισμα}
\newtheorem{conjecture}[theorem]{Εικασία}
\newtheorem{problem}[theorem]{Πρόβλημα}
\newtheorem*{claim}{Ισχυρισμός}
\newtheorem{observation}[theorem]{Παρατήρηση}
\newtheorem{definition}[theorem]{Ορισμός}
\newtheorem{question}[theorem]{Ερώτηση}
\newtheorem*{questions}{Ερωτήματα}
\newtheorem{example}[theorem]{Παράδειγμα}
\newtheorem{exercise}{Άσκηση}

\newtheorem*{combInterlude}{Ιντερλούδιο Συνδυαστικής}
\newtheorem*{example_cont}{Παράδειγμα~6.6}
\newtheorem*{digression_la}{Παρέκβαση Γραμμικής Άλγεβρας}
\newtheorem*{thm}{Θεώρημα}

\theoremstyle{remark}
\newtheorem*{remark}{Παρατήρηση}

% fixes the correct numbering of environments
\numberwithin{theorem}{section}
\numberwithin{exercise}{section}
\numberwithin{equation}{section}

% math ops (you might need to change the path)
% In this macro I define all of my math operators

% fields
\newcommand{\NN}{\mathbbmss{N}} 
\newcommand{\ZZ}{\mathbbmss{Z}} 
\newcommand{\QQ}{\mathbbmss{Q}} 
\newcommand{\RR}{\mathbbmss{R}} 
\newcommand{\CC}{\mathbbmss{C}} 
\newcommand{\KK}{\mathbbmss{K}} 
\newcommand{\FF}{\mathbbmss{F}} 

% symmetric group
\newcommand{\fS}{\mathfrak{S}}  

% calligraphic 
\newcommand{\aA}{\mathcal{A}} 
\newcommand{\bB}{\mathcal{B}}
\newcommand{\cC}{\mathcal{C}}
\newcommand{\dD}{\mathcal{D}}
\newcommand{\eE}{\mathcal{E}}
\newcommand{\fF}{\mathcal{F}}
\newcommand{\hH}{\mathcal{H}}
\newcommand{\iI}{\mathcal{I}}
\newcommand{\lL}{\mathcal{L}}
\newcommand{\oO}{\mathcal{O}}
\newcommand{\pP}{\mathcal{P}}
\newcommand{\sS}{\mathcal{S}}
\newcommand{\mM}{\mathcal{M}}
\newcommand{\uU}{\mathcal{U}}

% bold
\newcommand{\bfa}{\mathbf{a}}
\newcommand{\bfe}{\mathbf{e}}
\newcommand{\bfF}{\pmb{F}}
\newcommand{\bfR}{\pmb{R}}
\newcommand{\bfv}{\mathbf{v}}
%\newcommand{\bfx}{\bm{x}}
%\newcommand{\bfx}{\mathbf{x}} 
\newcommand{\bfx}{\pmb{x}}
\newcommand{\bfX}{\pmb{X}}
\newcommand{\bfy}{\pmb{y}}
\newcommand{\bfz}{\pmb{z}}

% roman
\newcommand{\rmA}{\mathrm{A}}
\newcommand{\rmB}{\mathrm{B}}
\newcommand{\rmC}{\mathrm{C}}
\newcommand{\rmD}{\mathrm{D}} 
\newcommand{\rmI}{\mathrm{I}} 
\newcommand{\rmK}{\mathrm{K}}
\newcommand{\rmM}{\mathrm{M}}
\newcommand{\rmP}{\mathrm{P}}  
\newcommand{\rmp}{\mathrm{p}}  
\newcommand{\rmQ}{\mathrm{Q}}  
\newcommand{\rmR}{\mathrm{R}}
\newcommand{\rmS}{\mathrm{S}}
\newcommand{\rmT}{\mathrm{T}}
\newcommand{\rmU}{\mathrm{U}}
\newcommand{\rmV}{\mathrm{V}}
\newcommand{\rmY}{\mathrm{Y}}
\newcommand{\rmZ}{\mathrm{Z}}
\newcommand{\rmz}{\mathrm{z}}

% greek letters
% I'm renewing some commands in order to appear in upright font
% If I want to change it later, I don't have to do it manually, I just change it from here.
% \newcommand{\uaa}{\alphaup}
% \renewcommand{\alpha}{\alphaup}
% \renewcommand{\beta}{\betaup}
% \renewcommand{\gamma}{\gammaup}
% \renewcommand{\delta}{\deltaup}
% \renewcommand{\epsilon}{\epsilonup}
% \newcommand{\ee}{\epsilon}
% \renewcommand{\varepsilon}{\varepsilonup}
% \renewcommand{\theta}{\thetaup}
% \renewcommand{\lambda}{\lambdaup}
% \newcommand{\ull}{\lambda}
% \renewcommand{\mu}{\muup}
% \renewcommand{\nu}{\nuup}
% \renewcommand{\pi}{\piup}
% \renewcommand{\rho}{\rhoup}
% \renewcommand{\varrho}{\varrhoup}
% \renewcommand{\sigma}{\sigmaup}
% \renewcommand{\tau}{\tauup} 
% \renewcommand{\phi}{\phiup}
% \renewcommand{\chi}{\chiup}
% \renewcommand{\psi}{\psiup}
% \renewcommand{\omega}{\omegaup}

% arrows and symbols 
\renewcommand{\to}{\rightarrow}
\newcommand{\toto}{\longrightarrow}
\newcommand{\mapstoto}{\longmapsto}
\newcommand{\then}{\Rightarrow}
\newcommand{\IFF}{\Leftrightarrow}
\newcommand{\tl}{\tilde}
\newcommand{\wtl}{\widetilde}
\newcommand{\ol}{\overline}
\newcommand{\ul}{\underline}
\newcommand{\oldemptyset}{\emptyset}
\renewcommand{\emptyset}{\varnothing}
\DeclareMathSymbol{\Arg}{\mathbin}{AMSa}{"39} % for arguments 
\newcommand{\onto}{\ensuremath{\twoheadrightarrow}}
\newcommand{\tle}{\trianglelefteq}
\newcommand{\tge}{\trianglerighteq}

% absolute value symbol
\usepackage{mathtools} 
\DeclarePairedDelimiter\abs{\lvert}{\rvert}%
\DeclarePairedDelimiter\norm{\lVert}{\rVert}%
\makeatletter
\let\oldabs\abs
\def\abs{\@ifstar{\oldabs}{\oldabs*}}

% tensor symbol
\newcommand{\tensor}[1]{%
  \mathbin{\mathop{\otimes}\limits_{#1}}%
}

% permutation cycle notation
\ExplSyntaxOn
\NewDocumentCommand{\cycle}{ O{\;} m }
 {
  (
  \alec_cycle:nn { #1 } { #2 }
  )
 }

\seq_new:N \l_alec_cycle_seq
\cs_new_protected:Npn \alec_cycle:nn #1 #2
 {
  \seq_set_split:Nnn \l_alec_cycle_seq { , } { #2 }
  \seq_use:Nn \l_alec_cycle_seq { #1 }
 }
\ExplSyntaxOff

% setminus symbol
\newcommand{\mysetminusD}{\hbox{\tikz{\draw[line width=0.6pt,line cap=round] (3pt,0) -- (0,6pt);}}}
\newcommand{\mysetminusT}{\mysetminusD}
\newcommand{\mysetminusS}{\hbox{\tikz{\draw[line width=0.45pt,line cap=round] (2pt,0) -- (0,4pt);}}}
\newcommand{\mysetminusSS}{\hbox{\tikz{\draw[line width=0.4pt,line cap=round] (1.5pt,0) -- (0,3pt);}}}
\newcommand{\sm}{\mathbin{\mathchoice{\mysetminusD}{\mysetminusT}{\mysetminusS}{\mysetminusSS}}}

% custom math operators
\newcommand{\Des}{\mathrm{Des}} 
\newcommand{\des}{\mathrm{des}} 
\newcommand{\Asc}{\mathrm{Asc}}
\newcommand{\asc}{\mathrm{asc}} 
\newcommand{\inv}{\mathrm{inv}}
\newcommand{\Inv}{\mathrm{Inv}}
\newcommand{\maj}{\mathrm{maj}} 
\newcommand{\comaj}{\mathrm{comaj}} 
\newcommand{\fix}{\mathrm{fix}} 
\newcommand{\Sym}{\mathrm{Sym}} 
\newcommand{\QSym}{\mathrm{QSym}}
\newcommand{\FQSym}{\mathrm{FQSym}} 
\newcommand{\End}{\mathrm{End}} 
\newcommand{\Rad}{\mathrm{Rad}} 
\newcommand{\rmMat}{\mathrm{Mat}} 
\newcommand{\rmdim}{\mathrm{dim}} 
\newcommand{\rmTop}{\mathrm{Top}} 
\newcommand{\rmCF}{\mathrm{CF}} 
\newcommand{\rmId}{\mathrm{Id}}
\newcommand{\rmid}{\mathrm{id}}
\newcommand{\rmtw}{\mathrm{tw}}
\newcommand{\trace}{\mathrm{tr}}
\newcommand{\Irr}{\mathrm{Irr}}
\newcommand{\Ind}{\mathrm{Ind}} % induction
\newcommand{\Res}{\mathrm{Res}} % restriction
\newcommand{\triv}{\mathrm{triv}} % trivial rep
\newcommand{\rmdef}{\mathrm{def}} % defining rep
\newcommand{\dom}{\triangleleft}
\newcommand{\domeq}{\trianglelefteq}
\newcommand{\lex}{\mathrm{lex}}
\newcommand{\sign}{\mathrm{sign}}
\newcommand{\SYT}{\mathrm{SYT}}
\renewcommand{\Im}{\mathrm{Im}}
\newcommand{\Ker}{\mathrm{Ker}}
\newcommand{\GL}{\mathrm{GL}}
\newcommand{\FL}{\mathrm{FL}}
\newcommand{\Span}{\mathrm{span}}
\newcommand{\pos}{\mathrm{pos}}
\newcommand{\Comp}{\mathrm{Comp}}
\newcommand{\Set}{\mathrm{Set}}
\newcommand{\std}{\mathrm{std}}
\newcommand{\cont}{\mathrm{cont}} %content of a SSYT
\newcommand{\SSYT}{\mathrm{SSYT}}
\newcommand{\ct}{\mathrm{ct}} % cycle type
\newcommand{\ch}{\mathrm{ch}} % Frobenius characteristic map
\newcommand{\height}{\mathrm{ht}}
\newcommand{\FPS}{\CC[\![\bfx]\!]} % formal power series
\newcommand{\FPSS}{\CC[\![\bfx,\bfy]\!]}
\newcommand{\reg}{\mathrm{reg}}
\newcommand{\hook}{\mathrm{h}}
\newcommand{\weight}{\mathrm{wt}}
\newcommand{\co}{\mathrm{co}}
\newcommand{\ps}{\mathrm{ps}}
\newcommand{\rmsum}{\mathrm{sum}}
\newcommand{\NSym}{\mathrm{NSym}}
\newcommand{\Hom}{\mathrm{Hom}}
\newcommand{\proj}{\mathrm{proj}}
\newcommand{\stat}{\mathrm{stat}}
\newcommand{\Par}{\mathrm{Par}}
\newcommand{\rmset}{\mathrm{set}}
\newcommand{\comp}{\mathrm{comp}}

% miscellaneous commands
\newcommand{\defn}[1]{{\color{mylightblue}{#1}}}
\newcommand{\toDo}{{\bf\color{red} TODO}}
\newcommand{\toCite}{{\bf\color{green} CITE}}
\newcommand*{\vertbar}{\rule[-1ex]{0.5pt}{2.5ex}} % for matrices with column vectors
\newcommand*{\horzbar}{\rule[.5ex]{2.5ex}{0.5pt}} % for matrices with row vectors
\newcommand{\myblue}[1]{{\color{iceberg}{#1}}}
\newcommand{\myred}[1]{{\color{burntorange}{#1}}}
\newcommand{\mygreen}[1]{{\color{applegreen}{#1}}}

% 
\newenvironment{nouppercase}{%
  \let\uppercase\relax%
  \renewcommand{\uppercasenonmath}[1]{}}{}

% titlepage
\title{Θ2.04: Θεωρία Αναπαραστάσεων και Συνδυαστική}
\author[Β.~Δ. Μουστακας]{Βασίλης Διονύσης Μουστάκας \\ Πανεπιστήμιο Κρήτης}
\date{4 Νοεμβρίου 2025}
% \urladdr{\href{https://sites.google.com/view/vasmous}{https://sites.google.com/view/vasmous}}

\begin{document}

\begingroup
\def\uppercasenonmath#1{} % this disables uppercase title
\let\MakeUppercase\relax % this disables uppercase authors
\maketitle
\endgroup


\setcounter{section}{8}
\setcounter{theorem}{0}
\begin{center}
    \textbf{8. Περιορισμός, επαγωγή και ο νόμος αντιστροφής Frobenius
}
\end{center}

Έστω $G$ πεπερασμένη ομάδα και $H$ υποομάδα της $G$.

\begin{que}
Αν γνωρίζουμε τις αναπαραστάσεις της $H$, τι μπορούμε να πούμε για τις αναπαραστάσεις της $G$ και αντίστροφα;
\end{que}

Αν $(\rho,V)$ είναι μια αναπαράσταση της $G$, τότε \textquote{περιορίζοντας} τη δράση στην $H$ παίρνουμε μια αναπαράσταση της $H$. Για παράδειγμα, αν $G = \fS_3$ και $H = \{\epsilon, \cycle{2,3}\}$ είναι η υποομάδα της $\fS_3$ που παράγεται από το $\cycle{2,3}$, τότε η αναπαράσταση καθορισμού $(\rho^\rmdef, \CC[1,2,3])$ της $\fS_3$ \textquote{περιορισμένη} στην $H$
\[
\rho^\rmdef(\epsilon) = 
\begin{pmatrix}
    1 & 0 & 0 \\
    0 & 1 & 0 \\
    0 & 0 & 1 \\
\end{pmatrix}, 
\quad 
\rho^\rmdef(\cycle{2,3}) = 
\begin{pmatrix}
    1 & 0 & 0 \\
    0 & 0 & 1 \\
    0 & 1 & 0 \\
\end{pmatrix}
\]
δίνει μια αναπαράσταση της $H$ διάστασης 3.
\begin{definition}
    \label{def:restriction}
    Η αναπαράσταση $\left(\rho\downarrow_H^G, V\right)$, όπου $\rho\downarrow_H^G : H \to \GL(V)$ ορίζεται θέτοντας 
    \[
    \rho\downarrow_H^G(h) = \rho(h)
    \]
    για κάθε $h \in H$ ονομάζεται \defn{περιορισμός} (restriction) της $(\rho, V)$ στην $H$. Ο χαρακτήρας της συμβολίζεται με $\chi^{\rho,V}\downarrow_H^G$ και ικανοποιεί 
    \[
    \chi^{\rho,V}\downarrow_H^G(h) = \chi^{\rho,V}(h)
    \]
    για κάθε $h \in H$.
\end{definition}

Παρόλο που περιορίζοντας μια αναπαράσταση δεν αλλάζει ο χαρακτήρας της, πρέπει να είναι κανείς προσεκτικός, καθώς 
\begin{itemize}
    \item ο περιορισμός μιας ανάγωγης αναπαράστασης δεν είναι κατ' ανάγκη ανάγωγη αναπαράσταση της $H$
    \item δυο συζυγή στοιχεία στην $G$ δεν είναι κατ' ανάγκη συζυγή στοιχεία στην $H$.
\end{itemize}
Για παράδειγμα, έστω $G=\fS_3$ και $H = \rmA_3 = \{\epsilon, \cycle{1,2,3}, \cycle{1,3,2}\}$ η εναλλάσσουσα υποομάδα. Επειδή $\rmA_3 \cong \rmC_3$, ο περιορισμός της συνήθους αναπαράστασης $\rho^\std\downarrow_{\rmA_3}^{\fS_3}$ δεν μπορεί να είναι ανάγωγη (γιατί;). Επίσης, η κλάση συζυγίας $\rmK_3 = \{\cycle{1,2,3}, \cycle{1,3,2}\}$ της $\fS_3$ \textquote{σπάει} σε δύο κλάσεις συζυγίας στην $\rmA_3$, η οποία ως κυκλική ομάδα έχει πίνακα χαρακτήρων 
\[
\renewcommand{\arraystretch}{1.2}
\begin{array}{l|c|c|c}
           & \{\epsilon\} & \{\cycle{1,2,3}\}   & \{\cycle{1,3,2}\} \\ \hline
    \chi_1 & 1            & 1       & 1 \\ \hline
    \chi_2 & 1            & \zeta   & \zeta^2 \\ \hline
    \chi_3 & 1            & \zeta^2 & \zeta
\end{array}\ ,
\]
όπου $\zeta$ είναι μια τρίτη ρίζα της μονάδας. Από τον Ορισμό~\ref{def:restriction}, έπεται ότι 
\[
\renewcommand{\arraystretch}{1.4} 
\begin{array}{c|c|c|c}
                                     & \{\epsilon\} & \{\cycle{1,2,3}\}   & \{\cycle{1,3,2}\} \\ \hline
\chi^\std\downarrow_{\rmA_3}^{\fS_3} & 2          & -1       & -1        
\end{array}
\]
και γι αυτό έχουμε την ισοτυπική διάσπαση 
\[
\chi^\std\downarrow_{\rmA_3}^{\fS_3} \ = \chi_2 + \chi_3,
\]
λόγω της σχέσης $\zeta + \zeta^2 = -1$ (γιατί;). 

Ας υποθέσουμε τώρα ότι $(\sigma,W)$ είναι μια αναπαράσταση της $H$. Ένας τρόπος να επεκτείνουμε την αναπαράσταση αυτή σε ολόκληρη την $G$ είναι με το να θεωρήσουμε \textquote{αντίτυπα} της δράσης της $\sigma$ σε κάθε \emph{συζυγές υποσύνολο}\footnote{Τα υποσύνολα της $G$ που είναι συζυγή ως προς την $H$ έχουν την μορφή $xHx^{-1} = \{xhx^{-1} : h \in H\}$ για κάθε $x \in G$.} της $H$ στην $G$. 

Πιο συγκεκριμένα, έστω $\{x_1, x_2, \dots, x_k\}$ ένα σύνολο αντιπροσώπων των αριστερών κλά\-σεων της $H$ στην $G$, δηλαδή 
\[
G = x_1H \biguplus x_2H \biguplus \cdots \biguplus x_kH.
\]
Θεωρούμε τον διανυσματικό χώρο 
\begin{equation}
\label{eq:induction_module}
\left(\CC[x_1]\otimes W\right) \oplus 
\left(\CC[x_2]\otimes W\right) \oplus \cdots \oplus 
\left(\CC[x_k]\otimes W\right),
\end{equation}
όπου $\CC[x_i]$ είναι ο διανυσματικός υπόχώρος της κανονικής αναπαράστασης της $G$ που παρά\-γεται από το στοιχείο $x_i$, για κάθε $1 \le i \le k$. Με άλλα λόγια, θεωρήσαμε το ευθύ άθροισμα $k$ \textquote{αντίτυπων} της $W$, ένα για κάθε αντιπρόσωπο των αριστερών κλάσεων της $H$ στην $G$. Αν $g \in G$, τότε για κάθε $1 \le j \le k$, υπάρχει (μοναδικό) $1 \le i \le k$ τέτοιο ώστε $gx_j \in x_iH$ και γι αυτό $gx_j = x_ih$ για κάποιο $h \in H$. Δίνουμε στον χώρο \eqref{eq:induction_module} τη δομή $G$-προτύπου θέτοντας 
\[
g \cdot\left( x_j \otimes w \right) \coloneqq x_i \otimes \left( \sigma(h)(w) \right)
\]
για κάθε $g \in G, 1 \le j \le k$ και $w \in W$. Με άλλα λόγια, το $g$ δρα στον προσθεταίο $\CC[x_j] \otimes W$ \textquote{στέλνοντάς} τον στο $\CC[x_i] \otimes W$ και δρώντας στο $W$ με την $\sigma(h)$.

\begin{example}
    \label{ex:induction_S3}
    Ας δούμε την παραπάνω κατασκευή για $G = \fS_3$ και $H = \{\epsilon, \cycle{2,3}\}$ την υποομάδα που παράγεται από το $\cycle{2,3}$. Έχουμε την διαμέριση
    \[
    \fS_3 = \underbrace{\myblue{\epsilon}H}_{=\,\{\epsilon, \, \cycle{2,3}\}} \biguplus \underbrace{\myred{\cycle{1,2}}H}_{= \, \{\cycle{1,2}, \, \cycle{1,2,3}\}}\biguplus \underbrace{\mygreen{\cycle{1,3}}H}_{=\, \{\cycle{1,3},\, \cycle{1,3,2}\}}.
    \]
    Έστω $(\sigma, W)$ μια αναπαράσταση της $H$. Για κάθε $g \in \fS_3$, θα υπολογίσουμε τον πίνακα της δράσης του $g$ στον χώρο 
    \[
    \left(\CC[\myblue{\epsilon}]\otimes W\right) \oplus 
    \left(\CC[\myred{\cycle{1,2}}]\otimes W\right) \oplus 
    \left(\CC[\mygreen{\cycle{1,3}}]\otimes W\right).
    \]
    Τους πίνακες αυτούς θα τους εκφράσουμε ως $(3\times{3})$-μπλοκ πίνακες, των οποίων οι γραμμές και οι στήλες καθορίζονται από το τι κάνει το $g$ στον προσθεταίο $\CC[x]\otimes W$, για κάθε $x \in \{\myblue{\epsilon}, \myred{\cycle{1,2}}, \mygreen{\cycle{1,3}}\}$.

    Για να το κάνουμε αυτό, πρέπει να λύσουμε τις εξισώσεις $gx_j = x_ih$, ως προς $1 \le i \le 3$ και $h \in H$. Έχουμε 
    \[
    \renewcommand{\arraystretch}{1.2}
    \begin{array}{c|ccc|ccc|ccc}
    g             & g\myblue{\epsilon} & x_i                   & h            & g\myred{\cycle{1,2}} & x_i                   & h           & g\mygreen{\cycle{1,3}}  & x_i & h \\ \hline
    \epsilon      & \epsilon           & \myblue{\epsilon}     & \epsilon     & \cycle{1,2}          & \myred{\cycle{1,2}}   & \epsilon    & \cycle{1,3}        & \mygreen{\cycle{1,3}} & \epsilon \\ \hline
    \cycle{1,2}   & \cycle{1,2}        & \myred{\cycle{1,2}}   & \epsilon     & \epsilon             & \myblue{\epsilon}     & \epsilon    & \cycle{1,3,2}      & \mygreen{\cycle{1,3}} & \cycle{2,3} \\ \hline
    \cycle{1,3}   & \cycle{1,3}        & \mygreen{\cycle{1,3}} & \epsilon     & \cycle{1,2,3}        & \myred{\cycle{1,2}}   & \cycle{2,3} & \epsilon           & \myblue{\epsilon}     & \epsilon \\ \hline
    \cycle{2,3}   & \cycle{2,3}        & \myblue{\epsilon}     & \cycle{2,3}  & \cycle{1,3,2}        & \mygreen{\cycle{1,3}} & \cycle{2,3} & \cycle{2,3}        & \myblue{\epsilon}     & \cycle{2,3} \\ \hline
    \cycle{1,2,3} & \cycle{1,2,3}      & \myred{\cycle{1,2}}   & \cycle{2,3}  & \cycle{1,3}          & \mygreen{\cycle{1,3}} & \epsilon    & \cycle{1,2}        & \myred{\cycle{1,2}}   & \epsilon \\ \hline
    \cycle{1,3,2} & \cycle{1,3,2}      & \mygreen{\cycle{1,3}} & \cycle{2,3}  & \cycle{2,3}          & \myblue{\epsilon}     & \cycle{2,3} & \cycle{1,2,3}      & \myred{\cycle{1,2}}   & \cycle{2,3} \\ 
\end{array}\ .
\]
Συνεπώς, οι ζητούμενοι πίνακες είναι 

\begin{align*}
\epsilon &\mapsto 
\begin{blockarray}{cccc}
    & \myblue{\epsilon}\otimes W & \myred{\cycle{1,2}}\otimes W & \mygreen{\cycle{1,3}}\otimes W \\
\begin{block}{r(ccc)}
  \myblue{\epsilon}\otimes W     & \sigma(\epsilon) & 0                 & 0 \\
  \myred{\cycle{1,2}}\otimes W   & 0                & \sigma(\epsilon)  & 0 \\
  \mygreen{\cycle{1,3}}\otimes W & 0                & 0                 & \sigma(\epsilon) \\
\end{block}
\end{blockarray}
\hspace{-1.5em}
\\
\cycle{1,2} &\mapsto 
\begin{blockarray}{cccc}
    & \myblue{\epsilon}\otimes W & \myred{\cycle{1,2}}\otimes W & \mygreen{\cycle{1,3}}\otimes W \\
\begin{block}{r(ccc)}
  \myblue{\epsilon}\otimes W     & 0                & \sigma(\epsilon) & 0 \\
  \myred{\cycle{1,2}}\otimes W   & \sigma(\epsilon) & 0                & 0 \\
  \mygreen{\cycle{1,3}}\otimes W & 0                & 0                & \sigma(\cycle{2,3}) \\
\end{block}
\end{blockarray}
\\
\cycle{1,3} &\mapsto 
\begin{blockarray}{cccc}
    & \myblue{\epsilon}\otimes W & \myred{\cycle{1,2}}\otimes W & \mygreen{\cycle{1,3}}\otimes W \\
\begin{block}{r(ccc)}
  \myblue{\epsilon}\otimes W     & 0                & 0                   & \sigma(\epsilon) \\
  \myred{\cycle{1,2}}\otimes W   & 0                & \sigma(\cycle{2,3}) & 0 \\
  \mygreen{\cycle{1,3}}\otimes W & \sigma(\epsilon) & 0                   & 0 \\
\end{block}
\end{blockarray}
\\
\cycle{2,3} &\mapsto 
\begin{blockarray}{cccc}
    & \myblue{\epsilon}\otimes W & \myred{\cycle{1,2}}\otimes W & \mygreen{\cycle{1,3}}\otimes W \\
\begin{block}{r(ccc)}
  \myblue{\epsilon}\otimes W     & \sigma(\cycle{2,3}) & 0                   & 0 \\
  \myred{\cycle{1,2}}\otimes W   & 0                   & 0                   & \sigma(\epsilon) \\
  \mygreen{\cycle{1,3}}\otimes W & 0                   & \sigma(\cycle{2,3}) & 0 \\
\end{block}
\end{blockarray}
\end{align*}
%
\begin{align*}
\cycle{1,2,3} &\mapsto 
\begin{blockarray}{cccc}
    & \myblue{\epsilon}\otimes W & \myred{\cycle{1,2}}\otimes W & \mygreen{\cycle{1,3}}\otimes W \\
\begin{block}{r(ccc)}
  \myblue{\epsilon}\otimes W     & 0                   & 0                   & \sigma(\epsilon) \\
  \myred{\cycle{1,2}}\otimes W   & \sigma(\cycle{2,3}) & 0                   & 0 \\
  \mygreen{\cycle{1,3}}\otimes W & 0                   & \sigma(\cycle{2,3}) & 0 \\
\end{block}
\end{blockarray}
\\
\cycle{1,3,2} &\mapsto 
\begin{blockarray}{cccc}
    & \myblue{\epsilon}\otimes W & \myred{\cycle{1,2}}\otimes W & \mygreen{\cycle{1,3}}\otimes W \\
\begin{block}{r(ccc)}
  \myblue{\epsilon}\otimes W     & 0                   & \sigma(\cycle{2,3}) & 0 \\
  \myred{\cycle{1,2}}\otimes W   & 0                   & 0                   & \sigma(\cycle{2,3}) \\
  \mygreen{\cycle{1,3}}\otimes W & \sigma(\cycle{2,3}) & 0                   & 0 \\
\end{block}
\end{blockarray}
\\
\end{align*}
Οι παραπάνω πίνακες είναι όλοι γινόμενα Kronecker (βλ. Παράδειγμα 6.6). Η διάστασή τους είναι $(3\dim(W))^2$. Τι παρατηρείται; Τι σας θυμίζουν;
\end{example}

\begin{definition}
    \label{def:induction}
    Το ζεύγος $(\sigma\uparrow_H^G, V)$, όπου $V$ είναι ο διανυσματικός χώρος \eqref{eq:induction_module} και η απεικόνιση $\sigma\uparrow_H^G : G \to \GL(V)$ ορίζεται θέτοντας το $(i,j)$-μπλοκ του $\sigma\uparrow_H^G(g)$ να είναι ίσο με  
    \[
    \begin{cases}
        \sigma(h), &\ \text{αν $gx_j = x_ih$} \\ 
        0, &\ \text{διαφορετικά} 
    \end{cases}
    \]
    ή ισοδύναμα 
    \[
    \begin{cases}
        \sigma(x_i^{-1}gx_j), &\ \text{αν $x_i^{-1}gx_j \in H$} \\ 
        0, &\ \text{διαφορετικά} 
    \end{cases}
    \]
    ονομάζεται \defn{επαγωγή} (induction) της $(\sigma, W)$ στην $G$. 
\end{definition}

\begin{proposition}
    \label{prop:induction}
    Η επαγωγή της $(\sigma, W)$ στην $G$ είναι αναπαράσταση της $G$ διάστασης $
    \frac{\abs{G}}{\abs{H}}\dim(W)$.
\end{proposition}
\begin{proof}[Απόδειξη]
    Αρχικά, η απεικόνιση $\sigma\uparrow_H^G$ είναι καλώς ορισμένη, δηλαδή 
    \[
    \sigma\uparrow_H^G(g) \in \GL(V)
    \]
    για κάθε $g \in G$. Αυτό έπεται από το ότι κάθε $\sigma\uparrow_H^G(g)$ είναι μπλοκ-πίνακας μετάθεσης (γιατί;). Πράγματι, από την διαμέριση 
    \[
    G = x_1H \biguplus x_2H \biguplus \cdots \biguplus x_kH
    \]
    η εξίσωση $gx_j = x_ih$ για σταθερό $j$ έχει μοναδική λύση για κάποιο $1 \le i \le k$ και $h \in H$. Συνεπώς, κάθε στήλη του $\sigma\uparrow_H^G(g)$ έχει μοναδικό μη μηδενικό μπλοκ. Ομοίως, για τις γραμμές και το ζητούμενο έπεται.

    Μένει λοιπόν, να δείξουμε ότι η απεικόνιση $\sigma\uparrow_H^G$ είναι ομομορφισμός ομάδων, ή ισοδύναμα ότι ικανοποιούνται οι ιδιότητες
    \begin{align*}
        \epsilon \cdot (x_j\otimes w) &= x_j \otimes w \\
        g \cdot \left(g' \cdot (x_j \otimes w)\right) &= (gg') \cdot (x_j \otimes w)\\
    \end{align*}
    για κάθε $g, g' \in G, 1 \le j \le k$ και $w \in W$. Η πρώτη ταυτότητα είναι προφανής (γιατί;). Για την δεύτερη, θεωρούμε (μοναδικά) $1 \le i, i' \le k$ και $h, h' \in H$ τέτοια ώστε 
    \begin{align}
        \label{eq:induction_help1}
        g'x_j &= x_{i'}h' \\
        \label{eq:induction_help2}
        gx_{i'} &= x_{i}h.
    \end{align}
    Τότε, έχουμε
    \[
    g \cdot \left(g' \cdot (x_j \otimes w)\right) 
    = g \cdot \left(x_{i'} \otimes (h'\cdot w)\right)
    = x_i \otimes \left((hh)'\cdot w\right).
    \]
    Από την άλλη, 
    \[
    gg'x_j = gx_{i'}h' = x_ihh'
    \]
    όπου η πρώτη (αντ. δεύτερη) ισότητα έπεται από την Ταυτότητα~\eqref{eq:induction_help1} (αντ. \eqref{eq:induction_help2}) και γι αυτό 
    \[
    (gg') \cdot (x_j \otimes w) = x_i \otimes \left((hh)'\cdot w\right)
    \]
    και το ζητούμενο έπεται.
\end{proof}

Αν $(\sigma, W)$ είναι η τετριμμένη αναπαράσταση της $H$, τότε η επαγωγή της στην $G$ δεν είναι παρά η αναπαράσταση συμπλόκου που συναντήσαμε στο Παράδειγμα 1.5 (γ). Πράγματι, στη θέση $(i,j)$ ο πίνακας ενός $g \in G$ στην αναπαράσταση συμπλόκου έχει 1 αν και μόνο αν 
\[
g \cdot x_jH = x_iH \Leftrightarrow x_i^{-1}gx_j \in H
\]
και 0 διαφορετικά, το οποίο είναι ακριβώς η συνθήκη του Ορισμού \ref{def:induction}. 

Στην περίπτωση της επαγωγής της τετριμμένης αναπαράστασης της $H$, σε επίπεδο χαρακτήρων, η περιγραφή είναι αρκετά απλή: παίρνουμε ένα αντίγραφο του $\chi^\triv$ για κάθε υποομάδα συζυγή με την $H$. Στο Παράδειγμα \ref{ex:induction_S3}, έχουμε 
\[
\renewcommand{\arraystretch}{1.2} 
\begin{array}{c|c|c|c}
                             & \rmK_{111} & \rmK_{21} & \rmK_{111}\\ \hline
\chi^\triv\uparrow_H^{\fS_3} & 3          & 1         & 1        
\end{array}
\]
η οποία δεν είναι άλλη από την αναπαράσταση καθρισμού της $\fS_3$. Η εικόνα που έχουμε κατά νου είναι 
\begin{figure}[H]
    \centering
    \includegraphics[scale=0.15]{rourou}
\end{figure}
\vspace{-50pt}
Γενικότερα, έχουμε το εξής αποτέλεσμα.
\begin{proposition}
    \label{prop:induction_character}
    Ο χαρακτήρας της επαγωγής της $(\sigma,W)$ στην $G$ δίνεται από 
    \begin{equation}
        \label{eq:induction_character}
        \chi^{\sigma,W}\uparrow_H^G(g) = 
        \frac{1}{\abs{H}} \sum_{\substack{x \in G \\ x^{-1}gx \, \in \, H}} \chi^{\sigma, W}(x^{-1}gx),
    \end{equation}
    για κάθε $g \in G$. Ειδικότερα, η επαγωγή ενός $H$-προτύπου στην $G$ δεν εξαρτάται από την επιλογή αντιπροσώπων των αριστερών κλάσεων της $H$ στην $G$.
\end{proposition}
\begin{proof}[Απόδειξη]
    Από τον Ορισμό~\ref{def:induction}, έπεται ότι 
    \[
    \chi^{\sigma,W}\uparrow_H^G(g) = \sum_{\substack{1 \le i \le k \\ x_i^{-1}gx_i \, \in \, H}} \chi^{\sigma, W}(x_i^{-1}gx_i).
    \]
    Επειδή ο $\chi^{\sigma, W}$ είναι συνάρτηση κλάσης της $H$ έχουμε 
    \[
    \chi^{\sigma, W}(x_i^{-1}gx_i) = \chi^{\sigma, W}\left(h^{-1}(x_i^{-1}gx_i)h\right),
    \]
    για κάθε $h \in H$. Όμως, 
    \[
    h^{-1}(x_i^{-1}gx_i)h \, \in \, H \ \Leftrightarrow \ 
    x_i^{-1}gx_i \, \in \, H
    \]
    για κάθε $h \in H$ και γι αυτό 
    \begin{align*}
        \chi^{\sigma,W}\uparrow_H^G(g)
        &= \sum_{\substack{1 \le i \le k \\ x_i^{-1}gx_i \, \in \, H}} \frac{1}{\abs{H}} \sum_{h \in H} \chi^{\sigma, W}\left(h^{-1}(x_i^{-1}gx_i)h\right) \\
        &= \frac{1}{\abs{H}} \sum_{\substack{1 \le i \le k \\ x_i^{-1}gx_i \, \in \, H}} \sum_{h \in H} \chi^{\sigma, W}\left((x_ih)^{-1}g(x_ih)\right) \\
        &= \frac{1}{\abs{H}} \sum_{\substack{x \in G \\ x^{-1}gx \, \in \, H}} \chi^{\sigma, W}(x^{-1}gx),
    \end{align*}
    όπου στο τελευταίο άθροισμα το $x = x_ih$ διατρέχει τα στοιχεία του $G$ με τον ίδιο τρόπο όπου στα προηγούμενα δυο αθροίσματα τα $i$ και $h$ διατρέχουν το $[k]$ και $H$, αντίστοιχα.
\end{proof}

Εφαρμόζοντας την Πρόταση \ref{prop:induction_character} στον χαρακτήρα της τετριμμένης αναπαράστασης προκύ\-πτει το εξής.

\begin{corollary}
    \label{cor:induction_trivial}
    Αν $\chi^\triv$ είναι ο χαρακτήρας της τετριμμένης αναπαράστασης της $H$, τότε 
    \[
    \chi^\triv\uparrow_H^G(g) = \frac{1}{\abs{H}}\abs{\{x \in G : x^{-1}gx \in H\}}
    \]
    για κάθε $g \in G$.
\end{corollary}

\begin{example}
    \label{ex:induction_of_trivial_defining}
    Έστω $(\fS_n)_n$ ο σταθεροποιητής του $n$ στην δράση καθορισμού της $\fS_n$ στο $[n]$. Όπως είδαμε στην Παράγραφο 1, $(\fS_n)_n \cong \fS_{n-1}$. Από το Πόρισμα \ref{cor:induction_trivial}, 
    \begin{align*}
        \chi^\triv\uparrow_{\fS_{n-1}}^{\fS_n}(\pi) 
        &= \frac{1}{(n-1)!} \abs{\{\sigma \in \fS_n : \sigma^{-1}(\pi(\sigma(n))) = n\}} \\
        &= \frac{1}{(n-1)!} \abs{\{\sigma \in \fS_n : \pi(\sigma(n)) = \sigma(n)\}} \\
        &= \abs{\{i \in [n] : \pi(i) = i\}} \\
        &= \fix(\pi) \\
        &= \chi^\rmdef(\pi),
    \end{align*}
    για κάθε $\pi \in \fS_n$. Άρα, η επαγωγή της τετριμμένης αναπαράστασης από την $\fS_{n-1}$ στην $\fS_n$ είναι ισόμορφη με την αναπαράσταση καθορισμού της $\fS_n$. Παρακάτω, θα υπολογίσουμε ταυτότητες για τους χαρακτήρες της επαγωγής της τετριμμένης αναπαράστασης από γενικότερες υποομάδες της $\fS_n$.
\end{example}

Ολοκληρώνουμε αυτή την παράγραφο, και κατά συνέπεια, τη συζήτηση για την θεωρία χαρακτήρων μιας πεπερασμένης ομάδας με το παρακάτω αποτέλεσμα, το οποίο θα μας φανεί πολύ χρήσιμο στο τρίτο μέρος του μαθήματος.
\begin{theorem}{\rm(Νόμος αντιστροφής Frobenius, 1898)}
    \label{thm:frobenius_reciprocity_law}
    Αν $\chi$ και $\psi$ είναι χαρακτήρες της $G$ και $H$, αντίστοιχα, τότε 
    \begin{equation}
        \label{eq:frobenius_reciprocity_law}
        \left(\psi\uparrow_H^G, \chi\right)_G = \left(\psi, \chi\downarrow_H^G\right)_H.
    \end{equation}
\end{theorem}
\begin{proof}[Απόδειξη]
    Έχουμε 
    \begin{align*}
        \left(\psi\uparrow_H^G, \chi\right)_G 
        &= \frac{1}{\abs{G}} \sum_{g \in G} \psi\uparrow_H^G(g)\chi(g^{-1}) \\ 
        &= \frac{1}{\abs{G}} \sum_{g \in G} \frac{1}{\abs{H}} \sum_{\substack{x \in G \\ x^{-1}gx \, \in \, H}} \psi(x^{-1}gx)\chi(g^{-1}) \\ 
        &= \frac{1}{\abs{G}\abs{H}} \sum_{x \in G} \sum_{\substack{g \in G \\ x^{-1}gx \, \in \, H}} \psi(x^{-1}gx)\chi(g^{-1}) \\
        &= \frac{1}{\abs{G}\abs{H}} \sum_{x \in G} \sum_{y \in H} \psi(y)\chi(xy^{-1}x^{-1}) \\
        &= \frac{1}{\abs{G}\abs{H}} \sum_{x \in G} \sum_{y \in H} \psi(y)\chi(y^{-1}) \\
        &= \frac{1}{\abs{H}}\sum_{y \in H} \psi(y)\chi(y^{-1}) \\
        &= \frac{1}{\abs{H}}\sum_{y \in H} \psi(y)\chi\downarrow_H^G(y^{-1}) \\
        &= \left(\psi, \chi\downarrow_H^G\right)_H,
    \end{align*}
    όπου η πρώτη ισότητα έπεται από την Άσκηση 2.2 (2), η δεύτερη ισότητα από την Πρόταση~\ref{prop:induction_character}, η τέταρτη ισότητα από την αλλαγή μεταβλητών $y \to x^{-1}gx$ και η πέμπτη ισότητα από το ότι ο $\chi$ είναι συνάρτηση κλάσης.
\end{proof}

Ο νόμος αντιστροφής Frobenius μας πληροφορεί ότι οι κατασκευές του περιορισμού και της επαγωγής συνδέονται από μια σχέση \textquote{δυϊκότητας}, η οποία μας θυμίζει τη σχέση που ικανοποιούν οι προσαρτημένοι πίνακες σε έναν διανυσματικό χώρο με εσωτερικό γινόμενο. Από αυτή την οπτική, η κατασκευή της επαγωγής μοιάζει \textquote{φυσική}.
\end{document}