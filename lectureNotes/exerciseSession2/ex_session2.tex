\documentclass[12pt,a4paper,reqno]{amsart}

% section handling
\usepackage{subfiles} 

% language
\usepackage[greek,english]{babel}
\usepackage[utf8]{inputenc}
\usepackage{alphabeta}

% change default names to greek
\addto\captionsenglish{
    \renewcommand{\contentsname}{Περιεχόμενα}
    \renewcommand{\refname}{Βιβλιογραφία}
    \renewcommand{\datename}{Ημερομηνία:}
    \renewcommand{\urladdrname}{Ιστοσελίδα}
}

% math 
\usepackage{amsmath,amsthm,amssymb,amscd}

% font
\usepackage[cal=euler]{mathalfa}
\usepackage{libertinus-type1}
% \usepackage{txfonts} % for upright greek letters
\usepackage{bm} % for bold symbols
\usepackage{bbm} % for the simply-looking bb symbols

% miscellaneous 
\usepackage{changepage} %for indenting environments
\usepackage{csquotes} % example: \textcquote{}
\usepackage{stmaryrd} % needed for \mapsfrom

% drawing
\usepackage{tikz,tikz-cd}
\usetikzlibrary{shapes.misc, patterns, matrix, calc, intersections,positioning}
\usepackage{graphics,graphicx}
\usepackage{float} % provides enhanced control and customization options for floating objects such as figures and tables

% colors
\usepackage{xcolor}
\definecolor{darkcandyapplered}{rgb}{0.64, 0.0, 0.0}
\definecolor{midnightblue}{rgb}{0.1, 0.1, 0.44}
\definecolor{mylightblue}{HTML}{336699}
\definecolor{burntorange}{rgb}{0.8, 0.33, 0.0}
\definecolor{iceberg}{rgb}{0.44, 0.65, 0.82}
\definecolor{applegreen}{rgb}{0.55, 0.71, 0.0}
\definecolor{canaryyellow}{rgb}{1.0, 0.94, 0.0}

% hrefs
\usepackage{hyperref}
\usepackage[noabbrev,capitalize]{cleveref}
\hypersetup{
    pdftoolbar=true,        
    pdfmenubar=true,        
    pdffitwindow=false,     
    pdfstartview={FitH},  % fits the width of the page to the window
    pdftitle={},
    pdfauthor={},
    pdfsubject={},
    pdfkeywords={},
    pdfnewwindow=true,  % links in new window
    colorlinks=true,  % false: boxed links; true: colored links
    linkcolor=darkcandyapplered,   % color of internal links
    citecolor=midnightblue,  % color of links to bibliography
    urlcolor=cyan,  % color of external links
    linktocpage=true  % changes the links from the section body to the page number
    }

% geometry
\textwidth=16cm 
\textheight=21cm 
\hoffset=-55pt 
\footskip=25pt

% thm envs
\theoremstyle{definition}
\newtheorem*{remarks}{Παρατηρήσεις}
\newtheorem*{example}{Παράδειγμα}
\newtheorem*{problem}{Πρόβλημα}
\newtheorem*{combInterlude}{Ιντερλούδιο Συνδυαστικής}

% math ops (you might need to change the path)
% In this macro I define all of my math operators

% fields
\newcommand{\NN}{\mathbbmss{N}} 
\newcommand{\ZZ}{\mathbbmss{Z}} 
\newcommand{\QQ}{\mathbbmss{Q}} 
\newcommand{\RR}{\mathbbmss{R}} 
\newcommand{\CC}{\mathbbmss{C}} 
\newcommand{\KK}{\mathbbmss{K}} 
\newcommand{\FF}{\mathbbmss{F}} 

% symmetric group
\newcommand{\fS}{\mathfrak{S}}  

% calligraphic 
\newcommand{\aA}{\mathcal{A}} 
\newcommand{\bB}{\mathcal{B}}
\newcommand{\cC}{\mathcal{C}}
\newcommand{\dD}{\mathcal{D}}
\newcommand{\eE}{\mathcal{E}}
\newcommand{\fF}{\mathcal{F}}
\newcommand{\hH}{\mathcal{H}}
\newcommand{\iI}{\mathcal{I}}
\newcommand{\lL}{\mathcal{L}}
\newcommand{\oO}{\mathcal{O}}
\newcommand{\pP}{\mathcal{P}}
\newcommand{\sS}{\mathcal{S}}
\newcommand{\mM}{\mathcal{M}}
\newcommand{\uU}{\mathcal{U}}

% bold
\newcommand{\bfa}{\mathbf{a}}
\newcommand{\bfe}{\mathbf{e}}
\newcommand{\bfF}{\pmb{F}}
\newcommand{\bfR}{\pmb{R}}
\newcommand{\bfv}{\mathbf{v}}
%\newcommand{\bfx}{\bm{x}}
%\newcommand{\bfx}{\mathbf{x}} 
\newcommand{\bfx}{\pmb{x}}
\newcommand{\bfX}{\pmb{X}}
\newcommand{\bfy}{\pmb{y}}
\newcommand{\bfz}{\pmb{z}}

% roman
\newcommand{\rmB}{\mathrm{B}}
\newcommand{\rmC}{\mathrm{C}}
\newcommand{\rmD}{\mathrm{D}} 
\newcommand{\rmI}{\mathrm{I}} 
\newcommand{\rmK}{\mathrm{K}}
\newcommand{\rmM}{\mathrm{M}}
\newcommand{\rmP}{\mathrm{P}}  
\newcommand{\rmQ}{\mathrm{Q}}  
\newcommand{\rmR}{\mathrm{R}}
\newcommand{\rmS}{\mathrm{S}}
\newcommand{\rmT}{\mathrm{T}}
\newcommand{\rmU}{\mathrm{U}}
\newcommand{\rmV}{\mathrm{V}}
\newcommand{\rmY}{\mathrm{Y}}
\newcommand{\rmZ}{\mathrm{Z}}

% greek letters
% I'm renewing some commands in order to appear in upright font
% If I want to change it later, I don't have to do it manually, I just change it from here.
% \newcommand{\uaa}{\alphaup}
% \renewcommand{\alpha}{\alphaup}
% \renewcommand{\beta}{\betaup}
% \renewcommand{\gamma}{\gammaup}
% \renewcommand{\delta}{\deltaup}
% \renewcommand{\epsilon}{\epsilonup}
% \newcommand{\ee}{\epsilon}
% \renewcommand{\varepsilon}{\varepsilonup}
% \renewcommand{\theta}{\thetaup}
% \renewcommand{\lambda}{\lambdaup}
% \newcommand{\ull}{\lambda}
% \renewcommand{\mu}{\muup}
% \renewcommand{\nu}{\nuup}
% \renewcommand{\pi}{\piup}
% \renewcommand{\rho}{\rhoup}
% \renewcommand{\varrho}{\varrhoup}
% \renewcommand{\sigma}{\sigmaup}
% \renewcommand{\tau}{\tauup} 
% \renewcommand{\phi}{\phiup}
% \renewcommand{\chi}{\chiup}
% \renewcommand{\psi}{\psiup}
% \renewcommand{\omega}{\omegaup}

% arrows and symbols 
\renewcommand{\to}{\rightarrow}
\newcommand{\toto}{\longrightarrow}
\newcommand{\mapstoto}{\longmapsto}
\newcommand{\then}{\Rightarrow}
\newcommand{\IFF}{\Leftrightarrow}
\newcommand{\tl}{\tilde}
\newcommand{\wtl}{\widetilde}
\newcommand{\ol}{\overline}
\newcommand{\ul}{\underline}
\newcommand{\oldemptyset}{\emptyset}
\renewcommand{\emptyset}{\varnothing}
\DeclareMathSymbol{\Arg}{\mathbin}{AMSa}{"39} % for arguments 
\newcommand{\onto}{\ensuremath{\twoheadrightarrow}}

% absolute value symbol
\usepackage{mathtools} 
\DeclarePairedDelimiter\abs{\lvert}{\rvert}%
\DeclarePairedDelimiter\norm{\lVert}{\rVert}%
\makeatletter
\let\oldabs\abs
\def\abs{\@ifstar{\oldabs}{\oldabs*}}

% tensor symbol
\newcommand{\tensor}[1]{%
  \mathbin{\mathop{\otimes}\limits_{#1}}%
}

% permutation cycle notation
\ExplSyntaxOn
\NewDocumentCommand{\cycle}{ O{\;} m }
 {
  (
  \alec_cycle:nn { #1 } { #2 }
  )
 }

\seq_new:N \l_alec_cycle_seq
\cs_new_protected:Npn \alec_cycle:nn #1 #2
 {
  \seq_set_split:Nnn \l_alec_cycle_seq { , } { #2 }
  \seq_use:Nn \l_alec_cycle_seq { #1 }
 }
\ExplSyntaxOff

% setminus symbol
\newcommand{\mysetminusD}{\hbox{\tikz{\draw[line width=0.6pt,line cap=round] (3pt,0) -- (0,6pt);}}}
\newcommand{\mysetminusT}{\mysetminusD}
\newcommand{\mysetminusS}{\hbox{\tikz{\draw[line width=0.45pt,line cap=round] (2pt,0) -- (0,4pt);}}}
\newcommand{\mysetminusSS}{\hbox{\tikz{\draw[line width=0.4pt,line cap=round] (1.5pt,0) -- (0,3pt);}}}
\newcommand{\sm}{\mathbin{\mathchoice{\mysetminusD}{\mysetminusT}{\mysetminusS}{\mysetminusSS}}}

% custom math operators
\newcommand{\Des}{\mathrm{Des}} 
\newcommand{\des}{\mathrm{des}} 
\newcommand{\Asc}{\mathrm{Asc}}
\newcommand{\asc}{\mathrm{asc}} 
\newcommand{\inv}{\mathrm{inv}}
\newcommand{\Inv}{\mathrm{Inv}}
\newcommand{\maj}{\mathrm{maj}} 
\newcommand{\comaj}{\mathrm{comaj}} 
\newcommand{\fix}{\mathrm{fix}} 
\newcommand{\Sym}{\mathrm{Sym}} 
\newcommand{\QSym}{\mathrm{QSym}}
\newcommand{\FQSym}{\mathrm{FQSym}} 
\newcommand{\End}{\mathrm{End}} 
\newcommand{\Rad}{\mathrm{Rad}} 
\newcommand{\rmMat}{\mathrm{Mat}} 
\newcommand{\rmdim}{\mathrm{dim}} 
\newcommand{\rmTop}{\mathrm{Top}} 
\newcommand{\rmCF}{\mathrm{CF}} 
\newcommand{\rmId}{\mathrm{Id}}
\newcommand{\rmid}{\mathrm{id}}
\newcommand{\rmtw}{\mathrm{tw}}
\newcommand{\trace}{\mathrm{tr}}
\newcommand{\Irr}{\mathrm{Irr}}
\newcommand{\Ind}{\mathrm{Ind}} % induction
\newcommand{\Res}{\mathrm{Res}} % restriction
\newcommand{\triv}{\mathrm{triv}} % trivial rep
\newcommand{\rmdef}{\mathrm{def}} % defining rep
\newcommand{\dom}{\triangleleft}
\newcommand{\domeq}{\trianglelefteq}
\newcommand{\lex}{\mathrm{lex}}
\newcommand{\sign}{\mathrm{sign}}
\newcommand{\SYT}{\mathrm{SYT}}
\renewcommand{\Im}{\mathrm{Im}}
\newcommand{\Ker}{\mathrm{Ker}}
\newcommand{\GL}{\mathrm{GL}}
\newcommand{\FL}{\mathrm{FL}}
\newcommand{\Span}{\mathrm{span}}
\newcommand{\pos}{\mathrm{pos}}
\newcommand{\Comp}{\mathrm{Comp}}
\newcommand{\Set}{\mathrm{Set}}
\newcommand{\std}{\mathrm{std}}
\newcommand{\cont}{\mathrm{cont}} %content of a SSYT
\newcommand{\SSYT}{\mathrm{SSYT}}
\newcommand{\rmz}{\mathrm{z}}
\newcommand{\ct}{\mathrm{ct}} % cycle type
\newcommand{\ch}{\mathrm{ch}} % Frobenius characteristic map
\newcommand{\height}{\mathrm{ht}}
\newcommand{\FPS}{\CC[\![\bfx]\!]} % formal power series
\newcommand{\FPSS}{\CC[\![\bfx,\bfy]\!]}
\newcommand{\reg}{\mathrm{reg}}
\newcommand{\hook}{\mathrm{h}}
\newcommand{\weight}{\mathrm{wt}}
\newcommand{\co}{\mathrm{co}}
\newcommand{\ps}{\mathrm{ps}}
\newcommand{\rmsum}{\mathrm{sum}}
\newcommand{\NSym}{\mathrm{NSym}}
\newcommand{\Hom}{\mathrm{Hom}}
\newcommand{\proj}{\mathrm{proj}}
\newcommand{\stat}{\mathrm{stat}}

% miscellaneous commands
\newcommand{\defn}[1]{{\color{mylightblue}{#1}}}
\newcommand{\toDo}{{\bf\color{red} TODO}}
\newcommand{\toCite}{{\bf\color{green} CITE}}

% 
\newenvironment{nouppercase}{%
  \let\uppercase\relax%
  \renewcommand{\uppercasenonmath}[1]{}}{}

\newcommand{\one}{\textcolor{iceberg}{1}}
\newcommand{\two}{\textcolor{burntorange}{2}}
\newcommand{\three}{\textcolor{applegreen}{3}}
\newcommand{\four}{\textcolor{canaryyellow}{4}}

% titlepage
\title{Θ2.04: Θεωρία Αναπαραστάσεων και Συνδυαστική}
\author[Β.~Δ. Μουστακας]{Βασίλης Διονύσης Μουστάκας \\ Πανεπιστήμιο Κρήτης}
\date{15 Οκτωβρίου 2025}
% \urladdr{\href{https://sites.google.com/view/vasmous}{https://sites.google.com/view/vasmous}}

\begin{document}

\begingroup
\def\uppercasenonmath#1{} % this disables uppercase title
\let\MakeUppercase\relax % this disables uppercase authors
\maketitle
\endgroup

% \setcounter{section}{}
\thispagestyle{empty}

\begin{center}
    \textbf{Η δράση συζυγίας}
\end{center}

Έστω $G$ ομάδα. Όπως είδαμε στο Παράδειγμα~1.3~(2), η $G$ δρα στον εαυτό της με \emph{συζυγία}: 
\[
g\cdot{x} = gxg^{-1}.
\]
Οι τροχιές αυτής της δράσης 
\[
\oO_x \coloneqq \{g\cdot{x} : g \in G\} = \{gxg^{-1} : g \in G\}
\]
ονομάζονται \emph{κλάσεις συζυγίας}. Δυο στοιχεία που ανήκουν στην ίδια τροχιά της δράσης συζυγίας ονομάζονται \emph{συζυγή}. Πιο συγκεκριμένα, αν $y \in \oO_x$, τότε 
\[
y = gxg^{-1}
\]
για κάποιο $g \in G$. 

\begin{remarks}
\leavevmode
\begin{itemize}
    \item[(1)] Αν $G = \GL_n(\FF)$ είναι η γενική γραμμική ομάδα, δηλαδή η ομάδα των αντιστρέψιμων πινάκων με στοιχεία στο $\FF$, τότε η συζυγία μεταξύ πινάκων δεν είναι τίποτα άλλο παρά η έννοια της \emph{ομοιότητας}.
    \item[(2)] Αν η $G$ είναι αβελιανή, τότε
    \[
    gxg^{-1} = x 
    \]
    για κάθε $g,x \in G$. Στην περίπτωση αυτή η δράση συζυγίας είναι η \emph{τετριμμένη} δράση. Επιπλέον, κάθε τροχιά της $G$ αποτελείται από ένα και μόνο στοιχείο, δηλαδή 
    \[
    \oO_x = \{gxg^{-1} : g\in G\} = \{x\}.
    \]
    \item[(3)] Η δράση συζυγίας δεν είναι \emph{μεταβατική}. Πράγματι,
    \[
    \oO_\epsilon = \{g\epsilon g^{-1} : g \in G\} = \{\epsilon\}
    \]
    και γι' αυτό αν $\abs{G} >1$, τότε η $G$ έχει τουλάχιστον δυο τροχιές.
    \item[(4)] Σε μια όχι απαραίτητα αβελιανή ομάδα, αν $\oO_x = \{x\}$, τότε $gxg^{-1} = x$, για κάθε $g \in G$. Το σύνολο των στοιχείων που έχουν αυτή την ιδιότητα, δηλαδή που μετατίθενται με όλα τα στοιχεία της ομάδας, ονομάζεται \emph{κέντρο} της $G$. 
\end{itemize}
\end{remarks}

\begin{example}{\rm(Κλάσεις συζυγίας της $\fS_3$)} 
    Στην $\fS_3$,
    \renewcommand{\arraystretch}{1.2}
    \[
    \begin{array}{c|c|c|c|c|c|c}
        \pi & \cycle{1}\cycle{2}\cycle{3} & \cycle{1,2}\cycle{3} & \cycle{1,3}\cycle{2} & \cycle{2,3}\cycle{1} & \cycle{1,2,3} & \cycle{1,3,2} \\ 
        \hline
        \pi^{-1} & \cycle{1}\cycle{2}\cycle{3} & \cycle{1,2}\cycle{3} & \cycle{1,3}\cycle{2} & \cycle{2,3}\cycle{1} & \cycle{1,3,2} & \cycle{1,2,3}
    \end{array}.
    \]
    Οπότε, υπολογίζουμε 
    \[
    \begin{array}{c|c|c|c|c|c|c}
        \pi & \cycle{1}\cycle{2}\cycle{3} & \cycle{1,2}\cycle{3} & \cycle{1,3}\cycle{2} & \cycle{2,3}\cycle{1} & \cycle{1,2,3} & \cycle{1,3,2} \\ 
        \hline
        \pi\cycle{1,2,3}\pi^{-1} & \cycle{1,2,3} & \cycle{1,3,2} & \cycle{1,3,2} & \cycle{1,3,2} & \cycle{1,2,3} & \cycle{1,3,2}
    \end{array}.
    \]
    Συνεπώς,
    \[
    \oO_{\cycle{1,2,3}} = \{\cycle{1,2,3},\cycle{1,3,2}\}.
    \]
    Ομοίως, υπολογίζουμε 
    \[
    \oO_{\cycle{1,2}\cycle{3}} = \{\cycle{1,2}\cycle{3}, \cycle{1,3}\cycle{2}, \cycle{2,3}\cycle{1}\}.
    \]

    Συνεπώς η $\fS_3$ έχει τρεις κλάσεις συζυγίας με 1, 2 και 3 στοιχεία αντίστοιχα και το κέντρο της είναι τετριμμένο, δηλαδή $\rmZ(\fS_3) = \{\epsilon\}$. Αν δούμε την $\fS_3$ ως ομάδα συμμετρίας του ισόπλευρου τριγώνου, τότε οι μη-τετριμμένες κλάσεις συζυγίας απαρτίζονται από τις στροφές και τις ανακλάσεις. Πως το ερμηνεύετε αυτό γεωμετρικά;
\end{example}

\begin{example}{\rm(Κλάσεις συζυγίας της $\rmD_8$)}
    Θυμίζουμε μια παράσταση της διεδρικής ομάδας 
    \[
    \rmD_8 = \langle r, s \, \vert \, r^4 = s^2 = \epsilon, rsr = s \rangle.
    \]
    Μια χρήσιμη ιδιότητα, η οποία υποδεικνύει πως μετατίθεται το $s$ με δυνάμεις του $r$, είναι 
    \[
    r^is = sr^{-i},
    \]
    για κάθε $1 \le i \le n-1$. Χρησιμοποιώντας αυτήν την ταυτότητα, υπολογίζουμε 
    \[
    \renewcommand{\arraystretch}{1.2}
    \begin{array}{c|c|c|c|c|c|c|c|c}
        g      & \epsilon & r   & r^2 & r^3 & s & sr & sr^2 & sr^3 \\ 
        \hline
        g^{-1} & \epsilon & r^3 & r^2 & r   & s & sr & sr^2 & sr^3
    \end{array}.
    \]
    Οπότε,
    \[
    \renewcommand{\arraystretch}{1.2}
    \begin{array}{c|c|c|c|c|c|c|c|c}
        g        & \epsilon & r & r^2 & r^3 & s   & sr  & sr^2 & sr^3 \\ 
        \hline
        grg^{-1} & r        & r & r   & r   & r^3 & r^3 & r^3  & r^3
    \end{array}
    \]
    και γι' αυτό 
    \[
    \oO_r = \{r, r^3\}.
    \]
    Ομοίως, υπολογίζουμε και τις υπόλοιπες κλάσεις συζυγίας 
    \[
    \oO_s = \{s, sr^2\}, \ \oO_{sr} = \{sr, sr^3\}.
    \]

    Συνεπώς, η $\rmD_8$ έχει πέντε κλάσεις συζυγίας με 1, 1, 2, 2, και 2 στοιχεία αντίστοιχα και το κέντρο της είναι 
    \[
    \rmZ(\rmD_8) = \{\epsilon, r^2\}.
    \]
    Πως το ερμηνεύετε αυτό γεωμετρικά κοιτάζοντας τις συμμετρίες του τετραγώνου;
\end{example}

Υποθέτουμε ότι η $G$ είναι πεπερασμένη. Πόσα στοιχεία έχει μια κλάση συζυγίας της $G$; Για να απαντήσουμε σε αυτό το ερώτημα θα θυμηθούμε την Ταυτότητα~(1.1) (\emph{ταυτότητα απαρίθμησης τροχιών}). Αν η $G$ δρα σε ένα σύνολο $S$, τότε
\begin{equation}
    \label{eq:orbit_counting_formula}
    \abs{\oO_s} = \frac{\abs{G}}{\abs{G_s}},
\end{equation}
όπου $G_s \coloneq \{g\cdot{s} : g \in G\}$ είναι ο σταθεροποητής του $s$ στην $G$. Ο σταθεροποιητής είναι υποομάδα της $G$. Πράγματι,
\[
(gx)\cdot{s} = g\cdot(x\cdot{s}) = g\cdot{s} = s,
\]
για κάθε $g, x \in G_s$ και γι' αυτό $gx \in G_s$. 

\begin{proof}[Απόδειξη Ταυτότητας~(1.1)]
    Αρχικά παρατηρούμε ότι\footnote{Αυτό είναι γνωστό ως Θεώρημα του Lagrange.} το δεξί μέλος απαριθμεί το πλήθος των αριστερών κλάσεων του σταθεροποιητή του $s$ στην $G$. Πράγματι, έστω $k$ το πλήθος των αριστερών κλάσεων του $G_s$ στην $G$. H απεικόνιση 
    \begin{align*}
        G_s &\to gG_s \\
        x &\mapsto gx
    \end{align*}
    είναι αμφιμονοσήμαντη (γιατί;) και γι' αυτό κάθε αριστερό σύμπλοκο του $G_s$ στην $G$ έχει $\abs{G_s}$ στοιχεία. Αφού το σύνολο των αριστερών κλάσεων του $G_s$ διαμερίζει την $G$, έπεται ότι 
    \[
    \abs{G} =\abs{G_s} k \ \Leftrightarrow k = \frac{\abs{G}}{\abs{G_s}}.
    \]

    Συνεπώς, για να αποδείξουμε την Ταυτότητα~(1.1), αρκεί να βρούμε μια αμφιμονοσήμαντη απεικόνιση $\phi$ μεταξύ της τροχιάς του $s$ και του συνόλου των αριστερών κλάσεων του $G_s$ στην $G$. Αν $x \in \oO_s$, τότε $x = g\cdot{s}$ για κάποιο $g \in G$ και θέτουμε 
    \[
    \phi(x) = gG_s.
    \]
    Η απεικόνιση αυτή είναι καλά ορισμένη (γιατί;). Για το ένα-προς-ένα, έστω $x, y \in G_s$ με $x = g\cdot{s}$ και $y = g'\cdot{s}$, για κάποια $g, g' \in G$ και υποθέτουμε ότι $\phi(x) = \phi(y)$. Τότε
    \[
    gG_s = g'G_s \ \Leftrightarrow \ (g')^{-1}g \in G_s \ \Leftrightarrow \ (g')^{-1}g = h \ \Leftrightarrow \ g = g'h,
    \]
    για κάποιο $h \in G_s$. Άρα,
    \[
    x = g\cdot{s} = (g'h)\cdot{s} = g'\cdot{h\cdot{s}} = g'\cdot{s} = y, 
    \]
    όπου η δεύτερη ισότητα έπεται από τις ιδιότητες της δράσης και η τέταρτη ισότητα έπεται από το ότι $h \in G_s$. Ομοίως αποδεικνύεται και το επί.
\end{proof}

Πίσω στη δράση συζυγίας, ο σταθεροποιητής ενός $x \in G$ 
\[
\oO_x = \{g \in G : gxg^{-1} = x\} 
\]
ονομάζεται \emph{κεντρικοποιητής} του $x$ στην $G$ και συχνά συμβολίζεται με $\rmZ_x$. Επομένως, αν $x_1, x_2, \dots, x_r$ είναι αντιπρόσωποι των κλάσεων συζυγίας της $G$ που \emph{δεν} ανήκουν στο κέντρο, τότε 
\[
G = \rmZ(G) \uplus \oO_{x_1} \uplus \oO_{x_2} \uplus \cdots \uplus \oO_{x_r}
\]
και παίρνοντας πληθαρίθμους και στα δυο μέλη προκύπτει η \emph{ταυτότητα κλάσης}
\begin{equation}
    \label{eq:class_equation}
    \abs{G} = \abs{\rmZ(G)} + \sum_{i=1}^r \frac{\abs{G}}{\abs{\rmZ_{x_i}}}.
\end{equation}

Για την $\fS_3$ και την $\rmD_8$, η ταυτότητα κλάσης μας πληροφορεί ότι $6 = 1 + 3 + 2$ και $8 = 2 + 2 + 2 + 2$, αντίστοιχα, σε συμφωνία με τους υπολογισμούς που κάναμε στα παραδείγματα.

\begin{problem}
Ποια είναι η εξίσωση κλάσης για την διεδρική ομάδα $\rmD_{2n}$; Ποιές είναι οι κλάσεις συζυγίας της; Ποιό είναι το κέντρο της;
\end{problem}

Παραμένοντας στο παράδειγμα της $\rmD_8$, είδαμε ότι έχει 5 κλάσεις συζυγίας, καθώς και 5 μη ισόμορφους ανάγωγους χαρακτήρες. Από αυτούς, οι τέσσερις είναι διάστασης 1 και ορίζονται θέτοντας
\[
\begin{aligned}
\chi_{11}(r)           &=  1, &\quad \chi_{11}(s)           &=  1, \\
\chi_{\overline{1}1}(r)  &= -1, &\quad \chi_{\overline{1}1}(s)  &=  1, \\
\chi_{1\overline{1}}(r)  &=  1, &\quad \chi_{1\overline{1}}(s)  &= -1, \\
\chi_{\overline{1}\overline{1}}(r) &= -1, &\quad \chi_{\overline{1}\overline{1}}(s) &= -1.
\end{aligned}
\]
Στην Άσκηση~1.1, υπολογίσαμε τον ανάγωγο χαρακτήρα, έστω $\chi$, διάστασης δυο. Ποιοί ήταν οι πίκανες της αντίστοιχης αναπαράστασης; Συνδυάζοντας όλα τα παραπάνω, ο πίνακας χαρακτήρων της $\rmD_8$ είναι 
\[
\begin{array}{c|c|c|c|c|c}
                        & \{\epsilon\} & \{r^2\}  & \{r, r^3\} & \{s, sr^2\} & \{sr, sr^3\} \\ \hline
    \chi_{11}           & 1            & 1        & 1          & 1           & 1            \\ \hline
    \chi_{\ol{1}1}      & 1            & 1        & -1         & 1           & -1           \\ \hline    
    \chi_{1\ol{1}}      & 1            & 1        & 1          & -1          & -1           \\ \hline    
    \chi_{\ol{1}\ol{1}} & 1            & 1        & -1         & -1          & 1            \\ \hline   
    \chi                & 2            & -2        & 0         & 0          & 0            \\    
\end{array}.
\]
\end{document}