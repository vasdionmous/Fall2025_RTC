\documentclass[12pt,a4paper,reqno]{amsart}

% section handling
\usepackage{subfiles} 

% language
\usepackage[english,greek]{babel}
\usepackage[utf8]{inputenc}
\usepackage{alphabeta}

% math 
\usepackage{amsmath,amsthm,amssymb,amscd}

% font
\usepackage[cal=euler,frak=mt]{mathalfa}
\usepackage{libertinus-type1}
\usepackage{txfonts} % for upright greek letters
\usepackage{bm} % for bold symbols
\usepackage{bbm} % for the simply-looking bb symbols

% miscellaneous 
\usepackage{changepage} %for indenting environments
\usepackage{csquotes} % example: \textcquote{}

% colors
\usepackage{xcolor}
\definecolor{darkcandyapplered}{rgb}{0.64, 0.0, 0.0}
\definecolor{midnightblue}{rgb}{0.1, 0.1, 0.44}
\definecolor{mylightblue}{HTML}{336699}

% hrefs
\usepackage{hyperref}
\usepackage[noabbrev,capitalize]{cleveref}
\hypersetup{
    pdftoolbar=true,        
    pdfmenubar=true,        
    pdffitwindow=false,     
    pdfstartview={FitH},  % fits the width of the page to the window
    pdftitle={},
    pdfauthor={},
    pdfsubject={},
    pdfkeywords={},
    pdfnewwindow=true,  % links in new window
    colorlinks=true,  % false: boxed links; true: colored links
    linkcolor=darkcandyapplered,   % color of internal links
    citecolor=midnightblue,  % color of links to bibliography
    urlcolor=cyan,  % color of external links
    linktocpage=true  % changes the links from the section body to the page number
    }

% geometry
\textwidth=16cm 
\textheight=21cm 
\hoffset=-55pt 
\footskip=25pt

% thm envs
% In this macro I define all of my math operators

% fields
\newcommand{\NN}{\mathbbmss{N}} 
\newcommand{\ZZ}{\mathbbmss{Z}} 
\newcommand{\QQ}{\mathbbmss{Q}} 
\newcommand{\RR}{\mathbbmss{R}} 
\newcommand{\CC}{\mathbbmss{C}} 
\newcommand{\KK}{\mathbbmss{K}} 
\newcommand{\FF}{\mathbbmss{F}} 

% symmetric group
\newcommand{\fS}{\mathfrak{S}}  

% calligraphic 
\newcommand{\aA}{\mathcal{A}} 
\newcommand{\bB}{\mathcal{B}}
\newcommand{\cC}{\mathcal{C}}
\newcommand{\dD}{\mathcal{D}}
\newcommand{\eE}{\mathcal{E}}
\newcommand{\fF}{\mathcal{F}}
\newcommand{\hH}{\mathcal{H}}
\newcommand{\iI}{\mathcal{I}}
\newcommand{\lL}{\mathcal{L}}
\newcommand{\oO}{\mathcal{O}}
\newcommand{\pP}{\mathcal{P}}
\newcommand{\sS}{\mathcal{S}}
\newcommand{\mM}{\mathcal{M}}
\newcommand{\uU}{\mathcal{U}}

% bold
\newcommand{\bfa}{\mathbf{a}}
\newcommand{\bfe}{\mathbf{e}}
\newcommand{\bfF}{\pmb{F}}
\newcommand{\bfR}{\pmb{R}}
\newcommand{\bfv}{\mathbf{v}}
%\newcommand{\bfx}{\bm{x}}
%\newcommand{\bfx}{\mathbf{x}} 
\newcommand{\bfx}{\pmb{x}}
\newcommand{\bfX}{\pmb{X}}
\newcommand{\bfy}{\pmb{y}}
\newcommand{\bfz}{\pmb{z}}

% roman
\newcommand{\rmA}{\mathrm{A}}
\newcommand{\rmB}{\mathrm{B}}
\newcommand{\rmC}{\mathrm{C}}
\newcommand{\rmD}{\mathrm{D}} 
\newcommand{\rmI}{\mathrm{I}} 
\newcommand{\rmK}{\mathrm{K}}
\newcommand{\rmM}{\mathrm{M}}
\newcommand{\rmP}{\mathrm{P}}  
\newcommand{\rmp}{\mathrm{p}}  
\newcommand{\rmQ}{\mathrm{Q}}  
\newcommand{\rmR}{\mathrm{R}}
\newcommand{\rmS}{\mathrm{S}}
\newcommand{\rmT}{\mathrm{T}}
\newcommand{\rmU}{\mathrm{U}}
\newcommand{\rmV}{\mathrm{V}}
\newcommand{\rmY}{\mathrm{Y}}
\newcommand{\rmZ}{\mathrm{Z}}
\newcommand{\rmz}{\mathrm{z}}

% greek letters
% I'm renewing some commands in order to appear in upright font
% If I want to change it later, I don't have to do it manually, I just change it from here.
% \newcommand{\uaa}{\alphaup}
% \renewcommand{\alpha}{\alphaup}
% \renewcommand{\beta}{\betaup}
% \renewcommand{\gamma}{\gammaup}
% \renewcommand{\delta}{\deltaup}
% \renewcommand{\epsilon}{\epsilonup}
% \newcommand{\ee}{\epsilon}
% \renewcommand{\varepsilon}{\varepsilonup}
% \renewcommand{\theta}{\thetaup}
% \renewcommand{\lambda}{\lambdaup}
% \newcommand{\ull}{\lambda}
% \renewcommand{\mu}{\muup}
% \renewcommand{\nu}{\nuup}
% \renewcommand{\pi}{\piup}
% \renewcommand{\rho}{\rhoup}
% \renewcommand{\varrho}{\varrhoup}
% \renewcommand{\sigma}{\sigmaup}
% \renewcommand{\tau}{\tauup} 
% \renewcommand{\phi}{\phiup}
% \renewcommand{\chi}{\chiup}
% \renewcommand{\psi}{\psiup}
% \renewcommand{\omega}{\omegaup}

% arrows and symbols 
\renewcommand{\to}{\rightarrow}
\newcommand{\toto}{\longrightarrow}
\newcommand{\mapstoto}{\longmapsto}
\newcommand{\then}{\Rightarrow}
\newcommand{\IFF}{\Leftrightarrow}
\newcommand{\tl}{\tilde}
\newcommand{\wtl}{\widetilde}
\newcommand{\ol}{\overline}
\newcommand{\ul}{\underline}
\newcommand{\oldemptyset}{\emptyset}
\renewcommand{\emptyset}{\varnothing}
\DeclareMathSymbol{\Arg}{\mathbin}{AMSa}{"39} % for arguments 
\newcommand{\onto}{\ensuremath{\twoheadrightarrow}}
\newcommand{\tle}{\trianglelefteq}
\newcommand{\tge}{\trianglerighteq}

% absolute value symbol
\usepackage{mathtools} 
\DeclarePairedDelimiter\abs{\lvert}{\rvert}%
\DeclarePairedDelimiter\norm{\lVert}{\rVert}%
\makeatletter
\let\oldabs\abs
\def\abs{\@ifstar{\oldabs}{\oldabs*}}

% tensor symbol
\newcommand{\tensor}[1]{%
  \mathbin{\mathop{\otimes}\limits_{#1}}%
}

% permutation cycle notation
\ExplSyntaxOn
\NewDocumentCommand{\cycle}{ O{\;} m }
 {
  (
  \alec_cycle:nn { #1 } { #2 }
  )
 }

\seq_new:N \l_alec_cycle_seq
\cs_new_protected:Npn \alec_cycle:nn #1 #2
 {
  \seq_set_split:Nnn \l_alec_cycle_seq { , } { #2 }
  \seq_use:Nn \l_alec_cycle_seq { #1 }
 }
\ExplSyntaxOff

% setminus symbol
\newcommand{\mysetminusD}{\hbox{\tikz{\draw[line width=0.6pt,line cap=round] (3pt,0) -- (0,6pt);}}}
\newcommand{\mysetminusT}{\mysetminusD}
\newcommand{\mysetminusS}{\hbox{\tikz{\draw[line width=0.45pt,line cap=round] (2pt,0) -- (0,4pt);}}}
\newcommand{\mysetminusSS}{\hbox{\tikz{\draw[line width=0.4pt,line cap=round] (1.5pt,0) -- (0,3pt);}}}
\newcommand{\sm}{\mathbin{\mathchoice{\mysetminusD}{\mysetminusT}{\mysetminusS}{\mysetminusSS}}}

% custom math operators
\newcommand{\Des}{\mathrm{Des}} 
\newcommand{\des}{\mathrm{des}} 
\newcommand{\Asc}{\mathrm{Asc}}
\newcommand{\asc}{\mathrm{asc}} 
\newcommand{\inv}{\mathrm{inv}}
\newcommand{\Inv}{\mathrm{Inv}}
\newcommand{\maj}{\mathrm{maj}} 
\newcommand{\comaj}{\mathrm{comaj}} 
\newcommand{\fix}{\mathrm{fix}} 
\newcommand{\Sym}{\mathrm{Sym}} 
\newcommand{\QSym}{\mathrm{QSym}}
\newcommand{\FQSym}{\mathrm{FQSym}} 
\newcommand{\End}{\mathrm{End}} 
\newcommand{\Rad}{\mathrm{Rad}} 
\newcommand{\rmMat}{\mathrm{Mat}} 
\newcommand{\rmdim}{\mathrm{dim}} 
\newcommand{\rmTop}{\mathrm{Top}} 
\newcommand{\rmCF}{\mathrm{CF}} 
\newcommand{\rmId}{\mathrm{Id}}
\newcommand{\rmid}{\mathrm{id}}
\newcommand{\rmtw}{\mathrm{tw}}
\newcommand{\trace}{\mathrm{tr}}
\newcommand{\Irr}{\mathrm{Irr}}
\newcommand{\Ind}{\mathrm{Ind}} % induction
\newcommand{\Res}{\mathrm{Res}} % restriction
\newcommand{\triv}{\mathrm{triv}} % trivial rep
\newcommand{\rmdef}{\mathrm{def}} % defining rep
\newcommand{\dom}{\triangleleft}
\newcommand{\domeq}{\trianglelefteq}
\newcommand{\lex}{\mathrm{lex}}
\newcommand{\sign}{\mathrm{sign}}
\newcommand{\SYT}{\mathrm{SYT}}
\renewcommand{\Im}{\mathrm{Im}}
\newcommand{\Ker}{\mathrm{Ker}}
\newcommand{\GL}{\mathrm{GL}}
\newcommand{\FL}{\mathrm{FL}}
\newcommand{\Span}{\mathrm{span}}
\newcommand{\pos}{\mathrm{pos}}
\newcommand{\Comp}{\mathrm{Comp}}
\newcommand{\Set}{\mathrm{Set}}
\newcommand{\std}{\mathrm{std}}
\newcommand{\cont}{\mathrm{cont}} %content of a SSYT
\newcommand{\SSYT}{\mathrm{SSYT}}
\newcommand{\ct}{\mathrm{ct}} % cycle type
\newcommand{\ch}{\mathrm{ch}} % Frobenius characteristic map
\newcommand{\height}{\mathrm{ht}}
\newcommand{\FPS}{\CC[\![\bfx]\!]} % formal power series
\newcommand{\FPSS}{\CC[\![\bfx,\bfy]\!]}
\newcommand{\reg}{\mathrm{reg}}
\newcommand{\hook}{\mathrm{h}}
\newcommand{\weight}{\mathrm{wt}}
\newcommand{\co}{\mathrm{co}}
\newcommand{\ps}{\mathrm{ps}}
\newcommand{\rmsum}{\mathrm{sum}}
\newcommand{\NSym}{\mathrm{NSym}}
\newcommand{\Hom}{\mathrm{Hom}}
\newcommand{\proj}{\mathrm{proj}}
\newcommand{\stat}{\mathrm{stat}}
\newcommand{\Par}{\mathrm{Par}}
\newcommand{\rmset}{\mathrm{set}}
\newcommand{\comp}{\mathrm{comp}}

% math ops
% In this macro I define all the theorem environments

\theoremstyle{definition}
\newtheorem{theorem}{Θεώρημα}
\newtheorem{proposition}[theorem]{Πρόταση}
\newtheorem{lemma}[theorem]{Λήμμα}
\newtheorem{corollary}[theorem]{Πόρισμα}
\newtheorem{conjecture}[theorem]{Εικασία}
\newtheorem{problem}[theorem]{Πρόβλημα}
\newtheorem*{claim}{Ισχυρισμός}
\newtheorem{observation}[theorem]{Παρατήρηση}
\newtheorem{definition}[theorem]{Ορισμός}
\newtheorem{question}[theorem]{Ερώτηση}
\newtheorem*{questions}{Ερωτήματα}
\newtheorem{example}[theorem]{Παράδειγμα}
\newtheorem{exercise}{Άσκηση}

\newtheorem*{combInterlude}{Ιντερλούδιο Συνδυαστικής}
\newtheorem*{example_cont}{Παράδειγμα~6.6}
\newtheorem*{digression_la}{Παρέκβαση Γραμμικής Άλγεβρας}
\newtheorem*{thm}{Θεώρημα}

\theoremstyle{remark}
\newtheorem*{remark}{Παρατήρηση}

% fixes the correct numbering of environments
\numberwithin{theorem}{section}
\numberwithin{exercise}{section}
\numberwithin{equation}{section}

% miscellaneous commands
\newcommand{\defn}[1]{{\color{mylightblue}{#1}}}
\newcommand{\toDo}{{\bf\color{red} TODO}}
\newcommand{\toCite}{{\bf\color{green} CITE}}

% redefine the titlepage labels


% titlepage
\title[]{Θεωρία Αναπαραστάσεων και Συνδυαστική \\ Φυλλάδιο Ασκήσεων 1}
%\author[]{Τμήμα Μαθηματικών και Εφαρμοσμένων Μαθηματικών, Πανεπιστήμιο Κρήτης \\ Χειμερινό Εξάμηνο 2025}

\begin{document}
\begingroup
\def\uppercasenonmath#1{} % this disables uppercase title
\let\MakeUppercase\relax % this disables uppercase authors
\maketitle
\endgroup

\setcounter{section}{1}
\thispagestyle{empty}

\begin{exercise}{(\texttt{Διεδρική ομάδα})}
    \\
    Θεωρούμε το τετράγωνο $T$ στον $\RR^2$ με κορυφές $(1,1), (1,-1)$, $(-1,-1)$ και $(-1,1)$.
    \begin{itemize}
        \item[(1)] Υπολογίστε την ομάδα συμμετρίας του $T$, την οποία συμβολίζουμε με $\rmD_8$.
        \item[(2)] Να δείξετε ότι η απεικόνιση $\rho : \rmD_8 \to \GL(\RR^2)$ που ορίζεται από 
        \begin{align*}
            \rho(r) &= 
                \begin{pmatrix}
                    0 & -1 \\
                    1 & 0
                \end{pmatrix} \\
            \rho(s) &= 
                \begin{pmatrix}
                    0 & 1 \\
                    1 & 0
                \end{pmatrix},
        \end{align*}
        όπου με $r$ και $s$ συμβολίζουμε την περιστροφή (με τη φορά του ρολογιού) κατά $\pi/2$ και την ανάκλαση ως προς την ευθεία $x_1 = x_2$ είναι ομομορφισμός ομάδων. 
        \item[(3)] Έστω η αναπαράσταση $(\rho, \RR^2)$ της $\rmD_8$ του προηγούμενου ερωτήματος. Υπολογίστε τους πίνακες $\rho(g)$, για κάθε $g \in \rmD_8$.
        \item[(4)] Η αναπαράσταση $(\rho, \RR^2)$ είναι ανάγωγη?
    \end{itemize}
\end{exercise}

\begin{exercise}
    \leavevmode
    \begin{itemize}
        \item[(1)] Δείξτε ότι η απεικόνιση $\fS_n\times\RR^n \to \RR^n$ που ορίζεται από 
        \[
        \pi\cdot(x_1, x_2, \dots, x_n) = (x_{\pi_1},x_{\pi_2}, \dots, x_{\pi_n})
        \]
        \emph{δεν} είναι δράση της $\fS_n$ στον $\RR^n$. Ειδικότερα, δείξτε ότι 
        \[
        \pi\cdot(\sigma\cdot \bfx) = (\sigma\pi)\cdot\bfx,
        \]
        για κάθε $\pi, \sigma \in \fS_n$ και $\bfx \in \RR^n$.
        \item[(2)] Δείξτε ότι η απεικόνιση $\fS_n\times\RR^n \to \RR^n$ που ορίζεται από 
        \[
        \pi\cdot(x_1, x_2, \dots, x_n) = (x_{\pi_1^{-1}},x_{\pi_2^{-1}}, \dots, x_{\pi_n^{-1}})
        \]
        \emph{είναι} δράση της $\fS_n$ στον $\RR^n$.
        \item[(3)] Δείξτε ότι η δράση του (2) δίνει στον $\RR^n$ την δομή $\fS_n$-προτύπου.
    \end{itemize}
\end{exercise}

\begin{exercise}{(\texttt{Το Θεώρημα του \textlatin{Maschke} παύει να ισχύει για άπειρες ομάδες})}
    \\
    Να δείξετε ότι η αναπαράσταση $(\rho, \RR^2)$ της $\ZZ$ με $\rho : \ZZ \to \GL(\RR^2)$ που ορίζεται από 
    \[
    \rho(n) = 
        \begin{pmatrix}
            1 & n \\
            0 & 1
        \end{pmatrix}
    \]
    για κάθε $n \in \ZZ$ δεν είναι πλήρως αναγώγιμη.
\end{exercise}

\begin{exercise}{(\texttt{Ομομορφισμοί μεταξύ προτύπων})}
    \\
    Έστω $G$ μια ομάδα και $V, W$ δυο $G$-πρότυπα. 
    \begin{itemize}
        \item[(1)] Αν $\{v_1, v_2, \dots, v_n\}$ και $\{w_1, w_2, \dots, w_m\}$ είναι βάσεις των $V$ και $W$, αντίστοιχα, να δείξετε ότι το σύνολο $\{\varphi_{ij} : 1 \le i \le n, 1 \le j \le m\}$, όπου 
        \[
        \varphi_{ij}(v_k) \coloneqq
        \begin{cases}
            w_j, &\ \text{αν $i = k$} \\
            0, &\ \text{διαφορετικά},
        \end{cases}
        \]
        αποτελεί βάση του $\Hom(V,W)$. Ειδικότερα, συνάγετε ότι 
        \[
        \dim\Hom(V,W) = \dim(V)\dim(W).
        \]
        \item[(2)] Να δείξετε ότι h $G$ δρα στο $\Hom(V,W)$ με 
        \[
        g \cdot \varphi(v) \coloneqq g\varphi(g^{-1}v)
        \]
        για κάθε $g \in G$, $\varphi \in \Hom(V,W)$ και $v \in V$ μετατρέποντάς το σε $G$-πρότυπο.
        \item[(3)] Να δείξετε ότι 
        \[
        \Hom_G(V,W) = \Hom(V,W)^G \coloneqq \{\varphi \in \Hom(V,W) : g\cdot\varphi = \varphi\}.
        \] 
    \end{itemize}
\end{exercise}

\begin{exercise}
    Έστω $G$ ομάδα και $\FF$ ένα αλγεβρικά κλειστό σώμα. Το σύνολο 
    \[
    \rmZ(G) \coloneqq \{g \in G : gh = hg, \ \text{για κάθε $h \in G$}\}
    \]
    ονομάζεται \defn{κέντρο} της $G$. Αν $(\rho, V)$ είναι ένα ανάγωγο $G$-πρότυπο και $g \in \rmZ(G)$, χρησιμοπιήστε το Λήμμα του \textlatin{Schur} για να δείξετε ότι 
    \[
    \rho(g) = c\rmid
    \]
    για κάποιο $c \in \FF$.
\end{exercise}

% [Notes, Ex8.7]
\begin{exercise}{(\textit{Ανάγωγες αναπαραστάσεις της κυκλικής ομάδας})}
    \\
    Έστω $\rmC_n$ η κυκλική ομάδα τάξης $n$ με γεννήτορα $x$.
    \begin{itemize}
        \item[(1)] Αν $\zeta$ είναι μια $n$-οστή ρίζα της μονάδας, τότε να δείξετε ότι η απεικόνιση
        \begin{align*}
            \rho : \ZZ_n &\to \GL(\CC) \\
            g^i &\mapsto \zeta^i
        \end{align*}
        για κάθε $1 \le i \le n-1$ είναι μια αναπαράσταση της $\rmC_n$.
        \item[(2)] Να δείξετε ότι κάθε άλλη μονοδιάστατη αναπαράσταση της $\rmC_n$ είναι της μορφής (1).
        \item[(3)] Πόσες κλάσεις ισομορφισμού μονοδιάστασων αναπραστάσεων της $\rmC_n$ (πάνω από το $\CC$) υπάρχουν?
        \item[(4)] Να δείξετε ότι αυτές είναι \emph{όλες} οι ανάγωγες αναπαραστάσεις της $\rmC_n$.
    \end{itemize}
\end{exercise}

\begin{exercise}
    Έστω $G$ μια πεπερασμένη ομάδα και $V, W$ δύο $G$-πρότυπα.
    \begin{itemize}
        \item[(1)] Αν $W_1, W_2, \dots, W_k$ είναι μια συλλογή από $G$-πρότυπα, να δείξετε ότι 
        \[
        \Hom_G(V,W_1 \oplus W_2 \oplus \cdots \oplus W_k) \cong 
        \Hom_G(V,W_1) \oplus \Hom_G(V,W_2) \oplus \cdots \oplus \Hom_G(V,W_k).
        \]
        \item[(2)] Αν $V_1, V_2, \dots, W_\ell$ είναι μια συλλογή από $G$-πρότυπα, να δείξετε ότι 
        \[
        \Hom_G(V_1 \oplus V_2 \oplus \cdots \oplus V_\ell,W) \cong 
        \Hom(V_1,W) \oplus \Hom(V_2,W) \oplus \cdots \oplus \Hom(V_\ell,W).
        \]
    \end{itemize}
\end{exercise}
\end{document}