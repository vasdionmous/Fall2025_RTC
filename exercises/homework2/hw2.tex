\documentclass[12pt,a4paper,reqno]{amsart}

% language
\usepackage[greek, english]{babel}
\usepackage[utf8]{inputenc}
\usepackage{alphabeta}

% change default names to greek
\addto\captionsenglish{
    \renewcommand{\contentsname}{Περιεχόμενα}
    \renewcommand{\refname}{Βιβλιογραφία}
    \renewcommand{\datename}{Ημερομηνία:}
    \renewcommand{\urladdrname}{Ιστοσελίδα}
}

% math 
\usepackage{amsmath,amsthm,amssymb,amscd}

% font
\usepackage[cal=euler,frak=mt]{mathalfa}
\usepackage{libertinus-type1}
\usepackage{txfonts} % for upright greek letters
\usepackage{bm} % for bold symbols
\usepackage{bbm} % for the simply-looking bb symbols

% miscellaneous 
\usepackage{changepage} %for indenting environments
\usepackage{csquotes} % example: \textcquote{}

% colors
\usepackage{xcolor}
\definecolor{darkcandyapplered}{rgb}{0.64, 0.0, 0.0}
\definecolor{midnightblue}{rgb}{0.1, 0.1, 0.44}
\definecolor{mylightblue}{HTML}{336699}
\definecolor{lightgray}{rgb}{0.83, 0.83, 0.83}

% hrefs
\usepackage{hyperref}
\usepackage[noabbrev,capitalize]{cleveref}
\hypersetup{
    pdftoolbar=true,        
    pdfmenubar=true,        
    pdffitwindow=false,     
    pdfstartview={FitH},  % fits the width of the page to the window
    pdftitle={},
    pdfauthor={},
    pdfsubject={},
    pdfkeywords={},
    pdfnewwindow=true,  % links in new window
    colorlinks=true,  % false: boxed links; true: colored links
    linkcolor=darkcandyapplered,   % color of internal links
    citecolor=midnightblue,  % color of links to bibliography
    urlcolor=cyan,  % color of external links
    linktocpage=true  % changes the links from the section body to the page number
    }

% geometry
\textwidth=16cm 
\textheight=21cm 
\hoffset=-55pt 
\footskip=25pt

% thm envs
% In this macro I define all the theorem environments

\theoremstyle{definition}
\newtheorem{theorem}{Θεώρημα}
\newtheorem{proposition}[theorem]{Πρόταση}
\newtheorem{lemma}[theorem]{Λήμμα}
\newtheorem{corollary}[theorem]{Πόρισμα}
\newtheorem{conjecture}[theorem]{Εικασία}
\newtheorem{problem}[theorem]{Πρόβλημα}
\newtheorem*{claim}{Ισχυρισμός}
\newtheorem{observation}[theorem]{Παρατήρηση}
\newtheorem{definition}[theorem]{Ορισμός}
\newtheorem{question}[theorem]{Ερώτηση}
\newtheorem{example}[theorem]{Παράδειγμα}
\newtheorem{exercise}{Άσκηση}

\theoremstyle{remark}
\newtheorem*{remark}{Παρατήρηση}

% fixes the correct numbering of environments
\numberwithin{theorem}{section}
\numberwithin{exercise}{section}
\numberwithin{equation}{section}

% math ops
% In this macro I define all of my math operators

% fields
\newcommand{\NN}{\mathbbmss{N}} 
\newcommand{\ZZ}{\mathbbmss{Z}} 
\newcommand{\QQ}{\mathbbmss{Q}} 
\newcommand{\RR}{\mathbbmss{R}} 
\newcommand{\CC}{\mathbbmss{C}} 
\newcommand{\KK}{\mathbbmss{K}} 
\newcommand{\FF}{\mathbbmss{F}} 

% symmetric group
\newcommand{\fS}{\mathfrak{S}}  

% calligraphic 
\newcommand{\aA}{\mathcal{A}} 
\newcommand{\bB}{\mathcal{B}}
\newcommand{\cC}{\mathcal{C}}
\newcommand{\dD}{\mathcal{D}}
\newcommand{\eE}{\mathcal{E}}
\newcommand{\fF}{\mathcal{F}}
\newcommand{\hH}{\mathcal{H}}
\newcommand{\iI}{\mathcal{I}}
\newcommand{\lL}{\mathcal{L}}
\newcommand{\oO}{\mathcal{O}}
\newcommand{\pP}{\mathcal{P}}
\newcommand{\sS}{\mathcal{S}}
\newcommand{\mM}{\mathcal{M}}
\newcommand{\uU}{\mathcal{U}}

% bold
\newcommand{\bfa}{\mathbf{a}}
\newcommand{\bfe}{\mathbf{e}}
\newcommand{\bfF}{\pmb{F}}
\newcommand{\bfR}{\pmb{R}}
\newcommand{\bfv}{\mathbf{v}}
%\newcommand{\bfx}{\bm{x}}
%\newcommand{\bfx}{\mathbf{x}} 
\newcommand{\bfx}{\pmb{x}}
\newcommand{\bfX}{\pmb{X}}
\newcommand{\bfy}{\pmb{y}}
\newcommand{\bfz}{\pmb{z}}

% roman
\newcommand{\rmB}{\mathrm{B}}
\newcommand{\rmC}{\mathrm{C}}
\newcommand{\rmD}{\mathrm{D}} 
\newcommand{\rmI}{\mathrm{I}} 
\newcommand{\rmK}{\mathrm{K}}
\newcommand{\rmM}{\mathrm{M}}
\newcommand{\rmP}{\mathrm{P}}  
\newcommand{\rmQ}{\mathrm{Q}}  
\newcommand{\rmR}{\mathrm{R}}
\newcommand{\rmS}{\mathrm{S}}
\newcommand{\rmT}{\mathrm{T}}
\newcommand{\rmU}{\mathrm{U}}
\newcommand{\rmV}{\mathrm{V}}
\newcommand{\rmY}{\mathrm{Y}}
\newcommand{\rmZ}{\mathrm{Z}}

% greek letters
% I'm renewing some commands in order to appear in upright font
% If I want to change it later, I don't have to do it manually, I just change it from here.
% \newcommand{\uaa}{\alphaup}
% \renewcommand{\alpha}{\alphaup}
% \renewcommand{\beta}{\betaup}
% \renewcommand{\gamma}{\gammaup}
% \renewcommand{\delta}{\deltaup}
% \renewcommand{\epsilon}{\epsilonup}
% \newcommand{\ee}{\epsilon}
% \renewcommand{\varepsilon}{\varepsilonup}
% \renewcommand{\theta}{\thetaup}
% \renewcommand{\lambda}{\lambdaup}
% \newcommand{\ull}{\lambda}
% \renewcommand{\mu}{\muup}
% \renewcommand{\nu}{\nuup}
% \renewcommand{\pi}{\piup}
% \renewcommand{\rho}{\rhoup}
% \renewcommand{\varrho}{\varrhoup}
% \renewcommand{\sigma}{\sigmaup}
% \renewcommand{\tau}{\tauup} 
% \renewcommand{\phi}{\phiup}
% \renewcommand{\chi}{\chiup}
% \renewcommand{\psi}{\psiup}
% \renewcommand{\omega}{\omegaup}

% arrows and symbols 
\renewcommand{\to}{\rightarrow}
\newcommand{\toto}{\longrightarrow}
\newcommand{\mapstoto}{\longmapsto}
\newcommand{\then}{\Rightarrow}
\newcommand{\IFF}{\Leftrightarrow}
\newcommand{\tl}{\tilde}
\newcommand{\wtl}{\widetilde}
\newcommand{\ol}{\overline}
\newcommand{\ul}{\underline}
\newcommand{\oldemptyset}{\emptyset}
\renewcommand{\emptyset}{\varnothing}
\DeclareMathSymbol{\Arg}{\mathbin}{AMSa}{"39} % for arguments 
\newcommand{\onto}{\ensuremath{\twoheadrightarrow}}

% absolute value symbol
\usepackage{mathtools} 
\DeclarePairedDelimiter\abs{\lvert}{\rvert}%
\DeclarePairedDelimiter\norm{\lVert}{\rVert}%
\makeatletter
\let\oldabs\abs
\def\abs{\@ifstar{\oldabs}{\oldabs*}}

% tensor symbol
\newcommand{\tensor}[1]{%
  \mathbin{\mathop{\otimes}\limits_{#1}}%
}

% permutation cycle notation
\ExplSyntaxOn
\NewDocumentCommand{\cycle}{ O{\;} m }
 {
  (
  \alec_cycle:nn { #1 } { #2 }
  )
 }

\seq_new:N \l_alec_cycle_seq
\cs_new_protected:Npn \alec_cycle:nn #1 #2
 {
  \seq_set_split:Nnn \l_alec_cycle_seq { , } { #2 }
  \seq_use:Nn \l_alec_cycle_seq { #1 }
 }
\ExplSyntaxOff

% setminus symbol
\newcommand{\mysetminusD}{\hbox{\tikz{\draw[line width=0.6pt,line cap=round] (3pt,0) -- (0,6pt);}}}
\newcommand{\mysetminusT}{\mysetminusD}
\newcommand{\mysetminusS}{\hbox{\tikz{\draw[line width=0.45pt,line cap=round] (2pt,0) -- (0,4pt);}}}
\newcommand{\mysetminusSS}{\hbox{\tikz{\draw[line width=0.4pt,line cap=round] (1.5pt,0) -- (0,3pt);}}}
\newcommand{\sm}{\mathbin{\mathchoice{\mysetminusD}{\mysetminusT}{\mysetminusS}{\mysetminusSS}}}

% custom math operators
\newcommand{\Des}{\mathrm{Des}} 
\newcommand{\des}{\mathrm{des}} 
\newcommand{\Asc}{\mathrm{Asc}}
\newcommand{\asc}{\mathrm{asc}} 
\newcommand{\inv}{\mathrm{inv}}
\newcommand{\Inv}{\mathrm{Inv}}
\newcommand{\maj}{\mathrm{maj}} 
\newcommand{\comaj}{\mathrm{comaj}} 
\newcommand{\fix}{\mathrm{fix}} 
\newcommand{\Sym}{\mathrm{Sym}} 
\newcommand{\QSym}{\mathrm{QSym}}
\newcommand{\FQSym}{\mathrm{FQSym}} 
\newcommand{\End}{\mathrm{End}} 
\newcommand{\Rad}{\mathrm{Rad}} 
\newcommand{\rmMat}{\mathrm{Mat}} 
\newcommand{\rmdim}{\mathrm{dim}} 
\newcommand{\rmTop}{\mathrm{Top}} 
\newcommand{\rmCF}{\mathrm{CF}} 
\newcommand{\rmId}{\mathrm{Id}}
\newcommand{\rmid}{\mathrm{id}}
\newcommand{\rmtw}{\mathrm{tw}}
\newcommand{\trace}{\mathrm{tr}}
\newcommand{\Irr}{\mathrm{Irr}}
\newcommand{\Ind}{\mathrm{Ind}} % induction
\newcommand{\Res}{\mathrm{Res}} % restriction
\newcommand{\triv}{\mathrm{triv}} % trivial rep
\newcommand{\rmdef}{\mathrm{def}} % defining rep
\newcommand{\dom}{\triangleleft}
\newcommand{\domeq}{\trianglelefteq}
\newcommand{\lex}{\mathrm{lex}}
\newcommand{\sign}{\mathrm{sign}}
\newcommand{\SYT}{\mathrm{SYT}}
\renewcommand{\Im}{\mathrm{Im}}
\newcommand{\Ker}{\mathrm{Ker}}
\newcommand{\GL}{\mathrm{GL}}
\newcommand{\FL}{\mathrm{FL}}
\newcommand{\Span}{\mathrm{span}}
\newcommand{\pos}{\mathrm{pos}}
\newcommand{\Comp}{\mathrm{Comp}}
\newcommand{\Set}{\mathrm{Set}}
\newcommand{\std}{\mathrm{std}}
\newcommand{\cont}{\mathrm{cont}} %content of a SSYT
\newcommand{\SSYT}{\mathrm{SSYT}}
\newcommand{\rmz}{\mathrm{z}}
\newcommand{\ct}{\mathrm{ct}} % cycle type
\newcommand{\ch}{\mathrm{ch}} % Frobenius characteristic map
\newcommand{\height}{\mathrm{ht}}
\newcommand{\FPS}{\CC[\![\bfx]\!]} % formal power series
\newcommand{\FPSS}{\CC[\![\bfx,\bfy]\!]}
\newcommand{\reg}{\mathrm{reg}}
\newcommand{\hook}{\mathrm{h}}
\newcommand{\weight}{\mathrm{wt}}
\newcommand{\co}{\mathrm{co}}
\newcommand{\ps}{\mathrm{ps}}
\newcommand{\rmsum}{\mathrm{sum}}
\newcommand{\NSym}{\mathrm{NSym}}
\newcommand{\Hom}{\mathrm{Hom}}
\newcommand{\proj}{\mathrm{proj}}
\newcommand{\stat}{\mathrm{stat}}

% miscellaneous commands
\newcommand{\defn}[1]{{\color{mylightblue}{#1}}}
\newcommand{\toDo}{{\bf\color{red} TODO}}
\newcommand{\toCite}{{\bf\color{green} CITE}}

% redefine the titlepage labels


% titlepage
\title[]{Θ2.04: Θεωρία Αναπαραστάσεων και Συνδυαστική \\ Φυλλάδιο Ασκήσεων 2}
% \author[]{9 Οκτωβρίου 2025}
\date{24 Οκτωβρίου 2025}

\begin{document}
\begingroup
\def\uppercasenonmath#1{} % this disables uppercase title
\let\MakeUppercase\relax % this disables uppercase authors
\maketitle
\endgroup

\setcounter{section}{2}
\thispagestyle{empty}

Σε ότι ακολουθεί $G$ είναι μια πεπερασμένη ομάδα, $n$ είναι ένας θετικός ακέραιος και όλοι οι διανυσματικοί χώροι είναι πεπερασμένης διάστασης πάνω από το $\CC$.

\begin{exercise}{(\texttt{Πίνακας χαρακτήρων της διεδρικής ομάδας})} \\
    Υποθέτουμε ότι $n\ge3$. Έστω 
    \[
    \rmD_{2n} = \langle r, s \ \vert \ r^n = s^2 = \epsilon, rsr=s \rangle
    \]
    η διεδρική ομάδα τάξης $2n$. Να δείξετε τα εξής.
    \begin{itemize}
        \item[(1)] Αν $n = 2k$, τότε οι κλάσεις συζυγίας της $\rmD_{2n}$ είναι 
        \[
        \{\epsilon\}, \
        \{r^k\}, \ 
        \{r^{\pm1}\}, \{r^{\pm2}\}, \dots, \{r^{\pm(k-1)}\}, \
        \{s, sr^2, \dots, sr^{2(k-1)}\}, \ 
        \{sr, sr^3, \dots, sr^{2(k-1)+1}\},
        \]
        ενώ αν $n = 2k+1$, τότε οι κλάσεις συζυγίας της $\rmD_{2n}$ είναι 
        \[
        \{\epsilon\}, \  
        \{r^{\pm1}\}, \{r^{\pm2}\}, \dots, \{r^{\pm k}\}, \ 
        \{s, sr, \dots, sr^{n-1}\}.
        \]
        \item[(2)] Για $1 \le k \le n-1$, η $\rho_k : \rmD_{2n} \to \GL_2(\CC)$ 
        \[
        \rho_k(r) = 
        \begin{pmatrix}
            \cos\left(2\pi{k}/n\right)  & \sin\left(2\pi{k}/n\right) \\
            -\sin\left(2\pi{k}/n\right) & \cos\left(2\pi{k}/n\right)
        \end{pmatrix},
        \quad 
        \text{και}
        \quad 
        \rho_k(s) = 
        \begin{pmatrix}
            0 & 1 \\
            1 & 0
        \end{pmatrix}.
        \]
        είναι ανάγωγη αναπαράσταση της $\rmD_{2n}$.
        \item[(3)] Αν ο $n$ είναι άρτιος (αντ. περιττός), τότε τα ζεύγη $(\rho_k,\rho_{n-k})$ είναι ισόμορφες αναπαραστάσεις για $1 \le k \le \frac{n}{2}-1$ (αντ. $1 \le k \le \frac{n-1}{2}$). 
        \item[(4)] Αν ο $n$ είναι άρτιος (αντ. περιττός), τότε το πλήθος των μη ισόμορφων αναπαρα\-στάσεων διάστασης 1 είναι ίσο με τέσσερα (αντ. δύο). Ποιές είναι αυτές;
        \item[(5)] Υπολογίστε τον πίνακα χαρακτήρων της $\rmD_{2n}$.
    \end{itemize}

    \textcolor{lightgray}{\small{\emph{Υπόδειξη}: Λύστε την άσκηση πρώτα για $n=4$ και $n=5$ και έπειτα γενικεύστε.}} 
\end{exercise}

\begin{exercise}{(\texttt{Ιδιότητες χαρακτήρων})} \\
    Έστω $\chi$ ο χαρακτήρας μιας αναπαράστασης $(\rho,V)$ της $G$. Να δείξετε τα εξής.
    \begin{itemize}
        \item[(1)] Αν $g \in G$ και $m$ ο ελάχιστος μη αρνητικός ακέραιος για τον οποίο $g^m=\epsilon$, τότε το $\chi(g)$ είναι άθροισμα $m$-οστών ριζών της μονάδας.
        \item[(2)] Για κάθε $g \in G$, ισχύει ότι $\chi(g^{-1}) = \ol{\chi(g)}$.
        \item[(3)] Αν $g$ και $g^{-1}$ είναι συζυγή στην $G$, τότε $\chi(g) \in \RR$.
    \end{itemize}

    \textcolor{lightgray}{\small{\emph{Υπόδειξη}: Για το (1), θεωρήστε την υποομάδα που παράγεται από το $g$, η οποία είναι ισόμορφη με την $\rmC_m$. Τι μπορείτε να πείτε για τον πίνακα $\rho(g)$ όταν \textquote{περιοριστούμε} σε αυτή τη δράση; Για το (2) (αντ. (3)), χρησιμοποιήστε το (1) (αντί. (2)).}} 
\end{exercise}

\begin{exercise}{(\texttt{Ο χαρακτήρας της αναπαράστασης μεταθέσεων και το Λήμμα του Burnside})}
    \label{ex:permutation_representation_character} 
    \noindent 
    Υποθέτουμε ότι η $G$ δρα σε ένα πεπερασμένο σύνολο $S$ και έστω $\chi$ ο χαρακτήρας της επαγόμενης αναπαράσταση μεταθέσεων. Να δείξετε τα εξής.
    \begin{itemize}
        \item[(1)] Για κάθε $g \in G$, ισχύει ότι $\chi(g) = \abs{S^g}$, όπου $S^g \coloneqq \{s \in S : g\cdot{s} = s\}$.
        \item[(2)] Η διάσταση του $\CC[S]^G$ ισούται με το πλήθος των τροχιών της δράσης της $G$ στο $S$.
        \item[(3)] Το πλήθος των τροχιών της δράσης της $G$ στο $S$ ισούται με 
        \[
        \frac{1}{\abs{G}}\sum_{g \in G}\abs{S^g}.
        \]
        \item[(4)] Πόσοι διαφορετικοί τρόποι υπάρχουν να χρωματίσουμε τις κορυφές ενός κανονικού εξαγώνου χρησιμοποιώντας 3 χρώματα, ως προς κυκλική συμμετρία; Δηλαδή, δυο χρωματισμοί θεωρούνται \textquote{ίδιοι} αν διαφέρουν κατά μια στροφή ως προς το κέντρο του εξαγώνου.
    \end{itemize}

    \textcolor{lightgray}{\small{\emph{Υπόδειξη}: Για το (1), δείξτε ότι ο πίνακας του $\rho(g)$ ως προς τη συνήθη βάση του $\CC[S]$ είναι πίνακας μετάθεσης. Για το (2), σκεφτείτε τι σημαίνει για ένα στοιχείο του $\CC[S]$ να είναι σταθερό από τη δράση της $G$. Για το (4), εξετάστε τη δράση της $\rmC_6$ στο σύνολο όλων των πιθανών χρωματισμών.}}
\end{exercise}

\begin{exercise}{(\texttt{Αναμενώμενος αριθμός σταθερών σημείων μιας τυχαίας μετάθεσης})} \\
    Να υπολογίσετε τα παρακάτω αθροίσματα
    \[
    \sum_{\pi \in \fS_n} \fix(\pi) \quad \text{και} \quad \sum_{\pi \in \fS_n} \fix(\pi)^2,
    \]
    όπου $\fix(\pi) \coloneqq \abs{\{i \in [n]:\pi_i=i\}}$. Ποιός είναι ο αναμενώμενος αριθμός των σταθερών σημείων μιας τυχαίας μετάθεσης;

    \textcolor{lightgray}{\small{\emph{Υπόδειξη}: Χρησιμοποιήστε τις σχέσεις ορθογωνιώτητας και την ισοτυπική διάσπαση της αναπαράστασης καθορισμού της $\fS_n$.}}
\end{exercise}

\begin{exercise}{(\texttt{Αθροίσματα γραμμών του πίνακα χαρακτήρων})} \\
    Έστω $\psi_G$ ο χαρακτήρας της αναπαράστασης μεταθέσεων που επάγεται από τη δράση της $G$ στον εαυτό της με συζυγία. Να δείξετε τα εξής.
    \begin{itemize}
        \item[(1)] Έχουμε 
        \[
        \psi_G = \sum_{\chi}\chi\ol{\chi},
        \]
        όπου στο άθροισμα το $\chi$ διατρέχει όλους τους ανάγωγους χαρακτήρες της $G$.
        \item[(2)] Αν $\chi$ είναι ένας ανάγωγος χαρακτήρας της $G$, τότε
        \[
        (\, \psi_G, \chi \,) = \sum_K \chi(K),
        \]
        όπου στο άθροισμα το $K$ διατρέχει όλες τις κλάσεις συζυγίας της $G$.
        \item[(3)] Τα αθροίσματα των στοιχείων των γραμμών του πίνακα χαρακτήρων της $G$ είναι μη αρνητικοί αριθμοί.
    \end{itemize}

    \textcolor{lightgray}{\small{\emph{Υπόδειξη}: Για το (1), χρησιμοποιήστε τις σχέσεις ορθογωνιώτητας και τον την Άσκηση~2.3~(1). Για το (2), χρησιμοποιήστε το (1). Για το (3), χρησιμοποιήστε το Πόρισμα 7.5.}}
\end{exercise}

% \begin{exercise}{(\texttt{Εναλλαγή περιορισμού και επαγωγής})}
%     Έστω $H$ μια υποομάδα της $G$ και $\chi$ (αντ. $\psi$) ένας χαρακτήρας της $G$ (αντ. $H$). Να δείξετε ότι 
%     \[
%     \left(\psi \, \left(\chi\downarrow_H\right) \right)\uparrow^G = \psi\uparrow^G \chi.
%     \]

%     \textcolor{lightgray}{\small{\emph{Υπόδειξη}: Υπολογίστε το εσωτερικό γινόμενο καθενός από τα δυο μέλη με έναν αυθαίρετο ανάγωγο χαρακτήρα και χρησιμοποιήστε το Θεώρημα Αντιστροφής του Frobenius.}}
% \end{exercise}
\end{document}