\documentclass[12pt,a4paper,reqno]{amsart}

% language
\usepackage[greek, english]{babel}
\usepackage[utf8]{inputenc}
\usepackage{alphabeta}

% change default names to greek
\addto\captionsenglish{
    \renewcommand{\contentsname}{Περιεχόμενα}
    \renewcommand{\refname}{Βιβλιογραφία}
    \renewcommand{\datename}{Ημερομηνία:}
    \renewcommand{\urladdrname}{Ιστοσελίδα}
}

% math 
\usepackage{amsmath,amsthm,amssymb,amscd}

% font
\usepackage[cal=euler,frak=mt]{mathalfa}
\usepackage{libertinus-type1}
\usepackage{txfonts} % for upright greek letters
\usepackage{bm} % for bold symbols
\usepackage{bbm} % for the simply-looking bb symbols

% miscellaneous 
\usepackage{changepage} %for indenting environments
\usepackage{csquotes} % example: \textcquote{}

% colors
\usepackage{xcolor}
\definecolor{darkcandyapplered}{rgb}{0.64, 0.0, 0.0}
\definecolor{midnightblue}{rgb}{0.1, 0.1, 0.44}
\definecolor{mylightblue}{HTML}{336699}
\definecolor{lightgray}{rgb}{0.83, 0.83, 0.83}

% hrefs
\usepackage{hyperref}
\usepackage[noabbrev,capitalize]{cleveref}
\hypersetup{
    pdftoolbar=true,        
    pdfmenubar=true,        
    pdffitwindow=false,     
    pdfstartview={FitH},  % fits the width of the page to the window
    pdftitle={},
    pdfauthor={},
    pdfsubject={},
    pdfkeywords={},
    pdfnewwindow=true,  % links in new window
    colorlinks=true,  % false: boxed links; true: colored links
    linkcolor=darkcandyapplered,   % color of internal links
    citecolor=midnightblue,  % color of links to bibliography
    urlcolor=cyan,  % color of external links
    linktocpage=true  % changes the links from the section body to the page number
    }

% geometry
\textwidth=16cm 
\textheight=21cm 
\hoffset=-55pt 
\footskip=25pt

% thm envs
% In this macro I define all the theorem environments

\theoremstyle{definition}
\newtheorem{theorem}{Θεώρημα}
\newtheorem{proposition}[theorem]{Πρόταση}
\newtheorem{lemma}[theorem]{Λήμμα}
\newtheorem{corollary}[theorem]{Πόρισμα}
\newtheorem{conjecture}[theorem]{Εικασία}
\newtheorem{problem}[theorem]{Πρόβλημα}
\newtheorem*{claim}{Ισχυρισμός}
\newtheorem{observation}[theorem]{Παρατήρηση}
\newtheorem{definition}[theorem]{Ορισμός}
\newtheorem{question}[theorem]{Ερώτηση}
\newtheorem*{questions}{Ερωτήματα}
\newtheorem{example}[theorem]{Παράδειγμα}
\newtheorem{exercise}{Άσκηση}

\newtheorem*{combInterlude}{Ιντερλούδιο Συνδυαστικής}
\newtheorem*{example_cont}{Παράδειγμα~6.6}
\newtheorem*{digression_la}{Παρέκβαση Γραμμικής Άλγεβρας}
\newtheorem*{thm}{Θεώρημα}

\theoremstyle{remark}
\newtheorem*{remark}{Παρατήρηση}

% fixes the correct numbering of environments
\numberwithin{theorem}{section}
\numberwithin{exercise}{section}
\numberwithin{equation}{section}

% math ops
% In this macro I define all of my math operators

% fields
\newcommand{\NN}{\mathbbmss{N}} 
\newcommand{\ZZ}{\mathbbmss{Z}} 
\newcommand{\QQ}{\mathbbmss{Q}} 
\newcommand{\RR}{\mathbbmss{R}} 
\newcommand{\CC}{\mathbbmss{C}} 
\newcommand{\KK}{\mathbbmss{K}} 
\newcommand{\FF}{\mathbbmss{F}} 

% symmetric group
\newcommand{\fS}{\mathfrak{S}}  

% calligraphic 
\newcommand{\aA}{\mathcal{A}} 
\newcommand{\bB}{\mathcal{B}}
\newcommand{\cC}{\mathcal{C}}
\newcommand{\dD}{\mathcal{D}}
\newcommand{\eE}{\mathcal{E}}
\newcommand{\fF}{\mathcal{F}}
\newcommand{\hH}{\mathcal{H}}
\newcommand{\iI}{\mathcal{I}}
\newcommand{\lL}{\mathcal{L}}
\newcommand{\oO}{\mathcal{O}}
\newcommand{\pP}{\mathcal{P}}
\newcommand{\sS}{\mathcal{S}}
\newcommand{\mM}{\mathcal{M}}
\newcommand{\uU}{\mathcal{U}}

% bold
\newcommand{\bfa}{\mathbf{a}}
\newcommand{\bfe}{\mathbf{e}}
\newcommand{\bfF}{\pmb{F}}
\newcommand{\bfR}{\pmb{R}}
\newcommand{\bfv}{\mathbf{v}}
%\newcommand{\bfx}{\bm{x}}
%\newcommand{\bfx}{\mathbf{x}} 
\newcommand{\bfx}{\pmb{x}}
\newcommand{\bfX}{\pmb{X}}
\newcommand{\bfy}{\pmb{y}}
\newcommand{\bfz}{\pmb{z}}

% roman
\newcommand{\rmA}{\mathrm{A}}
\newcommand{\rmB}{\mathrm{B}}
\newcommand{\rmC}{\mathrm{C}}
\newcommand{\rmD}{\mathrm{D}} 
\newcommand{\rmI}{\mathrm{I}} 
\newcommand{\rmK}{\mathrm{K}}
\newcommand{\rmM}{\mathrm{M}}
\newcommand{\rmP}{\mathrm{P}}  
\newcommand{\rmp}{\mathrm{p}}  
\newcommand{\rmQ}{\mathrm{Q}}  
\newcommand{\rmR}{\mathrm{R}}
\newcommand{\rmS}{\mathrm{S}}
\newcommand{\rmT}{\mathrm{T}}
\newcommand{\rmU}{\mathrm{U}}
\newcommand{\rmV}{\mathrm{V}}
\newcommand{\rmY}{\mathrm{Y}}
\newcommand{\rmZ}{\mathrm{Z}}
\newcommand{\rmz}{\mathrm{z}}

% greek letters
% I'm renewing some commands in order to appear in upright font
% If I want to change it later, I don't have to do it manually, I just change it from here.
% \newcommand{\uaa}{\alphaup}
% \renewcommand{\alpha}{\alphaup}
% \renewcommand{\beta}{\betaup}
% \renewcommand{\gamma}{\gammaup}
% \renewcommand{\delta}{\deltaup}
% \renewcommand{\epsilon}{\epsilonup}
% \newcommand{\ee}{\epsilon}
% \renewcommand{\varepsilon}{\varepsilonup}
% \renewcommand{\theta}{\thetaup}
% \renewcommand{\lambda}{\lambdaup}
% \newcommand{\ull}{\lambda}
% \renewcommand{\mu}{\muup}
% \renewcommand{\nu}{\nuup}
% \renewcommand{\pi}{\piup}
% \renewcommand{\rho}{\rhoup}
% \renewcommand{\varrho}{\varrhoup}
% \renewcommand{\sigma}{\sigmaup}
% \renewcommand{\tau}{\tauup} 
% \renewcommand{\phi}{\phiup}
% \renewcommand{\chi}{\chiup}
% \renewcommand{\psi}{\psiup}
% \renewcommand{\omega}{\omegaup}

% arrows and symbols 
\renewcommand{\to}{\rightarrow}
\newcommand{\toto}{\longrightarrow}
\newcommand{\mapstoto}{\longmapsto}
\newcommand{\then}{\Rightarrow}
\newcommand{\IFF}{\Leftrightarrow}
\newcommand{\tl}{\tilde}
\newcommand{\wtl}{\widetilde}
\newcommand{\ol}{\overline}
\newcommand{\ul}{\underline}
\newcommand{\oldemptyset}{\emptyset}
\renewcommand{\emptyset}{\varnothing}
\DeclareMathSymbol{\Arg}{\mathbin}{AMSa}{"39} % for arguments 
\newcommand{\onto}{\ensuremath{\twoheadrightarrow}}
\newcommand{\tle}{\trianglelefteq}
\newcommand{\tge}{\trianglerighteq}

% absolute value symbol
\usepackage{mathtools} 
\DeclarePairedDelimiter\abs{\lvert}{\rvert}%
\DeclarePairedDelimiter\norm{\lVert}{\rVert}%
\makeatletter
\let\oldabs\abs
\def\abs{\@ifstar{\oldabs}{\oldabs*}}

% tensor symbol
\newcommand{\tensor}[1]{%
  \mathbin{\mathop{\otimes}\limits_{#1}}%
}

% permutation cycle notation
\ExplSyntaxOn
\NewDocumentCommand{\cycle}{ O{\;} m }
 {
  (
  \alec_cycle:nn { #1 } { #2 }
  )
 }

\seq_new:N \l_alec_cycle_seq
\cs_new_protected:Npn \alec_cycle:nn #1 #2
 {
  \seq_set_split:Nnn \l_alec_cycle_seq { , } { #2 }
  \seq_use:Nn \l_alec_cycle_seq { #1 }
 }
\ExplSyntaxOff

% setminus symbol
\newcommand{\mysetminusD}{\hbox{\tikz{\draw[line width=0.6pt,line cap=round] (3pt,0) -- (0,6pt);}}}
\newcommand{\mysetminusT}{\mysetminusD}
\newcommand{\mysetminusS}{\hbox{\tikz{\draw[line width=0.45pt,line cap=round] (2pt,0) -- (0,4pt);}}}
\newcommand{\mysetminusSS}{\hbox{\tikz{\draw[line width=0.4pt,line cap=round] (1.5pt,0) -- (0,3pt);}}}
\newcommand{\sm}{\mathbin{\mathchoice{\mysetminusD}{\mysetminusT}{\mysetminusS}{\mysetminusSS}}}

% custom math operators
\newcommand{\Des}{\mathrm{Des}} 
\newcommand{\des}{\mathrm{des}} 
\newcommand{\Asc}{\mathrm{Asc}}
\newcommand{\asc}{\mathrm{asc}} 
\newcommand{\inv}{\mathrm{inv}}
\newcommand{\Inv}{\mathrm{Inv}}
\newcommand{\maj}{\mathrm{maj}} 
\newcommand{\comaj}{\mathrm{comaj}} 
\newcommand{\fix}{\mathrm{fix}} 
\newcommand{\Sym}{\mathrm{Sym}} 
\newcommand{\QSym}{\mathrm{QSym}}
\newcommand{\FQSym}{\mathrm{FQSym}} 
\newcommand{\End}{\mathrm{End}} 
\newcommand{\Rad}{\mathrm{Rad}} 
\newcommand{\rmMat}{\mathrm{Mat}} 
\newcommand{\rmdim}{\mathrm{dim}} 
\newcommand{\rmTop}{\mathrm{Top}} 
\newcommand{\rmCF}{\mathrm{CF}} 
\newcommand{\rmId}{\mathrm{Id}}
\newcommand{\rmid}{\mathrm{id}}
\newcommand{\rmtw}{\mathrm{tw}}
\newcommand{\trace}{\mathrm{tr}}
\newcommand{\Irr}{\mathrm{Irr}}
\newcommand{\Ind}{\mathrm{Ind}} % induction
\newcommand{\Res}{\mathrm{Res}} % restriction
\newcommand{\triv}{\mathrm{triv}} % trivial rep
\newcommand{\rmdef}{\mathrm{def}} % defining rep
\newcommand{\dom}{\triangleleft}
\newcommand{\domeq}{\trianglelefteq}
\newcommand{\lex}{\mathrm{lex}}
\newcommand{\sign}{\mathrm{sign}}
\newcommand{\SYT}{\mathrm{SYT}}
\renewcommand{\Im}{\mathrm{Im}}
\newcommand{\Ker}{\mathrm{Ker}}
\newcommand{\GL}{\mathrm{GL}}
\newcommand{\FL}{\mathrm{FL}}
\newcommand{\Span}{\mathrm{span}}
\newcommand{\pos}{\mathrm{pos}}
\newcommand{\Comp}{\mathrm{Comp}}
\newcommand{\Set}{\mathrm{Set}}
\newcommand{\std}{\mathrm{std}}
\newcommand{\cont}{\mathrm{cont}} %content of a SSYT
\newcommand{\SSYT}{\mathrm{SSYT}}
\newcommand{\ct}{\mathrm{ct}} % cycle type
\newcommand{\ch}{\mathrm{ch}} % Frobenius characteristic map
\newcommand{\height}{\mathrm{ht}}
\newcommand{\FPS}{\CC[\![\bfx]\!]} % formal power series
\newcommand{\FPSS}{\CC[\![\bfx,\bfy]\!]}
\newcommand{\reg}{\mathrm{reg}}
\newcommand{\hook}{\mathrm{h}}
\newcommand{\weight}{\mathrm{wt}}
\newcommand{\co}{\mathrm{co}}
\newcommand{\ps}{\mathrm{ps}}
\newcommand{\rmsum}{\mathrm{sum}}
\newcommand{\NSym}{\mathrm{NSym}}
\newcommand{\Hom}{\mathrm{Hom}}
\newcommand{\proj}{\mathrm{proj}}
\newcommand{\stat}{\mathrm{stat}}
\newcommand{\Par}{\mathrm{Par}}
\newcommand{\rmset}{\mathrm{set}}
\newcommand{\comp}{\mathrm{comp}}
\newcommand{\rmm}{\mathrm{m}}

% miscellaneous commands
\newcommand{\defn}[1]{{\color{mylightblue}{#1}}}
\newcommand{\toDo}{{\bf\color{red} TODO}}
\newcommand{\toCite}{{\bf\color{green} CITE}}

% titlepage
\title[]{Θ2.04: Θεωρία Αναπαραστάσεων και Συνδυαστική \\ Φυλλάδιο Ασκήσεων 4}
% \author[]{9 Οκτωβρίου 2025}
\date{2 Δεκεμβρίου 2025}

\begin{document}
\begingroup
\def\uppercasenonmath#1{} % this disables uppercase title
\let\MakeUppercase\relax % this disables uppercase authors
\maketitle
\endgroup

\setcounter{section}{4}
\thispagestyle{empty}

\begin{exercise}{(\texttt{Πίνακας χαρακτήρων της $\fS_4$ (ξανά)})} \\
    Για $\lambda \vdash n$, έστω $\rho^\lambda : \fS_n \to \GL_{f^\lambda}(\CC)$ η αναπαράσταση της $\fS_n$ που αντιστοιχεί στο πρότυπο Specht $\sS^\lambda$ και $\chi^\lambda$ ο χαρακτήρας της. Υποθέτουμε ότι $n=4$.
    \begin{itemize}
        \item[(1)] Για κάθε $\lambda \vdash 4$, να βρείτε τις τιμές του χαρακτήρα $\chi^\lambda$ υπολογίζοντας τους πίνακες $\rho^\lambda(\pi)$, για κάθε $\pi \in \fS_4$, στην βάση $\{\bfe_T : T \in \SYT(\lambda)\}$.
        \item[(2)] Χρησιμοποιώντας το (1), γράψτε τον πίνακα χαρακτήρων της $\sS_4$.
    \end{itemize} 
    \textcolor{lightgray}{\small{\emph{Υπόδειξη}: Για κάθε διαμέριση $\lambda$, αρκεί να υπολογίσετε τους πίνακες $\rho^\lambda$ για τις γειτονικές αντιμεταθέσεις και μετά να συνεχίσετε (γιατί;).}}
\end{exercise}

\begin{exercise}
    Έστω $\Pi_4$ το σύνολο όλων των διαμερίσεων του $[4]$. Να υπολογίσετε την ισοτυπική διάσπαση (σε πρότυπα Specht της $\fS_4$) της αναπαράστασης μεταθέσεων που επάγεται από τη δράση καθορισμού της $\fS_4$ στο $\Pi_4$.
\end{exercise}

\begin{exercise}
    Έστω $1 \le k \le n/2$ και $S$ το σύνολο όλων των υποσυνόλων του $[n]$ με $k$ στοιχεία. 
    \begin{itemize}
        \item[(1)] Να δείξετε ότι το πρότυπο Young $\rmM^{(n-k,k)}$ είναι ισόμορφο με το πρότυπο της αναπαράστασης μεταθέσεων που επάγεται από τη δράση καθορισμού της $\fS_n$ στο $S$.
        \item[(2)] Να υπολογίσετε το $f^{(n-k,k)}$, χωρίς να χρησιμοποιήσετε την hook-length formula.
        \item[(3)] Συμπεράνετε ότι $f^{(n,n)} = \frac{1}{n+1}\binom{2n}{n}$.
    \end{itemize}
    \textcolor{lightgray}{\small{\emph{Υπόδειξη}: Για το (2), το Πόρισμα 11.13 (σε συνδυασμό με το (1)) μπορεί να φανεί χρήσιμο.}}
\end{exercise}

% \begin{exercise}{(\texttt{Απαρίθμηση συνήθων Young ταμπλώ})}
%     \leavevmode
%     \begin{itemize}
%         \item[(1)] Για κάθε $1 \le k \le n-1$, να δείξετε ότι $f^{(n-k, 1^k)} = \binom{n-1}{k}$.
%         \item[(2)] Συμπεράνετε ότι τo πλήθος των συνήθων Young ταμπλώ περιεχομένου $[n]$ που έχουν σχήμα γάντζου ισούται με $2^{n-1}$. 
%         \item[(3)] Για κάθε $0 \le k \le n/2$, να δείξετε ότι $f^{(n-k,k)} = \binom{n}{k} - \binom{n}{k-1}$.
%         \item[(4)] Συμπεράνετε ότι $f^{(n,n)} = \frac{1}{n+1}\binom{2n}{n}$.
%         \item[(5)] Συμπεράνετε ότι τo πλήθος των συνήθων Young ταμπλώ περιεχομένου $[n]$ που έχουν το πολύ δύο γραμμές ισούται με $\binom{n}{\lfloor n/2 \rfloor}$.
%     \end{itemize}
% \end{exercise}

\begin{exercise}{(\texttt{Μεταβατικότητα της επαγωγής και του περιορισμού})} \\
    Έστω $G$ πεπερασμένη ομάδα, $K$ μια υποομάδα της $G$ και $H$ μια υποομάδα της $K$. 
    \begin{itemize}
        \item[(1)] Αν $\chi$ είναι ένας χαρακτήρας της $G$, να δείξετε ότι 
        \[
        \left(\chi\downarrow_K^G\right)\!\downarrow_H^K  \ = \chi\downarrow_H^G.
        \]
        \item[(2)] Αν $\psi$ είναι ένας χαρακτήρας της $H$, να δείξετε ότι 
        \[
        \left(\psi\uparrow_H^K\right)\uparrow_K^G \ = 
        \psi\uparrow_H^G
        \]
        χρησιμοποιώντας τον νόμο αντιστροφής Frobenius (βλ. Θεώρημα 8.8) και το (1).
        \item [(3)] Για την διαμέριση $\lambda = (4,4,3,1) \vdash 12$, να υπολογίσετε την ισοτυπική διάσπαση των $\sS^\lambda\!\downarrow_{\fS_9}^{\fS_{12}}$ και $\sS^\lambda\!\uparrow_{\fS_{12}}^{\fS_{15}}$.
    \end{itemize}
    
\end{exercise}

\begin{exercise}{(\texttt{H Ταυτότητα Frobenius-Young})} \\ 
    Για κάθε $\lambda = (\lambda_1, \lambda_2, \dots, \lambda_k) \vdash n$, να δείξετε ότι
    \[
    f^\lambda = \frac{n!}{\ell_1!\ell_2\cdots\ell_k!}\prod_{1 \le i < j \le k} (\ell_i - \ell_j),
    \]
    όπου $\ell_i \coloneqq \lambda_i + k -i$, για κάθε $1 \le i \le k$.

    \textcolor{lightgray}{\small{\emph{Υπόδειξη}: Χρησιμοποιήστε την hook-length formula (βλ. Θεώρημα 12.8).}}
\end{exercise}

Η επόμενη άσκηση είναι προεραιτική.
\begin{exercise}{(\texttt{Στοιχεία Jucys-Murphy})}\\
    Για κάθε $1 \le k \le n$, έστω
    \[
    \rmm_k \coloneqq \cycle{1,k} + \cycle{2,k} + \cdots + \cycle{k-1,k} \ \in \CC[\fS_n],
    \]
    όπου $\rmm_1 \coloneqq 0$ και 
    \[
    \nabla_{[k]} \coloneqq \sum_{\substack{\pi \in \fS_n \\ \pi(i) = i, \ \text{για $i \notin [k]$}}} \pi \ \in \CC[\fS_n].
    \] 
    Τα $\rmm_1, \rmm_2, \dots, \rmm_n$ ονομάζονται \defn{στοιχεία Jucys--Murphy} του $\CC[\fS_n]$. 
    \begin{itemize}
        \item[(0)] Γράψτε όλα τα $\rmm_k$ και $\nabla_{[k]}$ για $n=3$ και $n=4$.
        \item[(1)] Για κάθε $1 \le k \le n$, να δείξετε ότι 
        \[
        \nabla_{[k]} = \nabla_{[k-1]}\left(\epsilon + \rmm_k\right).
        \]
        \item[(2)] Να δείξετε ότι 
        \[
        \nabla_n \coloneqq (\epsilon + \rmm_1)(\epsilon + \rmm_2)\cdots(\epsilon + \rmm_n),
        \]
        όπου $\nabla_n \coloneqq \nabla_{[n]} = \sum_{\pi \in \fS_n}\pi$.
    \end{itemize}
\end{exercise}
\end{document}