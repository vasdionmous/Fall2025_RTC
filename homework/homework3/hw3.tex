\documentclass[12pt,a4paper,reqno]{amsart}

% language
\usepackage[greek, english]{babel}
\usepackage[utf8]{inputenc}
\usepackage{alphabeta}

% change default names to greek
\addto\captionsenglish{
    \renewcommand{\contentsname}{Περιεχόμενα}
    \renewcommand{\refname}{Βιβλιογραφία}
    \renewcommand{\datename}{Ημερομηνία:}
    \renewcommand{\urladdrname}{Ιστοσελίδα}
}

% math 
\usepackage{amsmath,amsthm,amssymb,amscd}

% font
\usepackage[cal=euler,frak=mt]{mathalfa}
\usepackage{libertinus-type1}
\usepackage{txfonts} % for upright greek letters
\usepackage{bm} % for bold symbols
\usepackage{bbm} % for the simply-looking bb symbols

% miscellaneous 
\usepackage{changepage} %for indenting environments
\usepackage{csquotes} % example: \textcquote{}

% colors
\usepackage{xcolor}
\definecolor{darkcandyapplered}{rgb}{0.64, 0.0, 0.0}
\definecolor{midnightblue}{rgb}{0.1, 0.1, 0.44}
\definecolor{mylightblue}{HTML}{336699}
\definecolor{lightgray}{rgb}{0.83, 0.83, 0.83}

% hrefs
\usepackage{hyperref}
\usepackage[noabbrev,capitalize]{cleveref}
\hypersetup{
    pdftoolbar=true,        
    pdfmenubar=true,        
    pdffitwindow=false,     
    pdfstartview={FitH},  % fits the width of the page to the window
    pdftitle={},
    pdfauthor={},
    pdfsubject={},
    pdfkeywords={},
    pdfnewwindow=true,  % links in new window
    colorlinks=true,  % false: boxed links; true: colored links
    linkcolor=darkcandyapplered,   % color of internal links
    citecolor=midnightblue,  % color of links to bibliography
    urlcolor=cyan,  % color of external links
    linktocpage=true  % changes the links from the section body to the page number
    }

% geometry
\textwidth=16cm 
\textheight=21cm 
\hoffset=-55pt 
\footskip=25pt

% thm envs
% In this macro I define all the theorem environments

\theoremstyle{definition}
\newtheorem{theorem}{Θεώρημα}
\newtheorem{proposition}[theorem]{Πρόταση}
\newtheorem{lemma}[theorem]{Λήμμα}
\newtheorem{corollary}[theorem]{Πόρισμα}
\newtheorem{conjecture}[theorem]{Εικασία}
\newtheorem{problem}[theorem]{Πρόβλημα}
\newtheorem*{claim}{Ισχυρισμός}
\newtheorem{observation}[theorem]{Παρατήρηση}
\newtheorem{definition}[theorem]{Ορισμός}
\newtheorem{question}[theorem]{Ερώτηση}
\newtheorem*{questions}{Ερωτήματα}
\newtheorem{example}[theorem]{Παράδειγμα}
\newtheorem{exercise}{Άσκηση}

\newtheorem*{combInterlude}{Ιντερλούδιο Συνδυαστικής}
\newtheorem*{example_cont}{Παράδειγμα~6.6}
\newtheorem*{digression_la}{Παρέκβαση Γραμμικής Άλγεβρας}
\newtheorem*{thm}{Θεώρημα}

\theoremstyle{remark}
\newtheorem*{remark}{Παρατήρηση}

% fixes the correct numbering of environments
\numberwithin{theorem}{section}
\numberwithin{exercise}{section}
\numberwithin{equation}{section}

% math ops
% In this macro I define all of my math operators

% fields
\newcommand{\NN}{\mathbbmss{N}} 
\newcommand{\ZZ}{\mathbbmss{Z}} 
\newcommand{\QQ}{\mathbbmss{Q}} 
\newcommand{\RR}{\mathbbmss{R}} 
\newcommand{\CC}{\mathbbmss{C}} 
\newcommand{\KK}{\mathbbmss{K}} 
\newcommand{\FF}{\mathbbmss{F}} 

% symmetric group
\newcommand{\fS}{\mathfrak{S}}  

% calligraphic 
\newcommand{\aA}{\mathcal{A}} 
\newcommand{\bB}{\mathcal{B}}
\newcommand{\cC}{\mathcal{C}}
\newcommand{\dD}{\mathcal{D}}
\newcommand{\eE}{\mathcal{E}}
\newcommand{\fF}{\mathcal{F}}
\newcommand{\hH}{\mathcal{H}}
\newcommand{\iI}{\mathcal{I}}
\newcommand{\lL}{\mathcal{L}}
\newcommand{\oO}{\mathcal{O}}
\newcommand{\pP}{\mathcal{P}}
\newcommand{\sS}{\mathcal{S}}
\newcommand{\mM}{\mathcal{M}}
\newcommand{\uU}{\mathcal{U}}

% bold
\newcommand{\bfa}{\mathbf{a}}
\newcommand{\bfe}{\mathbf{e}}
\newcommand{\bfF}{\pmb{F}}
\newcommand{\bfR}{\pmb{R}}
\newcommand{\bfv}{\mathbf{v}}
%\newcommand{\bfx}{\bm{x}}
%\newcommand{\bfx}{\mathbf{x}} 
\newcommand{\bfx}{\pmb{x}}
\newcommand{\bfX}{\pmb{X}}
\newcommand{\bfy}{\pmb{y}}
\newcommand{\bfz}{\pmb{z}}

% roman
\newcommand{\rmA}{\mathrm{A}}
\newcommand{\rmB}{\mathrm{B}}
\newcommand{\rmC}{\mathrm{C}}
\newcommand{\rmD}{\mathrm{D}} 
\newcommand{\rmI}{\mathrm{I}} 
\newcommand{\rmK}{\mathrm{K}}
\newcommand{\rmM}{\mathrm{M}}
\newcommand{\rmP}{\mathrm{P}}  
\newcommand{\rmp}{\mathrm{p}}  
\newcommand{\rmQ}{\mathrm{Q}}  
\newcommand{\rmR}{\mathrm{R}}
\newcommand{\rmS}{\mathrm{S}}
\newcommand{\rmT}{\mathrm{T}}
\newcommand{\rmU}{\mathrm{U}}
\newcommand{\rmV}{\mathrm{V}}
\newcommand{\rmY}{\mathrm{Y}}
\newcommand{\rmZ}{\mathrm{Z}}
\newcommand{\rmz}{\mathrm{z}}

% greek letters
% I'm renewing some commands in order to appear in upright font
% If I want to change it later, I don't have to do it manually, I just change it from here.
% \newcommand{\uaa}{\alphaup}
% \renewcommand{\alpha}{\alphaup}
% \renewcommand{\beta}{\betaup}
% \renewcommand{\gamma}{\gammaup}
% \renewcommand{\delta}{\deltaup}
% \renewcommand{\epsilon}{\epsilonup}
% \newcommand{\ee}{\epsilon}
% \renewcommand{\varepsilon}{\varepsilonup}
% \renewcommand{\theta}{\thetaup}
% \renewcommand{\lambda}{\lambdaup}
% \newcommand{\ull}{\lambda}
% \renewcommand{\mu}{\muup}
% \renewcommand{\nu}{\nuup}
% \renewcommand{\pi}{\piup}
% \renewcommand{\rho}{\rhoup}
% \renewcommand{\varrho}{\varrhoup}
% \renewcommand{\sigma}{\sigmaup}
% \renewcommand{\tau}{\tauup} 
% \renewcommand{\phi}{\phiup}
% \renewcommand{\chi}{\chiup}
% \renewcommand{\psi}{\psiup}
% \renewcommand{\omega}{\omegaup}

% arrows and symbols 
\renewcommand{\to}{\rightarrow}
\newcommand{\toto}{\longrightarrow}
\newcommand{\mapstoto}{\longmapsto}
\newcommand{\then}{\Rightarrow}
\newcommand{\IFF}{\Leftrightarrow}
\newcommand{\tl}{\tilde}
\newcommand{\wtl}{\widetilde}
\newcommand{\ol}{\overline}
\newcommand{\ul}{\underline}
\newcommand{\oldemptyset}{\emptyset}
\renewcommand{\emptyset}{\varnothing}
\DeclareMathSymbol{\Arg}{\mathbin}{AMSa}{"39} % for arguments 
\newcommand{\onto}{\ensuremath{\twoheadrightarrow}}
\newcommand{\tle}{\trianglelefteq}
\newcommand{\tge}{\trianglerighteq}

% absolute value symbol
\usepackage{mathtools} 
\DeclarePairedDelimiter\abs{\lvert}{\rvert}%
\DeclarePairedDelimiter\norm{\lVert}{\rVert}%
\makeatletter
\let\oldabs\abs
\def\abs{\@ifstar{\oldabs}{\oldabs*}}

% tensor symbol
\newcommand{\tensor}[1]{%
  \mathbin{\mathop{\otimes}\limits_{#1}}%
}

% permutation cycle notation
\ExplSyntaxOn
\NewDocumentCommand{\cycle}{ O{\;} m }
 {
  (
  \alec_cycle:nn { #1 } { #2 }
  )
 }

\seq_new:N \l_alec_cycle_seq
\cs_new_protected:Npn \alec_cycle:nn #1 #2
 {
  \seq_set_split:Nnn \l_alec_cycle_seq { , } { #2 }
  \seq_use:Nn \l_alec_cycle_seq { #1 }
 }
\ExplSyntaxOff

% setminus symbol
\newcommand{\mysetminusD}{\hbox{\tikz{\draw[line width=0.6pt,line cap=round] (3pt,0) -- (0,6pt);}}}
\newcommand{\mysetminusT}{\mysetminusD}
\newcommand{\mysetminusS}{\hbox{\tikz{\draw[line width=0.45pt,line cap=round] (2pt,0) -- (0,4pt);}}}
\newcommand{\mysetminusSS}{\hbox{\tikz{\draw[line width=0.4pt,line cap=round] (1.5pt,0) -- (0,3pt);}}}
\newcommand{\sm}{\mathbin{\mathchoice{\mysetminusD}{\mysetminusT}{\mysetminusS}{\mysetminusSS}}}

% custom math operators
\newcommand{\Des}{\mathrm{Des}} 
\newcommand{\des}{\mathrm{des}} 
\newcommand{\Asc}{\mathrm{Asc}}
\newcommand{\asc}{\mathrm{asc}} 
\newcommand{\inv}{\mathrm{inv}}
\newcommand{\Inv}{\mathrm{Inv}}
\newcommand{\maj}{\mathrm{maj}} 
\newcommand{\comaj}{\mathrm{comaj}} 
\newcommand{\fix}{\mathrm{fix}} 
\newcommand{\Sym}{\mathrm{Sym}} 
\newcommand{\QSym}{\mathrm{QSym}}
\newcommand{\FQSym}{\mathrm{FQSym}} 
\newcommand{\End}{\mathrm{End}} 
\newcommand{\Rad}{\mathrm{Rad}} 
\newcommand{\rmMat}{\mathrm{Mat}} 
\newcommand{\rmdim}{\mathrm{dim}} 
\newcommand{\rmTop}{\mathrm{Top}} 
\newcommand{\rmCF}{\mathrm{CF}} 
\newcommand{\rmId}{\mathrm{Id}}
\newcommand{\rmid}{\mathrm{id}}
\newcommand{\rmtw}{\mathrm{tw}}
\newcommand{\trace}{\mathrm{tr}}
\newcommand{\Irr}{\mathrm{Irr}}
\newcommand{\Ind}{\mathrm{Ind}} % induction
\newcommand{\Res}{\mathrm{Res}} % restriction
\newcommand{\triv}{\mathrm{triv}} % trivial rep
\newcommand{\rmdef}{\mathrm{def}} % defining rep
\newcommand{\dom}{\triangleleft}
\newcommand{\domeq}{\trianglelefteq}
\newcommand{\lex}{\mathrm{lex}}
\newcommand{\sign}{\mathrm{sign}}
\newcommand{\SYT}{\mathrm{SYT}}
\renewcommand{\Im}{\mathrm{Im}}
\newcommand{\Ker}{\mathrm{Ker}}
\newcommand{\GL}{\mathrm{GL}}
\newcommand{\FL}{\mathrm{FL}}
\newcommand{\Span}{\mathrm{span}}
\newcommand{\pos}{\mathrm{pos}}
\newcommand{\Comp}{\mathrm{Comp}}
\newcommand{\Set}{\mathrm{Set}}
\newcommand{\std}{\mathrm{std}}
\newcommand{\cont}{\mathrm{cont}} %content of a SSYT
\newcommand{\SSYT}{\mathrm{SSYT}}
\newcommand{\ct}{\mathrm{ct}} % cycle type
\newcommand{\ch}{\mathrm{ch}} % Frobenius characteristic map
\newcommand{\height}{\mathrm{ht}}
\newcommand{\FPS}{\CC[\![\bfx]\!]} % formal power series
\newcommand{\FPSS}{\CC[\![\bfx,\bfy]\!]}
\newcommand{\reg}{\mathrm{reg}}
\newcommand{\hook}{\mathrm{h}}
\newcommand{\weight}{\mathrm{wt}}
\newcommand{\co}{\mathrm{co}}
\newcommand{\ps}{\mathrm{ps}}
\newcommand{\rmsum}{\mathrm{sum}}
\newcommand{\NSym}{\mathrm{NSym}}
\newcommand{\Hom}{\mathrm{Hom}}
\newcommand{\proj}{\mathrm{proj}}
\newcommand{\stat}{\mathrm{stat}}
\newcommand{\Par}{\mathrm{Par}}
\newcommand{\rmset}{\mathrm{set}}
\newcommand{\comp}{\mathrm{comp}}

% miscellaneous commands
\newcommand{\defn}[1]{{\color{mylightblue}{#1}}}
\newcommand{\toDo}{{\bf\color{red} TODO}}
\newcommand{\toCite}{{\bf\color{green} CITE}}

% redefine the titlepage labels


% titlepage
\title[]{Θ2.04: Θεωρία Αναπαραστάσεων και Συνδυαστική \\ Φυλλάδιο Ασκήσεων 3}
% \author[]{9 Οκτωβρίου 2025}
\date{13 Νοεμβρίου 2025}

\begin{document}
\begingroup
\def\uppercasenonmath#1{} % this disables uppercase title
\let\MakeUppercase\relax % this disables uppercase authors
\maketitle
\endgroup

\setcounter{section}{3}
\thispagestyle{empty}

Σε ότι ακολουθεί $n$ είναι ένας θετικός ακέραιος, $G$ είναι μια πεπερασμένη ομάδα και όλοι οι διανυσματικοί χώροι είναι πεπερασμένης διάστασης πάνω από το $\CC$.

\begin{exercise}{(\texttt{Χώρος πηλίκο ως αναπαράσταση})}\\
    Έστω $V$ ένα $G$-πρότυπο και $W$ ένα υποπρότυπο του $V$. Να δείξετε τα εξής.
    \begin{itemize}
        \item[(1)] Ο (διανυσματικός) χώρος πηλίκο $V/W$ γίνεται $G$-πρότυπο θέτοντας 
        \[
        g \cdot\left(v + W\right) = gv + W
        \]
        για κάθε $g \in G$ και $v \in V$.
        \item[(2)] Έχουμε $V \cong_G W \oplus \left(V/W\right)$ (ως $G$-πρότυπα).
    \end{itemize}

    \textcolor{lightgray}{\small{\emph{Υπόδειξη}: Για το (2), μπορείτε να μιμηθείτε την απόδειξη του Λήμματος 7.3 ώστε να βρείτε μια $G$-αναλλοίωτη προβολή του $V$ με εικόνα $W$ και πυρήνα ισόμορφο με το $V/W$.}}
\end{exercise}

\begin{exercise}{(\texttt{Ορίζουσα του πίνακα χαρακτήρων})} \\
    Έστω $\rmK_1, \rmK_2, \dots, \rmK_r$ οι κλάσεις συζυγίας της $G$. Αν $X = (\chi(\rmK_j))_{\chi,j}$ είναι ο πίνακας χαρακτήρων της $G$, όπου το $\chi$ διατρέχει όλους τους διαφορετικούς ανάγωγους χαρακτήρες της $G$ και $1 \le j \le r$, τότε να δείξετε ότι 
    \[
    \abs{\det(X)}^2 = \frac{\abs{G}^r}{\abs{\rmK_1}\abs{\rmK_2}\cdots\abs{\rmK_r}}.
    \]

    \textcolor{lightgray}{\small{\emph{Υπόδειξη}: Χρησιμοποιήστε την ιδιότητα $\det(AB) = \det(A)\det(B)$ και τις σχέσεις ορθογωνιότητας ΙΙ (βλ. Θεώρημα 7.9).}}
\end{exercise}

\begin{exercise}{(\texttt{Εναλλαγή περιορισμού και επαγωγής})} \\
    Έστω $H$ μια υποομάδα της $G$ και $\chi$ (αντ. $\psi$) ένας χαρακτήρας της $G$ (αντ. $H$). Να δείξετε ότι 
    \[
    \left(\psi \, \left(\chi\downarrow_H\right) \right)\uparrow^G = \psi\uparrow^G \chi.
    \]

    \textcolor{lightgray}{\small{\emph{Υπόδειξη}: Υπολογίστε το εσωτερικό γινόμενο καθενός από τα δυο μέλη με έναν αυθαίρετο ανάγωγο χαρακτήρα της $G$ και χρησιμοποιήστε το Θεώρημα Αντιστροφής του Frobenius (βλ. Θεώρημα 8.8).}}
\end{exercise}

% \begin{exercise}{(\texttt{Ο περιορισμός και η επαγωγή είναι μεταβατικές})} \\
%     Έστω $H$ υποομάδα της $G$ και $K$ υποομάδα της $H$. Αν $\chi$ (αντ. $\psi$) είναι ένας χαρακτήρας της $G$ (αντ. $K$), τότε να δείξετε ότι 
%     \begin{align*}
%     \chi\downarrow_K^G &= \left(\chi\downarrow_H^G\right)\downarrow_K^H \\
%     \psi\uparrow_K^G &= \left(\psi\uparrow_K^H\right)\uparrow_K^G.
%     \end{align*}
% \end{exercise}

\begin{exercise}
    Έστω $\chi$ ένας χαρακτήρας του κέντρου $\rmZ(G)$ της $G$ και $m = \frac{\abs{G}}{\abs{\rmZ(G)}}$. Να δείξετε ότι 
    \[
    \chi\uparrow_{\rmZ(G)}^G(g) = 
    \begin{cases}
        m\chi(g), &\ \text{αν $g \in \rmZ(G)$} \\
        0, &\ \text{διαφορετικά}
    \end{cases}
    \]
    για κάθε $g \in G$.

    \textcolor{lightgray}{\small{\emph{Υπόδειξη}: Χρησιμοποιήστε την Πρόταση 8.5.}}
\end{exercise}

\newpage

\begin{exercise}{(\texttt{Σχέσεις Coxeter})} \\
    \label{ex:coxeter_relations}
    Έστω $s_i \coloneqq \cycle{i, i+1}$, για κάθε $1 \le i \le n-1$ και $\epsilon$ η ταυτοτική μετάθεση. Να δείξετε τα εξής.
    \begin{itemize}
        \item[(1)] Το σύνολο $\{s_1, s_2, \dots, s_{n-1}\}$ παράγει την $\fS_n$.
    \item[(2)] Οι γειτονικές αντιμεταθέσεις ικανοποιούν τις σχέσεις 
    \begin{align*}
    s_i^2 &= \epsilon,  \hspace{-70pt}& &\text{για κάθε $1 \le i \le n-1$}\\
    s_is_{i+1}s_i &= s_{i+1}s_is_{i+1}, \hspace{-70pt}&  &\text{για κάθε $1 \le i \le n-2$} \\
    s_is_j &= s_js_i, \hspace{-70pt}&  &\text{για κάθε $\abs{i-j}>1$}. 
    \end{align*}
    \end{itemize}

    \textcolor{lightgray}{\small{\emph{Υπόδειξη}: Για το (1), χρησιμοποιήστε την Πρόταση 9.2 και το γεγονός ότι η $\fS_n$ παράγεται από όλες τις αντιμεταθέσεις. Για παράδειγμα, ποιό είναι το $s_2s_1s_2^{-1}$;}}
\end{exercise}

\begin{exercise}{(\texttt{Αντιστροφές και πρόσημο})} \\
    Για $\pi \in \fS_n$, το ζεύγος $(i,j)$ με $1 \le i < j \le n$ τέτοια ώστε $\pi(i) > \pi(j)$ ονομάζεται \emph{αντιστροφή} (inversion) της $\pi$. Έστω $\Inv(\pi)$ το σύνολο όλων των αντιστροφών της $\pi$ και $\inv(\pi) \coloneqq \abs{\Inv(\pi)}$. Να δείξετε τα εξής.
    \begin{itemize}
        \item[(1)] Αν $s_i$ όπως στην Άσκηση~\ref{ex:coxeter_relations}, τότε
        \[
        \inv(\pi s_i) = 
        \begin{cases}
            \inv(\pi) + 1, &\ \text{αν $\pi_i < \pi_{i+1}$} \\
            \inv(\pi) - 1, &\ \text{αν $\pi_i > \pi_{i+1}$} \\
        \end{cases}
        \]
        για κάθε $1 \le i \le n-1$ και $\pi \in \fS_n$.
        \item[(2)] Κάθε $\pi \in \fS_n$ μπορεί να γραφεί ως γινόμενο $\inv(\pi)$ το πλήθος γειτονικών αντιμεταθέσεων.
        \item[(3)] Για κάθε $\pi \in \fS_n$, 
        \[
        \sign(\pi) = (-1)^{\inv(\pi)}.
        \]
        \item[(4)] Για κάθε $\pi, \sigma \in \fS_n$, 
        \begin{align*}
            \sign(\pi\sigma) &= \sign(\pi)\sign(\sigma) \\
            \sign(\pi^{-1}) &= \sign(\pi). \\
        \end{align*}
        \item[(5)] Ένας $k$-κύκλος είναι άρτιος αν και μόνο αν το $k$ είναι περιττός.
    \end{itemize}

    
    \textcolor{lightgray}{\small{\emph{Υπόδειξη}: Για το (1), παρατηρήστε ότι η $\pi s_i$ είναι η μετάθεση που προκύπτει από την $\pi = \pi_1 \pi_2 \cdots \pi_n$ εναλλάσσοντας τα $\pi_i$ και $\pi_j$.}}
\end{exercise}
\end{document}